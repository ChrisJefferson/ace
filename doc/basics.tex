%%%%%%%%%%%%%%%%%%%%%%%%%%%%%%%%%%%%%%%%%%%%%%%%%%%%%%%%%%%%%%%%%%%%%%%%%
%%
%W  basics.tex          ACE documentation - basics       Alexander Hulpke
%W                                                      Joachim Neub"user
%W                                                            Greg Gamble
%%
%H  $Id$
%%
%Y  Copyright (C) 2000  Centre for Discrete Mathematics and Computing
%Y                      Department of Computer Science & Electrical Eng.
%Y                      University of Queensland, Australia.
%%

%%%%%%%%%%%%%%%%%%%%%%%%%%%%%%%%%%%%%%%%%%%%%%%%%%%%%%%%%%%%%%%%%%%%%%%%%
\Chapter{Some Basics}

Throughout this manual for  the  use  of  {\ACE}  as  a  {\GAP}  share
package, we shall assume that the reader already knows the basic ideas
of coset enumeration, as can be  found  for  example  in~\cite{Neu82}.
There, a simple proof is given for the fact that a  coset  enumeration
for a subgroup of finite index in  a  finitely  presented  group  must
eventually terminate with the correct result, provided the enumeration
process obeys a simple condition (Mendelsohn's  condition)  formulated
in Lemma~1 and Theorem~2 of~\cite{Neu82}. This basic condition  leaves
room for a great variety  of  *strategies*\index{strategy}  for  coset
enumeration; two \lq{}classical' ones have been known for a long  time
as the *Felsch  strategy*\atindex{Felsch  strategy}{@Felsch  strategy}
and  the  *HLT  strategy*\atindex{HLT  strategy}{@HLT  strategy}  (for
Haselgrove, Leech and Trotter). Extensive experimental studies on many
strategies can be found in~\cite{CDHW73}, \cite{Hav91},  \cite{HR99a},
and \cite{HR99b}, in particular.

A few basic points should be particularly understood:

\beginlist

\item{--} \lq{}Subgroup (generator) and relator tables' that are  used
in the description of coset enumeration in \cite{Neu82}, and to  which
we will also occasionally refer in this manual,  do  *not*  physically
exist in the implementation of coset  enumeration  in  {\ACE}.  For  a
terminology that is closer to the actual implementation  and  also  to
the  formulations  in  the  manual  for  the  {\ACE}  standalone   see
\cite{CDHW73} and \cite{Hav91}.

\item{--} Coset    enumeration    proceeds    by    defining    *coset
numbers*\index{coset   numbers}   that    really    denote    possible
representatives for cosets written as words in the generators  of  the
group. At the time of their generation it is not guaranteed  that  any
two of these words do indeed represent different cosets. The state  of
an enumeration at any time is stored in a 2-dimensional array known as
a *coset table*\index{coset table} whose rows  are  indexed  by  coset
numbers and whose columns are indexed  by  the  group  generators  and
their inverses. Entries of the coset table that are  not  yet  defined
are known as *holes*\index{holes} (typically they are filled  with  0,
i.e.~an invalid coset number).

\item{--} It is customary in  talking about coset enumeration to speak
of *cosets*\index{cosets} when really coset numbers are  meant.  While
we try to avoid  this  in  this  interface  manual,  in  certain  word
combinations such as *coset application*\index{coset  application}  we
will follow this custom.

\item{--} The   definition   of   a   coset   number   may   lead   to
*deductions*\index{deduction}  from  the  \lq{}closing  of   rows   in
subgroup  or  relator  tables'.  These  are  kept  in   a   *deduction
stack*\index{deduction stack}.

\item{--} Also   it  may  be  found  that  (different)  words  in  the
generators defining different coset numbers really  lie  in  the  same
coset    of    the    given    subgroup.    This    is    called     a
*coincidence*\index{coincidence}  and  will  eventually  lead  to  the
elimination of the  larger  of  the  two  coset  numbers.  Until  this
elimination has been performed pending  coincidences  are  kept  in  a
*queue of coincidences*\index{coincidence queue}.

\item{--} A definition that will actually close a row in a subgroup or
relator table will immediately yield twice  as  many  entries  in  the
coset  table  as  other  definitions.  Such  definitions  are   called
*preferred definitions*\index{preferred definitions},  the  places  in
rows of a subgroup or relator table that they close are also  referred
to as \lq{}gaps of length one' or minimal gaps. Such gaps can be found
at little extra cost when \lq{}relators are traced from a given  coset
number'. {\ACE} keeps a selection of them in a  *preferred  definition
stack*\index{preferred definition stack} for use  in  some  definition
strategies (see~\cite{Hav91}).

\endlist

It will also be necessary to understand some further basic features of
the  implementation and  the corresponding  terminology which  we will
explain in the sequel.

%%%%%%%%%%%%%%%%%%%%%%%%%%%%%%%%%%%%%%%%%%%%%%%%%%%%%%%%%%%%%%%%%%%%%%%%%
\Section{Enumeration  Style}

The first main decision for any coset enumeration is in which sequence
to make definitions. When a new coset number has  to  be  defined,  in
{\ACE} there are basically three possible methods to choose from:

\beginlist

\item{--} One may fill the next empty entry  in  the  coset  table  by
scanning from the left/top of the coset table towards the right/bottom
--  that  is,  in  order  row  by  row.  This   is   called   *C-style
definition*\atindex{C-style  definition}{@C-style   definition}   (for
*C*oset Table Based definition) of coset numbers. In fact a  procedure
needs to follow a method like this to some extent for the proofs  that
coset enumeration eventually terminates in the case  of  finite  index
(see~\cite{Neu82}).

\item{--} In *R-style definition*\atindex{R-style definition}{@R-style
definition} (for *R*elator Based definition) the order in which  coset
numbers are defined is explicitly prescribed by  the  order  in  which
rows of (the subgroup generator tables and)  the  relator  tables  are
filled by making definitions.

\item{--} One may choose  definitions  from  a  *Preferred  Definition
Stack*\index{preferred definition stack}. In this stack  possibilities
for definition of coset numbers are stored that will close  a  certain
row    of    a    relator    table.     Using     these     *preferred
definitions*\index{preferred definitions} is sometimes  also  referred
to as a *minimal gaps  strategy*\index{strategy,  minimal  gaps}.  The
idea of using these is that by closing a row in a relator table, thus,
one will immediately get a consequence.  We  will  come  back  to  the
obvious question of where one obtains this  \lq{}preferred  definition
stack'.

\endlist

The *enumeration style* is mainly determined by  the  balance  between
C-style definitions and R-style definitions, which  is  controlled  by
the values of the `ct' and `rt' options (see~"option  ct"  and~"option
rt").

However this still leaves us with plenty of freedom for the design  of
definition strategies, freedom which can,  for  example,  be  used  to
great advantage in Felsch-type strategies. Though it is  not  strictly
necessary,  before  embarking  on  further  enumeration,   Felsch-type
programs generally start off  by  ensuring  that  each  of  the  given
subgroup generators produces a cycle of coset numbers at coset  1.  To
explain the idea, an example may help. Suppose  $a,b$  are  the  group
generators and $w=Abab$ is a subgroup generator, where $A$  represents
the inverse of $a$; then to say that \lq{}$(1,i,j,k)$ is  a  cycle  of
coset numbers produced at coset 1 by $w$' means  that  the  successive
application  of  the  \lq{}letters'  $A,b,a,b$  of   $w$   takes   one
successively from coset 1, through cosets $i$, $j$ and $k$,  and  back
to coset 1, i.e.~$A$ applied to coset 1  results  in  coset  $i$,  $b$
applied to coset $i$ results in coset $j$, $a$ applied  to  coset  $j$
results in coset $k$, and finally $b$ applied to coset  $k$  takes  us
back to coset $1$. In this  way,  a  hypothetical  subgroup  table  is
filled first. The use of this and other  possibilities  leads  to  the
following table of *enumeration styles*.

% \begin{table}
% \hrule
% \caption{The styles}
% \label{tab:sty}
% \smallskip
% \renewcommand{\arraystretch}{0.875}
% \begin{tabular*}{\textwidth}{@{\extracolsep{\fill}}crrlc} 
% \hline\hline
% & \ttt{Rt} value & \ttt{Ct} value & style name & \\
% \hline
% & $<\!0$ & $<\!0$ & R/C & \\
% & $<\!0$ & $0$    & R*  & \\
% & $<\!0$ & $>\!0$ & Cr  & \\
% & $0$    & $<\!0$ & C   & \\
% & $0$    & $0$    & R/C (defaulted) & \\
% & $0$    & $>\!0$ & C  & \\
% & $>\!0$ & $<\!0$ & Rc & \\
% & $>\!0$ & $0$    & R  & \\
% & $>\!0$ & $>\!0$ & CR & \\
% \hline\hline
% \end{tabular*}
% \end{table}
\begintt
Rt value     Ct value     style name
-----------------------------------------

   0           >0         C
  <0           >0         Cr
  >0           >0         CR

  >0            0         R
  <0            0         R*
  >0           <0         Rc
  <0           <0         R/C
   0            0         R/C (defaulted)

-----------------------------------------
\endtt

In *C-style*\atindex{C-style}{@C-style} most definitions are  made  in
the next empty coset table slot and are (in principle) tested  in  all
essentially different  positions  in  the  relators;  i.e.~this  is  a
Felsch-like mode.

However in C-Style some definitions may be made following a  preferred
definition strategy, controlled by the  `pmode'  and  `psize'  options
(see~"option pmode" and~"option psize").

*Cr-style*\atindex{Cr-style}{@Cr-style} is like C-style except that  a
single R-style pass is done after the initial C-style pass.

In *CR-style*\atindex{CR-style}{@CR-style} alternate passes of C-style
and R-style are performed.

In *R-style*\atindex{R-style}{@R-style} all the definitions  are  made
via relator scans; i.e.~this is an HLT-like mode.

*R\*-style*\atindex{R\*-style}{@R\*-style} makes definitions the  same
as R-style, but tests all definitions as for C-style.

*Rc-style*\atindex{Rc-style}{@Rc-style} is like R-style, except that a
single C-style pass is done after the initial R-style pass.

In *R/C-style*\atindex{R/C-style}{@R/C-style} we run in R-style  until
an overflow, perform a lookahead on the entire table, and then  switch
to CR-style.

*Defaulted     R/C-style*\atindex{Defaulted      R/C-style}{@Defaulted
R/C-style} is the default style,  used  if  you  call  {\ACE}  without
specifying options. In it we use R/C-style with `ct' set to  1000  and
`rt' set to approximately $2000$ divided by the total  length  of  the
relators in an attempt to balance R and C definitions when  we  switch
to CR-style.

%%%%%%%%%%%%%%%%%%%%%%%%%%%%%%%%%%%%%%%%%%%%%%%%%%%%%%%%%%%%%%%%%%%%%%
\Section{Finding Deductions, Coincidences, and Preferred Definitions} 

First, let us  broadly  discuss  strategies  and  how  they  influence
\lq{}definitions'.  By  *definition*\index{definition}  we  mean   the
allocation of a coset number. In a complete  coset  table  each  group
relator produces a cycle of cosets numbers at each  coset  number,  in
particular at coset 1, i.e.~for each relator $w$, and for  each  coset
number $i$, successive  application  of  the  letters  of  $w$,  trace
through a sequence of coset numbers that begins and ends in  $i$  (see
Section~"Enumeration Style" for an example). It has been found to be a
good general  rule  to  use  the  given  group  relators  as  subgroup
generators. This ensures the early definition  of  some  useful  coset
numbers, and is the  basis  of  the  `default'  strategy  (see~"option
default").  The  number  of  group  relators  included   as   subgroup
generators is determined by the `no' option (see~"option no"). Over  a
wide range of examples the use of group relators in this way has  been
shown to produce generally beneficial results in terms of the  maximum
number of cosets numbers defined at any one time and the total  number
of coset numbers defined. In~\cite{CDHW73}, it was reported  that  for
some Macdonald group $G(\alpha,\beta)$  examples,  (pure)  Felsch-type
strategies (that don't include the given group  relators  as  subgroup
generators) e.g.~the `felsch  :=  0'  strategy  (see~"option  felsch")
defined significantly more coset numbers than HLT-type (e.g.~the `hlt'
strategy,  see~"option  hlt")  strategies.  The  comparison  of  these
strategies  in  terms  of  total  number  of  coset  numbers  defined,
in~\cite{Hav91}, for the enumeration of the cosets of a certain  index
40 subgroup of the $G(3,21)$ Macdonald group were 91  for  HLT  versus
16067 for a pure Felsch-type strategy. For the  Felsch  strategy  with
the group relators included as subgroup generators, as for the `felsch
:= 1' strategy (see~"option felsch") the total number of coset numbers
defined reduced markedly to 59.

A *deduction*\index{deduction} occurs when the scanning of  a  relator
results in the assignment of a coset table  body  entry.  A  completed
table is only valid if every  table  entry  has  been  tested  in  all
essentially different positions in  all  relators.  This  testing  can
either be done directly (Felsch strategy) or via relator scanning (HLT
strategy). If it is done directly, then more than one deduction can be
waiting to be processed at any one time. The untested  deductions  are
stored in a stack. How this stack is  managed  is  determined  by  the
`dmode' option (see~"option dmode"), and its size is controlled by the
`dsize' option (see~"option dsize").

As already mentioned a *coincidence*\index{coincidence} occurs when it
is determined that two coset numbers in fact represent the same coset.
When this occurs  the  larger  coset  number  becomes  a  *dead  coset
number*\index{dead coset (number)} and the coincidence is placed in  a
queue. When and how these dead coset numbers are eventually eliminated
is  controlled  by  the  options  `dmode',  `path'  and   `compaction'
(see~"option dmode", "option path" and~"option compaction"). The  user
may  also  force   coincidences   to   occur   (see   Section~"Finding
Subgroups").

The key  to  performance  of  coset  enumeration  procedures  is  good
selection  of  the  next   coset   number   to   be   defined.   Leech
in~\cite{Lee77}  and~\cite{Lee84}  showed  how  a  number   of   coset
enumerations could be simplified by removing coset numbers  needlessly
defined by computer implementations. Human  enumerators  intelligently
choose which coset number should be defined next, based on  the  value
of each potential definition. In particular, definitions  which  close
relator cycles (or at least shorten gaps in cycles)  are  favoured.  A
definition which actually closes a relator  cycle  immediately  yields
twice as many table entries (deductions)  as  other  definitions.  The
value of the `pmode'  option  (see~"option  pmode")  determines  which
definitions are *preferred*; if the value of  the  `pmode'  option  is
non-zero, depending on the `pmode' value, gaps  of  length  one  found
during relator scans in Felsch style  are  either  filled  immediately
(subject to the value of `fill') or noted in the *preferred definition
stack*\index{preferred definition  stack}.  The  preferred  definition
stack is implemented as a ring  of  size  determined  by  the  `psize'
option (see~"option psize").  However,  making  preferred  definitions
carelessly  can  violate  the  conditions  required   for   guaranteed
termination of the coset enumeration procedure in the case  of  finite
index. To avoid such a violation {\ACE}  ensures  a  fraction  of  the
coset table is filled before  a  preferred  definition  is  made;  the
reciprocal of this fraction, the `fill factor', is manipulated via the
`fill' option (see~"option fill"). In~\cite{Hav91}, the `felsch :=  1'
type enumeration of the cosets of the certain index 40 subgroup of the
$G(3,21)$ Macdonald group was further  improved  to  require  a  total
number of coset numbers  of  just  43  by  incorporating  the  use  of
preferred definitions.

%%%%%%%%%%%%%%%%%%%%%%%%%%%%%%%%%%%%%%%%%%%%%%%%%%%%%%%%%%%%%%%%%%%%%%
\Section{Finding Subgroups}

The  {\ACE}  Share  Package,  via  its  interactive  {\ACE}  interface
functions   (described   in   Chapter~"Functions   for    Using    ACE
Interactively"), provides the possibility of searching for  subgroups.
To do this one starts  at  a  known  subgroup  (possibly  the  trivial
subgroup). Then one may augment it by adding new  subgroup  generators
either        explicitly        via         `ACEAddSubgroupGenerators'
(see~"ACEAddSubgroupGenerators")   or   implicitly   by    introducing
*coincidences*\index{coincidences}     (see     `ACECosetCoincidence':
"ACECosetCoincidence",           or           `ACERandomCoincidences':
"ACERandomCoincidences"). Also, one may descend to  smaller  subgroups
by  deleting  subgroup  generators  via  `ACEDeleteSubgroupGenerators'
(see~"ACEDeleteSubgroupGenerators").

%%%%%%%%%%%%%%%%%%%%%%%%%%%%%%%%%%%%%%%%%%%%%%%%%%%%%%%%%%%%%%%%%%%%%%
\Section{Coset Table Standardisation Schemes}

The standardisation scheme for coset tables that is the  default  used
in  {\GAP},  numbers  cosets   according   to   coset   representative
word-length in the group generators and lexical  ordering  imposed  by
the user-supplied  ordering  of  the  group  generators.  This  scheme
ensures that as one scans  the  columns  corresponding  to  the  group
generators (in user-supplied order) row by  row,  one  encounters  new
coset numbers in numeric order. As an example, consider the (infinite)
group with presentation $\langle x,  y,  a,  b  \mid  x^2,  y^3,  a^4,
b^2\rangle$. According to {\GAP}'s standardisation  scheme  the  coset
table  corresponding  to  this  presentation  (omitting  the   columns
corresponding to  inverse  generators,  which  play  no  part  in  the
numbering scheme) would begin as follows:

\begintt
 coset no. ||      x      y      a      b    rep've
-----------+--------------------------------------
         1 ||      2      3      4      5
         2 ||      1      6      7      8    x
         3 ||      9     10     11     12    y
         4 ||     13     14     15     16    a
         5 ||     17     18     19      1    b
         6 ||     20     21     22     23    xy
         7 ||     24     25      2     26    xa
         8 ||     27     28     29      2    xb
         9 ||      3     30     31     32    yx
        10 ||     33      1     34     35    yy
        11 ||     36     37     38     39    ya
        12 ||     40     41     42      3    yb
        13 ||      4     43     44     45    ax
        14 ||     46     47     48     49    ay
        15 ||     50     51     52     53    aa
        16 ||     54     55     56      4    ab
        17 ||      5     57     58     59    bx
        18 ||     60     61     62     63    by
        19 ||     64     65     66     67    ba
        20 ||      6     68     69     70    xyx
        21 ||     71      2     72     73    xyy
\endtt

Observe that the representatives are ordered according to  length  and
then the lexical ordering implied by defining $x\<y\<a\<b$ (with  some
words omitted due to their equivalence to words that precede  them  in
the ordering). Also observe that as one scans the body  of  the  table
row by row from left to right new  coset  numbers  appear  in  numeric
order without gaps (2, 3, 4, 5, then 1 which we have already seen,  6,
7, etc.).

A standardisation scheme that is often more convenient  than  {\GAP}'s
default    scheme    is    the     `lenlex'{\undoquotes\atindex{lenlex
standardisation scheme} {@`lenlex'  standardisation  scheme}}  scheme,
which is the  one  used  within  {\ACE}.  (With  the  `lenlex'  option
(see~"option lenlex"), the coset table output  by  `ACECosetTable'  or
`ACECosetTableFromGensAndRels'  is  standardised  according   to   the
`lenlex' scheme.) The `lenlex' scheme  is  very  similar  to  {\GAP}'s
default scheme except  inverse  generators  are  included,  with  each
inverse generator included  immediately  are  each  generator  in  the
alphabet ordering that defines the `lexical' ordering. By  doing  this
and including the columns for the inverse generators immediately after
the corresponding generators, in the coset table, one  finds  for  the
`lenlex' scheme that again when one scans the body of  the  table  new
coset  numbers  appear  in  numeric  order.  Let  us  take  the   same
presentation as before for an example. Then the first 21 lines of  its
coset  table  using  the  `lenlex'  scheme  for  coset  numbering  and
including columns for the inverse generators is as follows:

\begintt
 coset no. ||      x      X      y      Y      a      A      b      B   rep've
-----------+------------------------------------------------------------------
         1 ||      2      2      3      4      5      6      7      7
         2 ||      1      1      8      9     10     11     12     12    x
         3 ||     13     13      4      1     14     15     16     16    y
         4 ||     17     17      1      3     18     19     20     20    Y
         5 ||     21     21     22     23     24      1     25     25    a
         6 ||     26     26     27     28      1     24     29     29    A
         7 ||     30     30     31     32     33     34      1      1    b
         8 ||     35     35      9      2     36     37     38     38    xy
         9 ||     39     39      2      8     40     41     42     42    xY
        10 ||     43     43     44     45     46      2     47     47    xa
        11 ||     48     48     49     50      2     46     51     51    xA
        12 ||     52     52     53     54     55     56      2      2    xb
        13 ||      3      3     57     58     59     60     61     61    yx
        14 ||     62     62     63     64     65      3     66     66    ya
        15 ||     67     67     68     69      3     65     70     70    yA
        16 ||     71     71     72     73     74     75      3      3    yb
        17 ||      4      4     76     77     78     79     80     80    Yx
        18 ||     81     81     82     83     84      4     85     85    Ya
        19 ||     86     86     87     88      4     84     89     89    YA
        20 ||     90     90     91     92     93     94      4      4    Yb
        21 ||      5      5     95     96     97     98     99     99    ax
\endtt

In  the  table  each  inverse  generator   is   represented   by   the
corresponding uppercase letter ($X$  represents  the  inverse  of  $x$
etc.), and the lexical ordering of the representatives is that implied
by defining $x\<X\<y\<Y\<a\<A\<b\<B$. As before, observe that  as  one
scans the body of the table row by row from left to  right  new  coset
numbers are encountered in numeric  order  without  gaps  (2,  then  2
again, 3, 4, 5, 6, 7, then 7 again, etc.); but this time,  it  is  the
inclusion of the columns that correspond  to  the  inverse  generators
that makes it work.

%%%%%%%%%%%%%%%%%%%%%%%%%%%%%%%%%%%%%%%%%%%%%%%%%%%%%%%%%%%%%%%%%%%%%%%%%
%%
%E
