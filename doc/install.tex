%%%%%%%%%%%%%%%%%%%%%%%%%%%%%%%%%%%%%%%%%%%%%%%%%%%%%%%%%%%%%%%%%%%%%%%%%
%%
%W  install.tex     ACE documentation - installation     Alexander Hulpke
%W                                                      Joachim Neub"user
%W                                                            Greg Gamble
%%
%H  $Id$
%%
%Y  Copyright (C) 2000  Centre for Discrete Mathematics and Computing
%Y                      Department of Computer Science & Electrical Eng.
%Y                      University of Queensland, Australia.
%%

%%%%%%%%%%%%%%%%%%%%%%%%%%%%%%%%%%%%%%%%%%%%%%%%%%%%%%%%%%%%%%%%%%%%%%%%%
\Chapter{Installing and Loading the ACE Share Package}

%%%%%%%%%%%%%%%%%%%%%%%%%%%%%%%%%%%%%%%%%%%%%%%%%%%%%%%%%%%%%%%%%%%%%%
\Section{Installing the ACE Share Package}

To install, unpack the archive file as a sub-directory  in  the  `pkg'
hierarchy of your version  of  {\GAP}~4.  (This  might  be  the  `pkg'
directory of the {\GAP}~4 home directory; it is however also  possible
to keep an additional `pkg' directory in your private directories, see
Section~"ref:Installing Share  Packages"  of  the  {\GAP}~4  Reference
Manual for details on how to do this.) Go to the newly  created  `ace'
directory and call `./configure <path>' where <path> is  the  path  to
the {\GAP} home directory. So for example if you install  the  package
in the main `pkg' directory call

\begintt
./configure ../..
\endtt

This  will fetch  the  architecture  type for  which  {\GAP} has  been
compiled last and create a `Makefile'. Now simply call

\begintt
make
\endtt

to compile the binary and to install it in the appropriate place.

Note that the  current version of the configuration  process only sets
up  directory paths.  If you  need a  different compiler  or different
compiler options, you need to edit `src/Makefile.in'  yourself,  prior
to calling `make'.

If you use this installation of {\GAP} on different hardware platforms
you will have to compile the binary for each platform separately. This
is done by calling `configure',  editing  `src/Makefile.in'  possibly,
and calling `make' for the package anew  immediately  after  compiling
{\GAP} itself for the respective  architecture.  If  your  version  of
{\GAP} is already compiled (and has last been  compiled  on  the  same
architecture)  you  do  not  need  to  compile  {\GAP}  again,  it  is
sufficient  to  call  the  `configure'  script  in  the  {\GAP}   home
directory.

The manual you are currently reading describes how to use  the  {\ACE}
Share Package; it can be  found  in  the  `doc'  subdirectory  of  the
package. If your manual does not have a table of contents or index, or
has these but with invalid page numbers please re-generate the  manual
by executing

\begintt
./make_doc
\endtt

in the `doc' subdirectory.

The  subdirectory  `standalone-doc'  contains the  file  `ace3001.dvi'
which holds a version of the user manual for the {\ACE} standalone; it
forms part of~\cite{Ram99}).  You  should consult it if  you are going
to  switch to  the {\ACE}  standalone, e.g.~in  order to  directly use
interactive facilities.

The  `src' subdirectory  contains a  copy  of the  original source  of
{\ACE}.  (The  only modification  is  that  a  file `Makefile.in'  was
obtained from  the different `make.xyz' and  will be used  to create a
`Makefile'.)  You  can replace  the source by  a newer  version before
compiling.

If you encounter problems in installation please read the `README'.

%%%%%%%%%%%%%%%%%%%%%%%%%%%%%%%%%%%%%%%%%%%%%%%%%%%%%%%%%%%%%%%%%%%%%%
\Section{Loading the ACE Share Package}

To use the {\ACE} Share Package you have  to  request  it  explicitly.
This is done by calling

\beginexample
gap> RequirePackage( "ace" );
#I  Loading the ACE (Advanced Coset Enumerator) share package
#I           by George Havas <havas@csee.uq.edu.au> and
#I              Colin Ramsay <cram@csee.uq.edu.au>
#I                   ACE binary version: 3.000
true
\endexample

The      `RequirePackage'      command      is      described       in
Section~"ref:RequirePackage" in the {\GAP} Reference Manual.

If {\GAP} cannot find a working binary, the call  to  `RequirePackage'
will fail.

If you know you have a working {\ACE} binary, as well as  a  correctly
installed {\ACE} Share Package, it is possible to suppress the  `Info'
messages by temporarily setting the `InfoLevel' of `InfoWarning' to 0,
and a duplicated semicolon will suppress the `true' result:

\beginexample
gap> SetInfoLevel(InfoWarning, 0); RequirePackage( "ace" );;
gap> SetInfoLevel(InfoWarning, 1);
\endexample

If you want to load the {\ACE} package by default, you  can  put   the
`RequirePackage' command into your `.gaprc' file (see Section~"ref:The
.gaprc file" in the {\GAP} Reference Manual).

%%%%%%%%%%%%%%%%%%%%%%%%%%%%%%%%%%%%%%%%%%%%%%%%%%%%%%%%%%%%%%%%%%%%%%%%%
%%
%E
