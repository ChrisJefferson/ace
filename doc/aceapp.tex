%%%%%%%%%%%%%%%%%%%%%%%%%%%%%%%%%%%%%%%%%%%%%%%%%%%%%%%%%%%%%%%%%%%%%%%%%
%%
%W  aceapp.tex        ACE documentation appendices       Alexander Hulpke
%W                                                      Joachim Neub"user
%W                                                            Greg Gamble
%%
%H  $Id$
%%
%Y  Copyright (C) 2000  Centre for Discrete Mathematics and Computing
%Y                      Department of Computer Science & Electrical Eng.
%Y                      University of Queensland, Australia.
%%

%%%%%%%%%%%%%%%%%%%%%%%%%%%%%%%%%%%%%%%%%%%%%%%%%%%%%%%%%%%%%%%%%%%%%%%
\Chapter{The Meanings of ACE's output messages}

In this chapter, we discuss the meanings of the messages  that  appear
in output from the {\ACE} binary, the verbosity of which is determined
by the `messages' option (see~"Option `messages'"). Non-interactively,
these messages are directed to file `ACEData.outfile' (or  <filename>,
if option `aceoutfile := <filename>', or `ao := <filename>', is used).
They  may  also  be  seen  prepended   with   \lq{}`\#I   ''   (either
interactively of non-interactively) if one has set the `InfoLevel'  of
`InfoACE' to  at least 3, via

\beginexample
gap> SetInfoACELevel(3);
\endexample

Note that when {\ACE} is run non-interactively, the banner stating the
version number and date, which  will  be  observed  if  one  runs  the
standalone is re-directed to the file `ACEData.banner'.

For simplicity of exposition, from here on we  assume  non-interactive
use of {\ACE}.

What is first observed in the {\ACE} output file is a heading like:

\begintt
  #-- ACE 3.000: Run Parameters ---
\endtt

(where `3.000' may be replaced be some later version number)  followed
by the \lq{}input parameters' developed from the arguments and options
passed to `ACECosetTableFromGensAndRels' or  `ACEStats'.  After  these
appears a separator:

\begintt
  #--------------------------------
\endtt

followed by  any  *progress  messages*  (progress  messages  are  only
printed if `messages' is non-zero; recall that by default `messages' =
0),   followed   by   a   *results   message*.   In   the   case    of
`ACECosetTableFromGensAndRels', these messages  are  followed  by  yet
more progress messages (if `messages' is non-zero) and a coset  table.
Finally, the {\ACE} output file is terminated by {\ACE}'s exit banner,
which should look something like:

\begintt
=========================================
ACE 3.000        Sun Mar 12 17:25:37 2000
\endtt

Both *progress messages* and  the  *results  message*  consist  of  an
initial tag followed by a list of statistics. All messages have values
for the statistics `a', `r', `h', `n', `h',  `l'  and  `c'  (excepting
that the second `h', the one following  the  `n'  statistic,  is  only
given if hole monitoring has been turned on by setting `messages'  $\<
0$, which as noted above is expensive and  should  be  avoided  unless
really needed). Additionally, there may appear the statistics: `m' and
`t' (as for the results message); `d'; or `s', `d' and `c' (as for the
`DS' progress message). The meanings of  the  various  statistics  and
tags will follow later. The following is a sample progress message:

\begintt
AD: a=2 r=1 h=1 n=3; h=66.67% l=1 c=+0.00; m=2 t=2
\endtt

with tag `AD' and values for the statistics `a', `r',  `h',  `n',  `h'
(appears because `messages' $\<  0$),  `l',  `c',  `m'  and  `t'.  The
following is a sample results message:

\begintt
INDEX = 12 (a=12 r=16 h=1 n=16; l=3 c=0.01; m=14 t=15)
\endtt

which, in this case, declares a successful enumeration  of  the  coset
numbers of a subgroup of index 12 within a group,  and,  as  it  turns
out, values for the same statistics as the sample progress message.

In the following table we  list  the  statistics  that  can  follow  a
progress or results message tag, in order:

\begintt
--------------------------------------------------------------------
statistic   meaning
--------------------------------------------------------------------
a           number of active coset numbers
r           number of applied coset numbers
h           first (potentially) incomplete row
n           next coset number definition
h           percentage of holes in the table (if `messages'$ \< 0$) 
l           number of main loop passes
c           total CPU time
m           maximum number of active coset numbers
t           total number of coset numbers defined
s           new deduction stack size (with DS tag)
d           current deduction stack size, or
              no. of non-redundant deductions retained (with DS tag)
c           no. of redundant deductions discarded (with DS tag)
--------------------------------------------------------------------
\endtt

Now that we have discussed the various  meanings  of  the  statistics,
it's time to discuss the various types of progress messages possible.

%%%%%%%%%%%%%%%%%%%%%%%%%%%%%%%%%%%%%%%%%%%%%%%%%%%%%%%%%%%%%%%%%%%%%%%%
\Section{Progress Messages}

A progress message (and its tag) indicates the function just completed
by the enumerator. In the following table, the possible message `tag's
appear in the first column. In the `action' column,  a  `y'  indicates
the function is aggregated and counted. Every time this count  reaches
the value of `messages', a message line is printed and  the  count  is
zeroed. Those tags flagged  with  a  `y*'  are  only  present  if  the
appropriate option was included when the {\ACE} binary was compiled (a
default compilation includes the appropriate options; so normally read
`y*' as `y').

Tags with an `n' in the `action' column indicate the function  is  not
counted, and cause a message line to be output every time they  occur.
They also cause the action count to be reset.

\begintt
------------------------------------------------------------------
tag   action      meaning
------------------------------------------------------------------
AD         y      coset 1 application definition (`SG'/`RS' phase)
RD         y      R-style definition
RF         y      row-filling definition
CG         y      immediate gap-filling definition
CC         y*     coincidence processed
DD         y*     deduction processed
CP         y      preferred list gap-filling definition
CD         y      C-style definition
Lx         n      lookahead performed (type `x')
CO         n      table compacted
CL         n      complete lookahead (table as deduction stack)
UH         n      updated completed-row counter
RA         n      remaining coset numbers applied to relators
SG         n      subgroup generator phase
RS         n      relators in subgroup phase
DS         n      stack overflowed (compacted and doubled)
------------------------------------------------------------------
\endtt

% \begin{table}
% \hrule
% \caption{Possible progress messages}
% \label{tab:prog}
% \smallskip
% \renewcommand{\arraystretch}{0.875}
% \begin{tabular*}{\textwidth}{@{\extracolsep{\fill}}lll} 
% \hline\hline
% message & action & meaning \\
% \hline
% \ttt{AD} & y  & coset \#1 application definition 
% 			(\ttt{SG}/\ttt{RS} phase) \\
% \ttt{RD} & y  & R-style definition \\
% \ttt{RF} & y  & row-filling definition \\
% \ttt{CG} & y  & immediate gap-filling definition \\
% \ttt{CC} & y* & coincidence processed \\
% \ttt{DD} & y* & deduction processed \\
% \ttt{CP} & y  & preferred list gap-filling definition \\
% \ttt{CD} & y  & C-style definition \\
% \ttt{Lx} & n  & lookahead performed (type \ttt{x}) \\
% \ttt{CO} & n  & table compacted \\
% \ttt{CL} & n  & complete lookahead (table as deduction stack) \\
% \ttt{UH} & n  & updated completed-row counter \\
% \ttt{RA} & n  & remaining cosets applied to relators \\
% \ttt{SG} & n  & subgroup generator phase \\
% \ttt{RS} & n  & relators in subgroup phase \\
% \ttt{DS} & n  & stack overflowed (compacted and doubled) \\
% \hline\hline
% \end{tabular*}
% \end{table}

%%%%%%%%%%%%%%%%%%%%%%%%%%%%%%%%%%%%%%%%%%%%%%%%%%%%%%%%%%%%%%%%%%%%%%%
\Section{Results Messages}

The possible results are given in the following table; any result  not
listed represents an internal error and  should  be  reported  to  the
{\ACE} authors.

% The level column is omitted ... since it won't mean anything to a
% GAP user
\begintt
result tag           meaning 
------------------------------------------------------------------
INDEX = x            finite index of `x' obtained
OVERFLOW             out of table space
SG PHASE OVERFLOW    out of space (processing subgroup generators)
ITERATION LIMIT      `loop' limit triggered
TIME LIMT            `ti' limit triggered
HOLE LIMIT           `ho' limit triggered
INCOMPLETE TABLE     all coset numbers applied, but table has holes
MEMORY PROBLEM       out of memory (building data structures)
---------------------------------------------------------------------
\endtt

% \begin{table}
% \hrule
% \caption{Possible enumeration results}
% \label{tab:rslts}
% \smallskip
% \renewcommand{\arraystretch}{0.875}
% \begin{tabular*}{\textwidth}{@{\extracolsep{\fill}}lll} 
% \hline\hline
% result & level & meaning \\
% \hline
% \ttt{INDEX = x}         & 0 & finite index of \ttt{x} obtained \\
% \ttt{OVERFLOW}          & 0 & out of table space \\
% \ttt{SG PHASE OVERFLOW} & 0 & out of space (processing subgroup
% 				generators) \\
% \ttt{ITERATION LIMIT}   & 0 & \ttt{loop} limit triggered \\
% \ttt{TIME LIMT}         & 0 & \ttt{ti} limit triggered \\
% \ttt{HOLE LIMIT}        & 0 & \ttt{ho} limit triggered \\
% \ttt{INCOMPLETE TABLE}  & 0 & all cosets applied, but table has holes \\
% \ttt{MEMORY PROBLEM}    & 1 & out of memory (building data structures) \\
% \hline\hline
% \end{tabular*}
% \end{table}

*Notes*

Recall that hole monitoring is switched on by setting a negative value
for the `messages'(see~"Option `messages'") option, but note that hole
monitoring is expensive, so don't turn it on unless  you  really  need
it. If you wish to print out the presentation and the options, but not
the progress messages, then set `messages' non-zero, but  very  large.
(You'll still get the `SG', `DS', etc. messages,  but  not  the  `RD',
`DD', etc. ones.) You can  set  `messages'  to  $1$,  to  monitor  all
enumerator actions, but be warned  that  this  can  yield  very  large
output files.

%%%%%%%%%%%%%%%%%%%%%%%%%%%%%%%%%%%%%%%%%%%%%%%%%%%%%%%%%%%%%%%%%%%%%%%
\Chapter{Other ACE Options}

Here we will list  all  the  known  {\ACE}  options  that  users  will
normally only wish to use if generating an input file,  by  using  the
option `aceinfile' (see~"Option  `aceinfile'").  Most of  the  options
below are for interactive use with the  standalone;  from  the  {\GAP}
interface, we provide other functions for interactive use.

\beginitems

\>`aep:=<val>'{Option `aep'}&
Runs  the enumeration for `a'll `e'quivalent `p'resentations;
<val> is in the integer range 0 to 7.

The `aep' option runs  an  enumeration  for  combinations  of  relator
ordering, relator rotations, and relator inversions.

The argument <val> is considered as a binary number.  Its  three  bits
are treated as flags, and control relator rotations (the  $2^0$  bit),
relator inversions (the $2^1$ bit) and relator  orderings  (the  $2^2$
bit),  respectively;  where  $1$  means  \lq{}active'  and  $0$  means
\lq{}inactive'. (See below for an example).

The `aep' option first performs a \lq{}priming run' using the  options
as they stand. In particular, the `asis' and  `messages'  options  are
honoured.

It then turns `asis' on and `messages' off  (i.e.~sets  `messages'  to
0), and generates and tests the  requested  equivalent  presentations.
The maximum and minimum values attained by `m' (the maximum number  of
coset numbers defined at any stage) and `t' (the total number of coset
numbers defined) are tracked, and each  time  a  new  \lq{}record'  is
found, the relators used and the summary result line is  printed.  See
Appendix~"The Meanings of ACE's output messages" for a  discussion  of
the statistics `m' and `t'. To observe these messages either  set  the
`InfoLevel' of `InfoACE' to 3 or non-interactively you can peruse  the
{\ACE} output file (see~"Option `aceoutfile'").

The order in which the  equivalent  presentations  are  generated  and
tested has no particular significance, but note that the  presentation
as given *after* the initial priming run) is the  *last*  presentation
to be generated and tested, so that  the  group's  relators  are  left
`unchanged' by running the `aep' option, (not that  a  non-interactive
user cares).

As discussed by Cannon, Dimino, Havas  and  Watson  \cite{CDHW73}  and
Havas and Ramsay \cite{HR99b} such equivalent presentations can  yield
large variations in  the  number  of  coset  numbers  required  in  an
enumeration. For this command, we are interested in this variation.

After  the  final  presentation  is  run,   some   additional   status
information messages are printed to the {\ACE} output file:

\beginlist
\item{--}  the number of runs which yielded a finite index; 
\item{--}  the total number of runs (excluding the priming run); and 
\item{--}  the range of values observed for `m' and `t'.
\endlist

As an example (drawn from the discussion in \cite{HR99a}) consider the
enumeration   of   the   $448$   coset   numbers   of   the   subgroup
$\langle  a^2,Ab \rangle$ of the group
$$ (8,7 \mid 2,3) 
    = \langle a,b \mid a^8 = b^7 = (ab)^2 = (Ab)^3 = 1 \rangle. $$
There are $4!=24$  relator  orderings  and  $2^4=16$  combinations  of
relator or inverted relator. Exponents are  taken  into  account  when
rotating relators, so the relators given give rise to 1, 1,  2  and  2
rotations respectively, for a total of $1.1.2.2=4$  combinations.  So,
for  `aep'  =  $7$   (resp.~$3$),   $24.16.4=1536$   (resp.~$16.4=64$)
equivalent presentations are tested.

*Notes:*
There is no way to stop the `aep'  option  before  it  has  completed,
other than killing the task. So do a reality check beforehand  on  the
size of the search space and the time for each enumeration. If you are
interested in  finding  a  \lq{}good'  enumeration,  it  can  be  very
helpful, in terms of running time, to put a tight limit on the  number
of coset numbers via the `max'  option.  You  may  also  have  to  set
`compaction' = $100$ to prevent time-wasting attempts to recover space
via compaction. This maximises throughput  by  causing  the  \lq{}bad'
enumerations, which are in  the  majority,  to  overflow  quickly  and
abort. If you wish to explore  a  very  large  search-space,  consider
firing up many copies of {\ACE}, and starting each with a \lq{}random'
equivalent  presentation.  Alternatively,  you  could  use  the  `rep'
command.

Interactively,             use              `ACEAllEquivPresentations'
(see~"ACEAllEquivPresentations").

\>`rep:=<val>'{Option `rep'}
\>`rep:=[<val>, <Npresentations>]'{Option `rep'}&
Runs  the enumeration for `r'andom `e'quivalent `p'resentations;
<val> is in the integer range 0 to 7;
<Npresentations> must be a positive integer.

The `rep' (random equivalent  presentations)  option  complements  the
`aep'  option.  It  generates  and  tests   some   random   equivalent
presentations. The argument <val>  acts  as  for  `aep'.  It  is  also
possible to set the number <Npresentations>  of  random  presentations
used (by default, eight  are  used),  by  using  the  extended  syntax
`rep:=[<val>,<Npresentations>]'.

The routine first  turns  `asis'  on  and  `messages'  off  (i.e.~sets
`messages' to 0), and then generates and tests the requested number of
random equivalent presentations. For each presentation,  the  relators
used and the  summary  result  line  are  printed.  To  observe  these
messages either set the `InfoLevel' of `InfoACE'  to  at  least  3  or
non-interactively you can peruse the {\ACE} output  file  (see~"Option
`aceoutfile'").

*Notes:*
The relator inversions and rotations are \lq{}genuinely'  random.  The
relator permuting is a little bit of a kludge, with the  \lq{}quality'
of the permutations tending to improve with successive  presentations.
When the `rep' command  completes,  the  presentation  active  is  the
*last* one generated, (not that the non-interactive user cares).

*Guru Notes:*
It might appear that neglecting to restore the  original  presentation
is an error. In fact, it is a useful feature! Suppose that  the  space
of equivalent presentations is too  large  to  exhaustively  test.  As
noted in the entry for `aep', we can start up multiple copies of `aep'
at random points in the  search-space.  Manually  generating  `random'
equivalent presentations to serve as starting-points  is  tedious  and
error-prone. The `rep' option provides a simple solution;  simply  run
`rep := 7' before `aep := 7'.

Interactively,            use            `ACERandomEquivPresentations'
(see~"ACERandomEquivPresentations").

\>`sg:=<wordList>'{Option `sg'}&
Adds the words in <wordList> to any  `s'ubgroup  `g'enerators  already
present; <wordList> must be a list of words in the group generators.

The enumeration must  be  (re)`start'ed  or  `redo'ne,  it  cannot  be
`continue'd.

Interactively,             use              `ACEAddSubgroupGenerators'
(see~"ACEAddSubgroupGenerators").

\>`rl:=<wordList>'{Option `rl'}&
Appends the `r'elator `l'ist,  <wordList>  to  the  existent  list  of
relators present; <wordList> must be a list  of  words  in  the  group
generators.

The enumeration must  be  (re)`start'ed  or  `redo'ne,  it  cannot  be
`continue'd.

Interactively, use `ACEAddRelators' (see~"ACEAddRelators").

\>`ai'{Option `ai'}
\>`ai:=<filename>'{Option `ai'}&
`A'lter `i'nput to standard input or <filename>; <filename> must be  a
string.

By default, commands to {\ACE} are read from standard input (i.e., the
keyboard). With no value `ai' causes {\ACE} to revert to reading  from
standard input; otherwise, the `ai' command closes the  current  input
file,  and  opens  `<filename>'  as  the  source   of   commands.   If
`<filename>' can't be opened, input reverts to standard input.

*Notes:*
If you switch to taking input from (another) file, remember to  switch
back before the end of that file; otherwise the `EOF' there will cause
{\ACE} to terminate.

\>`begin'{Option `begin'}
\>`start'{Option `start'}&
Start an enumeration. (Shortest abbreviation of `begin' is `beg'.)

Any existing information in the table is cleared, and the  enumeration
starts from coset 1 (i.e., the subgroup).

Interactively, use `ACEStart' (see~"ACEStart").

\>`bye'{Option `bye'}
\>`exit'{Option `exit'}
\>`qui'{Option `qui'}&
Quit {\ACE}. (Shortest abbreviation of `qui' is `q'.)

This quits {\ACE} nicely, printing the date and  the  time.  An  `EOF'
(end-of-file; i.e., `\^{}d') has the same effect, so proper termination
occurs if {\ACE} is taking its input from a script file.

Interactively, use `ACEQuit' (see~"ACEQuit").

Note  that  `qui'  actually  abbreviates  the   corresponding   {\ACE}
directive `quit', but since `quit' is  a  {\GAP}  keyword  it  is  not
available via the {\GAP} interface to {\ACE}.

\>`cc:=<val>'{Option `cc'}&
Make `c'oset <val> `c'oincide with coset 1; <val> should be a positive
integer.

Prints out the representative of coset `<val>', and  adds  it  to  the
subgroup generators; i.e., forces coset `<val>' to coincide with coset
1, the subgroup.

\>`check'{Option `check'}
\>`redo'{Option `redo'}&
`Redo' an extant enumeration, using the current parameters.

As opposed to `start' (see~"Option `start'"), which clears an existing
coset table, any existing information in the table  is  retained,  and
the enumeration is restarted from coset 1 (i.e., the subgroup).

Interactively, use `ACERedo' (see~"ACERedo").

*Notes:*
This option is really intended for the case where additional  relators
(option `rl'; see~"Option `rl'") and/or  subgroup  generators  (option
`sg'; see~"Option `sg'") have  been  introduced.  The  current  table,
which may be incomplete or exhibit a finite index, is  still  *valid*.
However, the additional data may allow the enumeration to complete, or
cause a collapse to a smaller index.

\>`continue'{Option `continue'}&
`Continue' the current enumeration, building upon the existing table.
(Shortest abbreviation: `cont'.)

If a previous run stopped without producing a finite index you can, in
principle, change any of the parameters and `continue' on. Of  course,
if you make any changes which invalidate the current table, you  won't
be allowed to `continue',  although  you  may  be  allowed  to  `redo'
(see~"Option `redo'"). If `redo' is not allowed, you  must  re-`start'
(see~"Option `start'").

Interactively, use `ACEContinue' (see~"ACEContinue").

\>`cycles'{Option `cycles'}&
Print out the table in `cycles'. (Shortest abbreviation: `cy'.)

This option prints out the permutation representation.

Interactively, use `ACECycles' (see~"ACECycles").

\>`ds:=<list>'{Option `ds'}&
`D'elete `s'ubgroup generators; <list> must  be  a  list  of  positive
integers.

This command  allows  subgroup  generators  to  be  deleted  from  the
presentation. If the generators are numbered from 1 in the output  of,
say, the `sr' command (see~"Option `sr'"), then the generators  listed
in `<list>' are  deleted;  `<list>'  must  be  a  strictly  increasing
sequence.

Interactively,            use            `ACEDeleteSubgroupGenerators'
(see~"ACEDeleteSubgroupGenerators").

\>`dr:=<list>'{Option `dr'}&
`D'elete relators; <list> must be a list of positive integers.

This  command  allows  group  relators  to   be   deleted   from   the
presentation. If the relators are numbered from 1 in  the  output  of,
say, the `sr' command (see~"Option `sr'"), then the relators listed in
`<list>' are deleted; `<list>' must be a strictly increasing sequence.

Interactively, use `ACEDeleteRelators' (see~"ACEDeleteRelators").

\>`dump'{Option `dump'}
\>`dump:=<level>'{Option `dump'}
\>`dump:=[<level>]'{Option `dump'}
\>`dump:=[<level>, <detail>]'{Option `dump'}&
`Dump's the internal variables of {\ACE}; <level> must be  an  integer
in the range 0 to 2, and <detail> must be 0 or 1.
(Shortest abbreviation: `d'.)

The value of <level> determines which of the three levels of {\ACE} to
dump. (You will need to read the standalone manual to understand  what
Levels 0, 1 and 2 are all about.) The value of <detail> determines the
amount of detail (`<detail> = 0' means less detail).  The  first  form
(with no arguments) selects `<level> = 0, <detail> =  0'.  The  second
form of this command makes `<detail> = 0'. This option is intended for
gurus; the source code should be consulted  to  see  what  the  output
means. Non-interactively, the output from `dump' is  directed  to  the
temporary file with path `ACEData.banner'.

Interactively, use `ACEDumpVariables' (see~"ACEDumpVariables").

\>`generators:=<wordList>'{Option `generators'}&
Define the subgroup `generators'; <wordList> must be a list  of  words
in the group generators.
(Shortest abbreviation: `gen'.)

There should *never* be a need to use this command, even when creating
a standalone input file. The subgroup generators should  be  input  as
one of the arguments of an  {\ACE}  interface  function.  By  default,
there are no subgroup generators and the  subgroup  is  trivial.  This
command allows a list of subgroup generating words to be entered.

\>`group:=<list>'{Option `group'}&
Define the `group' generators; <list> must be a list of  {\GAP}  group
generators.
(Shortest abbreviation: `gr'.)

There should *never* be a need to use this command, even when creating
a standalone input file. The group generators should be input  as  one
of the arguments of an {\ACE} interface function.  If  the  generators
each have names that as strings are single  lowercase  letters,  those
same strings are used to represent  the  same  generators  by  {\ACE};
otherwise,  {\ACE}  will  represent  each  generator  by  an  integer,
numbered sequentially from 1.

*Notes:*
Any use of the  `group'  command  which  actually  defines  generators
invalidates any previous enumeration, and stays in  effect  until  the
next `group' command. Any words for the  group  or  subgroup  must  be
entered using the nominated generator format, and  all  printout  will
use this format. A valid set of generators is the minimum  information
necessary before {\ACE} will attempt an enumeration.

*Guru Notes:*
The columns of the coset table are allocated in the same order as  the
generators are listed, insofar as this is  possible,  given  that  the
first two columns must be  a  generator/inverse  pair  or  a  pair  of
involutions. The ordering of the columns can, in  some  cases,  affect
the definition sequence of cosets and  impact  the  statistics  of  an
enumeration.

\>`relators:=<wordList>'{Option `relators'}&
Define the group `relators'; <wordList> must be a list of words in the
group generators.
(Shortest abbreviation: `rel'.)

There should *never* be a need to use this command, even when creating
a standalone input file. The group relators should be input as one  of
the arguments of an {\ACE} interface function.  If  <wordList>  is  an
empty list, the group is free.

\>`help'{Option `help'}&
Print the {\ACE} help screen. (Shortest abbreviation: `h'.)

This option prints the list of options of the {\ACE} binary. Note that
this list is longer than a standard screenful.

\>`mode'{Option `mode'}&
Prints the possible enumeration `mode's.
(Shortest abbreviation: `mo'.)

Prints the possible enumeration  `mode's  (i.e.~which  of  `continue',
`redo' or  `start'  are  possible  (see~"Option  `continue'",  "Option
`redo'" and "Option `start'").

\>`nc'{Option `nc'}
\>`nc:=<val>'{Option `nc'}
\>`normal'{Option `normal'}
\>`normal:=<val>'{Option `normal'}&
Check or attempt to enforce normal closure; <val> must be 0 or 1.

This option tests the subgroup for normal closure within the group. If
a conjugate of a subgroup generator by a generator, is  determined  to
belong to a coset other than coset 1, it is printed out, and if `<val>
=  1',  then  any  such  conjugate  is  also  added  to  the  subgroup
generators. With no argument or if `<val> = 0', {\ACE}  does  not  add
any new subgroup generators.

*Notes:*
The method of determination of  whether  a  conjugate  of  a  subgroup
generator is in the subgroup, is by testing whether it can  be  traced
from coset 1 to coset 1  (see  `trace':~"Option `trace'").

The resultant subgroup need not be  normally  closed  after  executing
option `nc'  with  the  value  1.  It  is  still  possible  that  some
conjugates of the newly added subgroup generators will not be elements
of the subgroup.

Interactively,       use       `ACEConjugatesForSubgroupNormalClosure'
(see~"ACEConjugatesForSubgroupNormalClosure").

\>`options'{Option `options'}&
Dump version information of the {\ACE} binary.
(Shortest abbreviation: `opt'.)

A rather unfortunate name for an option; this command dumps details of
the \lq{}options' included in the version of {\ACE}  when  the  {\ACE}
binary was compiled.

Use  the   command   `ACEVersion();'   (see~"ACEVersion")   for   this
information, instead, unless you want it in an {\ACE} standalone input
file.

A typical output, is as follows:

\begintt
Executable built:
  Sat Feb 27 15:57:59 EST 1999
Level 0 options:
  statistics package = on
  coinc processing messages = on
  dedn processing messages = on
Level 1 options:
  workspace multipliers = decimal
Level 2 options:
  host info = on
\endtt

\>`oo:=<val>'{Option `oo'}
\>`order:=<val>'{Option `order'}&
Print a coset representative of a coset number with order  a  multiple
of <val> modulo the subgroup; <val> must be an integer.

This option finds a coset with order a multiple  of  `|<val>|'  modulo
the subgroup, and prints out its coset representative.  If  `<val>  \<
0', then all coset numbers meeting the  requirement  are  printed.  If
`<val> > 0', then just the first coset number meeting the  requirement
is printed. Also, `<val> = 0' is permitted; this special value effects
the printing of the orders (modulo the subgroup) of all coset numbers.

Interactively, use `ACEOrders' (see~"ACEOrders"), for the case  `<val>
= 0', or `ACEOrder' (see~"ACEOrder"), otherwise.

\>`sr'{Option `sr'}
\>`sr:=<val>'{Option `sr'}&
Print out parameters of the current presentation; <val> must be  0  or
1.

No argument, or `<val> = 0', prints out the  `Group  Name',  `Subgroup
Name', the group's `relators'  and  the  subgroup's  `generators'.  If
`<val> = 1', then the current setting of the \lq{}run  parameters'  is
also printed. The printout is the same as that produced at  the  start
of a run when option `messages' (see~"Option `messages'") is non-zero.

Interactively,  use  `ACEGroupGenerators'  (see~"ACEGroupGenerators"),
`ACERelators'   (see~"ACERelators"),   `ACESubgroupGenerators'    (see
"ACESubgroupGenerators"), and `ACEParameters' (see~"ACEParameters").

*Notes:*
Only use *after* an enumeration run; otherwise, the value 0  for  some
options will be unreliable.

\>`print'{Option `print'}
\>`print:=<val>'{Option `print'}
\>`print:=[<val>]'{Option `print'}
\>`print:=[<val>, <last>]'{Option `print'}
\>`print:=[<val>, <last>, <by>]'{Option `print'}&
Compact and print the coset table;  <val>  must  be  an  integer,  and
<last> and <by> must be positive integers.
(Shortest abbreviation: `pr'.)

In the first (no value) form, `print' prints the entire  coset  table,
without orders or coset  representatives.  In  the  second  and  third
forms, the absolute value of <val> is taken to be the last line of the
table to be printed (and 1 is taken to be the first);  in  the  fourth
and fifth forms, `|<val>|' is taken to be the first line of the  table
to be printed, and <last> is taken to be the number of the  last  line
to be printed. In the last  form,  the  table  is  printed  from  line
`|<val>|' to line <last> in steps of <by>. If <val> is negative,  then
the  orders   modulo   the   subgroup   (if   available)   and   coset
representatives are printed also.

\>`rc:=<val>'{Option `rc'}
\>`rc:=[<val>]'{Option `rc'}
\>`rc:=[<val>, <attempts>]'{Option `rc'}&
Enforce `r'andom `c'oincidences; <val> and <attempts> must be positive
integers.

This option attempts upto <attempts> (or,  in  the  first  and  second
forms, 8) times to find nontrivial subgroups with index a multiple  of
<val> by repeatedly making random coset numbers coincident with  coset
1 and seeing what happens. The starting coset table must be non-empty,
but need not be complete. For each attempt, we repeatedly  add  random
coset representatives to the subgroup and `redo' the  enumeration.  If
the table becomes too small, the  attempt  is  aborted,  the  original
subgroup generators restored, and another attempt made. If an  attempt
succeeds, then the new set of subgroup generators is retained.

Interactively, use `ACERandomCoincidences' (see~"ACERandomCoincidences").

*Guru Notes:*
A coset number can have many different coset representatives. Consider
running `standard' before `rc', to canonicise the table and hence  the
coset representatives.

\>`recover'{Option `recover'}
\>`contiguous'{Option `contiguous'}&
`Recover' space used by dead coset numbers\index{dead coset (number)}.
(Shortest  abbreviation  of  `recover'   is   `reco',   and   shortest
abbreviation of `contiguous' is `contig'.)

This option invokes the compaction routine on the table to recover the
space used by any dead coset numbers. A `CO' message line  is  printed
if any cosets were recovered, and a  `co'  line  if  none  were.  This
routine is called automatically if  the  `cycles',  `nc',  `print'  or
`standard' options  (see~"Option  `cycles'",  "Option  `nc'",  "Option
`print'" and "Option `standard'") are invoked.

Interactively, use `ACERecover' (see~"ACERecover").

\>`sc:=<val>'{Option `sc'}
\>`stabilising:=<val>'{Option `stabilising'}&
Print out the coset numbers whose elements stabilise  (i.e.~normalise)
the subgroup; <val> must be an integer.
(Shortest abbreviation of `stabilising' is `stabil'.)

If `<val> > 0', the first `<val>' non-trivial (i.e.~other  than  coset
1) coset numbers whose elements stabilise the subgroup are printed. If
`<val> = 0', all non-trivial coset numbers  whose  elements  stabilise
the subgroup, plus their representatives, are printed.  If  `<val>  \<
0', the first  `|<val>|'  non-trivial  coset  numbers  whose  elements
stabilise the subgroup, plus their representatives, are printed.

Interactively,          use           `ACECosetsThatStabiliseSubgroup'
(see~"ACECosetsThatStabiliseSubgroup").

\>`standard'{Option `standard'}&
Compacts {\ACE}'s  coset  table  and  standardises  the  numbering  of
cosets. (Shortest abbreviation: `st'.)

For a given ordering of the generators in the columns of the table, it
produces a canonical numbering of the cosets. This function  does  not
display the new table; use  the  `print'  (see~"Option  `print'")  for
that. Such a table has the property that a row-major scan (i.e.~a scan
of the successive rows of the *body* of the table  row  by  row,  from
left to right) encounters previously unseen cosets in numeric order.

*Notes:*
In a canonic  table,  the  coset  representatives  are  ordered  first
according to length and then the lexicographic order  defined  by  the
order the generators and their inverses head the columns.  Note  that,
since {\ACE} avoids having an involutory generator in the first column
when it can, this lexicographic order does not necessarily  correspond
with the order in which the generators were first put to  {\ACE}.  Two
tables are equivalent only if their canonic forms are the same.  Also,
standardising the coset numbering within {\ACE} does *not* affect  the
{\GAP} coset table obtained via `ACECosetTable'.

*Guru Notes:*
In  half  of  the  ten  standard  enumeration   strategies   of   Sims
\cite{Sim94}, the table is standardised repeatedly. This is  expensive
computationally, but can result in fewer cosets being  necessary.  The
effect of doing this can be investigated  in  {\ACE}  by  (repeatedly)
halting  the  enumeration  (by  say,  imposing  timing  restrictions),
standardising the coset numbering, and continuing.

Interactively,             use             `ACEStandardCosetNumbering'
(see~"ACEStandardCosetNumbering").

\>`statistics'{Option `statistics'}
\>`stats'{Option `stats'}&
Dump enumeration statistics.
(Shortest abbreviation of `statistics' is `stat'.)

If the statistics package is compiled into the {\ACE} code,  which  it
is by default (see the `options'~"Option `options'" option), then this
option  dumps  the  statistics  accumulated  during  the  most  recent
enumeration. See the `enum.c' source  file  for  the  meaning  of  the
variables.

Interactively, use `ACEDumpStatistics' (see~"ACEDumpStatistics").

\>`style'{Option `style'}&
Print the current enumeration style.

This option prints the current enumeration style, as deduced from  the
current `ct' and `rt' parameters (see~"Enumeration Style").

Interactively, use `ACEStyle' (see~"ACEStyle").

\>`system:=<string>'{Option `system'}&
Do a shell escape, to execute <string>; <string> must be a string.
(Shortest abbreviation: `sys'.)

Since {\GAP} already provides `Exec()' for this purpose,  this  option
is unlikely to have a use.

\>`text:=<string>'{Option `text'}&
Prints <string> in the output; <string> must be a string.

This allows the user to add comments to the output from {\ACE}.

\>`tw:=[<val>, <word>]'{Option `tw'}
\>`trace:=[<val>, <word>]'{Option `trace'}&
Traces `<word>' through the coset table, starting  at  coset  `<val>';
<val> must be a positive integer, and <word> must be  a  word  in  the
group generators.

This option prints the final coset number of the trace, if  the  trace
completes.

Interactively, use `ACETraceWord' (see~"ACETraceWord").

\>`aceincomment:=<string>'{Option `aceincomment'}&
Print comment <string> in the {\ACE} input; <string> must be a string.
(Shortest abbreviation: `aceinc'.)

This option prints the comment <string> behind a sharp sign (`\#')  in
the input to {\ACE}. Only useful  for  adding  comments  (that  {\ACE}
ignores) to standalone input files.

\enditems

%%%%%%%%%%%%%%%%%%%%%%%%%%%%%%%%%%%%%%%%%%%%%%%%%%%%%%%%%%%%%%%%%%%%%%%
\Chapter{Examples}

In this chapter  we  collect  together  a  number  of  examples  which
illustrate the various ways in which the {\ACE} Share Package  may  be
used, and give some interactions with the  `ACEExample'  function.  In
the first few cases, we have set the `InfoLevel' of `InfoACE' to 3, so
that (except for {\ACE}'s banner in non-interactive use),  all  output
from {\ACE} is displayed, prepended by \lq{}`\#I ''. We  have  omitted
the line

\beginexample
gap> RequirePackage("ace");
\endexample

which is,  of  course,  required  at  the  beginning  of  any  session
requiring {\ACE}.

%%%%%%%%%%%%%%%%%%%%%%%%%%%%%%%%%%%%%%%%%%%%%%%%%%%%%%%%%%%%%%%%%%%%%%
\Section{Example where ACE is made the Standard Coset Enumerator}

If {\ACE} is made the standard coset enumerator, one simply  uses  the
method of passing arguments normally used  with  those  commands  that
invoke `CosetTableFromGensAndRels', but one is able to use all options
available via the {\ACE} interface. As an example  we  use  {\ACE}  to
compute the permutation representation of a  perfect  group  from  the
data library (in this case, an automorphic  extension  of  the  simple
alternating group $A_5$):

\beginexample
gap> SetInfoACELevel(3); # Just to see what's going on behind the scenes
gap> TCENUM:=ACETCENUM;; # Make ACE the standard coset enumerator
gap> G := PerfectGroup(IsPermGroup, 16*60, 1   # Arguments ... as per usual
>                      : max := 50, mess := 10 # ... but we use ACE options
>                      );
#I    #-- ACE 3.000: Run Parameters ---
#I  Group Name: G;
#I  Group Generators: abstuv;
#I  Group Relators: (a)^2, (s)^2, (t)^2, (u)^2, (v)^2, (b)^3, (st)^2, (uv)^2, 
#I    (su)^2, (sv)^2, (tu)^2, (tv)^2, asau, atav, auas, avat, Bvbu, Bsbvt, 
#I    Bubvu, Btbvuts, (ab)^5;
#I  Subgroup Name: H;
#I  Subgroup Generators: a, b;
#I  Wo:1000000; Max:50; Mess:10; Ti:-1; Ho:-1; Loop:0;
#I  As:0; Path:0; Row:1; Mend:0; No:21; Look:0; Com:10;
#I  C:0; R:0; Fi:11; PMod:3; PSiz:256; DMod:4; DSiz:1000;
#I    #--------------------------------
#I  SG: a=1 r=1 h=1 n=2; l=1 c=+0.00; m=1 t=1
#I  RD: a=11 r=1 h=1 n=12; l=2 c=+0.00; m=11 t=11
#I  RD: a=21 r=2 h=1 n=22; l=2 c=+0.00; m=21 t=21
#I  CC: a=29 r=4 h=1 n=31; l=2 c=+0.00; d=0
#I  CC: a=19 r=4 h=1 n=31; l=2 c=+0.00; d=0
#I  CC: a=19 r=6 h=1 n=36; l=2 c=+0.00; d=0
#I  INDEX = 16 (a=16 r=36 h=1 n=36; l=3 c=0.00; m=30 t=35)
#I  CO: a=16 r=17 h=1 n=17; c=+0.00
\endexample
\beginexample
#I   coset |      b      B      a      s      t      u      v
#I  -------+-------------------------------------------------
#I       1 |      1      1      1      2      3      4      5
#I       2 |     11     14      4      1      6      8      9
#I       3 |     13     15      5      6      1     10     11
#I       4 |      7      5      2      8     10      1      7
#I       5 |      4      7      3      9     11      7      1
#I       6 |      8     10      7      3      2     12     14
#I       7 |      5      4      6     15     16      5      4
#I       8 |     10      6      8      4     12      2     15
#I       9 |     16     12     10      5     14     15      2
#I      10 |      6      8      9     12      4      3     16
#I      11 |     14      2     11     14      5     16      3
#I      12 |      9     16     15     10      8      6     13
#I      13 |     15      3     13     16     15     14     12
#I      14 |      2     11     16     11      9     13      6
#I      15 |      3     13     12      7     13      9      8
#I      16 |     12      9     14     13      7     11     10
A5 2^4
gap> GeneratorsOfGroup(G); # Just to show we indeed have a perm'n rep'n
[ ( 2, 4)( 3, 5)( 7,12)( 9,11)(13,14)(15,16), 
  ( 2, 6,13)( 3,10,16)( 4,12, 5)( 7, 8,11)( 9,14,15), 
  ( 1, 2)( 3, 7)( 4, 8)( 5, 9)( 6,13)(10,14)(11,15)(12,16), 
  ( 1, 3)( 2, 7)( 4,11)( 5, 6)( 8,15)( 9,13)(10,16)(12,14), 
  ( 1, 4)( 2, 8)( 3,11)( 5,12)( 6,14)( 7,15)( 9,16)(10,13), 
  ( 1, 5)( 2, 9)( 3, 6)( 4,12)( 7,13)( 8,16)(10,15)(11,14) ]
gap> Order(G);
960
\endexample

%%%%%%%%%%%%%%%%%%%%%%%%%%%%%%%%%%%%%%%%%%%%%%%%%%%%%%%%%%%%%%%%%%%%%%
\Section{Example of Using ACECosetTableFromGensAndRels}

The following example calls {\ACE} for up to 800 coset  numbers  using
Mendelsohn style relator processing and sets the message level to 500.
The value of `table', i.e.~the {\GAP} coset table, immediately follows
the last {\ACE} message (\lq{}`\#I '') line, but both the coset  table
from {\ACE} and the  {\GAP}  coset  table  have  been  abbreviated.  A
slightly modified version of this example, which includes  the  `echo'
option is available on-line via `table  :=  ACEExample("perf602p5");'.
You may wish to peruse the notes  in  the  `ACEExample'  index  first,
however, by executing `ACEExample();'.

\beginexample
gap> SetInfoACELevel(3);
gap> G := PerfectGroup(2^5*60, 2);;
gap> fgens := FreeGeneratorsOfFpGroup(G);;
gap> table := ACECosetTableFromGensAndRels(
>                 # arguments
>                 fgens, RelatorsOfFpGroup(G), fgens{[1]}
>                 # options
>                 : mendelsohn, max:=800, mess:=500);
#I    #-- ACE 3.000: Run Parameters ---
#I  Group Name: G;
#I  Group Generators: abstuvd;
#I  Group Relators: (s)^2, (t)^2, (u)^2, (v)^2, (d)^2, aad, (b)^3, (st)^2, 
#I    (uv)^2, (su)^2, (sv)^2, (tu)^2, (tv)^2, Asau, Atav, Auas, Avat, Bvbu, 
#I    dAda, dBdb, (ds)^2, (dt)^2, (du)^2, (dv)^2, Bubvu, Bsbdvt, Btbvuts, 
#I    (ab)^5;
#I  Subgroup Name: H;
#I  Subgroup Generators: a;
#I  Wo:1000000; Max:800; Mess:500; Ti:-1; Ho:-1; Loop:0;
#I  As:0; Path:0; Row:1; Mend:1; No:28; Look:0; Com:10;
#I  C:0; R:0; Fi:13; PMod:3; PSiz:256; DMod:4; DSiz:1000;
#I    #--------------------------------
#I  SG: a=1 r=1 h=1 n=2; l=1 c=+0.00; m=1 t=1
#I  RD: a=321 r=68 h=1 n=412; l=5 c=+0.01; m=327 t=411
#I  CC: a=435 r=162 h=1 n=719; l=9 c=+0.01; d=0
#I  CL: a=428 r=227 h=1 n=801; l=13 c=+0.04; m=473 t=800
#I  DD: a=428 r=227 h=1 n=801; l=14 c=+0.00; d=534
#I  DD: a=428 r=227 h=1 n=801; l=14 c=+0.01; d=32
#I  CO: a=428 r=192 h=243 n=429; l=15 c=+0.00; m=473 t=800
#I  INDEX = 480 (a=480 r=210 h=484 n=484; l=18 c=0.08; m=480 t=855)
#I  CO: a=480 r=210 h=481 n=481; c=+0.00
#I   coset |      a      A      b      B      s      t      u      v      d
#I  -------+---------------------------------------------------------------
#I       1 |      1      1      7      6      2      3      4      5      1
#I       2 |      4      4     22     36      1      8     10     11      2
#I       3 |      5      5     30     23      8      1     12     13      3
#I       4 |      2      2     17     14     10     12      1      9      4
#I       5 |      3      3     15     19     11     13      9      1      5
... 470 lines omitted here ...
#I     476 |    469    469    407    406    478    472    466    480    476
#I     477 |    477    477    314    313    474    471    467    468    477
#I     478 |    473    473    380    379    476    467    471    470    478
#I     479 |    479    479    384    383    475    468    470    471    479
#I     480 |    480    480    421    420    470    469    475    476    480
[ [ 1, 7, 5, 6, 3, 4, 2, 27, 25, 26, 23, 24, 39, 21, 15, 18, 46, 16, 19, 51, 
      14, 52, 11, 12, 9, 10, 8, 68, 69, 66, 67, 75, 64, 59, 62, 77, 60, 79, 
... 30 lines omitted here ...
      478, 476, 441, 475, 473, 480, 472, 471, 477 ],
[ [ 1, 7, 5, 6, 3, 4, 2, 27, 25, 26, 23, 24, 39, 21, 15, 18, 46, 16, 19, 51, 
      14, 52, 11, 12, 9, 10, 8, 68, 69, 66, 67, 75, 64, 59, 62, 77, 60, 79, 
      478, 476, 441, 475, 473, 480, 472, 471, 477 ], 
... 396 lines omitted here ...
  [ 1, 2, 3, 4, 5, 6, 7, 8, 9, 10, 11, 12, 13, 14, 15, 16, 17, 18, 19, 20, 
      21, 22, 23, 24, 25, 26, 27, 28, 29, 30, 31, 32, 33, 34, 35, 36, 37, 38, 
... 30 lines omitted here ...
      472, 473, 474, 475, 476, 477, 478, 479, 480 ] ]
\endexample

%%%%%%%%%%%%%%%%%%%%%%%%%%%%%%%%%%%%%%%%%%%%%%%%%%%%%%%%%%%%%%%%%%%%%%
\Section{Example of Using ACE Interactively (Using ACEStart)}

Now we illustrate a simple interactive process, with an enumeration of
an index 12 subgroup (isomorphic to $C_5$) within $A_5$. Observe  that
we  have  relied  on  the  default  level  of  messaging  from  {\ACE}
(`messages' = 0) which gives a result  line  only,  without  parameter
information. Interactively, {\ACE}'s banner is not redirected  but  it
is only observed when the `InfoLevel'  of  `InfoACE'  is  4.  We  have
however used the option `echo', so that we can see how  the  interface
handled the arguments and options. On-line  try:  `SetInfoACELevel(3);
ACEExample("A5-C5", ACEStart);' to emulate the session  prior  to  the
`ACEQuit' command.

\beginexample
gap> SetInfoACELevel(3);
gap> F := FreeGroup("a","b");; a := F.1;;  b := F.2;;
gap> G := F / [a^2, b^3, (a*b)^5 ];
<fp group on the generators [ a, b ]>
gap> ACEStart(FreeGeneratorsOfFpGroup(G), RelatorsOfFpGroup(G), [a*b]
>          # Options
>          : echo, # Echo handled by GAP (not ACE)
>            enum := "A_5",  # Give the group G a meaningful name
>            subg := "C_5"); # Give the subgroup a meaningful name
ACEStart called with the following arguments:
 Group generators : [ a, b ]
 Group relators : [ a^2, b^3, a*b*a*b*a*b*a*b*a*b ]
 Subgroup generators : [ a*b ]
ACEStart called with the following options:
 echo := true (not passed to ACE)
 enum := A_5
 subg := C_5
#I  INDEX = 12 (a=12 r=16 h=1 n=16; l=3 c=0.00; m=14 t=15)
1
gap> # The return value on the last line identifies the interactive process
gap> # ... which we use with functions that need to interact with it:      
gap> ACEStats(1);    
rec( index := 12, cputime := 0, cputimeUnits := "10^-2 seconds", 
  maxcosets := 14, totcosets := 15 )
gap> # Actually, we didn't need to pass an argument to ACEStats()          
gap> # ... we could have relied on the default:                            
gap> ACEStats();                                                 
rec( index := 12, cputime := 0, cputimeUnits := "10^-2 seconds", 
  maxcosets := 14, totcosets := 15 )
gap> # Similarly, we can use ACECosetTable() (which returns the 
gap> # `standardized' coset table) with or without an argument:  
gap> ACECosetTable(); # Interactive version of ACECosetTableFromGensAndRels()
#I  CO: a=12 r=13 h=1 n=13; c=+0.00
#I   coset |      b      B      a
#I  -------+---------------------
#I       1 |      3      2      2
#I       2 |      1      3      1
#I       3 |      2      1      4
#I       4 |      8      5      3
#I       5 |      4      8      6
#I       6 |      9      7      5
#I       7 |      6      9      8
#I       8 |      5      4      7
#I       9 |      7      6     10
#I      10 |     12     11      9
#I      11 |     10     12     12
#I      12 |     11     10     11
[ [ 2, 1, 4, 3, 6, 5, 8, 7, 10, 9, 12, 11 ], 
  [ 2, 1, 4, 3, 6, 5, 8, 7, 10, 9, 12, 11 ], 
  [ 3, 1, 2, 5, 7, 8, 4, 9, 6, 11, 12, 10 ], 
  [ 2, 3, 1, 7, 4, 9, 5, 6, 8, 12, 10, 11 ] ]
gap> # To terminate the interactive process we do:
gap> ACEQuit(1); # Again, we could have omitted the 1
gap> # If we had more than one interactive process we could have 
gap> # terminated them all in one go with:
gap> ACEQuitAll();
\endexample

%%%%%%%%%%%%%%%%%%%%%%%%%%%%%%%%%%%%%%%%%%%%%%%%%%%%%%%%%%%%%%%%%%%%%%
\Section{Fun with ACEExample}

First let's see the `ACEExample' index  (obtained  with  no  argument,
with  `"index"'  as  argument,  or  with  a  non-existent  example  as
argument):

\beginexample
gap> ACEExample();
#I                             ACEExample Index
#I                             ----------------
#I  This index is displayed when calling ACEExample with no arguments, or
#I  with the argument: "index", or with a non-existent example name.
#I  
#I  The following ACE examples are available (in each case, for a subgroup
#I  H of a group G, the cosets of H in G are enumerated):
#I  
#I    Example          G                      H              strategy
#I    -------          -                      -              --------
#I    "A5"             A_5                    Id             default
#I    "A5-C5"          A_5                    C_5            default
#I    "C5-fel0"        C_5                    Id             felsch := 0
#I    "F27-purec"      F(2,7) = C_29          Id             purec
#I    "F27-fel0"       F(2,7) = C_29          Id             felsch := 0
#I    "F27-fel1"       F(2,7) = C_29          Id             felsch := 1
#I    "M12-hlt"        M_12 (Matthieu group)  Id             hlt
#I    "M12-fel1"       M_12 (Matthieu group)  Id             felsch := 1
#I    "SL219-hard"     SL(2,19)               ||G : H|| = 180  hard
#I    "perf602p5"      PerfectGroup(60*2^5,2) ||G : H|| = 480  default
#I  * "2p17-fel1"      ||G|| = 2^17             ||G : H|| = 1    felsch := 1
#I  * "2p18-fel1"      ||G|| = 2^18             ||G : H|| = 2    felsch := 1
#I  * "big-fel1"       ||G|| = 2^18.3           ||G : H|| = 6    felsch := 1
#I  * "big-hard"       ||G|| = 2^18.3           ||G : H|| = 6    hard
#I    "2p17-id-fel1"   ||G|| = 2^17             Id             felsch := 1
#I    "2p17-2p14-fel1" ||G|| = 2^17             ||G : H|| = 2^14 felsch := 1
#I    "2p17-2p3-fel1"  ||G|| = 2^17             ||G : H|| = 2^3  felsch := 1
#I    "2p17-fel1a"     ||G|| = 2^17             ||G : H|| = 1    felsch := 1
#I  
#I  Notes
#I  -----
#I  1. The example (first) argument of  ACEExample()  is  a  string; each
#I     example above is in double quotes to remind you to include them.
#I  2. The enumeration for each of the *-ed examples fails. (See Note 3.)
#I  3. Try altering the ACE function used, by calling  ACEExample with  a
#I     2nd argument; choose from: ACECosetTableFromGensAndRels (default),
#I     or ACEStats, or ACEStart. The 2nd argument is *not* quoted.
#I  4. Try `SetInfoACELevel(3);' before calling  ACEExample,  to  see the
#I     effect of setting the "mess" (= "messages") option.
#I  5. To suppress a long output, use a double semicolon (`;;') after the
#I     ACEExample command.
#I  6. Also, try `SetInfoACELevel(2);' before calling ACEExample.
gap>
\endexample

Observe that the example we first met in Section~"Using  ACE  Directly
to Generate a Coset Table", the Fibonacci group F(2,7),  is  available
via examples `"F27-purec"',  `"F27-fel0"',  and  `"F27-fel1"',  except
each of these enumerate the cosets of its trivial subgroup  (of  index
29). Let's experiment with the first of these F(2,7)  examples;  since
this  example  uses  the  `messages'  option,  we  ought  to  set  the
`InfoLevel' of `InfoACE' to  3,  first,  but  since  the  coset  table
(default output) is  quite  long,  we'll  pass  `ACEStats'  as  second
argument:

\beginexample
gap> SetInfoACELevel(3);
gap> ACEExample("F27-purec", ACEStats);
#I  # ACEExample "F27-purec" : enumeration of cosets of H in G,
#I  # where G = F(2,7) = C_29, H = Id, using purec strategy.
#I  #
#I  # F, G, a, b, c, d, e, x, y are local to ACEExample
#I  # We define F(2,7) on 7 generators
#I  F := FreeGroup("a","b","c","d","e", "x", "y"); 
#I       a := F.1;  b := F.2;  c := F.3;  d := F.4; 
#I       e := F.5;  x := F.6;  y := F.7;
#I  G := F / [a*b*c^-1, b*c*d^-1, c*d*e^-1, d*e*x^-1, 
#I            e*x*y^-1, x*y*a^-1, y*a*b^-1];
#I  ACEStats(
#I      FreeGeneratorsOfFpGroup(G), 
#I      RelatorsOfFpGroup(G), 
#I      [] # Generators of identity subgroup (empty list)
#I      # Options that don't affect the enumeration
#I      : echo, enum := "F(2,7), aka C_29", subg := "Id",
#I      # Other options
#I      wo := "2M", mess := 25000, purec);
ACEStats called with the following arguments:
 Group generators : [ a, b, c, d, e, x, y ]
 Group relators : [ a*b*c^-1, b*c*d^-1, c*d*e^-1, d*e*x^-1, e*x*y^-1, 
  x*y*a^-1, y*a*b^-1 ]
 Subgroup generators : [  ]
ACEStats called with the following options:
 echo := true (not passed to ACE)
 enum := F(2,7), aka C_29
 subg := Id
 wo := 2M
 mess := 25000
 purec (no value)
#I    #-- ACE 3.000: Run Parameters ---
#I  Group Name: F(2,7), aka C_29;
#I  Group Generators: abcdexy;
#I  Group Relators: abC, bcD, cdE, deX, exY, xyA, yaB;
#I  Subgroup Name: Id;
#I  Subgroup Generators: ;
#I  Wo:2M; Max:142855; Mess:25000; Ti:-1; Ho:-1; Loop:0;
#I  As:0; Path:0; Row:0; Mend:0; No:0; Look:0; Com:100;
#I  C:1000; R:0; Fi:1; PMod:0; PSiz:256; DMod:4; DSiz:1000;
#I    #--------------------------------
#I  DD: a=5290 r=1 h=1050 n=5291; l=8 c=+0.03; d=2
#I  CD: a=10410 r=1 h=2149 n=10411; l=13 c=+0.03; m=10410 t=10410
#I  DD: a=15428 r=1 h=3267 n=15429; l=18 c=+0.03; d=0
#I  DD: a=20430 r=1 h=4386 n=20431; l=23 c=+0.03; d=1
#I  DD: a=25397 r=1 h=5519 n=25399; l=28 c=+0.02; d=1
#I  CD: a=30313 r=1 h=6648 n=30316; l=33 c=+0.03; m=30313 t=30315
#I  DS: a=32517 r=1 h=7326 n=33240; l=36 c=+0.02; s=2000 d=997 c=4
#I  DS: a=31872 r=1 h=7326 n=33240; l=36 c=+0.01; s=4000 d=1948 c=53
#I  DS: a=29077 r=1 h=7326 n=33240; l=36 c=+0.01; s=8000 d=3460 c=541
#I  DS: a=23433 r=1 h=7326 n=33240; l=36 c=+0.02; s=16000 d=5940 c=2061
#I  DS: a=4163 r=1 h=7326 n=33240; l=36 c=+0.08; s=32000 d=447 c=15554
#I  INDEX = 29 (a=29 r=1 h=33240 n=33240; l=37 c=0.41; m=33237 t=33239)
rec( index := 29, cputime := 41, cputimeUnits := "10^-2 seconds", 
  maxcosets := 33237, totcosets := 33239 )
gap>
\endexample

Observe that the first group of `Info' lines list  the  commands  that
were executed; these lines are followed by the result  of  the  `echo'
option (see~"Option `echo'"); which in turn  are  followed  by  `Info'
messages from {\ACE} courtesy of the non-zero value of the  `messages'
option (and we see these because  we  first  set  the  `InfoLevel'  of
`InfoACE' to 3); and finally,  we  get  the  output  (record)  of  the
`ACEStats' command.

Observe also that {\ACE} has used the same generators as  were  input;
this will always occur if you stick to single  lowercase  letters  for
your generator names. Note, also that capitalisation is used by {\ACE}
as a short-hand for inverses, e.g.~`C = c^-1' (see `Group Relators' in
the {\ACE} \lq{}Run Parameters' block).

Now let's observe that we can add some new  options,  even  ones  that
over-ride the example's options; but first we'll reduce the  output  a
bit by setting the `InfoLevel' of `InfoACE' to 2 and since we are  not
going toobserve any progress mesages form {\ACE} with  that  `InfoACE'
level we'll set `messages := 0'; also we'll use the  default  function
`ACECosetTableFromGensAndRels' and so it's like  our  first  encounter
with F(2,7) we'll add the subgroup generator `c'  via  `sg  :=  ["c"]'
(see "Option `sg'"). This is a bit of a cheat. Observe that `"c"' is a
string not a {\GAP} group generator; so a warning message  appears  in
the output. Nevertheless, the {\ACE} binary is passed the string `"c"'
which is what it uses to  identify  the  {\GAP}  group  generator;  so
{\ACE} itself doesn't complain. (You *can* rely on this,  so  long  as
all the {\GAP} group generators are single lower case letter  strings;
and for this example, it's fortunate that we  could  since  the  group
generators of each `ACEExample' example are *local* variables and  are
so not accessible.) For  good  measure,  we  also  change  the  string
identifying the subgroup (since it  will  no  longer  be  the  trivial
group), via the `subgroup' option (see "Option `subgroup'").

\beginexample
gap> SetInfoACELevel(2);                                                       
gap> ACEExample("F27-purec" : sg := ["c"], subgroup := "< c >", messages := 0);
#I  # ACEExample "F27-purec" : enumeration of cosets of H in G,
#I  # where G = F(2,7) = C_29, H = Id, using purec strategy.
#I  #
#I  # F, G, a, b, c, d, e, x, y are local to ACEExample
#I  # We define F(2,7) on 7 generators
#I  F := FreeGroup("a","b","c","d","e", "x", "y"); 
#I       a := F.1;  b := F.2;  c := F.3;  d := F.4; 
#I       e := F.5;  x := F.6;  y := F.7;
#I  G := F / [a*b*c^-1, b*c*d^-1, c*d*e^-1, d*e*x^-1, 
#I            e*x*y^-1, x*y*a^-1, y*a*b^-1];
#I  ACECosetTableFromGensAndRels(
#I      FreeGeneratorsOfFpGroup(G), 
#I      RelatorsOfFpGroup(G), 
#I      [] # Generators of identity subgroup (empty list)
#I      # Options that don't affect the enumeration
#I      : echo, enum := "F(2,7), aka C_29", subg := "Id", 
#I      # Other options
#I      wo := "2M", mess := 25000, purec, 
#I      # User Options
#I        sg := [ "c" ],
#I        subgroup := "< c >",
#I        messages := 0);
ACECosetTableFromGensAndRels called with the following arguments:
 Group generators : [ a, b, c, d, e, x, y ]
 Group relators : [ a*b*c^-1, b*c*d^-1, c*d*e^-1, d*e*x^-1, e*x*y^-1, 
  x*y*a^-1, y*a*b^-1 ]
 Subgroup generators : [  ]
ACECosetTableFromGensAndRels called with the following options:
 aceexampleoptions := true (inserted by ACEExample, not passed to ACE)
 echo := true (not passed to ACE)
 enum := F(2,7), aka C_29
 wo := 2M
 purec (no value)
 sg := [ "c" ] (brackets are not passed to ACE)
#I  ACE Warning: [ "c" ]: possibly not an allowed value of sg
 subgroup := < c >
 messages := 0
#I  INDEX = 1 (a=1 r=2 h=2 n=2; l=4 c=0.00; m=332 t=332)
[ [ 1 ], [ 1 ], [ 1 ], [ 1 ], [ 1 ], [ 1 ], [ 1 ], [ 1 ], [ 1 ], [ 1 ], 
  [ 1 ], [ 1 ], [ 1 ], [ 1 ] ]
gap>
\endexample

Observe that in the block of `Info' output all  the  original  example
options are listed along with our new options `sg := [ "c" ], messages
:= 0' after the tag \lq{}`\#  User  Options''.  Following  the  `Info'
block there is the block due to `echo'; in its listing of the  options
first up there is `aceexampleoptions' alerting us that we passed  some
`ACEExample' options. Observe that `subg := Id' and  `mess  :=  25000'
have disappeared (they were over-ridden by  `subgroup  :=  "\<  c  >",
messages := 0', but the quotes for the value  of  `subgroup'  are  not
visible); note that we didn't have to use the  same  abbreviations  to
over-ride them. Also observe that our  new  options  *are*  last,  and
observe the warning due to the value of `sg' not being a {\GAP}  group
word.

Now following on from our last example we shall  demonstrate  how  one
can recover  from  a  `break'-loop{\undoquotes  \atindex  {break-loop}
{@`break'-loop}} (see Section~"Using ACE Directly to Generate a  Coset
Table"). To force the `break'-loop we pass  `max  :=  2'  (see~"Option
`max'"),  while  using   the   default   {\ACE}   interface   function
`ACECosetTableFromGensAndRels' of `ACEExample'; in  this  way,  {\ACE}
will not be able to complete  the  enumeration,  and  hence  enters  a
`break'-loop when it tries to provide a complete  coset  table.  While
we're at it we'll pass the `hlt' (see~"Option `hlt'") strategy  option
(which will over-ride `purec'). (The `InfoACE' level is still 2.) Note
that there are some \lq{}user-input' comments inserted at  the  `brk>'
prompt.

\beginexample
gap> ACEExample("F27-purec" : sg := ["c"], subgroup := "< c >", max := 2, hlt);
#I  # ACEExample "F27-purec" : enumeration of cosets of H in G,
#I  # where G = F(2,7) = C_29, H = Id, using purec strategy.
#I  #
#I  # F, G, a, b, c, d, e, x, y are local to ACEExample
#I  # We define F(2,7) on 7 generators
#I  F := FreeGroup("a","b","c","d","e", "x", "y"); 
#I       a := F.1;  b := F.2;  c := F.3;  d := F.4; 
#I       e := F.5;  x := F.6;  y := F.7;
#I  G := F / [a*b*c^-1, b*c*d^-1, c*d*e^-1, d*e*x^-1, 
#I            e*x*y^-1, x*y*a^-1, y*a*b^-1];
#I  ACECosetTableFromGensAndRels(
#I      FreeGeneratorsOfFpGroup(G), 
#I      RelatorsOfFpGroup(G), 
#I      [] # Generators of identity subgroup (empty list)
#I      # Options that don't affect the enumeration
#I      : echo, enum := "F(2,7), aka C_29", subg := "Id", 
#I      # Other options
#I      wo := "2M", mess := 25000, purec, 
#I      # User Options
#I        sg := [ "c" ],
#I        subgroup := "< c >",
#I        max := 2,
#I        hlt := true);
ACECosetTableFromGensAndRels called with the following arguments:
 Group generators : [ a, b, c, d, e, x, y ]
 Group relators : [ a*b*c^-1, b*c*d^-1, c*d*e^-1, d*e*x^-1, e*x*y^-1, 
  x*y*a^-1, y*a*b^-1 ]
 Subgroup generators : [  ]
ACECosetTableFromGensAndRels called with the following options:
 aceexampleoptions := true (inserted by ACEExample, not passed to ACE)
 echo := true (not passed to ACE)
 enum := F(2,7), aka C_29
 wo := 2M
 mess := 25000
 purec (no value)
 sg := [ "c" ] (brackets are not passed to ACE)
#I  ACE Warning: [ "c" ]: possibly not an allowed value of sg
 subgroup := < c >
 max := 2
 hlt (no value)
#I  OVERFLOW (a=2 r=1 h=1 n=3; l=4 c=0.00; m=2 t=2)
Error : No coset table ... at
Error( ": No coset table ..." );
#I  The `ACE' coset enumeration failed with the result:
#I  OVERFLOW (a=2 r=1 h=1 n=3; l=4 c=0.00; m=2 t=2)
#I  Try relaxing any restrictive options:
#I  type: 'DisplayACEOptions();' to see current ACE options;
#I  type: 'SetACEOptions(:<option1> := <value1>, ...);'
#I  to set <option1> to <value1> etc.
#I  (i.e. pass options after the ':' in the usual way)
#I  ... and then, type: 'return;' to continue.
#I  Otherwise, type: 'quit;' to quit the enumeration.
Entering break read-eval-print loop, you can 'quit;' to quit to outer loop,
or you can return to continue
brk> # Let's give ACE enough coset numbers to work with ...                    
brk> # and while we're at it see the effect of 'echo := 2' :                   
brk> SetACEOptions(: max := 0, echo := 2);                                     
brk> # Let's check what the options are now:                                   
brk> DisplayACEOptions();                                                      
rec(
  enum := "F(2,7), aka C_29",
  wo := "2M",
  mess := 25000,
  purec := true,
  sg := [ "c" ],
  subgroup := "< c >",
  hlt := true,
  max := 0,
  echo := 2 )
brk> # That's ok ... so now we 'return;' to escape the break-loop              
brk> return;                                                                   
ACECosetTableFromGensAndRels called with the following arguments:
 Group generators : [ a, b, c, d, e, x, y ]
 Group relators : [ a*b*c^-1, b*c*d^-1, c*d*e^-1, d*e*x^-1, e*x*y^-1, 
  x*y*a^-1, y*a*b^-1 ]
 Subgroup generators : [  ]
ACECosetTableFromGensAndRels called with the following options:
 aceexampleoptions := true (inserted by ACEExample, not passed to ACE)
 enum := F(2,7), aka C_29
 wo := 2M
 mess := 25000
 purec (no value)
 sg := [ "c" ] (brackets are not passed to ACE)
#I  ACE Warning: [ "c" ]: possibly not an allowed value of sg
 subgroup := < c >
 hlt (no value)
 max := 0
 echo := 2 (not passed to ACE)
Other options set via ACE defaults:
 asis := 0
 compaction := 10
 ct := 0
 dmode := 0
 dsize := 1000
 fill := 1
 hole := -1
 lookahead := 1
 loop := 0
 mendelsohn := 0
 no := 0
 path := 0
 pmode := 0
 psize := 256
 row := 1
 rt := 1000
 time := -1
#I  INDEX = 1 (a=1 r=2 h=2 n=2; l=3 c=0.01; m=2049 t=3127)
[ [ 1 ], [ 1 ], [ 1 ], [ 1 ], [ 1 ], [ 1 ], [ 1 ], [ 1 ], [ 1 ], [ 1 ], 
  [ 1 ], [ 1 ], [ 1 ], [ 1 ] ]
gap>
\endexample

Observe that  `purec'  did  *not*  disappear  from  the  option  list;
nevertheless, it *is* over-ridden by the `hlt' option (at  the  {\ACE}
level). Observe the \lq{}`Other options set via ACE defaults'' list of
options  that  has  resulted  from  having  the  `echo  :=  2'  option
(see~"Option `echo'"). Observe, also, that `hlt' is  nowhere  near  as
good, here, as  `purec'  (refer  to  Section~"Using  ACE  Directly  to
Generate  a  Coset  Table"):  whereas  `purec'  completed   the   same
enumeration with a total number of coset numbers  of  332,  the  `hlt'
strategy required 3127.

Of course, running `ACEExample' with  `ACEStart'  as  second  argument
opens up far more flexibility. Try it! Have fun!  Play  with  as  many
options as you can. Also, note that the `*'-ed examples of  the  index
fail  to  give  a  coset  table;  so  these  give  you  non-artificial
`break'-loop examples for you to try.

%%%%%%%%%%%%%%%%%%%%%%%%%%%%%%%%%%%%%%%%%%%%%%%%%%%%%%%%%%%%%%%%%%%%%%%%%
%%
%E
