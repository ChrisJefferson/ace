%%%%%%%%%%%%%%%%%%%%%%%%%%%%%%%%%%%%%%%%%%%%%%%%%%%%%%%%%%%%%%%%%%%%%%%%%
%%
%W  examples.tex        ACE appendix - examples          Alexander Hulpke
%W                                                      Joachim Neub"user
%W                                                            Greg Gamble
%%
%H  $Id$
%%
%Y  Copyright (C) 2000  Centre for Discrete Mathematics and Computing
%Y                      Department of Computer Science & Electrical Eng.
%Y                      University of Queensland, Australia.
%%

%%%%%%%%%%%%%%%%%%%%%%%%%%%%%%%%%%%%%%%%%%%%%%%%%%%%%%%%%%%%%%%%%%%%%%%
\Chapter{Examples}

In this chapter  we  collect  together  a  number  of  examples  which
illustrate the various ways in which the {\ACE} Share Package  may  be
used, and give some interactions with the `ACEExample' function. In  a
number of cases, we have set the `InfoLevel' of  `InfoACE'  to  3,  so
that all output from {\ACE} is displayed,  prepended  by  ```\#I  '''.
Recall that to also see the commands directed *to*  {\ACE}  (behind  a
```ToACE> ''' prompt), you will need to set the `InfoACE' level to  4.
We have omitted the line

\beginexample
gap> RequirePackage("ace");
\endexample

which is,  of  course,  required  at  the  beginning  of  any  session
requiring {\ACE}.

%%%%%%%%%%%%%%%%%%%%%%%%%%%%%%%%%%%%%%%%%%%%%%%%%%%%%%%%%%%%%%%%%%%%%%
\Section{Example where ACE is made the Standard Coset Enumerator}

If {\ACE} is made the standard coset enumerator, one simply  uses  the
method of passing arguments normally used  with  those  commands  that
invoke `CosetTableFromGensAndRels', but one is able to use all options
available via the {\ACE} interface. As an example  we  use  {\ACE}  to
compute the permutation representation of a  perfect  group  from  the
data library ({\GAP}'s perfect group library stores for each  group  a
presentation together with generators of a subgroup as  words  in  the
group generators such  that  the  permutation  representation  on  the
cosets  of  this  subgroup  will  be  a  (nice)  faithful  permutation
representation for the perfect group). The example we have  chosen  is
an automorphic extension of the simple alternating group $A_5$.

\beginexample
gap> TCENUM:=ACETCENUM;; # Make ACE the standard coset enumerator
gap> G := PerfectGroup(IsPermGroup, 16*60, 1   # Arguments ... as per usual
>                      : max := 50, mess := 10 # ... but we use ACE options
>                      );
A5 2^4
gap> GeneratorsOfGroup(G); # Just to show we indeed have a perm'n rep'n
[ (2,4)(3,5)(7,15)(8,14)(10,13)(12,16), (2,6,7)(3,11,12)(4,14,5)(8,9,13)(10,
    15,16), (1,2)(3,8)(4,9)(5,10)(6,7)(11,15)(12,14)(13,16), 
  (1,3)(2,8)(4,13)(5,6)(7,10)(9,16)(11,12)(14,15), 
  (1,4)(2,9)(3,13)(5,14)(6,15)(7,11)(8,16)(10,12), 
  (1,5)(2,10)(3,6)(4,14)(7,8)(9,12)(11,16)(13,15) ]
gap> Order(G);
960
\endexample

The call to `PerfectGroup' produced an output string  that  identifies
the group `G', but we didn't see  how  {\ACE}  became  involved  here.
Let's redo that part of the above  example  after  first  setting  the
`InfoLevel' of `InfoACE' to 3, so that we may get  to  glimpse  what's
going on behind the scenes.

\beginexample
gap> SetInfoACELevel(3); # Just to see what's going on behind the scenes
gap> # Recall that we did: TCENUM:=ACETCENUM;;
gap> G := PerfectGroup(IsPermGroup, 16*60, 1   # Arguments ... as per usual
>                      : max := 50, mess := 10 # ... but we use ACE options
>                      );
#I  ACE 3.000        Wed Oct 18 11:07:24 2000
#I  =========================================
#I  Host information:
#I    name = boronia
#I    #-- ACE 3.000: Run Parameters ---
#I  Group Name: G;
#I  Group Generators: abstuv;
#I  Group Relators: (a)^2, (s)^2, (t)^2, (u)^2, (v)^2, (b)^3, (st)^2, (uv)^2, 
#I    (su)^2, (sv)^2, (tu)^2, (tv)^2, asau, atav, auas, avat, Bvbu, Bsbvt, 
#I    Bubvu, Btbvuts, (ab)^5;
#I  Subgroup Name: H;
#I  Subgroup Generators: a, b;
#I  Wo:1000000; Max:50; Mess:10; Ti:-1; Ho:-1; Loop:0;
#I  As:0; Path:0; Row:1; Mend:0; No:21; Look:0; Com:10;
#I  C:0; R:0; Fi:11; PMod:3; PSiz:256; DMod:4; DSiz:1000;
#I    #--------------------------------
#I  SG: a=1 r=1 h=1 n=2; l=1 c=+0.00; m=1 t=1
#I  RD: a=11 r=1 h=1 n=12; l=2 c=+0.00; m=11 t=11
#I  RD: a=21 r=2 h=1 n=22; l=2 c=+0.00; m=21 t=21
#I  CC: a=29 r=4 h=1 n=31; l=2 c=+0.00; d=0
#I  CC: a=19 r=4 h=1 n=31; l=2 c=+0.00; d=0
#I  CC: a=19 r=6 h=1 n=36; l=2 c=+0.00; d=0
#I  INDEX = 16 (a=16 r=36 h=1 n=36; l=3 c=0.00; m=30 t=35)
#I  ***
#I  CO: a=16 r=17 h=1 n=17; c=+0.00
#I   coset ||      b      B      a      s      t      u      v
#I  -------+-------------------------------------------------
#I       1 ||      1      1      1      2      3      4      5
#I       2 ||     11     14      4      1      6      8      9
#I       3 ||     13     15      5      6      1     10     11
#I       4 ||      7      5      2      8     10      1      7
#I       5 ||      4      7      3      9     11      7      1
#I       6 ||      8     10      7      3      2     12     14
#I       7 ||      5      4      6     15     16      5      4
#I       8 ||     10      6      8      4     12      2     15
#I       9 ||     16     12     10      5     14     15      2
#I      10 ||      6      8      9     12      4      3     16
#I      11 ||     14      2     11     14      5     16      3
#I      12 ||      9     16     15     10      8      6     13
#I      13 ||     15      3     13     16     15     14     12
#I      14 ||      2     11     16     11      9     13      6
#I      15 ||      3     13     12      7     13      9      8
#I      16 ||     12      9     14     13      7     11     10
A5 2^4
\endexample

%%%%%%%%%%%%%%%%%%%%%%%%%%%%%%%%%%%%%%%%%%%%%%%%%%%%%%%%%%%%%%%%%%%%%%
\Section{Example of Using ACECosetTableFromGensAndRels}

The following example calls {\ACE} for up to 800 coset  numbers  (`max
:= 800') using Mendelsohn style relator processing (`mendelsohn')  and
sets progress messages to be printed every 500  iterations  (`messages
:=500'); we do ```SetInfoACELevel(3);''' so  that  we  may  see  these
messages.  The  value  of  `table',  i.e.~the  {\GAP}   coset   table,
immediately follows the last {\ACE}  message  (```\#I ''')  line,  but
both the coset table from {\ACE} and the {\GAP} coset table have  been
abbreviated. A  slightly  modified  version  of  this  example,  which
includes  the  `echo'  option  is  available  on-line  via  `table  :=
ACEExample("perf602p5");'. You may wish to peruse  the  notes  in  the
`ACEExample' index first, however, by executing `ACEExample();'. (Note
that the final table output here is `lenlex'  standardised,  since  we
used {\GAP} 4.3; with {\GAP} 4.2 the final table will be  `semilenlex'
standardised.)

\beginexample
gap> SetInfoACELevel(3);           # So we can see the progress messages
gap> G := PerfectGroup(2^5*60, 2);;# See previous example:
gap>                               # "Example where ACE is made the
gap>                               #  Standard Coset Enumerator"
gap> fgens := FreeGeneratorsOfFpGroup(G);;
gap> table := ACECosetTableFromGensAndRels(
>                 # arguments
>                 fgens, RelatorsOfFpGroup(G), fgens{[1]}
>                 # options
>                 : mendelsohn, max:=800, mess:=500);
#I  ACE 3.000        Wed Oct 18 11:23:13 2000
#I  =========================================
#I  Host information:
#I    name = boronia
#I    #-- ACE 3.000: Run Parameters ---
#I  Group Name: G;
#I  Group Generators: abstuvd;
#I  Group Relators: (s)^2, (t)^2, (u)^2, (v)^2, (d)^2, aad, (b)^3, (st)^2, 
#I    (uv)^2, (su)^2, (sv)^2, (tu)^2, (tv)^2, Asau, Atav, Auas, Avat, Bvbu, 
#I    dAda, dBdb, (ds)^2, (dt)^2, (du)^2, (dv)^2, Bubvu, Bsbdvt, Btbvuts, 
#I    (ab)^5;
#I  Subgroup Name: H;
#I  Subgroup Generators: a;
#I  Wo:1000000; Max:800; Mess:500; Ti:-1; Ho:-1; Loop:0;
#I  As:0; Path:0; Row:1; Mend:1; No:28; Look:0; Com:10;
#I  C:0; R:0; Fi:13; PMod:3; PSiz:256; DMod:4; DSiz:1000;
#I    #--------------------------------
#I  SG: a=1 r=1 h=1 n=2; l=1 c=+0.00; m=1 t=1
#I  RD: a=321 r=68 h=1 n=412; l=5 c=+0.00; m=327 t=411
#I  CC: a=435 r=162 h=1 n=719; l=9 c=+0.01; d=0
#I  CL: a=428 r=227 h=1 n=801; l=13 c=+0.00; m=473 t=800
#I  DD: a=428 r=227 h=1 n=801; l=14 c=+0.00; d=534
#I  DD: a=428 r=227 h=1 n=801; l=14 c=+0.01; d=32
#I  CO: a=428 r=192 h=243 n=429; l=15 c=+0.00; m=473 t=800
#I  INDEX = 480 (a=480 r=210 h=484 n=484; l=18 c=0.02; m=480 t=855)
#I  ***
#I  CO: a=480 r=210 h=481 n=481; c=+0.00
#I   coset ||      a      A      b      B      s      t      u      v      d
#I  -------+---------------------------------------------------------------
#I       1 ||      1      1      7      6      2      3      4      5      1
#I       2 ||      4      4     22     36      1      8     10     11      2
... 476 lines omitted here ...
#I     479 ||    479    479    384    383    475    468    470    471    479
#I     480 ||    480    480    421    420    470    469    475    476    480
[ [ 1, 8, 13, 6, 7, 4, 5, 2, 34, 35, 32, 33, 3, 48, 49, 46, 47, 57, 59, 28, 
      21, 25, 62, 64, 22, 26, 66, 20, 67, 69, 74, 11, 12, 9, 10, 89, 65, 87, 
... 30 lines omitted here ...
      477, 438, 478, 446, 475, 479, 471, 473, 476, 469 ], 
  [ 1, 8, 13, 6, 7, 4, 5, 2, 34, 35, 32, 33, 3, 48, 49, 46, 47, 57, 59, 28, 
      21, 25, 62, 64, 22, 26, 66, 20, 67, 69, 74, 11, 12, 9, 10, 89, 65, 87, 
... 30 lines omitted here ...
      477, 438, 478, 446, 475, 479, 471, 473, 476, 469 ], 
... 363 lines omitted here ...
  [ 1, 2, 3, 4, 5, 6, 7, 8, 9, 10, 11, 12, 13, 14, 15, 16, 17, 18, 19, 20, 
      21, 22, 23, 24, 25, 26, 27, 28, 29, 30, 31, 32, 33, 34, 35, 36, 37, 38, 
... 30 lines omitted here ...
      472, 473, 474, 475, 476, 477, 478, 479, 480 ] ]
\endexample

%%%%%%%%%%%%%%%%%%%%%%%%%%%%%%%%%%%%%%%%%%%%%%%%%%%%%%%%%%%%%%%%%%%%%%
\Section{Example of Using ACE Interactively (Using ACEStart)}

Now we illustrate a simple interactive process, with an enumeration of
an index 12 subgroup (isomorphic to $C_5$) within $A_5$. Observe  that
we  have  relied  on  the  default  level  of  messaging  from  {\ACE}
(`messages' = 0) which gives a result line (the  ```\#I  INDEX''' line
here) only, without parameter information. The result line is  visible
in the `Info'-ed component of the output  below  because  we  set  the
`InfoLevel' of `InfoACE' to a value of at least 2 (in fact we  set  it
to 3; doing ```SetInfoACELevel(2);''' would  make  *only*  the  result
line visible). We have however used the option `echo', so that we  can
see how the interface handled the arguments and options. On-line  try:
`SetInfoACELevel(3); ACEExample("A5-C5", ACEStart);' (this  is  nearly
equivalent to the sequence following, but the variables `F', `a', `b',
`G' are not accessible, being ``local'' to `ACEExample').

\beginexample
gap> SetInfoACELevel(3); # So we can see output from ACE binary
gap> F := FreeGroup("a","b");; a := F.1;;  b := F.2;;
gap> G := F / [a^2, b^3, (a*b)^5 ];
<fp group on the generators [ a, b ]>
gap> ACEStart(FreeGeneratorsOfFpGroup(G), RelatorsOfFpGroup(G), [a*b]
>          # Options
>          : echo, # Echo handled by GAP (not ACE)
>            enum := "A_5",  # Give the group G a meaningful name
>            subg := "C_5"); # Give the subgroup a meaningful name
ACEStart called with the following arguments:
 Group generators : [ a, b ]
 Group relators : [ a^2, b^3, a*b*a*b*a*b*a*b*a*b ]
 Subgroup generators : [ a*b ]
#I  ACE 3.000        Wed Oct 18 14:48:43 2000
#I  =========================================
#I  Host information:
#I    name = boronia
ACEStart called with the following options:
 echo := true (not passed to ACE)
 enum := A_5
 subg := C_5
#I  ***
#I  INDEX = 12 (a=12 r=16 h=1 n=16; l=3 c=0.00; m=14 t=15)
1
\endexample

The return value on the last line is an ``index'' that identifies  the
interactive process; we use this ``index'' with functions that need to
interact with the interactive {\ACE} process; we now demonstrate  this
with the interactive version of `ACEStats':

\beginexample
gap> ACEStats(1);
rec( index := 12, cputime := 0, cputimeUnits := "10^-2 seconds", 
  activecosets := 12, maxcosets := 14, totcosets := 15 )
gap> # Actually, we didn't need to pass an argument to ACEStats()
gap> # ... we could have relied on the default:
gap> ACEStats();
rec( index := 12, cputime := 0, cputimeUnits := "10^-2 seconds", 
  activecosets := 12, maxcosets := 14, totcosets := 15 )
\endexample

Similarly, we may use `ACECosetTable' with 0 or 1 arguments, which  is
the interactive version of `ACECosetTableFromGensAndRels',  and  which
returns a standard table (with {\GAP} 4.2 that  means  a  `semilenlex'
standard table, but with {\GAP} 4.3, as  below,  a  `lenlex'  standard
table is returned).

\beginexample
gap> ACECosetTable(); # Interactive version of ACECosetTableFromGensAndRels()
#I  CO: a=12 r=13 h=1 n=13; c=+0.00
#I   coset ||      b      B      a
#I  -------+---------------------
#I       1 ||      3      2      2
#I       2 ||      1      3      1
#I       3 ||      2      1      4
#I       4 ||      8      5      3
#I       5 ||      4      8      6
#I       6 ||      9      7      5
#I       7 ||      6      9      8
#I       8 ||      5      4      7
#I       9 ||      7      6     10
#I      10 ||     12     11      9
#I      11 ||     10     12     12
#I      12 ||     11     10     11
[ [ 2, 1, 4, 3, 7, 8, 5, 6, 10, 9, 12, 11 ], 
  [ 2, 1, 4, 3, 7, 8, 5, 6, 10, 9, 12, 11 ], 
  [ 3, 1, 2, 5, 6, 4, 8, 9, 7, 11, 12, 10 ], 
  [ 2, 3, 1, 6, 4, 5, 9, 7, 8, 12, 10, 11 ] ]
gap> # To terminate the interactive process we do:
gap> ACEQuit(1); # Again, we could have omitted the 1
gap> # If we had more than one interactive process we could have
gap> # terminated them all in one go with:
gap> ACEQuitAll();
\endexample

%%%%%%%%%%%%%%%%%%%%%%%%%%%%%%%%%%%%%%%%%%%%%%%%%%%%%%%%%%%%%%%%%%%%%%
\Section{Fun with ACEExample}

First let's see the `ACEExample' index  (obtained  with  no  argument,
with  `"index"'  as  argument,  or  with  a  non-existent  example  as
argument):

\beginexample
gap> ACEExample();
#I                             ACEExample Index
#I                             ----------------
#I  This index is displayed when calling ACEExample with no arguments, or
#I  with the argument: "index", or with a non-existent example name.
#I  
#I  The following ACE examples are available (in each case, for a subgroup
#I  H of a group G, the cosets of H in G are enumerated):
#I  
#I    Example          G                      H              strategy
#I    -------          -                      -              --------
#I    "A5"             A_5                    Id             default
#I    "A5-C5"          A_5                    C_5            default
#I    "C5-fel0"        C_5                    Id             felsch := 0
#I    "F27-purec"      F(2,7) = C_29          Id             purec
#I    "F27-fel0"       F(2,7) = C_29          Id             felsch := 0
#I    "F27-fel1"       F(2,7) = C_29          Id             felsch := 1
#I    "M12-hlt"        M_12 (Matthieu group)  Id             hlt
#I    "M12-fel1"       M_12 (Matthieu group)  Id             felsch := 1
#I    "SL219-hard"     SL(2,19)               ||G : H|| = 180  hard
#I    "perf602p5"      PerfectGroup(60*2^5,2) ||G : H|| = 480  default
#I  * "2p17-fel1"      ||G|| = 2^17             ||G : H|| = 1    felsch := 1
#I  * "2p18-fel1"      ||G|| = 2^18             ||G : H|| = 2    felsch := 1
#I  * "big-fel1"       ||G|| = 2^18.3           ||G : H|| = 6    felsch := 1
#I  * "big-hard"       ||G|| = 2^18.3           ||G : H|| = 6    hard
#I    "2p17-id-fel1"   ||G|| = 2^17             Id             felsch := 1
#I    "2p17-2p14-fel1" ||G|| = 2^17             ||G : H|| = 2^14 felsch := 1
#I    "2p17-2p3-fel1"  ||G|| = 2^17             ||G : H|| = 2^3  felsch := 1
#I    "2p17-fel1a"     ||G|| = 2^17             ||G : H|| = 1    felsch := 1
#I  
#I  Notes
#I  -----
#I  1. The example (first) argument of ACEExample() is a string; each
#I     example above is in double quotes to remind you to include them.
#I  2. By default, ACEExample applies ACECosetTableFromGensAndRels to the
#I     chosen example. You may alter the ACE function used, by calling
#I     ACEExample with a 2nd argument; choose from: ACEStats, or ACEStart,
#I     e.g. `ACEExample("A5", ACEStats);'
#I  3. You may call ACEExample with additional ACE options (entered after a
#I     colon in the usual way for options), e.g. `ACEExample("A5" : hlt);' 
#I  4. Try the *-ed examples to explore how to modify options when an
#I     enumeration fails (just follow the instructions you get within the
#I     break-loop, or see Notes 2. and 3.).
#I  5. Try `SetInfoACELevel(3);' before calling ACEExample, to see the
#I     effect of setting the "mess" (= "messages") option.
#I  6. To suppress a long output, use a double semicolon (`;;') after the
#I     ACEExample command. (However, this does not suppress Info-ed output.)
#I  7. Also, try `SetInfoACELevel(2);' or `SetInfoACELevel(4);' before 
#I     calling ACEExample.
gap> 
\endexample

Notice  that the example we first met in Section~"Using  ACE  Directly
to Generate a Coset Table", the Fibonacci group F(2,7),  is  available
via examples `"F27-purec"',  `"F27-fel0"',  and  `"F27-fel1"',  except
each of these enumerate the cosets of its trivial subgroup  (of  index
29). Let's experiment with the first of these F(2,7)  examples;  since
this  example  uses  the  `messages'  option,  we  ought  to  set  the
`InfoLevel' of `InfoACE' to  3,  first,  but  since  the  coset  table
(default output) is  quite  long,  we'll  pass  `ACEStats'  as  second
argument.

Before exhibiting the example we list a few observations  that  should
be made. Observe that  the  first  group  of  `Info'  lines  list  the
commands that are executed; these lines are followed by the result  of
the `echo' option (see~"option echo"); which in turn are  followed  by
`Info' messages from {\ACE} courtesy of  the  non-zero  value  of  the
`messages'  option  (and  we  see  these  because  we  first  set  the
`InfoLevel' of `InfoACE'  to  3);  and  finally,  we  get  the  output
(record) of the `ACEStats' command.

Observe also that {\ACE} uses the same generators as  are input;  this
will always occur if you stick to single lowercase  letters  for  your
generator names. Note, also that capitalisation is used by {\ACE} as a
short-hand for inverses, e.g.~`C = c^-1' (see `Group Relators' in  the
{\ACE} ``Run Parameters'' block).

\beginexample
gap> SetInfoACELevel(3);
gap> ACEExample("F27-purec", ACEStats);
#I  # ACEExample "F27-purec" : enumeration of cosets of H in G,
#I  # where G = F(2,7) = C_29, H = Id, using purec strategy.
#I  #
#I  # F, G, a, b, c, d, e, x, y are local to ACEExample
#I  # We define F(2,7) on 7 generators
#I  F := FreeGroup("a","b","c","d","e", "x", "y"); 
#I       a := F.1;  b := F.2;  c := F.3;  d := F.4; 
#I       e := F.5;  x := F.6;  y := F.7;
#I  G := F / [a*b*c^-1, b*c*d^-1, c*d*e^-1, d*e*x^-1, 
#I            e*x*y^-1, x*y*a^-1, y*a*b^-1];
#I  ACEStats(
#I      FreeGeneratorsOfFpGroup(G), 
#I      RelatorsOfFpGroup(G), 
#I      [] # Generators of identity subgroup (empty list)
#I      # Options that don't affect the enumeration
#I      : echo, enum := "F(2,7), aka C_29", subg := "Id", 
#I      # Other options
#I      wo := "2M", mess := 25000, purec);
ACEStats called with the following arguments:
 Group generators : [ a, b, c, d, e, x, y ]
 Group relators : [ a*b*c^-1, b*c*d^-1, c*d*e^-1, d*e*x^-1, e*x*y^-1, 
  x*y*a^-1, y*a*b^-1 ]
 Subgroup generators : [  ]
ACEStats called with the following options:
 echo := true (not passed to ACE)
 enum := F(2,7), aka C_29
 subg := Id
 wo := 2M
 mess := 25000
 purec (no value, passed to ACE via option: pure c)
#I  ACE 3.000        Wed Oct 18 17:55:31 2000
#I  =========================================
#I  Host information:
#I    name = boronia
#I    #-- ACE 3.000: Run Parameters ---
#I  Group Name: F(2,7), aka C_29;
#I  Group Generators: abcdexy;
#I  Group Relators: abC, bcD, cdE, deX, exY, xyA, yaB;
#I  Subgroup Name: Id;
#I  Subgroup Generators: ;
#I  Wo:2M; Max:142855; Mess:25000; Ti:-1; Ho:-1; Loop:0;
#I  As:0; Path:0; Row:0; Mend:0; No:0; Look:0; Com:100;
#I  C:1000; R:0; Fi:1; PMod:0; PSiz:256; DMod:4; DSiz:1000;
#I    #--------------------------------
#I  DD: a=5290 r=1 h=1050 n=5291; l=8 c=+0.03; d=2
#I  CD: a=10410 r=1 h=2149 n=10411; l=13 c=+0.02; m=10410 t=10410
#I  DD: a=15428 r=1 h=3267 n=15429; l=18 c=+0.03; d=0
#I  DD: a=20430 r=1 h=4386 n=20431; l=23 c=+0.03; d=1
#I  DD: a=25397 r=1 h=5519 n=25399; l=28 c=+0.03; d=1
#I  CD: a=30313 r=1 h=6648 n=30316; l=33 c=+0.02; m=30313 t=30315
#I  DS: a=32517 r=1 h=7326 n=33240; l=36 c=+0.02; s=2000 d=997 c=4
#I  DS: a=31872 r=1 h=7326 n=33240; l=36 c=+0.01; s=4000 d=1948 c=53
#I  DS: a=29077 r=1 h=7326 n=33240; l=36 c=+0.01; s=8000 d=3460 c=541
#I  DS: a=23433 r=1 h=7326 n=33240; l=36 c=+0.02; s=16000 d=5940 c=2061
#I  DS: a=4163 r=1 h=7326 n=33240; l=36 c=+0.08; s=32000 d=447 c=15554
#I  INDEX = 29 (a=29 r=1 h=33240 n=33240; l=37 c=0.40; m=33237 t=33239)
#I  ***
rec( index := 29, cputime := 40, cputimeUnits := "10^-2 seconds", 
  activecosets := 29, maxcosets := 33237, totcosets := 33239 )
\endexample

Now let's see that we  can  add  some  new  options,  even  ones  that
over-ride the example's options; but first we'll reduce the  output  a
bit by setting the `InfoLevel' of `InfoACE' to 2 and since we are  not
going to observe any progress messages from {\ACE} with that `InfoACE'
level we'll set `messages := 0'; also we'll use the  default  function
`ACECosetTableFromGensAndRels' and so it's like  our  first  encounter
with F(2,7) we'll add the subgroup generator `c'  via  `sg  :=  ["c"]'
(see "option sg"). Observe that `"c"' is a string not a  {\GAP}  group
generator; to convert a list of {\GAP}  words  <sgens>  in  generators
<fgens>, suitable for  an  assignment  of  the  `sg'  option  use  the
construction: `ToACEWords(<fgens>, <sgens>)' (see~"ToACEWords").  Note
again that if  only  single  lowercase  letter  strings  are  used  to
identify the {\GAP} group generators, the same  strings  are  used  to
identify those generators in {\ACE}. (It's actually fortunate that  we
could pass the value of `sg' as a string here, since the generators of
each `ACEExample'  example  are  *local*  variables  and  so  are  not
accessible, though  we  could  call  `ACEExample'  with  2nd  argument
`ACEStart' and use `ACEGroupGenerators' to  get  at  them.)  For  good
measure, we also change the string identifying the subgroup (since  it
will no longer be the trivial group), via the `subgroup'  option  (see
"option subgroup").

In considering the example following, observe that in the `Info' block
all the original example options are listed along with our new options
`sg := [ "c" ], messages := 0' after the tag  ```\#  User  Options'''.
Following the `Info' block there is a block  due  to  `echo';  in  its
listing of the options first up there is `aceexampleoptions'  alerting
us that we passed some `ACEExample' options; observe also that in this
block `subg := Id' and `mess := 25000' disappear (they are over-ridden
by `subgroup := "\< c >", messages := 0', but the quotes for the value
of `subgroup' are not visible); note that we don't  have  to  use  the
same abbreviations for options to over-ride them.  Also  observe  that
our new options are *last*.

\beginexample
gap> SetInfoACELevel(2);
gap> ACEExample("F27-purec" : sg := ["c"], subgroup := "< c >",
>                             messages := 0);
#I  # ACEExample "F27-purec" : enumeration of cosets of H in G,
#I  # where G = F(2,7) = C_29, H = Id, using purec strategy.
#I  #
#I  # F, G, a, b, c, d, e, x, y are local to ACEExample
#I  # We define F(2,7) on 7 generators
#I  F := FreeGroup("a","b","c","d","e", "x", "y"); 
#I       a := F.1;  b := F.2;  c := F.3;  d := F.4; 
#I       e := F.5;  x := F.6;  y := F.7;
#I  G := F / [a*b*c^-1, b*c*d^-1, c*d*e^-1, d*e*x^-1, 
#I            e*x*y^-1, x*y*a^-1, y*a*b^-1];
#I  ACECosetTableFromGensAndRels(
#I      FreeGeneratorsOfFpGroup(G), 
#I      RelatorsOfFpGroup(G), 
#I      [] # Generators of identity subgroup (empty list)
#I      # Options that don't affect the enumeration
#I      : echo, enum := "F(2,7), aka C_29", subg := "Id", 
#I      # Other options
#I      wo := "2M", mess := 25000, purec, 
#I      # User Options
#I        sg := [ "c" ],
#I        subgroup := "< c >",
#I        messages := 0);
ACECosetTableFromGensAndRels called with the following arguments:
 Group generators : [ a, b, c, d, e, x, y ]
 Group relators : [ a*b*c^-1, b*c*d^-1, c*d*e^-1, d*e*x^-1, e*x*y^-1, 
  x*y*a^-1, y*a*b^-1 ]
 Subgroup generators : [  ]
ACECosetTableFromGensAndRels called with the following options:
 aceexampleoptions := true (inserted by ACEExample, not passed to ACE)
 echo := true (not passed to ACE)
 enum := F(2,7), aka C_29
 wo := 2M
 purec (no value, passed to ACE via option: pure c)
 sg := [ "c" ] (brackets are not passed to ACE)
 subgroup := < c >
 messages := 0
#I  INDEX = 1 (a=1 r=2 h=2 n=2; l=4 c=0.00; m=332 t=332)
[ [ 1 ], [ 1 ], [ 1 ], [ 1 ], [ 1 ], [ 1 ], [ 1 ], [ 1 ], [ 1 ], [ 1 ], 
  [ 1 ], [ 1 ], [ 1 ], [ 1 ] ]
\endexample

\atindex{break-loop}{@\noexpand`break'-loop}
Now following on from our last example we shall  demonstrate  how  one
can recover from a `break'-loop (see Section~"Using  ACE  Directly  to
Generate a Coset Table"). To force the `break'-loop we pass `max := 2'
(see~"option max"), while using the default {\ACE} interface  function
`ACECosetTableFromGensAndRels' of `ACEExample'; in  this  way,  {\ACE}
will not be able to complete  the  enumeration,  and  hence  enters  a
`break'-loop when it tries to provide a complete  coset  table.  While
we're at it we'll pass the `hlt' (see~"option  hlt")  strategy  option
(which will over-ride `purec'). (The `InfoACE' level is still 2.) Note
that there are some ``user-input'' comments  inserted  at  the  `brk>'
prompt.

\beginexample
gap> ACEExample("F27-purec" : sg := ["c"], subgroup := "< c >",
>                             max := 2, hlt);
#I  # ACEExample "F27-purec" : enumeration of cosets of H in G,
#I  # where G = F(2,7) = C_29, H = Id, using purec strategy.
#I  #
#I  # F, G, a, b, c, d, e, x, y are local to ACEExample
#I  # We define F(2,7) on 7 generators
#I  F := FreeGroup("a","b","c","d","e", "x", "y"); 
#I       a := F.1;  b := F.2;  c := F.3;  d := F.4; 
#I       e := F.5;  x := F.6;  y := F.7;
#I  G := F / [a*b*c^-1, b*c*d^-1, c*d*e^-1, d*e*x^-1, 
#I            e*x*y^-1, x*y*a^-1, y*a*b^-1];
#I  ACECosetTableFromGensAndRels(
#I      FreeGeneratorsOfFpGroup(G), 
#I      RelatorsOfFpGroup(G), 
#I      [] # Generators of identity subgroup (empty list)
#I      # Options that don't affect the enumeration
#I      : echo, enum := "F(2,7), aka C_29", subg := "Id", 
#I      # Other options
#I      wo := "2M", mess := 25000, purec, 
#I      # User Options
#I        sg := [ "c" ],
#I        subgroup := "< c >",
#I        max := 2,
#I        hlt := true);
ACECosetTableFromGensAndRels called with the following arguments:
 Group generators : [ a, b, c, d, e, x, y ]
 Group relators : [ a*b*c^-1, b*c*d^-1, c*d*e^-1, d*e*x^-1, e*x*y^-1, 
  x*y*a^-1, y*a*b^-1 ]
 Subgroup generators : [  ]
ACECosetTableFromGensAndRels called with the following options:
 aceexampleoptions := true (inserted by ACEExample, not passed to ACE)
 echo := true (not passed to ACE)
 enum := F(2,7), aka C_29
 wo := 2M
 mess := 25000
 purec (no value, passed to ACE via option: pure c)
 sg := [ "c" ] (brackets are not passed to ACE)
 subgroup := < c >
 max := 2
 hlt (no value)
#I  OVERFLOW (a=2 r=1 h=1 n=3; l=4 c=0.00; m=2 t=2)
Error, no coset table ...
 the `ACE' coset enumeration failed with the result:
 OVERFLOW (a=2 r=1 h=1 n=3; l=4 c=0.00; m=2 t=2)
Entering break read-eval-print loop ...
 try relaxing any restrictive options.
 Type: 'DisplayACEOptions();' to see current ACE options;
 type: 'SetACEOptions(:<option1> := <value1>, ...);'
 to set <option1> to <value1> etc.
 (i.e. pass options after the ':' in the usual way)
 ... and then, type: 'return;' to continue.
 Otherwise, type: 'quit;' to quit the enumeration.
brk> # Let's give ACE enough coset numbers to work with ...                    
brk> # and while we're at it see the effect of 'echo := 2' :                   
brk> SetACEOptions(: max := 0, echo := 2);                                     
brk> # Let's check what the options are now:                                   
brk> DisplayACEOptions();                                                      
rec(
  enum := "F(2,7), aka C_29",
  wo := "2M",
  mess := 25000,
  purec := true,
  sg := [ "c" ],
  subgroup := "< c >",
  hlt := true,
  max := 0,
  echo := 2 )
brk> # That's ok ... so now we 'return;' to escape the break-loop              
brk> return;                                                                   
ACECosetTableFromGensAndRels called with the following arguments:
 Group generators : [ a, b, c, d, e, x, y ]
 Group relators : [ a*b*c^-1, b*c*d^-1, c*d*e^-1, d*e*x^-1, e*x*y^-1, 
  x*y*a^-1, y*a*b^-1 ]
 Subgroup generators : [  ]
ACECosetTableFromGensAndRels called with the following options:
 aceexampleoptions := true (inserted by ACEExample, not passed to ACE)
 enum := F(2,7), aka C_29
 wo := 2M
 mess := 25000
 purec (no value, passed to ACE via option: pure c)
 sg := [ "c" ] (brackets are not passed to ACE)
 subgroup := < c >
 hlt (no value)
 max := 0
 echo := 2 (not passed to ACE)
Other options set via ACE defaults:
 asis := 0
 compaction := 10
 ct := 0
 dmode := 0
 dsize := 1000
 fill := 1
 hole := -1
 lookahead := 1
 loop := 0
 mendelsohn := 0
 no := 0
 path := 0
 pmode := 0
 psize := 256
 row := 1
 rt := 1000
 time := -1
#I  INDEX = 1 (a=1 r=2 h=2 n=2; l=3 c=0.01; m=2049 t=3127)
[ [ 1 ], [ 1 ], [ 1 ], [ 1 ], [ 1 ], [ 1 ], [ 1 ], [ 1 ], [ 1 ], [ 1 ], 
  [ 1 ], [ 1 ], [ 1 ], [ 1 ] ]
\endexample

Observe that  `purec'  did  *not*  disappear  from  the  option  list;
nevertheless, it *is* over-ridden by the `hlt' option (at  the  {\ACE}
level). Observe the  ```Other options set via ACE defaults''' list  of
options  that  has  resulted  from  having  the  `echo  :=  2'  option
(see~"option echo"). Observe, also, that  `hlt'  is  nowhere  near  as
good, here, as  `purec'  (refer  to  Section~"Using  ACE  Directly  to
Generate  a  Coset  Table"):  whereas  `purec'  completed   the   same
enumeration with a total number of coset numbers  of  332,  the  `hlt'
strategy required 3127.

Of course, running `ACEExample' with  `ACEStart'  as  second  argument
opens up far more flexibility. Try it! Have fun!  Play  with  as  many
options as you can. Also, note that the `*'-ed examples of  the  index
fail  to  give  a  coset  table;  so  these  give  you  non-artificial
`break'-loop examples for you to try.

%%%%%%%%%%%%%%%%%%%%%%%%%%%%%%%%%%%%%%%%%%%%%%%%%%%%%%%%%%%%%%%%%%%%%%
\Section{Using ACEReadResearchExample}

Here I will describe how  `ACEReadResearchExample'  defines  functions
such  as  `PGRelFind',  and  how  these   functions   were   used   in
\cite{CHHR00} to show that the group $L_3(5)$ has deficiency 0.

%%%%%%%%%%%%%%%%%%%%%%%%%%%%%%%%%%%%%%%%%%%%%%%%%%%%%%%%%%%%%%%%%%%%%%%%%
%%
%E
