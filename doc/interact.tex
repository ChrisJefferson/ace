%%%%%%%%%%%%%%%%%%%%%%%%%%%%%%%%%%%%%%%%%%%%%%%%%%%%%%%%%%%%%%%%%%%%%%%%%
%%
%W  interact.tex   ACE documentation - interactive fns   Alexander Hulpke
%W                                                      Joachim Neub"user
%W                                                            Greg Gamble
%%
%H  $Id$
%%
%Y  Copyright (C) 2000  Centre for Discrete Mathematics and Computing
%Y                      Department of Computer Science & Electrical Eng.
%Y                      University of Queensland, Australia.
%%

%%%%%%%%%%%%%%%%%%%%%%%%%%%%%%%%%%%%%%%%%%%%%%%%%%%%%%%%%%%%%%%%%%%%%%%%%
\Chapter{Functions for Using ACE Interactively}

The user will probably benefit most from interactive use of {\ACE}  by
setting   the   `InfoLevel'   of   `InfoACE'    to    at    least    3
(see~"SetInfoACELevel"),  particularly  if  she  uses  the  `messages'
option with a non-zero value.

All functions that manipulate an interactive process  (that  has  been
initiated by `ACEStart', with three arguments), have a form where  the
first  argument  is  the  integer  <i>  returned  by  the   initiating
`ACEStart' command, and a second form with one fewer arguments  (where
the integer <i> is  discovered  by  a  default  mechanism,  namely  by
determining the least integer <i>  for  which  there  is  a  currently
active interactive {\ACE} process). In each case, it is an  error,  if
<i> is not the index of an active interactive process, or there are no
current active interactive processes.

*Notes*: 

The global method of passing options (via `PushOptions'),  should  not
be  used  with  any  of  the  interactive  functions.  In  fact,   the
`OptionsStack' should be empty at the  time  any  of  the  interactive
functions is called.

On `quit'ting {\GAP}, `ACEQuitAll();' is  executed,  which  terminates
all active interactive {\ACE} processes. If {\GAP} is  killed  without
`quit'ting, before all interactive {\ACE}  processes  are  terminated,
*zombie* processes (still living  *child*  processes  whose  *parents*
have died), will result. Since zombie processes do consume  resources,
in such an event, the responsible computer user should  seek  out  and
kill the still living `ace' children (e.g.~by piping the output  of  a
`ps' with appropriate options, usually `aux' or `ef', to  `grep  ace',
to find the process ids, and then using `kill'; try `man ps' and  `man
kill' if these hints are unhelpful).

%%%%%%%%%%%%%%%%%%%%%%%%%%%%%%%%%%%%%%%%%%%%%%%%%%%%%%%%%%%%%%%%%%%%%%%
\Section{Starting and Stopping Interactive ACE Processes}

\beginitems

\>ACEStart( 0 ) F
\>ACEStart( <fgens>, <rels>, <sgens> [:<options>] ) F
\>ACEStart( <i> [:<options>] ) F
\>ACEStart( [:<options>] ) F

The first form of `ACEStart' (on the  one  argument:  0)  is  used  to
initiate an interactive {\ACE} process, but does  nothing  more.  This
form is mainly for gurus who are familiar with the  {\ACE}  standalone
and who wish, at least initially, to communicate with {\ACE} using the
primitive  read/write  tools  of  Section~"Primitive  ACE   Read/Write
Functions". In this case, after the group  generators,  relators,  and
subgroup generators have been set in the {\ACE}  process,  invocations
of    any    of    `ACEGroupGenerators'    (see~"ACEGroupGenerators"),
`ACERelators'       (see~"ACERelators"),       `ACESubgroupGenerators'
(see~"ACESubgroupGenerators"),           or            `ACEParameters'
(see~"ACEParameters") will establish the corresponding {\GAP}  values.
Be warned, though, that unless one of the modes `ACEStart' (without  a
zero   argument),   `ACERedo'   (see~"ACERedo")    or    `ACEContinue'
(see~"ACEContinue"), or their equivalent  for  the  standalone  {\ACE}
(`start;', `redo;', or `continue;'), has been invoked since  the  last
change of any parameter options (see Section~"ACE Parameter Options"),
some  of  the  values  reported  by  `ACEParameters'   may   well   be
*incorrect*.

The second form of `ACEStart' (on three arguments) is used to start an
interactive process; here <fgens> is a list of free generators, <rels>
a list of words in these generators giving  relators  for  a  finitely
presented group, and <sgens> the list of  subgroup  generators,  again
expressed as words in the free generators. All these are given in  the
standard {\GAP} format (See Chapter~"ref:Finitely Presented Groups" of
the {\GAP} Reference Manual).

When `ACEStart' is called with one positive integer  argument  <i>  it
starts a new enumeration on the <i>th running process, i.e.~it  scrubs
a previously generated table and starts from  scratch  with  the  same
parameters (i.e.~the same arguments and options); except that  if  new
options are included these will modify  those  given  previously.  The
only reason for doing such a thing, without new options, is to perhaps
compare timings of runs (a second run is quicker  because  memory  has
already been allocated).  If  you  are  interested  in  this  sort  of
information, however, you may be better off dealing directly with  the
standalone.

When `ACEStart' is  called  with  no  arguments  it  finds  the  least
positive integer <i> for which an interactive process is  running  and
applies `ACEStart(<i>)'. (Most users will  only  run  one  interactive
process at a time. Hence, `ACEStart()' will be a useful  shortcut  for
`ACEStart(1)'.)

If you intend to use options these are  listed  behind  a  colon;  any
selection  of  the  options   available   for   the   interface   (see
Chapters~"Options for ACE" and~"Strategy  Options  for  ACE")  can  be
given,  separated  by  commas  like  record  components.   Note   that
strategies are simply  special  options  that  set  a  number  of  the
options, detailed in Chapter~"Options  for  ACE",  all  at  once.  The
reader is strongly encouraged to read  the  introductory  sections  of
Chapter~"Options for ACE". The global mechanism (via `PushOptions') of
passing options is *not* recommended  for  use  with  the  interactive
{\ACE} interface functions; please ensure the `OptionsStack' is  empty
before calling an interactive {\ACE} interface function.

The return value (for all four forms  of  `ACEStart')  is  an  integer
(numbering from  1)  which  represents  the  running  process.  It  is
possible to have more than one interactive process  running  at  once.
The integer returned may be used to index which of these processes  an
interactive {\ACE} interface function should be applied to.

\>ACEQuit( <i> ) F
\>ACEQuit() F

terminate an interactive {\ACE} process,  where  <i>  is  the  integer
returned by `ACEStart' when the process was  started.  If  the  second
form is used (i.e.~without arguments) then the interactive process  of
least index that is still running is terminated.

As a convenience, to terminate all active interactive {\ACE} processes
at once, we provide:

\>ACEQuitAll() F

\enditems

%%%%%%%%%%%%%%%%%%%%%%%%%%%%%%%%%%%%%%%%%%%%%%%%%%%%%%%%%%%%%%%%%%%%%%
\Section{Primitive ACE Read/Write Functions}

For those familiar with the  workings  of  the  {\ACE}  standalone  we
provide primitive read/write tools to  communicate  directly  with  an
interactive {\ACE} process,  started  via  `ACEStart'  (possibly  with
argument 0, but this is not essential). For the most part, it is up to
the user to translate the output  strings  from  {\ACE}  into  a  form
useful in {\GAP}. However, after the group generators,  relators,  and
subgroup  generators  have  been  set  in  the  {\ACE}  process,   via
`ACEWrite',    invocations    of    any    of     `ACEGroupGenerators'
(see~"ACEGroupGenerators"),     `ACERelators'     (see~"ACERelators"),
`ACESubgroupGenerators'       (see~"ACESubgroupGenerators"),        or
`ACEParameters' (see~"ACEParameters") will establish the corresponding
{\GAP} values.  Be  warned  though,  that  unless  one  of  the  modes
`ACEStart'  (without  a  zero  argument;  see~"ACEStart"),   `ACERedo'
(see~"ACERedo")  or  `ACEContinue'   (see~"ACEContinue"),   or   their
equivalent  for  the  standalone   {\ACE}   (`start;',   `redo;',   or
`continue;'), has been invoked since the last change of any  parameter
options (see Section~"ACE Parameter  Options"),  some  of  the  values
reported by `ACEParameters' may well be *incorrect*.

\beginitems

\>ACEWrite( <i>, <string> ) F
\>ACEWrite( <string> ) F

write <string> to the <i>th or  default  interactive  {\ACE}  process;
<string> must be in exactly the form the  {\ACE}  standalone  expects.
The command is  echoed  via  `Info'  at  `InfoACE'  level  4  (with  a
\lq{}`ToACE> '' prompt); i.e.~do `SetInfoACELevel(4);' to see what  is
transmitted  to  the  {\ACE}  binary.  `ACEWrite'  returns  `true'  if
successful in writing to the stream of the interactive {\ACE} process,
and `fail' otherwise.

*Note:*
If `ACEWrite' returns `fail' (which means that the {\ACE} process  has
died), you may like to try resurrecting the interactive {\ACE} process
via `ACEResurrectProcess' (see~"ACEResurrectProcess").

\>ACERead( <i> ) F
\>ACERead() F

read a complete line of {\ACE}  output,  from  the  <i>th  or  default
interactive {\ACE} process, if there is output to be read and  returns
`fail' otherwise. When successful, the line is returned  as  a  string
complete with trailing newline  character.  Please  note  that  it  is
possible to be \lq{}too quick' (i.e.~the return can be  `fail'  purely
because the output from {\ACE} is not there  yet),  but  if  `ACERead'
finds any output at all, it waits for a complete line.

\>ACEReadAll( <i> ) F
\>ACEReadAll() F

read and return as many *complete* lines of {\ACE}  output,  from  the
<i>th or default interactive {\ACE} process, as there are to be  read,
*at the time of the call*, as a list  of  strings  with  the  trailing
newlines removed and returns the empty  list  otherwise.  `ACEReadAll'
also writes each line read via `Info' at `InfoACE' level  3.  Whenever
`ACEReadAll' finds only a partial line,  it  waits  for  the  complete
line, thus increasing the probability that it  has  captured  all  the
output to be had from {\ACE}.

\>ACEReadUntil( <i>, <IsMyLine> ) F
\>ACEReadUntil( <IsMyLine> ) F
\>ACEReadUntil( <i>, <IsMyLine>, <Modify> ) F
\>ACEReadUntil( <IsMyLine>, <Modify> ) F

read complete lines of  {\ACE}  output,  from  the  <i>th  or  default
interactive  {\ACE}  process,  \lq{}chomps'  them  (i.e.~removes   any
trailing newline character), emits them to `Info' at  `InfoACE'  level
3, and applies the function  <Modify>  (where  <Modify>  is  just  the
identity map/function for the first two forms) until  a  \lq{}chomped'
line <line>  for  which  `<IsMyLine>(  <Modify>(<line>)  )'  is  true.
`ACEReadUntil' returns the list  of  <Modify>-ed  \lq{}chomped'  lines
read.

*Notes:* 

When provided by the user, <Modify> should be a function that  accepts
a single string argument.

<IsMyLine> should be a function that is able to accept the  output  of
<Modify> (or take a  single  string  argument  when  <Modify>  is  not
provided) and should return a boolean.

If `<IsMyLine>( <Modify>(<line>) )' is never true, `ACEReadUntil' will
wait indefinitely.

\enditems

%%%%%%%%%%%%%%%%%%%%%%%%%%%%%%%%%%%%%%%%%%%%%%%%%%%%%%%%%%%%%%%%%%%%%%
\Section{Interactive ACE Process Utility Functions and Interruption of
an Interactive ACE Process}

\beginitems

\>ACEProcessIndex( <i> ) F
\>ACEProcessIndex() F

With argument <i>, which must be a positive integer, `ACEProcessIndex'
returns <i> if it corresponds to an  active  interactive  process,  or
raises an error. With no  arguments  it  returns  the  default  active
interactive process or returns `fail' and emits a warning  message  to
`Info' at `InfoACE' or `InfoWarning' level 1.

\>ACEProcessIndices() F

returns the list of integer indices of all active  interactive  {\ACE}
processes.

\index{interruption of an interactive ACE process} If  the  user  does
not yet have a `gap>' prompt then usually {\ACE} is still  away  doing
something and an {\ACE} interface function  is  still  waiting  for  a
reply from {\ACE}. Typing a `Ctrl-C' (i.e. holding down the `Ctrl' key
and typing  `c')  will  stop  the  waiting  and  send  {\GAP}  into  a
`break'-loop{\undoquotes \atindex {break-loop} {@`break'-loop}},  from
which one has no option but to `quit;'.  Typing  `Ctrl-C'  in  such  a
circumstance will usually cause the  stream  of  the  the  interactive
{\ACE} process to die; to check this  we  provide  `IsACEProcessAlive'
(see~"IsACEProcessAlive"). If the  stream  of  an  interactive  {\ACE}
process, indexed by <i>, say, has died, it may still  be  possible  to
recover enough of the state before the death of the  stream  from  the
information stored in the `ACEData.io[<i>]' record  (see  Section~"The
ACEData   Record").   For   such   a   purpose,   we   have   provided
`ACEResurrectProcess' (see~"ACEResurrectProcess").

\>IsACEProcessAlive( <i> ) F
\>IsACEProcessAlive() F

return `true' if the stream of  the  <i>th  (or  default)  interactive
{\ACE} process started by  `ACEStart'  is  alive  (i.e.~can  still  be
written to), or `false', otherwise.

\>ACEResurrectProcess( <i> [: options] ) F
\>ACEResurrectProcess( [: options] ) F

re-generate the stream of the <i>th (or  default)  interactive  {\ACE}
process, started by `ACEStart' (see~"ACEStart"), and tries to  recover
as much as possible of the previous state from  saved  values  of  the
process's arguments and parameter options. The possible <options> here
are `use' and `useboth' which are described in detail below.

The arguments of the <i>th interactive {\ACE} process  are  stored  in
`ACEData.io[<i>].args', a  record  with  fields  `fgens',  `rels'  and
`sgens', which are the {\GAP} group generators, relators and  subgroup
generators, respectively (see Section~"The  ACEData  Record").  Option
information is saved in `ACEData.io[<i>].options' when a user uses  an
interactive  {\ACE}  interface   function   with   options   or   uses
`SetACEOptions' (see~"SetACEOptions"). Option information is saved  in
`ACEData.io[<i>].parameters' if `ACEParameters'  (see~"ACEParameters")
is used to extract from  {\ACE}  the  current  values  of  the  {\ACE}
parameter options (this is generally less reliable unless one  of  the
{\ACE} modes (see Section~"General and  Experimentation  ACE  Modes"),
has been run previously).

By   default,   `ACEResurrectProcess'   recovers   parameter    option
information from `ACEData.io[<i>].options' if it  is  bound,  or  from
`ACEData.io[<i>].parameters'   if    is    bound,    otherwise.    The
`ACEData.io[<i>].options'  record,  however,  is  first  filtered  for
parameter and strategy options (see Sections~"ACE  Parameter  Options"
and~"The ACEStrategyOptions list"). To alter this behaviour, the  user
is provided two options:

\beginitems

\quad`use := <useList>'& <useList>  may   contain   one  or  both   of
`"options"'  and  `"parameters"'.  By  default,  `use  =   ["options",
"parameters"]'.

\quad`useboth' & (A boolean option). By default, `useboth = false'.

\enditems

If `useboth = true', `SetACEOptions' (see~"SetACEOptions") is  applied
to   the    <i>-th    interactive    {\ACE}    process    with    each
`ACEData.io[<i>].(<field>)'   for   each   <field>   (`"options"'   or
`"parameters"') that is bound and in <useList>, in the  order  implied
by <useList>. If `useboth = false', `SetACEOptions'  is  applied  with
`ACEData.io[<i>].(<field>)' for only the first <field> that  is  bound
in <useList>.

*Notes:*
Do not use general {\ACE}  options  with  `ACEResurrectProcess';  they
will  only   be   superseded   by   those   options   recovered   from
`ACEData.io[<i>].options'     and/or     `ACEData.io[<i>].parameters'.
Instead, call `SetACEOptions' first (despite a warning that the stream
is dead, the `ACEData.io[<i>].options' will be updated).

\enditems

%%%%%%%%%%%%%%%%%%%%%%%%%%%%%%%%%%%%%%%%%%%%%%%%%%%%%%%%%%%%%%%%%%%%%%
\Section{General and Experimentation ACE Modes}

For our purposes, we define an interactive {\ACE} interface command to
be  an  {\ACE}  mode,  if  it  delivers  an  enumeration  result  (see
Section~"Results messages"). Thus, `ACEStart' (see~"ACEStart"), except
when called with the  argument  0,  and  the  commands  `ACERedo'  and
`ACEContinue' which we describe below are {\ACE} modes; we call  these
\lq{}general' {\ACE} modes. Additionally, there are two other commands
which   deliver   enumeration   results:    `ACEAllEquivPresentations'
(see~"ACEAllEquivPresentations")   and   `ACERandomEquivPresentations'
(see~"ACERandomEquivPresentations");       we        call        these
\lq{}experimentation' {\ACE} modes (they  implement  respectively  the
`aep' and `rep' {\ACE} standalone commands).

*Guru Note:*
The  {\ACE}  standalone  defines  three  modes:  `start',  `redo'  and
`continue',  which  double  as  (\lq{}modal')  command   names   that,
respectively, build an initial coset table, modify an existing  table,
and build on an existing table. Each of  these  commands  delivers  as
their final output, an enumeration result. For the  {\ACE}  interface,
we have generalised the term \lq{}mode' to include  any  command  that
produces enumeration results,  and  ignored  the  semantic  difference
between a \lq{}mode' and a \lq{}modal command'.

After changing any of  {\ACE}'s  parameters,  one  of  three  *general
modes*  is  possible:  one  may  be   able   to   \lq{}continue'   via
`ACEContinue'  (see~"ACEContinue"),  or   \lq{}redo'   via   `ACERedo'
(see~"ACERedo"), or if neither of these is possible one  may  have  to
re-\lq{}start'  the  enumeration  via   `ACEStart'   (see~"ACEStart").
Generally, the appropriate mode is invoked automatically when  options
are changed; so most users should be  able  to  ignore  the  following
three functions.

\beginitems

\>ACEModes( <i> ) F
\>ACEModes() F

for the <i>th (or default) interactive {\ACE} process, return a record
whose fields are the modes `ACEStart',  `ACEContinue'  and  `ACERedo',
and whose values are `true' if the mode is possible  for  the  process
and `false' otherwise.

\>ACEContinue( <i> [:<options>] ) F
\>ACEContinue( [:<options>] ) F

for the <i>th (or  default)  interactive  {\ACE}  process,  apply  any
<options> and then \lq{}continue' the  current  enumeration,  building
upon the existing table. If a previous run stopped without producing a
finite index you can, in principle, change  any  of  the  options  and
continue on. Of course, if you make any changes which  invalidate  the
current table, you won't be allowed to `ACEContinue' and an error will
be raised. However, after `quit'ting the `break'-loop, the interactive
{\ACE} process should normally still be active; after  doing  so,  run
`ACEModes' (see~"ACEModes") to see which of `ACERedo' or `ACEStart' is
possible.

\>ACERedo( <i> [:<options>] ) F
\>ACERedo( [:<options>] ) F

for the <i>th (or  default)  interactive  {\ACE}  process,  apply  any
<options> and then \lq{}redo' the current  enumeration;  any  existing
information in the table is retained, and the enumeration is restarted
from coset 1 (i.e., the subgroup).

*Notes:*

This command is really intended for the case where additional relators
and/or subgroup generators have been introduced.  The  current  table,
which  may  be  incomplete  or  exhibit  a  finite  index,  is   still
\lq{}valid'. However, the  new  data  may  allow  the  enumeration  to
complete, or cause a collapse to  a  smaller  index.  In  some  cases,
`ACERedo' may not be possible and an error will  be  raised;  in  this
case, `quit' the `break'-loop, and try `ACEStart', which will  discard
the current table and re-\lq{}start' the enumeration.

Now we describe the two *experimentation modes*.

\>ACEAllEquivPresentations( <i>, <val> ) F
\>ACEAllEquivPresentations( <val> ) F

for the <i>th (or default) interactive {\ACE} process,  generates  and
tests an enumeration for combinations  of  relator  ordering,  relator
rotations, and relator inversions; <val> is in the integer range 0  to
7.

The argument <val> is considered as a binary number.  Its  three  bits
are treated as flags, and control relator rotations (the  $2^0$  bit),
relator inversions (the $2^1$ bit) and relator  orderings  (the  $2^2$
bit),  respectively;  where  $1$  means  \lq{}active'  and  $0$  means
\lq{}inactive'. (See below for an example).

Before we describe the {\GAP} output of `ACEAllEquivPresentations' let
us spend some time considering what happens before the  {\ACE}  binary
output is parsed.

The `ACEAllEquivPresentations' command first performs  a  \lq{}priming
run' using the options as they stand. In particular,  the  `asis'  and
`messages' options are honoured.

It  then  turns  `asis'  (see~"option   asis")   on   and   `messages'
(see~"option messages") off (i.e.~sets `messages' to 0), and generates
and tests the requested  equivalent  presentations.  The  maximum  and
minimum values attained by `m' (the maximum number  of  coset  numbers
defines at any stage) and `t'  (the  total  number  of  coset  numbers
defined) are tracked, and each time a new \lq{}record' is  found,  the
relators  used  and  the  summary  result  line   are   printed.   See
Appendix~"The Meanings of ACE's output messages" for a  discussion  of
the statistics  `m'  and  `t'.  To  observe  these  messages  set  the
`InfoLevel' of `InfoACE' to 3.

The order in which the  equivalent  presentations  are  generated  and
tested has no particular significance, but note that the  presentation
as given *after* the initial priming run) is the  *last*  presentation
to be generated and tested, so that  the  group's  relators  are  left
\lq{}unchanged' by running the `ACEAllEquivPresentations' command.

As discussed by Cannon, Dimino, Havas  and  Watson  \cite{CDHW73}  and
Havas and Ramsay \cite{HR99b} such equivalent presentations can  yield
large variations in  the  number  of  coset  numbers  required  in  an
enumeration. For this command, we are interested in this variation.

After  the  final  presentation  is  run,   some   additional   status
information messages are printed to the {\ACE} output:

\beginlist
\item{--}  the number of runs which yielded a finite index; 
\item{--}  the total number of runs (excluding the priming run); and 
\item{--}  the range of values observed for `m' and `t'.
\endlist

As an example (drawn from the discussion in \cite{HR99a}) consider the
enumeration   of   the   $448$   coset   numbers   of   the   subgroup
$\langle  a^2,Ab \rangle$ of the group
$$ (8,7 \mid 2,3) 
    = \langle a,b \mid a^8 = b^7 = (ab)^2 = (Ab)^3 = 1 \rangle. $$
There are $4!=24$  relator  orderings  and  $2^4=16$  combinations  of
relator or inverted relator. Exponents are  taken  into  account  when
rotating relators, so the relators given give rise to 1, 1,  2  and  2
rotations respectively, for a total of $1.1.2.2=4$  combinations.  So,
for  <val>${} = 7$   (resp.~$3$),   $24.16.4=1536$   (resp.~$16.4=64$)
equivalent presentations are tested.

Now we describe the output  of  `ACEAllEquivPresentations';  it  is  a
record with fields:

\beginitems

\quad`primingResult' & the  {\ACE}  enumeration  result  message  (see
Section~"Results Messages") of the priming run;

\quad`primingStats' & the enumeration result of the priming run  as  a
{\GAP}  record  with  fields   `index',   `cputime',   `cputimeUnits',
`activecosets', `maxcosets' and `totcosets', exactly as for the record
returned by `ACEStats' (see~"ACEStats");

\quad`equivRuns' & a list of data records, one  for  each  run,  where
each record has fields:

\beginitems

\quad\quad`rels'& the relators in the order used for the run,

\quad\quad`enumResult'& the {\ACE}  enumeration  result  message  (see
Section~"Results Messages") of the run, and

\quad\quad`stats'& the enumeration result as a {\GAP}  record  exactly
like the record returned by `ACEStats' (see~"ACEStats");

\enditems

\quad`summary' & a record with fields:

\beginitems
\quad\quad`successes'& the total  number  of  successful  (i.e.~having
finite enumeration index) runs,

\quad\quad`runs'& the total number  of  equivalent  presentation  runs
executed,

\quad\quad`maxcosetsRange'& the range  of  values  as  a  {\GAP}  list
inside which each `equivRuns[<i>].maxcosets' lies, and

\quad\quad`totcosetsRange'& the range  of  values  as  a  {\GAP}  list
inside which each `equivRuns[<i>].totcosets' lies.

\enditems

\enditems

*Notes:*
There is no way to stop the `ACEAllEquivPresentations' command  before
it has completed, other than killing the task. So do a  reality  check
beforehand on the size of the search  space  and  the  time  for  each
enumeration.  If  you  are  interested   in   finding   a   \lq{}good'
enumeration, it can be very helpful, in terms of running time, to  put
a tight limit on the number of coset numbers via the `max' option. You
may also have to set `compaction'  =  $100$  to  prevent  time-wasting
attempts to recover space via compaction. This maximises throughput by
causing the \lq{}bad' enumerations, which  are  in  the  majority,  to
overflow quickly and abort. If  you  wish  to  explore  a  very  large
search-space, consider firing up many copies of {\ACE},  and  starting
each with a \lq{}random' equivalent presentation.  Alternatively,  you
could use the `ACERandomEquivPresentations' command.


\>ACERandomEquivPresentations( <i>, <val> ) F
\>ACERandomEquivPresentations( <val> ) F
\>ACERandomEquivPresentations( <i>, [<val>] ) F
\>ACERandomEquivPresentations( [<val>] ) F
\>ACERandomEquivPresentations( <i>, [<val>, <Npresentations>] ) F
\>ACERandomEquivPresentations( [<val>, <Npresentations>] ) F

for the <i>th (or default) interactive {\ACE} process,  generates  and
tests up to <Npresentations> (or 8,  in  the  first  4  forms)  random
presentations; <val>, an integer in the range 0  to  7,  acts  as  for
`ACEAllEquivPresentations' and <Npresentations>, when given, should be
a positive integer.

The routine first turns `asis' (see~"option asis") on  and  `messages'
(see~"option messages") off (i.e.~sets  `messages'  to  0),  and  then
generates  and  tests  the  requested  number  of  random   equivalent
presentations. For  each  presentation,  the  relators  used  and  the
summary result line are printed by {\ACE}. To observe these messages set
the `InfoLevel' of `InfoACE' to at least 3. 

`ACERandomEquivPresentations' parses the {\ACE} messages,  translating
them to {\GAP}, and thus returns a list of  records  (similar  to  the
field     `equivRuns'     of     the      returned      record      of
`ACEAllEquivPresentations'). Each record of the returned list  is  the
data derived from a presentation run and has fields:

\beginitems

\quad`rels'& the relators in the order used for the run,

\quad`enumResult'&  the  {\ACE}  enumeration   result   message   (see
Section~"Results Messages") of the run, and

\quad`stats'& the enumeration result as a {\GAP} record  exactly  like
the record returned by `ACEStats' (see~"ACEStats").

\enditems

*Notes:*
The relator inversions and rotations are \lq{}genuinely'  random.  The
relator permuting is a little bit of a kludge, with the  \lq{}quality'
of the permutations tending to improve with successive  presentations.
When  the   `ACERandomEquivPresentations'   command   completes,   the
presentation active is the *last* one generated.

*Guru Notes:*
It might appear that neglecting to restore the  original  presentation
is an error. In fact, it is a useful feature! Suppose that  the  space
of equivalent presentations is too  large  to  exhaustively  test.  As
noted in the entry for `ACEAllEquivPresentations',  we  can  start  up
multiple copies of `ACEAllEquivPresentations' at random points in  the
search-space. Manually generating `random' equivalent presentations to
serve   as   starting-points   is   tedious   and   error-prone.   The
`ACERandomEquivPresentations'  command  provides  a  simple  solution;
simply    run    `ACERandomEquivPresentations(<i>,     7);'     before
`ACEAllEquivPresentations(<i>, 7);'.

\enditems

%%%%%%%%%%%%%%%%%%%%%%%%%%%%%%%%%%%%%%%%%%%%%%%%%%%%%%%%%%%%%%%%%%%%%%%%
\Section{Interactive Query Functions and an Option Setting Function}

\beginitems

\>ACEGroupGenerators( <i> ) F
\>ACEGroupGenerators() F

return  the  {\GAP}  group  generators,  of  the  <i>th  (or  default)
interactive {\ACE} process. If no generators have been saved  for  the
interactive {\ACE} process, possibly because the process  was  started
via   `ACEStart(0);'   (see~"ACEStart"),   the   {\ACE}   process   is
interrogated,  the  equivalent  in  {\GAP}  is  saved  and   returned.
Essentially,  `ACEGroupGenerators(<i>)',   interrogates   {\ACE}   and
establishes `ACEData.io[<i>].args.fgens', if  necessary,  and  returns
`ACEData.io[<i>].args.fgens'.  As  a  side-effect,  if  any   of   the
remaining       fields       of       `ACEData.io[<i>].args'        or
`ACEData.io[<i>].acegens' are unset, they are  also  set.  Note  that,
{\GAP} provides  `GroupWithGenerators'  (see~"ref:GroupWithGenerators"
in the {\GAP} Reference Manual) to establish a free group on  a  given
set of already-defined generators.

\>ACERelators( <i> ) F
\>ACERelators() F

return the {\GAP} relators, of  the  <i>th  (or  default)  interactive
{\ACE} process. If no relators have been  saved  for  the  interactive
{\ACE}  process,  possibly  because  the  process  was   started   via
`ACEStart(0);' (see~"ACEStart"), the {\ACE} process  is  interrogated,
the  equivalent  in  {\GAP}  is  saved  and   returned.   Essentially,
`ACERelators(<i>)',    interrogates     {\ACE}     and     establishes
`ACEData.io[<i>].args.rels',     if     necessary,     and     returns
`ACEData.io[<i>].args.rels'. As a side-effect, if any of the remaining
fields  of  `ACEData.io[<i>].args'  or  `ACEData.io[<i>].acegens'  are
unset, they are also set.

\>ACESubgroupGenerators( <i> ) F
\>ACESubgroupGenerators() F

return the {\GAP} subgroup  generators,  of  the  <i>th  (or  default)
interactive {\ACE} process. If no subgroup generators have been  saved
for the interactive {\ACE} process, possibly because the  process  was
started via `ACEStart(0);' (see~"ACEStart"),  the  {\ACE}  process  is
interrogated,  the  equivalent  in  {\GAP}  is  saved  and   returned.
Essentially,  `ACESubgroupGenerators(<i>)',  interrogates  {\ACE}  and
establishes `ACEData.io[<i>].args.sgens', if  necessary,  and  returns
`ACEData.io[<i>].args.sgens'.  As  a  side-effect,  if  any   of   the
remaining       fields       of       `ACEData.io[<i>].args'        or
`ACEData.io[<i>].acegens' are unset, they are also set.

\>DisplayACEOptions( <i> ) F
\>DisplayACEOptions() F

display the options of the  <i>th  (or  default)  process  started  by
`ACEStart'. In fact, `DisplayACEOptions(<i>)' is just a pretty-printer
of   the   `ACEData.io[<i>].options'   record.   Use   `GetACEOptions'
(see~"GetACEOptions") in assignments. Please note that no-value {\ACE}
options  will   appear   with   the   assigned   value   `true'   (see
Section~"Interpretation of ACE Options" for how the  {\ACE}  interface
functions interpret such options). Note, however, that any options set
via `ACEWrite' (see~"ACEWrite") will *not* be displayed.

\>GetACEOptions( <i> ) F
\>GetACEOptions() F

return a record of the current  options  of  the  <i>th  (or  default)
process started  by  `ACEStart'.  Please  note  that  no-value  {\ACE}
options  will   appear   with   the   assigned   value   `true'   (see
Section~"Interpretation of ACE Options" for how the  {\ACE}  interface
functions interpret such options). Note, however, that any options set
via `ACEWrite' (see~"ACEWrite") will *not* be included in the returned
record.

\>SetACEOptions( <i> [:<options>] ) F
\>SetACEOptions( [:<options>] ) F

modify the current options of the <i>th (or default)  process  started
by `ACEStart'. Please ensure that the `OptionsStack' is  empty  before
calling `SetACEOptions', otherwise the options already present on  the
`OptionsStack'  will  also  be  \lq{}seen'.  All  interactive   {\ACE}
interface functions that accept options,  actually  call  an  internal
version of `SetACEOptions'; so, it is generally important to keep  the
`OptionsStack' clear while working with {\ACE} interactively.

After  setting  the  options  passed,  the  first  available  mode  of
`ACEContinue'  (see~"ACEContinue"),   `ACERedo'   (see~"ACERedo")   or
`ACEStart' (see~"ACEStart"), is automatically invoked.

Since a user will sometimes have options  in  the  form  of  a  record
(e.g.~via   `GetACEOptions'),   we   provide   a    `PushOptions'-like
alternative to the behind-the-colon syntax for the passing of  options
via `SetACEOptions':

\>SetACEOptions( <i>, <optionsRec> ) F
\>SetACEOptions( <optionsRec> ) F

In this form, the record <optionsRec> is used to  update  the  current
options of the <i>th (or default) process started by `ACEStart'.  Note
that since <optionsRec> is a record each field must have  an  assigned
value; in particular, no-value {\ACE} options should be  assigned  the
value `true' (see Section~"Interpretation  of  ACE  Options").  Please
don't mix these two forms of `SetACEOptions'  with  the  previous  two
forms; i.e.~do *not* pass both a record argument  and  options,  since
this will lead to options appearing in the wrong order; if you want to
do this, make two separate calls to `SetACEOptions', e.g.

\beginexample
gap> SetACEOptions( rec(echo := 2) );
gap> SetACEOptions( : hlt);
\endexample

*Notes:*

When `ACECosetTableFromGensAndRels' enters a  `break'-loop{\undoquotes
\atindex {break-loop} {@`break'-loop}} local versions  of  the  second
form  of  each  of  `DisplayACEOptions'  and  `SetACEOptions'   become
available. (Even though the names are similar and  their  function  is
analogous they are in fact different functions.)

\>ACEParameters( <i> ) F
\>ACEParameters() F

return a record of the current values of the {\ACE} Parameter  Options
(see Section~"ACE  Parameter  Options")  of  the  <i>th  (or  default)
process started by `ACEStart', according to {\ACE}. Please  note  that
some options may be reported with incorrect values if they  have  been
changed  recently  without  following  up  with  one  of   the   modes
`ACEContinue',  `ACERedo'  or  `ACEStart'.   Together   the   commands
`ACEGroupGenerators',   `ACERelators',   `ACESubgroupGenerators'   and
`ACEParameters'  give  the  equivalent  {\GAP}  information  that   is
obtained in {\ACE} with `sr := 1'  (see~"option  sr"),  which  is  the
\lq{}Run  Parameters'  block  obtained   in   the   messaging   output
(observable when the `InfoLevel' of `InfoACE' is set to at  least  3),
when `messages' (see~"option messages") is set a non-zero value.

*Notes:*
One use for this function might be to determine the  options  required
to replicate a previous run,  but  be  sure  that,  if  this  is  your
purpose, that any recent change in the  parameter  option  values  has
been   followed   by   an   invocation   of   one   of   `ACEContinue'
(see~"ACEContinue"),   `ACERedo'   (see~"ACERedo")    or    `ACEStart'
(see~"ACEStart").

As a side-effect, for  {\ACE}  process  <i>,  any  of  the  fields  of
`ACEData.io[<i>].args' or `ACEData.io[<i>].acegens'  that  are  unset,
are set.

\>ACEVersion( <i> ) F
\>ACEVersion() F

for the <i>th  (or  default)  process  started  by  `ACEStart',  print
version details of  the  {\ACE}  binary  you  are  currently  running,
including what compiler flags were set when the executable was  built;
essentially the information obtained is what is obtained via  {\ACE}'s
`options'   option   (see~"option   options").   A   typical   output,
illustrating the default build, is:

\beginexample
gap> ACEVersion();
#I  ACE 3.000
#I  Executable built:
#I    Sat Feb 27 15:57:59 EST 1999
#I  Level 0 options:
#I    statistics package = on
#I    coinc processing messages = on
#I    dedn processing messages = on
#I  Level 1 options:
#I    workspace multipliers = decimal
#I  Level 2 options:
#I    host info = on
\endexample

*Note:*
Unlike other {\ACE} interface functions, the information obtained  via
`ACEVersion();' is absolutely independent of any enumeration. For this
reason, we make it permissible to run `ACEVersion();' when  there  are
no currently active interactive {\ACE} processes; and, in such a case,
`ACEVersion();' initiates (and closes again) its own stream to  obtain
the information from the {\ACE} binary.

The next two functions of this section are really intended for  {\ACE}
standalone gurus. To fully understand their output you  will  need  to
consult the standalone manual and the C source code.

\>ACEDumpVariables( <i> ) F
\>ACEDumpVariables() F
\>ACEDumpVariables( <i>, [<level>] ) F
\>ACEDumpVariables( [<level>] ) F
\>ACEDumpVariables( <i>, [<level>, <detail>] ) F
\>ACEDumpVariables( [<level>, <detail>] ) F

dump the internal variables  of  {\ACE}  of  the  <i>th  (or  default)
process started by `ACEStart'; <level> should be one of 0,  1,  or  2,
and <detail> should be 0 or 1.

The value of <level> determines which of the three levels of {\ACE} to
dump. (You will need to read the standalone manual to understand  what
Levels 0, 1 and 2 are all about.) The value of <detail> determines the
amount of detail (`<detail> = 0' means less  detail).  The  first  two
forms of `ACEDumpVariables' (with no list argument) selects `<level> =
0, <detail> = 0'. The third and fourth forms  (with  a  list  argument
containing the integer <level>) makes `<detail> = 0'. This command  is
intended for gurus; the source code should be consulted  to  see  what
the output means.

\>ACEDumpStatistics( <i> ) F
\>ACEDumpStatistics() F

dump {\ACE}'s internal statistics accumulated during the  most  recent
enumeration of the <i>th (or default) process started  by  `ACEStart',
provided the {\ACE} binary  was  built  with  the  statistics  package
(which it is by default). Use  `ACEVersion();'  (see~"ACEVersion")  to
check for the inclusion of the statistics package.  See  the  `enum.c'
source file for the meaning of the variables.

\>ACEStyle( <i> ) F
\>ACEStyle() F

returns the current enumeration style as one of  the  strings:  `"C"',
`"Cr"', `"CR"', `"R"', `"R*"', `"Rc"', `"R/C"', or `"R/C (defaulted)"'
(see Section~"Enumeration Style").

\>ACEDisplayCosetTable( <i> ) F
\>ACEDisplayCosetTable() F
\>ACEDisplayCosetTable( <i>, [<val>] ) F
\>ACEDisplayCosetTable( [<val>] ) F
\>ACEDisplayCosetTable( <i>, [<val>, <last>] ) F
\>ACEDisplayCosetTable( [<val>, <last>] ) F
\>ACEDisplayCosetTable( <i>, [<val>, <last>, <by>] ) F
\>ACEDisplayCosetTable( [<val>, <last>, <by>] ) F

compact and display the coset table of the <i>th (or default)  process
started by `ACEStart'; <val> must be an integer, and <last>  and  <by>
must be positive integers. In the first two forms of the command,  the
entire  coset  table   is   displayed,   without   orders   or   coset
representatives. In the third and fourth forms, the absolute value  of
<val> is taken to be the last line of the table to be displayed (and 1
is taken to be the first); in the fifth and sixth forms, `|<val>|'  is
taken to be the first line of the table to be displayed, and <last> is
taken to be the number of the last line to be displayed. In  the  last
two forms, the table is displayed from line `|<val>|' to  line  <last>
in steps of <by>. If <val> is negative, then  the  orders  modulo  the
subgroup (if available) and coset representatives are displayed also.

\>IsCompleteACECosetTable( <i> ) F
\>IsCompleteACECosetTable() F

return, for the <i>th (or  default)  process  started  by  `ACEStart',
`true' if {\ACE}'s current coset table is complete (as  determined  by
the index of the current enumeration), and `false' otherwise.

\>ACECosetRepresentative( <i>, <n> ) F
\>ACECosetRepresentative( <n> ) F

return, for the <i>th (or default) process started by `ACEStart',  the
coset representative of coset <n> of the current coset table  held  by
{\ACE}, where <n> must be a positive integer.

\>ACECosetRepresentatives( <i> ) F
\>ACECosetRepresentatives() F

return, for the <i>th (or default) process started by `ACEStart',  the
list of coset representatives of  the  current  coset  table  held  by
{\ACE}.

\>ACETransversal( <i> ) F
\>ACETransversal() F

return, for the <i>th (or default) process started by `ACEStart',  the
list of coset representatives of  the  current  coset  table  held  by
{\ACE}, if the  current  table  is  complete,  and  `fail'  otherwise.
Essentially, `ACETransversal(<i>) = ACECosetRepresentatives(<i>)'  for
a complete table.

\>ACECycles( <i> ) F
\>ACECycles() F
\>ACEPermutationRepresentation( <i> ) F
\>ACEPermutationRepresentation() F

return, for the <i>th (or default) process started  by  `ACEStart',  a
list of permutations corresponding to the group generators, (i.e., the
permutation representation), if the current coset table held by {\ACE}
is complete or `fail', otherwise. In the event of failure a message is
emitted to `Info' at `InfoACE' or `InfoWarning' level 1.

\>ACETraceWord( <i>, <n>, <word> ) F
\>ACETraceWord( <n>, <word> ) F

for the <i>th (or  default)  interactive  {\ACE}  process  started  by
`ACEStart', trace <word> through {\ACE}'s  coset  table,  starting  at
coset <n>, and return the final coset number if the  trace  completes,
and `fail' otherwise. In Group  Theory  terms,  if  the  cosets  of  a
subgroup $H$ in a group $G$ are  the  subject  of  interactive  {\ACE}
process <i> and the coset identified by that process  by  the  integer
<n> corresponds to some coset $Hx$, for some $x$ in  $G$,  and  <word>
represents the element $g$ of $G$, then, providing the  current  coset
table is complete enough, `ACETraceWord( <i>, <n>, <word>  )'  returns
the integer identifying the coset $Hxg$.

*Notes:*
You may wish to compact {\ACE}'s coset table first, either  explicitly
via `ACERecover' (see~"ACERecover"), or, implicitly, via any  function
call that invokes {\ACE}'s compaction routine (see~"ACERecover" note).

If you actually wanted {\ACE}'s  coset  representative,  then,  for  a
*compact*   table,   feed   the   output    of    `ACETraceWord'    to
`ACECosetRepresentative' (see~"ACECosetRepresentative").

\>ACEOrders( <i> ) F
\>ACEOrders() F
\>ACEOrders( <i> : suborder := <suborder> ) F
\>ACEOrders(: suborder := <suborder> ) F

for the <i>th (or  default)  interactive  {\ACE}  process  started  by
`ACEStart', search for coset  numbers  whose  representatives'  orders
(modulo the subgroup) are either  finite,  or,  if  invoked  with  the
`suborder' option,  are  multiples  of  <suborder>,  where  <suborder>
should be a positive integer. `ACEOrders' returns a  (possibly  empty)
list of records, each with fields `coset', `order'  and  `rep',  which
are respectively, the coset number, its order modulo the subgroup, and
a representative for each coset number satisfying the criteria of  the
search. 

*Note:*
You may wish to compact {\ACE}'s coset table first, either  explicitly
via `ACERecover' (see~"ACERecover"), or, implicitly, via any  function
call that invokes {\ACE}'s compaction routine (see~"ACERecover" note).

\>ACEOrder( <i>, <suborder> ) F
\>ACEOrder( <suborder> ) F

for the <i>th (or  default)  interactive  {\ACE}  process  started  by
`ACEStart', search for coset numbers whose coset representatives  have
order modulo the subgroup a multiple of <suborder>. When <suborder> is
a positive integer, `ACEOrder' returns a record with  fields  `coset',
`order' and `rep', which are respectively, the coset number, its order
modulo the subgroup, and a representative for the first  coset  number
satisfying the criteria of the search, or `fail' if there is  no  such
coset number. The value of <suborder> may also be a negative  integer,
in  which  case,  `ACEOrder(  <i>,  <suborder>  )'  is  equivalent  to
`ACEOrders( <i> : suborder := |<suborder>|)';  or  <suborder>  may  be
zero, in which case, `ACEOrder( <i>, 0 )' is equivalent to `ACEOrders(
<i> )'.

*Note:*
You may wish to compact {\ACE}'s coset table first, either  explicitly
via `ACERecover' (see~"ACERecover"), or, implicitly, via any  function
call that invokes {\ACE}'s compaction routine (see~"ACERecover" note).

\>ACECosetsThatStabiliseSubgroup( <i>, <n> ) F
\>ACECosetsThatStabiliseSubgroup( <n> ) F
\>ACECosetsThatNormaliseSubgroup( <i>, <n> ) F
\>ACECosetsThatNormaliseSubgroup( <n> ) F

for the <i>th (or  default)  interactive  {\ACE}  process  started  by
`ACEStart', determine non-trivial  (i.e.~other  than  coset  1)  coset
numbers whose representatives stabilise (i.e.~normalise) the subgroup.

\beginlist

\item{--} If <n> $> 0$, the list of the first  <n>  non-trivial  coset
numbers whose representatives normalise the subgroup is returned.

\item{--} If <n> $\< 0$, a list of records  with  fields  `coset'  and
`rep'  which  represent  the  coset  number  and   a   representative,
respectively,  of  the  first  <n>  non-trivial  coset  numbers  whose
representatives normalise the subgroup is returned.

\item{--} If <n> $= 0$, a list of  records  with  fields  `coset'  and
`rep'  which  represent  the  coset  number  and   a   representative,
respectively, of all non-trivial coset numbers  whose  representatives
normalise the subgroup is returned.

\endlist

*Note:*
You may wish to compact {\ACE}'s coset table first, either  explicitly
via `ACERecover' (see~"ACERecover"), or, implicitly, via any  function
call that invokes {\ACE}'s compaction routine (see~"ACERecover" note).

\enditems

%%%%%%%%%%%%%%%%%%%%%%%%%%%%%%%%%%%%%%%%%%%%%%%%%%%%%%%%%%%%%%%%%%%%%%
\Section{Interactive Versions of Non-interactive ACE Functions}

\beginitems

\>ACECosetTable( <i> [:<options>] ) F
\>ACECosetTable( [:<options>] ) F

return a coset table  as  a  {\GAP}  object,  in  standard  form  (for
{\GAP}).   These   functions   perform   the    same    function    as
`ACECosetTableFromGensAndRels' and `ACECosetTable' on three arguments,
albeit interactively, on the <i>th (or  default)  process  started  by
`ACEStart'.  If  options  are  passed  then  an  internal  version  of
`ACEModes'  is  run   to   determine   which   of   the   modes   (see
Section~"General  and  Experimentation  ACE   Modes")   `ACEContinue',
`ACERedo' or `ACEStart' is possible; and (an internal version of)  the
first mode of these  that  is  allowed  is  executed,  to  ensure  the
resultant table is correct for the current options.

\>ACEStats( <i> [:<options>] ) F
\>ACEStats( [:<options>] ) F

perform the same function as `ACEStats'  on  three  arguments,  albeit
interactively,  on  the  <i>th  (or  default)   process   started   by
`ACEStart'.  If  options  are  passed  then  an  internal  version  of
`ACEModes'  is  run   to   determine   which   of   the   modes   (see
Section~"General  and  Experimentation  ACE   Modes")   `ACEContinue',
`ACERedo' or `ACEStart' is possible; and (an internal version of)  the
first mode of these  that  is  allowed  is  executed,  to  ensure  the
resultant statistics are correct for the current options.

See Section~"Example of Using ACE Interactively (Using ACEStart)"  for
an example demonstrating both these functions  within  an  interactive
process.

\enditems

%%%%%%%%%%%%%%%%%%%%%%%%%%%%%%%%%%%%%%%%%%%%%%%%%%%%%%%%%%%%%%%%%%%%%%
\Section{Steering ACE Interactively}

\beginitems

\>ACERecover( <i> ) F
\>ACERecover() F

invokes on the coset table, of  the  <i>th  (or  default)  interactive
{\ACE} process  started  by  `ACEStart',  the  compaction  routine  to
recover the space used by  the  dead  coset  numbers\index{dead  coset
(number)}. A `CO' message line is printed if any  rows  of  the  coset
table were recovered, and a `co' line if none were. (See Appendix~"The
Meanings  of  ACE's  output  messages"  for  the  meanings  of   these
messages.)

*Note:*
The  compaction  routine  is  called   automatically   when   any   of
`ACEDisplayCosetTable'                   (see~"ACEDisplayCosetTable"),
`ACECosetRepresentative'               (see~"ACECosetRepresentative"),
`ACECosetRepresentatives'             (see~"ACECosetRepresentatives"),
`ACETransversal'          (see~"ACETransversal"),          `ACECycles'
(see~"ACECycles"),                         `ACEStandardCosetNumbering'
(see~"ACEStandardCosetNumbering"),                     `ACECosetTable'
(see~"ACECosetTable")    or    `ACEConjugatesForSubgroupNormalClosure'
(see~"ACEConjugatesForSubgroupNormalClosure"), is invoked.

\>ACEStandardCosetNumbering( <i> ) F
\>ACEStandardCosetNumbering() F

compacts and then standardises the numbering of cosets  in  the  coset
table of the <i>th (or default) interactive {\ACE} process started  by
`ACEStart'.  This  function  does  not  display  the  new  table;  use
`ACEDisplayCosetTable' (see~"ACEDisplayCosetTable") for that. That is,
for a given ordering of the generators in the columns of the table, it
produces a canonic table.  Such  a  table  has  the  property  that  a
row-major scan (i.e.~a scan of the successive rows of  the  *body*  of
the table row by row, from left to right) encounters previously unseen
cosets in numeric order.

*Notes:*
In a canonic  table,  the  coset  representatives  are  ordered  first
according to length and then the lexicographic order  defined  by  the
order the generators and their inverses head the columns.  Note  that,
since {\ACE} avoids having an involutory generator in the first column
when it can, this lexicographic order does not necessarily  correspond
with the order in which the generators were first put to  {\ACE}.  Two
tables are equivalent only if their canonic forms are the same.  Also,
standardising the coset numbering within {\ACE} does *not* affect  the
{\GAP} coset table obtained via `ACECosetTable'.

*Guru Notes:*
In  half  of  the  ten  standard  enumeration   strategies   of   Sims
\cite{Sim94}, the table is standardised repeatedly. This is  expensive
computationally, but can result in fewer cosets being  necessary.  The
effect of doing this can be investigated  in  {\ACE}  by  (repeatedly)
halting  the  enumeration  (by  say,  imposing  timing  restrictions),
standardising the coset numbering, and continuing.

\>ACEAddRelators( <i>, <wordlist> ) F
\>ACEAddRelators( <wordlist> ) F

add, for the <i>th (or default) interactive {\ACE} process started  by
`ACEStart', the words in the list <wordlist> to any  relators  already
present, and automatically invokes  either  `ACERedo'  or  `ACEStart'.
Note that, {\ACE} sorts the resultant relator list, unless the  `asis'
option (see~"option asis") has been set to  1;  don't  assume,  unless
`asis = 1', that the new relators have been appended in  user-provided
order to the previously existing relator list.  `ACEAddRelators'  also
returns the new relator list. Use `ACERelators' (see~"ACERelators") to
determine the current relator list.

\>ACEAddSubgroupGenerators( <i>, <wordlist> ) F
\>ACEAddSubgroupGenerators( <wordlist> ) F

add, for the <i>th (or default) interactive {\ACE} process started  by
`ACEStart',  the  words  in  the  list  <wordlist>  to  any   subgroup
generators already present.  The  enumeration  must  be  restarted  or
redone, it cannot be  continued.  add,  for  the  <i>th  (or  default)
interactive {\ACE} process started by `ACEStart',  the  words  in  the
list <wordlist>  to  any  subgroup  generators  already  present,  and
automatically invokes  either  `ACERedo'  or  `ACEStart'.  Note  that,
{\ACE} sorts the resultant subgroup generator list, unless the  `asis'
option (see~"option asis") has been set to  1;  don't  assume,  unless
`asis = 1', that the new subgroup generators  have  been  appended  in
user-provided order to  the  previously  existing  subgroup  generator
list.  `ACEAddSubgroupGenerators'  also  returns  the   new   subgroup
generator          list.          Use          `ACESubgroupGenerators'
(see~"ACESubgroupGenerators")  to  determine  the   current   subgroup
generator list.

\>ACEDeleteRelators( <i>, <list> ) F
\>ACEDeleteRelators( <list> ) F

for the <i>th (or  default)  interactive  {\ACE}  process  started  by
`ACEStart', delete <list> from the current relators, if list is a list
of words in the group generators, or those current relators indexed by
the integers in <list>, if <list> is a list of positive integers,  and
automatically invoke `ACEStart'. `ACEDeleteRelators' also returns  the
new relator list. Use `ACERelators' (see~"ACERelators")  to  determine
the current relator list.

\>ACEDeleteSubgroupGenerators( <i>, <list> ) F
\>ACEDeleteSubgroupGenerators( <list> ) F

for the <i>th (or  default)  interactive  {\ACE}  process  started  by
`ACEStart', delete <list> from the  current  subgroup  generators,  if
list is a list of words in the  group  generators,  or  those  current
subgroup generators indexed by the integers in <list>, if <list> is  a
list  of  positive  integers,  and  automatically  invoke  `ACEStart'.
`ACEDeleteSubgroupGenerators' also returns the new subgroup  generator
list.  Use  `ACESubgroupGenerators'  (see~"ACESubgroupGenerators")  to
determine the current subgroup generator list.

\>ACECosetCoincidence( <i>, <n> ) F
\>ACECosetCoincidence( <n> ) F

for the <i>th (or  default)  interactive  {\ACE}  process  started  by
`ACEStart', return the representative of coset <n>, where <n> must  be
a positive integer, and add  it  to  the  subgroup  generators;  i.e.,
equates  this  coset  with  coset  1,  the  subgroup.   `ACERedo'   is
automatically invoked.

\>ACERandomCoincidences( <i>, <subindex> ) F
\>ACERandomCoincidences( <subindex> ) F
\>ACERandomCoincidences( <i>, [<subindex>] ) F
\>ACERandomCoincidences( [<subindex>] ) F
\>ACERandomCoincidences( <i>, [<subindex>, <attempts>] ) F
\>ACERandomCoincidences( [<subindex>, <attempts>] ) F

for the <i>th (or  default)  interactive  {\ACE}  process  started  by
`ACEStart', attempt upto <attempts> (or, in the first four  forms,  8)
times to find nontrivial subgroups with index a multiple of <subindex>
by repeatedly making random coset numbers coincident with coset 1  and
seeing what happens. The starting coset table must be  non-empty,  but
should not be complete. For each attempt,  we  repeatedly  add  random
coset representatives to the subgroup and `redo' the  enumeration.  If
the table becomes too small, the  attempt  is  aborted,  the  original
subgroup generators restored, and another attempt made. If an  attempt
succeeds, then  the  new  set  of  subgroup  generators  is  retained.
`ACERandomCoincidences' returns the list of  new  subgroup  generators
added. Use  `ACESubgroupGenerators'  (see~"ACESubgroupGenerators")  to
determine the current subgroup generator list.

*Notes:* 
`ACERandomCoincidences' may add subgroup generators even if it  failed
to  determine  a  nontrivial  subgroup  with  index  a   multiple   of
<subindex>; in such a case, the original status  may  be  restored  by
applying                                 `ACEDeleteSubgroupGenerators'
(see~"ACEDeleteSubgroupGenerators")  with   the   list   returned   by
`ACERandomCoincidences'.

It  makes  no  sense  to  invoke   `ACERandomCoincidences'   when   an
enumeration has already obtained a finite index (either, <subindex> is
already a divisor of that finite index, or the request is impossible).
Thus an invocation of `ACERandomCoincidences', in  this  case,  is  an
error.

*Guru  Notes:*  A  coset  can  have  many  different  representatives.
Consider              running              `ACEStandardCosetNumbering'
(see~"ACEStandardCosetNumbering") before  `ACERandomCoincidences',  to
canonicise the table and the representatives.

\>ACEConjugatesForSubgroupNormalClosure( <i> ) F
\>ACEConjugatesForSubgroupNormalClosure() F
\>ACEConjugatesForSubgroupNormalClosure( <i> : add ) F
\>ACEConjugatesForSubgroupNormalClosure(: add ) F

for the <i>th (or  default)  interactive  {\ACE}  process  started  by
`ACEStart', tests each conjugate of a subgroup generator  by  a  group
generator for membership of the subgroup, and  returns  the  (possibly
empty) list of conjugates that were determined  to  belong  to  cosets
other than coset 1 (the subgroup);  and,  if  called  with  the  `add'
option, these conjugates are  also  added  to  the  existing  list  of
subgroup generators.

*Notes:* A conjugate of a subgroup generator is tested for  membership
of the subgroup, by checking whether it can be traced from coset 1  to
coset 1  (see  `ACETraceWord':~"ACETraceWord").  For  an  *incomplete*
coset  table,  such  a  trace  may  not  complete,   in   which   case
`ACEConjugatesForSubgroupNormalClosure' may return an empty list  even
though the subgroup is *not* normally closed within the group.

The `add' option does *not* guarantee that the resultant  subgroup  is
normally closed. It is still possible  that  some  conjugates  of  the
newly added subgroup generators will not be elements of the subgroup.

\enditems

%%%%%%%%%%%%%%%%%%%%%%%%%%%%%%%%%%%%%%%%%%%%%%%%%%%%%%%%%%%%%%%%%%%%%%%%%
%%
%E
