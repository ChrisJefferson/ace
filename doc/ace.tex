%%%%%%%%%%%%%%%%%%%%%%%%%%%%%%%%%%%%%%%%%%%%%%%%%%%%%%%%%%%%%%%%%%%%%%%%%
%%
%W  ace.tex                GAP documentation             Alexander Hulpke
%%
%H  $Id$
%%
%Y  Copyright (C) 1998, School of Math & Comp. Sci., St Andrews, Scotland
%%

%%%%%%%%%%%%%%%%%%%%%%%%%%%%%%%%%%%%%%%%%%%%%%%%%%%%%%%%%%%%%%%%%%%%%%%%%
\Chapter{The ACE share package}

The ACE share package permits to call the Adaptive Coset Enumerator written
by George Havas and Colin Ramsay from within GAP. 
\index{ace}

\begintt
ACE coset enumerator (C) 1995-1998 by George Havas and Colin Ramsay
    http://www.it.uq.edu.au/~havas/ace2.tar.gz
\endtt

The {\GAP} interface has been written by Alexander Hulpke.

The package interfaces to an external binary and therefore is only usable
under UNIX. It will not work on Windows or the Macintosh. It requires
{\GAP}4 in at least beta version 5.

%%%%%%%%%%%%%%%%%%%%%%%%%%%%%%%%%%%%%%%%%%%%%%%%%%%%%%%%%%%%%%%%%%%%%%%%%
\Section{Installing the ACE share package}

To install unpack the archive file in a directory in the `pkg' hierarchy of
your version of {\GAP}4. (This might be the `pkg' directory of the
GAP4 home directory; it is however also possible to keep an additional `pkg'
directory in you private directories, see section "ref:Installing Share
Packages" of the {\GAP}4 reference manual for details on how to do this.)
Go to the newly created `ACE' directory and call `configure <path>' where
<path> is the path to the GAP home directory. So for example if you install
the package in the main `pkg' directory call
\begintt
configure ../..
\endtt
This will fetch the architecture type for which GAP has been compiled last
and create a `Makefile'. 
Now simply call
\begintt
make
\endtt
to compile the binary and to install it in the appropriate place.

If you use this installation of GAP on different hardware platforms you will
have to compile the binary for each platform separately. This is done by
calling `configure' and `make' for the package anew immediately after
compiling GAP itself for the respective architecture.
If your version of GAP is already compiled (and has last been compiled on
the same architecture) you do not need to compile GAP
again, it is sufficient to call the `configure' script in the GAP home
directory.

For details on how to use the `ACE' package see the documentation in the
`doc' subdirectory.

The `src' subdirectory contains a copy of the original source of `ACE'. (The
only modification is that a make-file was copied to `Makefile'.) You can
replace it by a newer version before compiling.

%%%%%%%%%%%%%%%%%%%%%%%%%%%%%%%%%%%%%%%%%%%%%%%%%%%%%%%%%%%%%%%%%%%%%%%%%
\Section{Loading the ACE share package}

To use the `ACE' package you have to request it explicitly. This is done by
calling 
\begintt
gap> RequirePackage("ace");
  Loading the `ACE' share package version 1.0
  Using ACE binary version 2.000
\endtt

If {\GAP} cannot find a working binary, the call to `RequirePackage' will
fail.

If you want to load the `ACE' package by default, you can put this
`RequirePackage' command in your `gaprc' file.

If you want to use the `ACE' coset enumerator as a default for all coset
enumerations done by GAP, you can achieve this by setting the global
variable `TCENUM' to `ACETCENUM'.
\begintt
gap> TCENUM:=ACETCENUM;;
\endtt

You
can switch back to the coset enumerator built into the GAP library by
assining `TCENUM' to `GAPTCENUM'.
\begintt
gap> TCENUM:=GAPTCENUM;
\endtt


%%%%%%%%%%%%%%%%%%%%%%%%%%%%%%%%%%%%%%%%%%%%%%%%%%%%%%%%%%%%%%%%%%%%%%%%%
\Section{The ACE interface}

To call the `ACE' binary directly, you may call

\>ACETC.CosetTableFromGensAndRels(<fgens>,<rels>,<sgens>)

where <fgens> is a list of free generators, <rels> a list of words in these
generators giving relators for a finitely presented group and <sgens>
subgroup generators, again expressed as words in the free generators. The
operation calls the ACE coset enumerator and returns the coset table
obtaines or `fail' if the coset enumeration did not finish in the preset
limits. (See section "Options for the ACE enumerator" on how to extend these
limits and how to set enumeration options.)

\begintt
gap> g:=PerfectGroup(2^5*60,2);;
gap> f:=FreeGeneratorsOfFpGroup(g);;
gap> CallACE(f,RelatorsOfFpGroup(g),f{[1,2]});;
\endtt

%%%%%%%%%%%%%%%%%%%%%%%%%%%%%%%%%%%%%%%%%%%%%%%%%%%%%%%%%%%%%%%%%%%%%%%%%
\Section{Options for the ACE enumerator}

The `ACE' program offers a wide variety of options to direct and guide a
coset enumeration. You can use the option mechanism of GAP (see
chapter~"ref:Options Stack" in the reference manual) to pass such options to
the C standalone.

The options currently supported are:

\beginitems
`com:=<val>'&Sets the percentage of dead cosets needed to trigger
compaction. The default is 20.

`workspace:=<val>'&Sets the size of the workspace in words.
The default is 200000.

`max:=<val>'&Sets the maximum number of cosets to be defined. The syntax
`max:=[<val1>,<val2>,<val3>]' is also permitted.

`rt:=<val>'&Sets the RT factor. The syntax
`rt:=[<val1>,<val2>,<val3>]' is also permitted.

`ct:=<val>'&Sets the CT factor. The syntax
`ct:=[<val1>,<val2>,<val3>]' is also permitted.

`time:=<val>'&Sets maximum execution time in seconds.
The default is 0 which is no time limit.

`fill:=<val>'&Sets the fill factor. The syntax
`fill:=[<val1>,<val2>,<val3>]' is also permitted.

`mendelsohn'&Turns on Mendelsohn style relator processing.

`asis'&Turns off cyclic reduction and sorting of relators.

`mess'&Displays the output of `ACE' on screen. By assigning `mess' to
integers, the internal message level of `ACE' is changed.

\enditems

See the `ACE' `README' file for a more detailled description of the function of
these options.

The following example calls ACE for up to 500 cosets using
Mendelsohn style relator processing and sets the message level to 200
\begintt
gap> CallACE(f,RelatorsOfFpGroup(g),f{[1]}:mendelsohn,max:=500,mess:=200);;
Start ACE v1.1 on Fri Oct  9 14:55:49 1998

Enumeration: G
Group Generators: abcdefg;
Group Relators: (c)^2, (d)^2, (e)^2, (f)^2, (g)^2, aaG, (b)^3, (CD)^2, 
  (EF)^2, (CE)^2, (CF)^2, (DE)^2, (DF)^2, AcaE, AdaF, AeaC, AfaD, BfbE, 
  GAga, GBgb, (GC)^2, (GD)^2, (GE)^2, (GF)^2, BebFE, BcbGFD, BdbFEDC, 
  (ab)^5;
Subgroup name: H
Subgroup Generators: a;
Max=500 CTF=0 RTF=0 FillF=13 NRinS=0 Mendel=1 Adapt=0
RD a=200 m=200 t=234 knr=29 knc=1 cpu=0.00
RD a=400 m=400 t=486 knr=68 knc=1 cpu=0.01
CO a=296 m=407 t=500 knr=67 knc=158 cpu=0.03
INDEX = 480  (MAX=480 TOT=684 KNR=67 KNC=481 CPU=0.06)
#DEF: gap=183 knc=1 ff=0 dead pdl=0; knr=499 fill=0 (sg=46)
#GAP: tot=2714 full=1082 #DED: CT=811 RT=605 LA=366 discarded=23***
#COINC: CT=0 RT=74 LA=65 ap=1
\endtt

If `ACE' is made the standard coset enumerator the same method of passing
arguments may be used with all other commands and will affect coset
enumerations. As an example we use the `ACE' enumerator to compute the
permutation representation of a perfect group from the data library:

\begintt
gap> TCENUM:=ACETCENUM;;
gap> PerfectGroup(IsPermGroup,16*60,1:max:=[50,500,40],mess);
Start ACE 2.000 on Fri Oct 23 10:09:53 1998
host name: muir

Enumeration: G;
INDEX = 16 (MAX=30 TOT=35 KNR=36 KNC=1 CPU=0.00)
INDEX = 16 (MAX=30 TOT=35 KNR=36 KNC=1 CPU=0.00)
INDEX = 16 (MAX=30 TOT=35 KNR=36 KNC=1 CPU=0.00)
INDEX = 16 (MAX=30 TOT=35 KNR=36 KNC=1 CPU=0.00)
INDEX = 16 (MAX=30 TOT=35 KNR=36 KNC=1 CPU=0.00)
INDEX = 16 (MAX=30 TOT=35 KNR=36 KNC=1 CPU=0.00)
INDEX = 16 (MAX=30 TOT=35 KNR=36 KNC=1 CPU=0.01)
INDEX = 16 (MAX=30 TOT=35 KNR=36 KNC=1 CPU=0.00)
INDEX = 16 (MAX=30 TOT=35 KNR=36 KNC=1 CPU=0.00)
INDEX = 16 (MAX=30 TOT=35 KNR=36 KNC=1 CPU=0.00)
INDEX = 16 (MAX=30 TOT=35 KNR=36 KNC=1 CPU=0.00)
INDEX = 16 (MAX=30 TOT=35 KNR=36 KNC=1 CPU=0.00)
A5 2^4
\endtt
