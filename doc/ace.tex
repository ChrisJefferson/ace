%%%%%%%%%%%%%%%%%%%%%%%%%%%%%%%%%%%%%%%%%%%%%%%%%%%%%%%%%%%%%%%%%%%%%%%%%
%%
%W  ace.tex         ACE documentation introduction       Alexander Hulpke
%W                                                      Joachim Neub"user
%W                                                            Greg Gamble
%%
%H  $Id$
%%
%Y  Copyright (C) 2000  Centre for Discrete Mathematics and Computing
%Y                      Department of Computer Science & Electrical Eng.
%Y                      University of Queensland, Australia.
%%

%%%%%%%%%%%%%%%%%%%%%%%%%%%%%%%%%%%%%%%%%%%%%%%%%%%%%%%%%%%%%%%%%%%%%%%%%
\Chapter{The ACE Share Package}

The \lq{}Advanced Coset Enumerator' {\ACE}:

\begintt
ACE coset enumerator (C) 1995-1999 by George Havas and Colin Ramsay
    http://www.csee.uq.edu.au/~havas/ace3.tar.gz
\endtt

can  be called  from within  {\GAP} through  an interface,  written by
Alexander Hulpke and Greg Gamble, which is described in this manual.

The interface links to an external binary and therefore is only usable
under UNIX (see Section~"Installing the ACE Share Package" for how  to
install {\ACE}).  It will not work  on  Windows  or the Macintosh.  It
requires {\GAP}~4.2.

{\ACE} can be used through this interface in a variety  of  ways:

\beginlist

\item{--} one may supplant the  usual  {\GAP}  coset  enumerator  (see
Section~"Using ACE as a Default for Coset Enumerations"),

\item{--}  one  may  generate  a  coset  table  using  {\ACE}  without
redefining the usual {\GAP} coset enumerator (see  Section~"Using  ACE
Directly to Generate a Coset Table"),

\item{--} one  may  simply  test  whether  a  coset  enumeration  will
terminate (see Section~"Using ACE Directly to  Test  whether  a  Coset
Enumeration Terminates"),

\item{--} one may  use  {\GAP}  to  write  a  script  for  the  {\ACE}
standalone  (see  Section~"Writing  ACE  Standalone  Input  Files   to
Generate a Coset Table"), and

\item{--} one may interact with  the  {\ACE}  standalone  from  within
{\GAP} (see Section~"Using ACE Interactively").  Among  other  things,
the   interactive   {\ACE}   interface   functions    (described    in
Chapter~"Functions for Using ACE Interactively") enable  the  user  to
search for subgroups of a group (see the note  of  Section~"Using  ACE
Interactively").

\endlist

Each of these ways gives the user access to a welter of options, which
are discussed in full  in  Chapters~"Options  for  ACE"  and~"Strategy
Options for ACE".

*Note*: Some care needs to be taken with  options;  be  sure  to  read
Section~"General Warnings  regarding  the  Use  of  Options"  and  the
introductory sections of Chapter~"Options for ACE" for  some  warnings
regarding them and a general discussion of their use, before using any
of the functions provided by this interface to the {\ACE} binary.

%%%%%%%%%%%%%%%%%%%%%%%%%%%%%%%%%%%%%%%%%%%%%%%%%%%%%%%%%%%%%%%%%%%%%%
\Section{Using ACE as a Default for Coset Enumerations}

After loading  {\ACE}   (see Section~"Loading the ACE Share Package"),
if you  want to use the {\ACE}  coset enumerator as a  default for all
coset  enumerations  done  by   {\GAP}  (which  may  also  get  called
indirectly),  you can  achieve  this by  setting  the global  variable
`TCENUM' to `ACETCENUM'.

\beginexample
gap> TCENUM:=ACETCENUM;;
\endexample

This    sets    the    function    `CosetTableFromGensAndRels'    (see
Section~"ref:Coset  Tables  and  Coset  Enumeration"  in  the   {\GAP}
Reference Manual) to be  the  function  `ACECosetTableFromGensAndRels'
(described in Section~"Using ACE Directly to Generate a Coset Table"),
which then can be called with all the options defined for  the  {\ACE}
interface, not just the options `max' and `silent'. If `TCENUM' is set
to `ACETCENUM' without any  further  action,  the  `default'  strategy
(option) of the {\ACE} enumerator will be used (see  Chapter~"Strategy
Options for ACE").

You can switch back to the coset  enumerator  built  into  the  {\GAP}
library by assigning `TCENUM' to `GAPTCENUM'.

\beginexample
gap> TCENUM:=GAPTCENUM;;
\endexample

%%%%%%%%%%%%%%%%%%%%%%%%%%%%%%%%%%%%%%%%%%%%%%%%%%%%%%%%%%%%%%%%%%%%%%
\Section{Using ACE Directly to Generate a Coset Table}

If, on the other hand you do not want to set up  {\ACE}  globally  for
your coset enumerations, you may call the {\ACE}  interface  directly,
which will allow you to decide for yourself, for each such call, which
options you want to use for running {\ACE}. Please note  the  warnings
regarding options in Section~"General Warnings regarding  the  Use  of
Options". The functions discussed in this and  the  following  section
(`ACECosetTableFromGensAndRels' and `ACEStats')  are  non-interactive,
i.e.~by their use, a file with your  input  data  in  {\ACE}  readable
format will be handed to {\ACE} and you will get the  answer  back  in
{\GAP} format. At that moment however the {\ACE}  job  is  terminated,
that is, you cannot send any further questions or requests  about  the
result of that job to {\ACE}. For an interactive use  of  {\ACE}  from
{\GAP} see Section~"Using ACE  Interactively"  and  Chapter~"Functions
for Using ACE Interactively".

\beginitems

Using the {\ACE} interface directly to generate a coset table is  done
by either of

\>ACECosetTableFromGensAndRels( <fgens>, <rels>, <sgens> [: <options>] ) F
\>ACECosetTable( <fgens>, <rels>, <sgens> [: <options>] ) F

Here <fgens> is a list of free generators, <rels> a list of  words  in
these generators giving relators for a finitely presented  group,  and
<sgens> the list of subgroup generators, again expressed as  words  in
the free generators. All these are given in the standard {\GAP} format
(see Chapter~"ref:Finitely Presented Groups" of the  {\GAP}  Reference
Manual). Note that `ACECosetTable' may be called with 0 or 1 arguments
after  a  call  to  `ACEStart',  in  which  case  it  is  being   used
interactively   (see    Section~"Using    ACE    Interactively"    and
Chapter~"Functions for Using ACE Interactively").

Behind the colon any  selection  of  the  options  available  for  the
interface (see Chapters~"Options for ACE"  and~"Strategy  Options  for
ACE") can be given, separated by commas like record components.  These
can be used e.g.~to preset limits of space and time  to  be  used,  to
modify input and output and to modify the enumeration procedure.  Note
that strategies are simply special options that set a  number  of  the
options, detailed in Chapter~"Options for ACE", all at once.

Please see Section~"General Warnings regarding the Use of Options" for
a discussion regarding global and local passing of  options,  and  the
non-orthogonal nature of {\ACE}'s options.

The function calls the {\ACE} binary and  returns  the  (standardized)
coset table obtained.

The example given  below  is  the  call  for  a  presentation  of  the
Fibonacci group F(2,7) for  which  we  shall  discuss  the  impact  of
various options in Appendix~"Examples". Observe that  in  the  example
below, no options  are  passed,  which  means  that  {\ACE}  uses  the
`default' strategy (see Chapter~"Strategy Options for ACE").

\beginexample
gap> F:= FreeGroup( "a", "b", "c", "d", "e", "x", "y");;
gap> a:= F.1;; b:= F.2;; c:= F.3;; d:= F.4;; e:= F.5;; x:= F.6;; y:= F.7;;
gap> fgens:= [a, b, c, d, e, x, y,];;
yap> rels:= [ a*b*c^-1, b*c*d^-1, c*d*e^-1, d*e*x^-1, 
>             e*x*y^-1, x*y*a^-1, y*a*b^-1];;
gap> ACECosetTable(fgens, rels, [c]);;
\endexample

In computing  the  coset  table,  `ACECosetTableFromGensAndRels'  must
first do a coset enumeration (which is where {\ACE} comes in!). If the
coset  enumeration  does  not  finish   in   the   preset   limits   a
`break'-loop{\undoquotes\atindex{break-loop}{@`break'-loop}}        is
entered, unless the `silent' option (see~"option silent") is  set,  in
which case it returns `fail'. In the  event  that  a  `break'-loop  is
entered, don't despair or  be  frightened  by  the  word  `Error';  by
tweaking  {\ACE}'s  options  via  the  `SetACEOptions'  function  that
becomes available in the `break'-loop and then typing `return;' it may
be possible to help {\ACE} complete the coset enumeration  (and  hence
successfully compute the coset table); if not, you will end up in  the
`break'-loop again, and you can have another go (or `quit;' if  you've
had enough). The `SetACEOptions' function is a  no-argument  function;
it's there *purely* to pass *options* (which, of  course,  are  listed
behind a colon (`:') with record components  syntax).  Let's  continue
the Fibonacci example above, redoing the last  command  but  with  the
option `max := 2' (see~"option max"), so that  the  coset  enumeration
has only two coset numbers to play with and hence is bound to fail  to
complete, putting us in a `break'-loop.

\beginexample
gap> ACECosetTable(fgens, rels, [c] : max := 2);
Error : No coset table ... at
Error( ": No coset table ..." );
#I  The `ACE' coset enumeration failed with the result:
#I  OVERFLOW (a=2 r=1 h=1 n=3; l=5 c=0.00; m=2 t=2)
#I  Try relaxing any restrictive options:
#I  type: 'DisplayACEOptions();' to see current ACE options;
#I  type: 'SetACEOptions(:<option1> := <value1>, ...);'
#I  to set <option1> to <value1> etc.
#I  (i.e. pass options after the ':' in the usual way)
#I  ... and then, type: 'return;' to continue.
#I  Otherwise, type: 'quit;' to quit the enumeration.
Entering break read-eval-print loop, you can 'quit;' to quit to outer loop,
or you can return to continue
brk> SetACEOptions(: max := 0);
brk> return;
[ [ 1 ], [ 1 ], [ 1 ], [ 1 ], [ 1 ], [ 1 ], [ 1 ], [ 1 ], [ 1 ], [ 1 ], 
  [ 1 ], [ 1 ], [ 1 ], [ 1 ] ]
gap>
\endexample

Observe how the `Info' lines (the  ones  starting  with  \lq{}`\#I '')
tell you *exactly* what to do. At the `break'-loop  prompt  `brk>'  we
relaxed all restrictions on `max' (by re-setting it to  0)  and  typed
`return;' to leave the `break'-loop. The coset  enumeration  was  then
successful, allowing the computation  of  what  turned  out  to  be  a
trivial coset table. Despite the fact that the  eventual  coset  table
only has one line (i.e.~there is  exactly  one  coset  number)  {\ACE}
*did* need to define more than 2 coset numbers. To find out  just  how
many  were  required  before  the  final  collapse,  let's   set   the
`InfoLevel' of `InfoACE' (see~"SetInfoACELevel") to  2,  so  that  the
{\ACE} enumeration result is printed.

\beginexample
gap> SetInfoACELevel(2);
gap> ACECosetTable(fgens, rels, [c]);
#I  INDEX = 1 (a=1 r=2 h=2 n=2; l=6 c=0.01; m=2049 t=3127)
[ [ 1 ], [ 1 ], [ 1 ], [ 1 ], [ 1 ], [ 1 ], [ 1 ], [ 1 ], [ 1 ], [ 1 ], 
  [ 1 ], [ 1 ], [ 1 ], [ 1 ] ]
\endexample

The enumeration result line is the `Info' line beginning \lq{}`\#I ''.
Appendix~"The Meanings of  ACE's  Output  Messages"  explains  how  to
interpret such output messages from {\ACE}. In particular, it explains
that `t=3127' tells us that a `t'otal number  of  3127  coset  numbers
needed to be defined before the final  collapse  to  1  coset  number.
Using some of the many options that {\ACE} provides, one  may  achieve
this  result  more  efficiently,  e.g.~with   the   `purec'   strategy
(see~"option purec"):

\beginexample
gap> ACECosetTable(fgens, rels, [c] : purec);
#I  INDEX = 1 (a=1 r=2 h=2 n=2; l=4 c=0.00; m=332 t=332)
[ [ 1 ], [ 1 ], [ 1 ], [ 1 ], [ 1 ], [ 1 ], [ 1 ], [ 1 ], [ 1 ], [ 1 ], 
  [ 1 ], [ 1 ], [ 1 ], [ 1 ] ]
\endexample

{\ACE} needs to define a `t'otal number of only  (relatively-speaking)
332 coset numbers before the final collapse to 1 coset number.

\enditems

%%%%%%%%%%%%%%%%%%%%%%%%%%%%%%%%%%%%%%%%%%%%%%%%%%%%%%%%%%%%%%%%%%%%%%
\Section{Using ACE  Directly  to  Test  whether  a  Coset  Enumeration
Terminates}

\beginitems

If you only want to  test, whether a coset enumeration terminates, and
don't want to  transfer the whole coset table  to {\GAP}, you can call

\>ACEStats( <fgens>, <rels>, <sgens> [: <options>] ) F

which returns a record `rec(index=<i>, cputime=<c>, cputimeUnits=<cu>,
activecosets=<a>, maxcosets=<m>, totcosets=<t>)' for which <i> is  the
index of the subgroup $\langle <sgens> \rangle$  in  $\langle  <fgens>
\mid <rels> \rangle$  (or  $0$,  whenever  the  enumeration  does  not
succeed); <c> is the total CPU time used as  an  integer;  <cu>  is  a
string representing the units of <c>, e.g.~`"10^-2 seconds"';  <a>  is
the number of currently active  coset  numbers;  <m>  is  the  maximum
number  of  \lq{}alive'  coset  numbers  at  any  one  time   in   the
enumeration; and <t> is the total number of coset  numbers  that  were
defined in the enumeration. Options (see  Chapters~"Options  for  ACE"
and~"Strategy Options for ACE") are used in exactly the  same  way  as
for `ACECosetTableFromGensAndRels', discussed in the previous section;
and the same warnings alluded to previously,  regarding  options  (see
Section~"General Warnings regarding the Use of Options"), apply.

\enditems

%%%%%%%%%%%%%%%%%%%%%%%%%%%%%%%%%%%%%%%%%%%%%%%%%%%%%%%%%%%%%%%%%%%%%%
\Section{Writing ACE Standalone Input Files to Generate a Coset Table}

If you want to use {\ACE} as a standalone with its own syntax, you can
write an {\ACE} standalone input file by calling `ACECosetTable'  with
three arguments (see "ACECosetTableFromGensAndRels")  and  the  option
`aceinfile := <filename>' (see~"option aceinfile"). This will keep the
input file for the {\ACE} standalone produced by the {\GAP}  interface
under the file name <filename> (and  just  return)  so  that  you  can
perform interactive work in the standalone.

%%%%%%%%%%%%%%%%%%%%%%%%%%%%%%%%%%%%%%%%%%%%%%%%%%%%%%%%%%%%%%%%%%%%%%
\Section{Using ACE Interactively}

\beginitems

An interactive {\ACE} process is initiated with the command

\>ACEStart( <fgens>, <rels>, <sgens> [:<options>] ) F

whose    arguments    and    options    are     exactly     as     for
`ACECosetTableFromGensAndRels'  and  `ACEStats',   as   discussed   in
Sections~"Using ACE Directly to Generate a Coset Table" and~"Using ACE
Directly to Test whether a Coset Enumeration  Terminates".  The  usual
warnings  regarding  options  apply  (see  Section~"General   Warnings
regarding the Use of Options").

The return value is an integer (numbering from 1) which represents the
running process. (It is possible to have  more  than  one  interactive
process running at once.) The integer returned may be  used  to  index
which of these  processes  an  interactive  function  {\ACE}  function
should be applied to.

An interactive{\ACE} process is terminated with the command

\>ACEQuit( <i> ) F

where <i> is the integer returned by `ACEStart' when the  process  was
begun.

We discuss each of these commands, as well as the range  of  functions
which enable one to access  features  of  the  {\ACE}  standalone  not
available non-interactively, in depth in Chapter~"Functions for  Using
ACE interactively".

\enditems

*Note:*

{\ACE} not only allows one to do a coset enumeration of a group  by  a
given (and then fixed) subgroup but it also allows one to  search  for
subgroups by starting from a given one (possibly the trivial subgroup)
and then augmenting  it  by  adding  new  subgroup  generators  either
explicitly               via                `ACEAddSubgroupGenerators'
(see~"ACEAddSubgroupGenerators")   or   implicitly   by    introducing
*coincidences*\index{coincidences}     (see     `ACECosetCoincidence':
"ACECosetCoincidence",           or           `ACERandomCoincidences':
"ACERandomCoincidences");  or  one  can  find  smaller  subgroups   by
deleting   subgroup   generators   via   `ACEDeleteSubgroupGenerators'
(see~"ACEDeleteSubgroupGenerators").

%%%%%%%%%%%%%%%%%%%%%%%%%%%%%%%%%%%%%%%%%%%%%%%%%%%%%%%%%%%%%%%%%%%%%%
\Section{Accessing ACE Examples with ACEExample}

\beginitems

There are a number of examples available in the `examples'  directory,
which may be accessed via

\>ACEExample() F
\>ACEExample( <examplename> [:<options>] ) F
\>ACEExample( <examplename>, <ACEfunc> [:<options>] ) F

where  <examplename>  is  a  string,  the  name  of  an  example  (and
corresponding file in the `examples' directory); and <ACEfunc> is  the
{\ACE} function with which the example is to be executed. 

If `ACEExample' is called with no arguments,  or  with  the  argument:
`"index"', or with a non-existent example name, an index of  available
examples is displayed.

By default, examples are executed via  `ACECosetTableFromGensAndRels'.
However, if `ACEExample' is called with a second argument (choose from
the (other) alternatives: `ACEStats', or `ACEStart'), the  example  is
executed using that function, instead. Note that,  whereas  the  first
argument appears in double quotes (since it's a  string),  the  second
argument does not (since it's a function).

`ACEExample' also accepts user options, which  may  be  passed  either
globally  (i.e.~by  using  `PushOptions'  to  push   them   onto   the
`OptionsStack') or locally  behind  a  colon  after  the  `ACEExample'
arguments, and they are passed  to  `ACECosetTableFromGensAndRels'  or
<ACEfunc> as if they  were  *appended*  to  the  existing  options  of
<examplename>; in this way, the user may *over-ride* any or all of the
options  of  <examplename>.  This  is  done  by  passing   an   option
`aceexampleoptions' (see~"option aceexampleoptions"), which sets up  a
mechanism to reverse the usual order in which options  of  recursively
called functions  are  pushed  onto  the  `OptionsStack'.  The  option
`aceexampleoptions' is *not* a user option; it is  intended  only  for
*internal* use by `ACEExample', for the above purpose. In the  portion
of the output due to the `echo' option, if one has passed  options  to
`ACEExample', one will see `aceexampleoptions' listed  first  and  the
result  of  the  interaction  of  <examplename>'s  options   and   the
additional options.

Consider the example `"A5"'. The effect of running

\beginexample
gap> ACEExample("A5", ACEStats);
\endexample

is essentially equivalent to executing:

\beginexample
file := Filename(DirectoriesPackageLibrary("ace", "examples"), "A5");
ACEfunc := ACEStats;
ReadAsFunction(file)();
\endexample

except that  some  internal  \lq{}magic'  of  `ACEExample'  edits  the
example file and  displays  equivalent  commands  a  user  \lq{}would'
execute. If the user has passed options to `ACEExample'  these  appear
in a \lq{}`\# User Options'' block after the original options  of  the
example in the `Info' portion of the output.  By  comparing  with  the
portion of the output from the `echo'  option  (unless  the  user  has
over-ridden the `echo' option), the  user  may  directly  observe  any
over-riding effects of user-passed options.

Please see Section~"Fun with ACEExample" for some sample  interactions
with `ACEExample'.

*Notes*

Most examples use the `mess'  ($=  `messages'$)  option.  To  see  the
effect of this, it is recommended to do: `SetInfoACELevel(3);'  before
calling `ACEExample', with an example.

The coset table output from `ACEExample', when called with many of the
examples     and     with     the     default     {\ACE}      function
`ACECosetTableFromGensAndRels' is often quite long.  Recall  that  the
output may be suppressed by following the (`ACEExample') command  with
a double semicolon (`;;').

Also, try `SetInfoACELevel(2);' before  calling `ACEExample', with  an
example.

If you unexpectedly observe `aceexampleoptions' in your  output,  then
most probably you have unintentionally passed options  by  the  global
method, by having a non-empty `OptionsStack'. One possible  remedy  is
to use `FlushOptionsStack();' (see~"FlushOptionsStack"), before trying
your `ACEExample' call again.

As discussed in Section~"Interpretation  of  ACE  Options",  there  is
generally no sensible meaning that can  be  attributed  to  setting  a
strategy option (see Chapter~"Strategy Options for ACE")  to  `false';
if you wish to nullify the effect of a strategy option,  pass  another
strategy  option,  e.g.~pass  the  `default'  (see~"option   default")
strategy option.

\enditems

%%%%%%%%%%%%%%%%%%%%%%%%%%%%%%%%%%%%%%%%%%%%%%%%%%%%%%%%%%%%%%%%%%%%%%
\Section{General Warnings regarding the Use of Options}
Firstly, let us mention that an {\ACE} strategy is  merely  a  special
option of {\ACE} that sets  a  number  of  the  options  described  in
Chapter~"Options for ACE" all at once.

\beginitems

One can set options globally using  the  function  `PushOptions'  (see
Chapter~"ref:Options Stack" in the {\GAP} Reference  Manual);  options
pushed onto `OptionsStack' this way, remain there  until  an  explicit
`PopOptions()' call is made. However, the usual way to pass options is
behind a colon following a  function's  arguments  (see  "ref:Function
Calls" in the {\GAP} Reference Manual, or for an example refer to  the
function               `ACECosetTableFromGensAndRels'               in
Section~"ACECosetTableFromGensAndRels"); options passed this  way  are
local, and  disappear  from  `OptionsStack'  after  the  function  has
executed successfully. Beware, however the function will see any other
options on the `OptionsStack' (in particular,  any  pushed  there  via
`PushOptions', prior to the function call), unless over-ridden  by  an
option passed via the function. Also note that duplication  of  option
names for different programs may lead to misinterpretations.  You  can
use `DisplayOptionsStack'  (see  Chapter~"ref:DisplayOptionsStack"  in
the {\GAP} Reference Manual) to ensure that there is no  such  danger.
However, considering how much of a potential  mine-field  a  non-empty
`OptionsStack' might be for the unwary user, we provide

\>FlushOptionsStack()

which simply executes `PopOptions()' until `OptionsStack' is empty.

\enditems

It  is  important  to  realize  that  {\ACE}'s   options   (even   the
non-strategy options) are not orthogonal, i.e.\  the  order  in  which
they are put to {\ACE} can be important. For this reason, except for a
few options that have no effect on the course of an  enumeration,  the
order in which options are passed to the {\ACE} interface is preserved
when those same options are passed to the  {\ACE}  binary.  Also  note
that except for limitations  imposed  by  {\GAP}  e.g.\  clashes  with
{\GAP} keywords and blank spaces not allowed in keywords, the  options
of the {\ACE} interface are the  same  as  for  the  binary;  so,  for
example, the options can appear in upper or  lower  case  (or  indeed,
mixed case) and most may be abbreviated. Note that  any  options  that
the {\ACE} binary doesn't understand are simply ignored and a  warning
appears in the {\ACE} output. If this occurs, you may  wish  to  check
the input fed to {\ACE} and the output from {\ACE}, which when  {\ACE}
is run non-interactively are stored in files whose full path names are
recorded in the record fields `ACEData.infile' and  `ACEData.outfile',
respectively. Alternatively, both interactively and  non-interactively
one can set the `InfoLevel' of `InfoACE' to 3 (see~"SetInfoACELevel"),
to see the output from {\ACE}, or  to  4  to  also  see  the  commands
directed to {\ACE}.

%%%%%%%%%%%%%%%%%%%%%%%%%%%%%%%%%%%%%%%%%%%%%%%%%%%%%%%%%%%%%%%%%%%%%%
\Section{The ACEData Record}

Essential data for an {\ACE} session within {\GAP} is  stored  in  the
`ACEData'{\undoquotes\atindex{ACEData}{@`ACEData'   record}}   record,
whose fields are:

\beginitems

\quad`binary' & the path of the {\ACE} binary;

\quad`tmpdir' & the path of the  temporary  directory  containing  the
non-interactive {\ACE} input and output files;

\quad`io'     & list  of  data  records  for  `ACEStart'   (see  below 
and~"ACEStart") processes;

\quad`infile' & the full path of the  (non-interactive)  {\ACE}  input
file;

\quad`outfile'& the full path of the (non-interactive)  {\ACE}  output
file; and

\quad`version'& the version of the current {\ACE} binary.

\enditems

Each time an interactive {\ACE} process is  initiated  via  `ACEStart'
(see~"ACEStart"), an identifying number <ioIndex> is generated for the
interactive process and  a  record  `ACEData.io[<ioIndex>]'  with  the
following fields is created:

\beginitems

\quad`args'   & a record with fields: `fgens', `rels',  `sgens'  whose
values  are  the  corresponding   arguments   passed   originally   to
`ACEStart';

\quad`options'& the current options record of the interactive process;

\quad`acegens'& a list of strings representing the generators used  by
{\ACE} (if the names of the generators passed via the  first  argument
<fgens> of `ACEStart' were all lowercase alphabetic  characters,  then
`acegens' is the `String' equivalent of <fgens>,  i.e.~`acegens[1]   =
String(<fgens>[1])' etc.);

\quad`stream' & the IOStream opened for an interactive {\ACE}  process
initiated via `ACEStart'; and

\quad`enumResult' 
              & the enumeration result (string) without  the  trailing
newline, output from {\ACE};

\quad`stats'  & a record as output by the function `ACEStats'.

\enditems

%%%%%%%%%%%%%%%%%%%%%%%%%%%%%%%%%%%%%%%%%%%%%%%%%%%%%%%%%%%%%%%%%%%%%%
\Section{Setting the Verbosity of ACE via Info and InfoACE}

\beginitems

The output of the {\ACE} binary is, by default, not displayed. However
the user may choose to see some, or all, of this output. This is  done
via the `Info' mechanism  (see  Chapter~"ref:Info  Functions"  in  the
{\GAP} Reference Manual). For this purpose, there is  the  <InfoClass>
`InfoACE'. Each line of output from {\ACE} is directed to  a  call  to
`Info' and will be displayed for the user to see if the `InfoLevel' of
`InfoACE' is high enough. By default, the `InfoLevel' of `InfoACE'  is
1, and it is recommended that you leave it at this level,  or  higher.
Only messages which we think that the user will really want to see are
directed to `Info' at `InfoACE' level 1. To turn off  *all*  `InfoACE'
messaging, set the `InfoACE' level to 0  (see~"SetInfoACELevel").  For
convenience, we provide the function

\>InfoACELevel()

which returns the current `InfoLevel' of `InfoACE' (i.e.~it is  simply
a shorthand for `InfoLevel(InfoACE)'.

To set the `InfoLevel' of `InfoACE' we also provide

\>SetInfoACELevel( <level> )
\>SetInfoACELevel()

which for a non-negative integer  <level>,  sets  the  `InfoLevel'  of
`InfoACE'    to    <level>    (i.e.~it    is    a    shorthand     for
`SetInfoLevel(InfoACE, <level>)'),  or  with  no  arguments  sets  the
`InfoACE' level to the default value 1.  Currently,  information  from
{\ACE} is directed to `Info' at four `InfoACE' levels: 1, 2, 3 and  4.
The command

\begintt
gap> SetInfoACELevel(2);
\endtt

enables the display of the results line of an enumeration from {\ACE},
whereas

\begintt
gap> SetInfoACELevel(3);
\endtt

enables the display of all of  the  output  from  {\ACE},  except  the
{\ACE} banner (including the host machine information); in particular,
the progress messages, emitted by {\ACE} when  the  `messages'  option
(see~"option messages") is set to a non-zero value, will be  displayed
via `Info'. Finally,

\begintt
gap> SetInfoACELevel(4);
\endtt

enables the display of all the input  directed  to  {\ACE}  (behind  a
\lq{}`ToACE> '' prompt, so you can distinguish it from  other  output)
and, interactively, the {\ACE}  banner  (including  the  host  machine
information). The `InfoACE' level of 4 is really  for  gurus  who  are
familiar with the {\ACE} standalone.

\enditems

%%%%%%%%%%%%%%%%%%%%%%%%%%%%%%%%%%%%%%%%%%%%%%%%%%%%%%%%%%%%%%%%%%%%%%
\Section{Acknowledgement}

Large parts of this manual,  in  particular  the  description  of  the
options for running {\ACE}, are directly copied  from  the  respective
description in the manual~\cite{Ram99} for the standalone  version  of
{\ACE} by Colin Ramsay.  Most  of  the  examples,  in  the  `examples'
directory, and accessed  via  the  `ACEExample'  function  are  direct
translations  of  Colin  Ramsay's `test\#\#\#.in' files  in  the `src'
directory.

%%%%%%%%%%%%%%%%%%%%%%%%%%%%%%%%%%%%%%%%%%%%%%%%%%%%%%%%%%%%%%%%%%%%%%
\Section{Changes from earlier versions}

A function, some options and a data record variable have been changed
from older versions of the {\ACE} Share Package:

\beginlist

\item{--} Function   `CallACE'{\undoquotes\atindex{CallACE}{@`CallACE'
(deprecated: use  `ACECosetTableFromGensAndRels')}}  (deprecated:  use
`ACECosetTable' or `ACECosetTableFromGensAndRels').

\item{--} Option     `outfile'{\undoquotes\atindex{option     outfile}
{@option `outfile' (deprecated: use  `aceinfile')}}  (deprecated:  use
`aceinfile').

\item{--} Option    `messfile'{\undoquotes\atindex{option    messfile}
{@option `messfile' (deprecated: use `aceoutfile')}} (deprecated:  use
`aceoutfile').

\item{--} Data      record      `ACEinfo'{\undoquotes\atindex{ACEinfo}
{@`ACEinfo' (deprecated: old name for `ACEData' record)}} (deprecated:
old name for `ACEData' record). This change was made because there  is
now a new `InfoClass': `InfoACE' (and it was  too  confusing  to  have
both it and `ACEinfo'!)

\endlist

%%%%%%%%%%%%%%%%%%%%%%%%%%%%%%%%%%%%%%%%%%%%%%%%%%%%%%%%%%%%%%%%%%%%%%%%%
%%
%E
