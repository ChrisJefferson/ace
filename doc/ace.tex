%%%%%%%%%%%%%%%%%%%%%%%%%%%%%%%%%%%%%%%%%%%%%%%%%%%%%%%%%%%%%%%%%%%%%%%%%
%%
%W  ace.tex                ACE documentation             Alexander Hulpke
%W                                                      Joachim Neub"user
%%
%H  $Id$
%%
%Y  Copyright (C) 2000, School of Math & Comp. Sci., St Andrews, Scotland
%%

\def\ACE{{\sf ACE}}
%%%%%%%%%%%%%%%%%%%%%%%%%%%%%%%%%%%%%%%%%%%%%%%%%%%%%%%%%%%%%%%%%%%%%%%%%
\Chapter{The ACE Share Package}

The  'Adaptable Coset Enumerator' {\ACE}:

\begintt
ACE coset enumerator (C) 1995-1999 by George Havas and Colin Ramsay
    http://www.csee.uq.edu.au/~havas/ace3.tar.gz
\endtt

can  be called  from within  {\GAP} through  an interface,  written by
Alexander Hulpke, which is described in this manual.

The interface links to an external binary and therefore is only usable
under UNIX. It will not work  on Windows or the Macintosh. It requires
{\GAP}~4.

{\ACE} can be used through this interface in different ways:

%%%%%%%%%%%%%%%%%%%%%%%%%%%%%%%%%%%%%%%%%%%%%%%%%%%%%%%%%%%%%%%%%%%%%%%%%
\Section{Using ACE as a Default for Coset Enumerations}

If you  want to use the {\ACE}  coset enumerator as a  default for all
coset  enumerations  done  by   {\GAP}  (which  may  also  get  called
indirectly),  you can  achieve  this by  setting  the global  variable
`TCENUM' to `ACETCENUM'.

\begintt
gap> TCENUM:=ACETCENUM;;
\endtt

If this  is done without any  further action, the  default strategy of
the {\ACE} enumerator (see~"parameter default") will be used.  You may
however want to use the  {\ACE} enumerator globally with some other of
the strategies  or options that are described  in Chapter~"Options for
the ACE  Enumerator".  You  can set these  options globally  using the
function `PushOptions' (see  Chapter~"ref:Options Stack" in the {\GAP}
Reference  Manual).   Note however  the  explicit  warnings about  the
working  of this  mechanism in  this Chapter  that always  the topmost
record of an  option stack is used, which means  that later setting of
other options  may invalidate  your request and  that in  turn setting
your  options for {\ACE}  will invalidate  other options  set earlier.
Also note that duplication of  option names for different programs can
lead  to misinterpretations.  You  can use  'DisplayOptionsStack' (see
Chapter~"ref:DisplayOptionsStack") to make sure  that there is no such
danger.

*TO  GREG: 
HERE THE  APPROPRIATE WARNINGS ABOUT  INTERPLAY OF OPTIONS HAVE  TO BE
INSERTED THAT YOU MENTION IN YOUR LETTER OF FEB. 11. DETAILS ABOUT THE
INTERPLAY  OF DIFFERENT  OPTIONS CAN  BE LEFT  TO THE  CHAPTER  ON THE
OPTIONS, TO WHICH A POINTER SHOULD BE GIVEN*

You  can switch back  to the  coset enumerator  built into  the {\GAP}
library by assigning `TCENUM' to `GAPTCENUM'.


\begintt
gap> TCENUM:=GAPTCENUM;;
\endtt


%%%%%%%%%%%%%%%%%%%%%%%%%%%%%%%%%%%%%%%%%%%%%%%%%%%%%%%%%%%%%%%%%%%%%%%%%
\Section{Calling ACE Directly}

If on  the other hand you  do not want  to set up {\ACE}  globally for
your coset  enumerations, you may call the  {\ACE} interface directly,
which  will allow  you  to decide  yourself  in each  such call  which
options you  want to use  for running {\ACE}.   By such a call  a file
with  your input  data in  {\ACE} readable  format will  be  handed to
{\ACE} and you will get the answer back in some {\GAP} format. At that
moment however the {\ACE} job  is terminated, that is, you cannot send
any  further questions or  requests about  the result  of that  job to
{\ACE}, you  are not using  {\ACE} interactively.  For  an interactive
use of {\ACE} from {\GAP} see Section "Using ACE Interactively".

Calling the {\ACE} interface directly is done by

\>CallACE( <fgens>, <rels>, <sgens> :<options> )

Here <fgens> is  a list of free generators, <rels> a  list of words in
these generators  giving relators for a finitely  presented group, and
<sgens> the list  of subgroup generators, again expressed  as words in
the free generators. All these are given in the standard {\GAP} format
(See Chapter~"ref:Finitely  Presented Groups" of  the {\GAP} Reference
Manual).

Behind  the colon  any  selection  of the  options  available for  the
interface (see Chapter~"Options for the ACE Enumerator") can be given,
separated by kommas like record components.  These can be used e.g. to
preset limits of space and time to be used, to modify input and output
and to modify the enumeration procedure.

*TO GREG:  
PLEASE ADD  A PASSAGE  HERE THAT EXPLAINS  HOW OPTIONS ARE  PARSED AND
THAT  DEPENDENCIES OF  THE OPTIONS  MUST  BE CONSIDERED.  IT MIGHT  BE
ENOUGH TO  EXPLAIN THIS  IS PRINCIPLE HERE  WITH A STRONG  WARNING AND
LEAVE  THE   DETAILS  TO  THE   CHAPTER  IN  WHICH  THE   OPTIONS  ARE
DESCRIBED. THEN A POINTER TO THAT CHAPTER MUST BE GIVEN HERE.*

The  operation  calls the  {\ACE}  coset  enumerator  and returns  the
(standardized) coset table obtained.

If the coset enumeration did not finish in the preset limits it raises
an  error,   respectively  returns  `fail'  if   the  `silent'  option
(see~"parameter silent") is set.

The  example  given  below is  the  call  for  a presentation  of  the
Fibonacci  group F(2,7)  for  which  we shall  discuss  the impact  of
various options in Chapter~"Examples".

\begintt
gap> F:= FreeGroup( "a", "b", "c", "d", "e", "f", "g");;
gap> a:= F.1;; b:= F.2;; c:= F.3;; d:= F.4;; e:= F.5;; f:= F.6;; g:= F.7;;
gap> fgens:= [a, b, c, d, e, f, g,];;
gap> rels:= [
> a*b*c^-1,
> b*c*d^-1,
> c*d*e^-1,
> d*e*f^-1,
> e*f*g^-1,
> f*g*a^-1,
> g*a*b^-1];;
gap> CallACE(fgens, rels, [c]);;
\endtt

(The  variable  `ACETCENUM.CosetTableFromGensAndRels'  is assigned  to
`CallACE'.)

If you only want to  test, whether a coset enumeration terminates, you
don't want to  transfer the whole coset table  to {\GAP}. Instead, you
can call

\>ACEStats(<fgens>,<rels>,<sgens>:<options>)

which    calls    the   {\ACE}    binary    and    returns   a    list
`[<index>,<c>,<m>,<t>]'  where <index>  is the  index of  the subgroup
(and $0$, if the enumeration does not succeed), <c> gives you the time
used for the  enumeration, <m> is the maximum  number of 'alive' coset
numbers at any time in the enumeration , while <t> is the total number
of  coset numbers  that  were  defined in  the  enumeration. See  also
Section "Information about the Enumeration".

%%%%%%%%%%%%%%%%%%%%%%%%%%%%%%%%%%%%%%%%%%%%%%%%%%%%%%%%%%%%%%%%%%%%%%
\Section{Writing ACE Standalone Input Files}

If you want  to use ACE as  a standalone with its own  syntax, you can
write  a  {\GAP}   input  and  can  use  `CallACE'   with  the  option
`outfile:=<filename>' (see~"parameter  outfile").  This will  keep the
input file for the {\ACE}  standalone produced by the {\GAP} interface
under the  file name <filename> (and  just return `fail')  so that you
can perform  interactive work in the standalone.


%%%%%%%%%%%%%%%%%%%%%%%%%%%%%%%%%%%%%%%%%%%%%%%%%%%%%%%%%%%%%%%%%%%%%%
\Section{Calling ACE Interactively}




%The  {\ACE} standalone  provides further  features, in  particular 
%
%\beginlist

%\item{--} the possibility for  interactive work. This is presently not
%available  from {\GAP} through  the interface,  since {\GAP}  does not
%provide  (yet) bidirectional  streams.  It is  hoped  that these  will
%become  available in  the future  and when  this will  be the  case an
%update  of  the  interface  should  also provide  the  possibility  of
%interactive use of {\ACE} from {\GAP}.

%\item{--} the possibility to work with incomplete tables and to try to
%get information out of these.  This is not (yet) available from {\GAP}
%since {\GAP} does not have data structures for such incomplete tables.

%\endlist


%%%%%%%%%%%%%%%%%%%%%%%%%%%%%%%%%%%%%%%%%%%%%%%%%%%%%%%%%%%%%%%%%%%%%%

\Section{Acknowledgement}

Large parts of  this manual, in particular of  the description of the
options for  running {\ACE}, are  directly copied from  the respective
description in  the manual \cite{HR99a} for the  standalone version of
{\ACE} by George Havas and Colin Ramsay.



%%%%%%%%%%%%%%%%%%%%%%%%%%%%%%%%%%%%%%%%%%%%%%%%%%%%%%%%%%%%%%%%%%%%%%%%%
\Chapter{Installing and Loading the ACE Share Package}

%%%%%%%%%%%%%%%%%%%%%%%%%%%%%%%%%%%%%%%%%%%%%%%%%%%%%%%%%%%%%%%%%%%%%%%%%
\Section{Installing}

To  install, unpack  the  archive file  in  a directory  in the  `pkg'
hierarchy  of your  version  of  {\GAP}~4. (This  might  be the  `pkg'
directory of the {\GAP}~4 home  directory; it is however also possible
to keep an additional `pkg' directory in your private directories, see
section  "ref:Installing  Share Packages"  of  the {\GAP}~4  Reference
Manual for details  on how to do this.) Go to  the newly created `ACE'
directory and  call `./configure <path>'  where <path> is the  path to
the {\GAP} home  directory. So for example if  you install the package
in the main `pkg' directory call

\begintt
./configure ../..
\endtt

This  will fetch  the  architecture  type for  which  {\GAP} has  been
compiled last and create a `Makefile'. Now simply call

\begintt
make
\endtt

to compile the binary and to install it in the appropriate place.

Note that the  current version of the configuration  process only sets
up  directory paths.  If you  need a  different compiler  or different
compiler options, you need  to edit `src/Makefile.in' prior to calling
`make' yourself.

If you use this installation of {\GAP} on different hardware platforms
you will have to compile the binary for each platform separately. This
is  done  by calling  `configure'  and  `make'  for the  package  anew
immediately   after  compiling  {\GAP}   itself  for   the  respective
architecture.  If your version of  {\GAP} is already compiled (and has
last  been compiled  on  the same  architecture)  you do  not need  to
compile {\GAP} again, it is  sufficient to call the `configure' script
in the {\GAP} home directory.

The manual you are currently reading describes how to use the {\ACE}
share package; it can be found in the `doc' subdirectory of the package.

The subdirectory `acedoc' contains the file `acedoc.ps' which holds a
version of the user manual for the {\ACE} standalone;
it forms part of~\cite{HR99a}).
You should consult it if you are going to switch to the {\ACE}
standalone, e.g., in order to use interactive facilities.

The  `src' subdirectory  contains a  copy  of the  original source  of
{\ACE}.  (The  only modification  is  that  a  file `Makefile.in'  was
obtained from  the different `make.xyz' and  will be used  to create a
`Makefile'.)  You  can replace  the source by  a newer  version before
compiling.


%%%%%%%%%%%%%%%%%%%%%%%%%%%%%%%%%%%%%%%%%%%%%%%%%%%%%%%%%%%%%%%%%%%%%%%%%
\Section{Loading}

To use  the {\ACE} package you  have to request it  explicitly. This is
done by calling
\begintt
gap> RequirePackage( "ace" );
\endtt
See~"ref:RequirePackage" in the {\GAP} Reference Manual.

If {\GAP} cannot find a working binary, the call to `RequirePackage' will
fail.

If you want to load the {\ACE} package by default, you can put this
`RequirePackage' command into your `gaprc' file
(see~"ref:The .gaprc file" in the {\GAP} Reference Manual).




%%%%%%%%%%%%%%%%%%%%%%%%%%%%%%%%%%%%%%%%%%%%%%%%%%%%%%%%%%%%%%%%%%%%%%%%%
\Chapter{Some Basics}

Throughout this manual for the use of {\ACE} as a {\GAP} share package
we shall assume that the reader already knows the basic ideas of coset
enumeration,  as  can be  found  for  example in~\cite{Neu82}.   There
i.a. a simple proof is given for the fact that a coset enumeration for
a  subgroup  of  finite  index  in a  finitely  presented  group  must
eventually terminate with the correct result, provided the enumeration
process obeys  a simple  condition (Mendelsohn's theorem).  This basic
condition leaves  room for a  great variety of 'strategies'  for coset
enumeration, the  two 'classical'  ones are known  for a long  time as
'Felsch  strategy'  and  'HLT  strategy' (for  Haselgrove,  Leech  and
Trotter).  Extensive  experimental studies  on many strategies  can be
found in particular in \cite{CDHW73}, \cite{Hav91}, and \cite{HR99}.

A few basic points  should in particular be understood:

\beginlist

\item{--} 'Subgroup(generator)  and relator  tables' that are  used in
the description of coset enumeration  in \cite{Neu82}, and to which we
will  also occasionally  refer in  this manual,  are  *not* physically
existing in the implementation of  coset enumeration in {\ACE}.  For a
terminology that  is closer to  the actual implementation and  also to
the  formulations  in  the   manual  for  the  {\ACE}  standalone  see
\cite{CDHW73} and \cite{Hav91}.

\item{--} Coset enumeration proceeds  by defining 'coset numbers' that
really represent  possible representatives for cosets of  which at the
time of their generation it is  not guaranteed that any two of them do
indeed represent different cosets.

\item{--} The  definition of a  coset number may lead  to 'deductions'
from the 'closing  of rows in subgroup or  relator tables'. These are
kept in a 'deduction stack'.

\item{--}  Also it may  be found  that 'representatives'  described by
different 'coset  numbers' really lie in  the same coset  of the given
subgroup. This is  called a 'coincidence' and will  eventually lead to
the elimination  of the larger of  the two coset  numbers.  Until this
elimination  has been  performed pending  coincidences are  kept  in a
'queue of coincidences'.

\item{--} A definition that will actually close a row in a subgroup or
relator  table will  immediately yield  twice as  many entries  in the
coset  table  as  other  definitions.   Such  definitions  are  called
'preferred definitions', the  places in rows of a  subgroup or relator
table that they close are also  referred to as 'gaps of length one' or
minimal  gaps.  Such  gaps  can be  found  at little  extra cost  when
'relators  are traced  from a  given  coset number'.   {\ACE} keeps  a
selection of  them in a 'preferred  definition stack' for  use in some
definition strategies. (see~\cite{Hav91}).

\endlist

It will also be necessary to understand some further basic features of
the  implementation and  the corresponding  terminology which  we will
explain in the sequel.

%%%%%%%%%%%%%%%%%%%%%%%%%%%%%%%%%%%%%%%%%%%%%%%%%%%%%%%%%%%%%%%%%%%%%%%%%
\Section{Enumeration  Style}

The first main decision for any coset enumeration is in which sequence
definitions are  made. When a new  coset number has to  be defined, in
{\ACE} there are basically three possibilities to choose it:

\beginlist

\item{--}  One  may fill  the  next empty  entry  in  the coset  table
scanning  this  from the  left/top  of  the  coset table  towards  the
right/bottom -- that is, in order  row by row. This is called *C-style
definition* (for *C*oset Table Based definition) of coset numbers.  In
fact a procedure needs to follow a method like this to some extent for
the proofs that coset enumeration eventually terminates in the case of
finite index (see~\cite{Neu82}).

\item{--} In *R-style definition* (for *R*elator Based definition) the
order  in which  cosets are  defined is  explicitly prescribed  by the
order in which rows of (the subgroup generator tables and) the relator
tables are filled by making definitions.

\item{--}  One may  choose  definitions from  a *Preferred  Definition
Stack*.  In  this stack possibilities for definition  of coset numbers
are stored  that will close a  certain row of a  relator table.  Using
these  'preferred definitions'  is  sometimes also  referred  to as  a
'minimal gaps strategy'.  The idea of using these is of course that by
thus closing  a row  in a  relator table, one  will immediately  get a
consequence. We will come back  to the obvious question where one gets
this 'preferred definition stack' from.

\endlist

The *enumeration  style* is mainly  determined by the  balance between
C-style definitions  and R-style  definitions, which is  controlled by
the  values  of  the  `ct'  and `rt'  parameters  (see~"parameter  ct"
and~"parameter rt").

However this still leaves us with  plenty of freedom for the design of
definition  strategies,  freedom which  can  e.g.   be  used to  great
advantage  in  Felsch-type  strategies.   Though it  is  not  strictly
necessary, Felsch-type  programs generally start off  by ensuring that
each of the given subgroup generators  forms a cycle at coset 1 before
embarking  on   further  enumeration.  The  use  of   this  and  other
possibilities leads to the following table of *enumeration styles*.


% \begin{table}
% \hrule
% \caption{The styles}
% \label{tab:sty}
% \smallskip
% \renewcommand{\arraystretch}{0.875}
% \begin{tabular*}{\textwidth}{@{\extracolsep{\fill}}crrlc} 
% \hline\hline
% & \ttt{Rt} value & \ttt{Ct} value & style name & \\
% \hline
% & $<\!0$ & $<\!0$ & R/C & \\
% & $<\!0$ & $0$    & R*  & \\
% & $<\!0$ & $>\!0$ & Cr  & \\
% & $0$    & $<\!0$ & C   & \\
% & $0$    & $0$    & R/C (defaulted) & \\
% & $0$    & $>\!0$ & C  & \\
% & $>\!0$ & $<\!0$ & Rc & \\
% & $>\!0$ & $0$    & R  & \\
% & $>\!0$ & $>\!0$ & CR & \\
% \hline\hline
% \end{tabular*}
% \end{table}
\begintt
Rt value     Ct value     style name
---------------------------------------

   0           >0         C
  <0           >0         Cr
  >0           >0         CR

  >0            0         R
  <0            0         R*
  >0           <0         Rc
  <0           <0         R/C
   0            0         R/C (default)

---------------------------------------
\endtt

In *C style*  most definitions are made in the  next empty coset table
slot  and  are (in  principle)  tested  in  all essentially  different
positions in the relators; i.e., this is a Felsch-like mode.

However in C Style some  definitions may be made following a preferred
definition strategy, see the `pmode' and `psize' options below.

*Cr style* is like  C style except that a single R  style pass is done
after the initial C style pass.

In *CR style* alternate passes of C style and R style are performed.


In *R  style* all  the definitions are  made via relator  scans; i.e.,
this is an HLT-like mode.

*R\*  style* makes  definitions the  same as  R style,  but  tests all
definitions as for C style.

*Rc style* is like R style, except  that a single C style pass is done
after the initial R style pass.

In  *R/C  style* we  run  in  R style  until  an  overflow, perform  a
lookahead on the entire table, and then switch to CR style.

*Defaulted R/C  style* is the default  style, used if  you call {\ACE}
without specifying  options. In it we  use R/C style with  `ct' set to
1000 and `rt' set to  approximately $2000$ divided by the total length
of the relators  in an attempt to balance R and  C definitions when we
switch to CR style.

%%%%%%%%%%%%%%%%%%%%%%%%%%%%%%%%%%%%%%%%%%%%%%%%%%%%%%%%%%%%%%%%%%%%%%
\Section{Finding Deductions, Coincidences, and Preferred Definitions} 

*TO BE WRITTEN*

%%%%%%%%%%%%%%%%%%%%%%%%%%%%%%%%%%%%%%%%%%%%%%%%%%%%%%%%%%%%%%%%%%%%%%

\Chapter{Options for the ACE Enumerator}

{\ACE} offers a default strategy as well as several further predefined
strategies.   Such  strategies  can  be chosen  by  setting  'strategy
options'  (see Chapter~"Predefined Strategies"  for the  definition of
these startegies and  how they are chosen). However  you may define to
some  extent your  own  strategy or  modify  predefined strategies  by
combining the options described in this chapter.

{\ACE} offers  a wide  range of  options to direct  and guide  a coset
enumeration,  most of  which are  available from  {\GAP}  through this
interface.  In this chapter we describe those options and how they are
passed syntactically from {\GAP}.

Further options can also be used to specify technical parameters (such
as the  amount of  memory to  be used), to  specify in  detail certain
parameters  that  influence  the   strategy  (e.g.   the  sequence  of
definition  of coset  numbers)  or  the input  or  output format  (See
Chapter~"Options for the ACE Enumerator").

*Strategy Options are parsed by  the {\GAP} interface before any other
option,  it is therefore  possible to  modify parameters  on top  of a
strategy.*  Note that the  order in  which the  options are  passed to
{\ACE} is  implicit in the interface,  the order in  which the options
are given in {\GAP} is ignored!

*TO  GREG: 
THE PRECEDING PASSAGE IS CERTAINLY  NO LONGER TRUE AS YOU EXPLAINED IN
YOUR  LETTER OF FEB.   11, PLEASE  REPLACE THAT  PASSAGE BY  A CORRECT
STATEMENT HERE*

Options  are   passed  via  the   option  mechanism  of   {\GAP}  (see
chapter~"ref:Options Stack"  in the Reference Manual):  

\beginlist

\item{--}  Options  can either  be  set  globally  using the  function
`PushOptions', in particular  if you want to use  {\ACE} for all coset
enumerations in a {\GAP} run  (see Section~"Using ACE as a Default for
Coset Enumerations" for details), or


\item{--} Options can be appended to the argument list of any function
call,  separated by a  colon from  the argument  list.  They  are then
passed on recursively to any subsequent inner function call (but these
inner calls  could set options  by themself which then  might override
options set  before for the scope  of this inner  function call). This
possibility of passing options with the arguments of a function can in
particular  be used  if  {\ACE}  is called  directly  by the  function
`CallACE' (see Section~"Calling ACE directly").

\item{--} If {\ACE} is to be used interactively .....
*TO BE FILLED IN*

\endlist

Options  are  given  like  record  components,  separated  by  commas.
Options that take a parameter are  given as an assignement to the name
(such  as `time:=<val>').  As  a convenience  for boolean  options the
assignment can  be left out, defaulting  to a value of  `true'.  If a
boolean option is not given at  all, it defaults to `false'. 


So  for example to  use the  `hard' strategy  option and  increase the
workspace to $10^7$ words for the example given above, one would call:

\begintt
gap> CallACE(fgens,rels,[c]:hard,workspace:=10^7);;
\endtt

Except for  the `number' option, all  option names are the  same as in
the {\ACE} standalone.

%%%%%%%%%%%%%%%%%%%%%%%%%%%%%%%%%%%%%%%%%%%%%%%%%%%%%%%%%%%%%%%%%%%%%%%%%
\Section{An Option that Modifies the Input to the Enumeration}


\beginitems
\>`asis'{parameter asis}&
Do not reduce relators.
\enditems

By default, {\ACE} freely  and cyclically reduces the relators, freely
reduces  the  subgroup generators,  and  sorts  relators and  subgroup
generators in length-increasing  order.  If you do not  want this, you
can switch it off by setting the `asis' option.

*Notes:* As well as allowing you  to use the presentation *as* it *is*
given,  this  is  useful for  forcing  definitions  to  be made  in  a
prespecified  order,  by  introducing  dummy  (i.e.,  freely  trivial)
subgroup generators.   (Note that the  exact form of  the presentation
can  have a significant  impact on  the enumeration  statistics.)  For
some fine points of the influence of `asis' being set on the treatment
of involutory generators see the {\ACE} standalone manual.


%%%%%%%%%%%%%%%%%%%%%%%%%%%%%%%%%%%%%%%%%%%%%%%%%%%%%%%%%%%%%%%%%%%%%%%%%
\Section{Parameters that Modify the Enumeration Process}

\beginitems

\>`ct:=<val>'{parameter ct}&
Number of C-style definitions per pass.

\>`rt:=<val>'{parameter rt}&
Number of R-style definitions per pass.


The absolute value of these  parameters sets the number of definitions
C-style  or R-style,  respectively per  pass through  the enumerator's
main  loop.  The  sign of  these parameters  sets the  style,  and the
possible  combinations are given  in the  table of  enumeration styles
in Section~"Enumeration Style".


\>`number:=<val>'{parameter  number}&   The  'number  of   relators  in
subgroup' parameter (called `no' in the standalone manual).

It  is sometimes  helpful to  include relators  into the  list  of the
subgroup generators, in  the sense that they are  applied to coset \#1
at the  start of an enumeration.  A  value of 0 for  this option turns
this feature  off and  the (default) argument  of -1 includes  all the
relators.   A  positive  argument  includes the  specified  number  of
relators,  in order.   The `number'  option affects  only the  C style
procedures.


\>`mendelsohn'{parameter mendelsohn}&
Turns on mendelsohn processing.

Mendelsohn style processing  during relator scanning/closing is turned
on  by  giving  this option.   Off  is  the  default, and  here  coset
applications  are done  only  at the  start  (and end)  of a  relator.
Mendelsohn  'on'  means  that  coset  applications  are  done  at  all
(different)  cyclic permutations of  the relator.   

The effect  of Mendelsohn style  processing is case-specific.   It can
mean the difference  between success or failure, or  it can impact the
number  of  cosets   required,  or  it  can  have   no  effect  on  an
enumeration's statistics.

*Note:* Processing all cyclic permutations of the relators can be very
time-consuming,  especially if  the  presentation is  large.  So,  all
other things being equal, the  Mendelsohn flag should normally be left
off.

*Parameters Modifying C Style Definition*

The  next  three  options  are   relevant  only  for  making  C  style
definitions.  Making definitions in C style, that is filling the coset
table line  by line, it can  be very advantageous to  switch to making
definitions from the preferred definition stack.  Possible definitions
can  be  extracted  from  this  stack  in various  ways  and  the  two
parameters `pmode'  and `psize' (see~"parameter  pmode" and ~"parameter
psize"  respectively)  regulate this.  However  it  should be  clearly
understood that  making  all definitions  from  a preferred  definition
stack one may  violate the condition of Mendelsohn's  theorem, and the
parameter `fill' (see~"parameter fill") can be used to avoid this.

\>`fill:=<val>'{parameter fill}&
Controls the preferred definition strategy by setting the fill factor.

Unless prevented by  the fill factor, gaps of  length one found during
deduction   testing   are   preferentially  filled   (see~\cite{Hav91}).
However,  this potentially  violates the  formal requirement  that all
rows in the coset table  are eventually filled (and tested against the
relators).   The fill  factor is  used  to ensure  that some  constant
proportion of the coset table  is always kept filled.  Before defining
a coset  to fill a  gap of length  one, the enumerator  checks whether
`fill' times  the completed part  of the table  is at least  the total
size of  the table  and, if  not, fills coset  table rows  in standard
order (C style) instead of filling gaps.

An  argument of  0  selects  the default  value  of $\lfloor  5(n+2)/4
\rfloor$,  where $n$  is the  number of  columns in  the  table.  This
default  fill factor  allows  a moderate  amount  of gap-filling.   If
`fill' is  1, then there is  no gap-filling.  A large  value of `fill'
can cause  what is in effect  infinite looping (resolved  by the coset
enumeration failing).   However, in general,  a large value  does work
well.  The  effects of the various gap-filling  stategies vary widely.
It is  not clear  which values are  good general defaults  or, indeed,
whether any strategy is always ``not too bad''.


\>`pmode:=<val>'{parameter pmode}&
Parameter for preferred definitions.

The  value of  the  `pmode' option  determines  which definitions  are
preferred.  If  the argument is  0, then Felsch style  definitions are
made using  the next empty table  slot.  If the  argument is non-zero,
then gaps of length one found during relator scans in Felsch style are
preferentially  filled  (subject to  the  value  of  `fill').  If  the
argument  is 1,  they are  filled  immediately, and  if it  is 2,  the
consequent deduction  is also made  immediately (of course,  these are
also put on the deduction stack).  If the argument is 3, then the gaps
of length one are noted in the preferred definition queue.

Provided such a gap survives  (and no coincidence occurs, which causes
the queue to be discarded) the  next coset will be defined to fill the
oldest  gap of  length  one.  The  default  value is  either  0 or  3,
depending   on  the   strategy  selected   (see   Section  "Predefined
Strategies").  If you want to know more details, read the code.


\>`psize:=<val>'{parameter psize}&
Size of preferred definition queue.

The  preferred definition  queue is  implemented as  a  ring, dropping
earliest  entries.  Its  size *must*  be  $2^n$, for  some $n>0$.   An
argument  of 0 selects  the default  size of  $256$.  Each  queue slot
takes two words (i.e., 8 bytes), and the queue can store up to $2^n-1$
entries.

*Parameters for R style Definition*

\>`norow'{parameter norow}&
Set the `row filling' parameter to 0.

When making  HLT-style definitions, it is  normal to scan  each row of
the coset table after its coset  has been applied to all relators, and
make definitions  to fill  any holes  in that row  of the  coset table
encountered. This  will in particular guarantee that  the condition of
Mendelsohn's Theorem  will be fulfilled.   Failure to do so  can cause
even simple enumerations to overflow.   To turn this row filling off,
you can use `norow'.

\>`look:=<val>'{parameter look}&
Lookahead
  
Although HLT-style strategies  are fast, they are local,  in the sense
that  the  implications   of  any  definitions/deductions  made  while
applying cosets may not become  apparent until much later.  One way to
alleviate this problem is to perform lookaheads occasionally; that is,
to  test the  information  in  the table,  looking  for deductions  or
concidences.  {\ACE} can perform a lookahead when the table overflows,
before the compaction routine is  called.  Lookahead can be done using
the entire  table or  only that  part of the  table above  the current
coset, and it can be  done R-style (scanning cosets from the beginning
of relators)  or C-style (testing  all definitions in  all essentially
different positions).

The `look' option sets the value for lookahead, a value of 0 disables
lookahead.

\beginlist

\item{--} A value of 1 does a partial table, R-style lookahead; 
\item{--} 2 does all the table, C-style; 
\item{--} 3 does all the table, R-style; 
\item{--} 4 does a partial table, C-style.  

\endlist

The  default  is  either  0  or  1  depending  on  the  strategy;  see
Section~"Predefined Strategies".

*Further Parameters*

\>`dmode:=<val>'{parameter dmode}&
Deduction mode.

A completed table  is only valid if every table  entry has been tested
in all essentially different  positions in all relators.  This testing
can either be done directly  (Felsch strategy) or via relator scanning
(HLT strategy).  If it is  done directly, then more than one deduction
(i.e., table  entry) can be waiting  to be processed at  any one time.
So the untested deductions are stored in a stack.  Normally this stack
is fairly small but, during a collapse, it can become very large.

This  command allows  the user  to  specify how  deductions should  be
handled.  The value <val> has the following interpretations:

\beginlist

\item{--} <val>  = $0$:  
discard deductions if there is no stack space left; in addition purges
redundant cosets off the top of the stack at every coincidence.

\item{--} <val> = $1$: 
like <val> = $0$, but purge  redundant cosets off the top of the stack
at every coincidence,

\item{--} <val> = $2$: 
again as <val> = 0, but  purges all redundant cosets from the stack at
every coincidence.

\item{--} <val> = $3$:
 discard the entire stack if it overflows;

\item{--} <val> = $4$:
if  the stack  overflows, then  double the  stack size  and  purge all
redundant cosets from the stack.

\endlist

The  default deduction mode  is either  $0$ or  $4$, depending  on the
strategy  selected (see  Section~"Predefined Strategies"),  and  it is
recommended that you stay with the  default.  If you want to know more
details, read the code.

*Notes:*
If deductions are discarded for any reason, then a final relator check
phase  will be run  automatically at  the end  of the  enumeration, if
necessary, to check the result.

\>`dsize:=<val>'{parameter dsize}&
Deduction stack size.

Sets the  size of  the (initial) allocation  for the  deduction stack.
The size is  in terms of the number of  deductions, with one deduction
taking two words (i.e., 8 bytes).  The default size, of $1000$, can be
selected  by  a value  of  0.   See the  `dmod'  entry  for a  (brief)
discussion of deduction handling.


\enditems


%%%%%%%%%%%%%%%%%%%%%%%%%%%%%%%%%%%%%%%%%%%%%%%%%%%%%%%%%%%%%%%%%%%%%%%%%
\Section{Technical Options}

The following options do not affect how the coset enumeration is done,
but how it  uses the computer's resources. They  might thus affect the
runtime as  well as  the range of  problems that  can be tackled  on a
given machine.

\beginitems

\>`workspace:=<val>'{parameter workspace}&
Workspace size in words (default $10^6$).


By default, {\ACE} has a physical table size of $10^6$ words (i.e., $4
\times 10^6$ bytes in the  default 32-bit environment).  The number of
cosets  in the  table  is the  table  size divided  by  the number  of
columns.  Although the  number of cosets is limited  to $2^{31}-1$ (if
the C `int'  type is 32 bits), the table size  can exceed the $4$GByte
32-bit limit if a suitable machine is used.


\>`time:=<val>'{parameter time}&
Maximum execution time in seconds.

The `time' command  puts a time limit (in seconds) on  the length of a
run.  The  default is 0  which is no  time limit.  If the  argument is
$\ge0$ then the total elapsed time for this call is checked at the end
of each pass through the enumerator's main loop, and if it's more than
the limit the run is stopped and the current table returned.

Note that a limit of $0$ performs exactly one pass through the main
loop, since $0 \ge 0$.

\begintt
???????????????????????????????????????????????????????????????????
*GREG:

DOES THIS HAVE RELEVANCE TO THE INTERACTIVE MODE?*

If the enumerator is run in the continue mode, this allows a form of
  ``single-stepping''\kern-1.5pt.

??????????????????????????????????????????????????????????????????
\endtt

The time  limit is approximate, in  the sense that  the enumerator may
run for a longer, but never a shorter, time.  So, if there is, e.g., a
big collapse (so that the time round the loop becomes very long), then
the run may run over the limit by a large amount.

*Notes:*
The time  limit is  CPU-time, not wall-time.   As in all  timing under
Unix, the clock's granularity (usually  $10$ mSec) and the system load
can affect  the timing;  so the  number of main  loop iterations  in a
given time may vary.

\begintt
*GREG:

I WOULD  LIKE TO HAVE THE  `loop' PARAMETER BEEN MADE  AVAILABLE AS AN
OPTION  TOO. IT  SHOULD THEN  PROBABLY GO  INTO THIS  PLACE  AMONG THE
TECHNICAL OPTIONS. A FORMULASTION OF THE FUNCTION CALL HAS TO BE GIVEN
IN  LINE WITH  THE OTHERS  IN TH  INTERFACE. THE  TEXT FOR  IT  IN THE
STANDALONE MANUAL SAYS:*
\endtt


The core enumerator is organised as a state machine, with each step
performing an 'action' (i.e., lookahead, compaction) or a block of
actions (i.e., `ct' coset definitions, `rt' coset applications).
The number of passes through the main loop (i.e., steps) is counted, and
the enumerator can make an early return when this count hits the
value of `loop'.
A value of $0$, the default, turns this feature off.

*Guru Notes:*
You can do lots of really neat things using this feature, but you need
some understanding of the internals of {\ACE} to get real benefit from
it.


\>`path'{parameter path}&
Turns on path compression.

To  correctly  process  multiple  concidences, a  union-find  must  be
performed.  If both path compression and weighted union are used, then
this can be  done in essentially linear time  (see, e.g., \cite{CLR}).
Weighted union alone, in the  worst-case, is worse than linear, but is
subquadratic.  In  practice, path  compression is expensive,  since it
involves many coset table  accesses.  So, by default, path compression
is turned off; it can be turned on by `path'.  It has no effect on the
result, but may effect the running time and the internal statistics.

*Guru Notes:*
The whole question of the best way to handle large coincidence forests
is problematic.  Formally, {\ACE} does  not do a weighted union, since
it is constrained to replace the higher-numbered of a coincident pair.
In practice,  this seems  to amount to  much the same  thing!  Turning
path  compression on  cuts down  the  amount of  data movement  during
coincidence processing at the expense of having to trace the paths and
compress them.  In general, it does not seem to be worthwhile.

\>`com:=<val>'{parameter com}&
Percentage of dead cosets to trigger compaction.

The parameter `com'  controls compaction of the coset  table during an
enumeration.  It sets the percentage  of dead cosets needed to trigger
compaction.  The default is 10  or 100, depending on the strategy used
(see Section~"Predefined Strategies").

Compaction recovers the space allocated to cosets which are flagged as
dead  (i.e., which  were found  to be  coincident  with lower-numbered
cosets).   It results  in  a table  where  all the  active cosets  are
numbered contiguously  from \#1, and  with the remainder of  the table
available for new cosets.

The  coset table  is compacted  when a  coset definition  is required,
there is  no space for  a new coset  available, and provided  that the
given  percentage  of  the  coset  table contains  dead  cosets.   For
example, `com =  20' means compaction will occur only  if 20\% or more
of the cosets in the table  are dead.  The argument can be any integer
in  the range  $0--100$,  and the  default  value is  10  or 100;  see
Section~"Predefined  Strategies".   An  argument  of  100  means  that
compaction is  never performed,  while an argument  of 0  means always
compact, no matter how few dead cosets there are (provided there is at
least one, of course).

Compaction may be performed  multiple times during an enumeration, and
the table that results from an  enumeration may or may not be compact,
depending on whether or not there have been any coincidences since the
last compaction (or  from the start of the  enumeration, if there have
been no compactions).

*Notes:*
In some strategies  (e.g., `hlt') a lookahead phase  may be run before
compaction  is   attempted.   In  other   strategies  (e.g.,  `sims3')
compaction may  be performed  while there are  outstanding deductions;
since deductions  are discarded  during compaction, a  final lookahead
phase will (automatically) be performed.

Compacting a  table 'destroys' information  and history, in  the sense
that the  coincidence list is deleted,  and the table  entries for any
dead cosets are deleted.

??????????????????????????????????????????????????????????????????????
*GREG:
WILL THIS BE RELEVANT FOR INTERACTIVE USE?*

If  messaging is  enabled (see~"Information  About  the Enumeration"),
then a progress  message (labelled {\tt CO}) is  printed each time the
compaction routine is  actually called (as opposed to  each time it is
potentially called).
??????????????????????????????????????????????????????????????????????


\>`max:=<val>'{parameter max}&
Maximum number of cosets to be defined.

Sets the maximum  number of cosets to be defined.   By default, all of
the workspace is used, if  necessary, in building the coset table.  So
the table size is  an upper bound on how many cosets  can be active at
any one  time.  The `max'  option allows a  limit to be placed  on how
much of the physical table  space is made available to the enumerator.
Enough  space for  at least  two cosets  (i.e., the  subgroup  and one
other) must  be made available.  An  argument of 0 selects  all of the
workspace.

\enditems

%%%%%%%%%%%%%%%%%%%%%%%%%%%%%%%%%%%%%%%%%%%%%%%%%%%%%%%%%%%%%%%%%%%%%%%%%
\Chapter{Predefined Strategies}


The versatility  of {\ACE}  means that it  can be difficult  to select
appropriate  parameters when  presented with  a new  enumeration.  The
problem is compounded  by the fact that no  generally applicable rules
exist to  predict, given a presentation, which  parameter settings are
'good'.   To  help  overcome  this problem,  {\ACE}  contains  various
commands which select particular enumeration strategies.  One or other
of these strategies may work and, if not, the results may indicate how
the parameters can be varied  to obtain a successful enumeration.  

By default  {\ACE} uses  a strategy `default'  which assumes  that the
enumeration is easy, and if it turns out not to be so, {\ACE} switches
to  a strategy designed  for more  difficult enumerations.   The other
straightforward  options for  beginning users  are `easy'  and `hard'.
Thus, `easy' will quickly succeed or fail (in the context of the given
resources); `default' may  quickly succeed, but then try  `hard' for a
while;  `hard' will  run  more  slowly from  the  beginning trying  to
succeed.

The following boolean  options can set strategies (as  also defined in
the   {\ACE}  standalone   manual):   `default',  `easy',   `felsch0',
`felsch1', `hard', `hlt', `purec', `purer', `sims1', `sims3', `sims5',
`sims7', and `sims9'.

Strategy  options  are parsed  by  the  interface  *before* any  other
option,  it is therefore  possible to  modify parameters  on top  of a
strategy.

The thirteen standard strategies are listed below, their definition is
given  in the following  table. For  all of  them the  following three
parameters  are the  same;  `path'  = `false',  `psize'  = $256$,  and
`dsize' = $1000$. Note;  `mend' stands for the parameter `mendelsohn',
`row' =  `n' means that `norow' has  been set, `row' =  `y' means that
`norow' has not been set.

% \begin{table}
% \hrule
% \caption{The Predefined Strategies}
% \label{tab:pred}
% \smallskip
% \renewcommand{\arraystretch}{0.875}
% \begin{tabular*}{\textwidth}{@{\extracolsep{\fill}}lrrrrrrrrrrrrr} 
% \hline\hline
%           & \multicolumn{13}{c}{parameter} \\ 
% \cline{2-14}
% strategy & path & row & mend & number & look & com & ct   & rt    & fill &
% pmode & psize & dmode & dsize \\ 
% \hline
% default    & n   & y   & n    & -1 & n    & 10  & 0    & 0     & 0  & 3    & 256  & 4    & 1000 \\
% easy   & n   & y   & n    & 0  & n    & 100 & 0    & 1000  & 1  & 0    & 256  & 0    & 1000 \\
% felsch0  & n   & n   & n    & 0  & n    & 10  & 1000 & 0     & 1  & 0    & 256  & 4    & 1000 \\
% felsch1  & n   & n   & n    & -1 & n    & 10  & 1000 & 0     & 0  & 3    & 256  & 4    & 1000 \\
% hard   & n   & y   & n    & -1 & n    & 10  & 1000 & 1     & 0  & 3    & 256  & 4    & 1000 \\
% hlt    & n   & y   & n    & 0  & 1    & 10  & 0    & 1000  & 1  & 0    & 256  & 0    & 1000 \\
% purec & n   & n   & n    & 0  & n    & 100 & 1000 & 0     & 1  & 0    & 256  & 4    & 1000 \\
% purer & n   & n   & n    & 0  & n    & 100 & 0    & 1000  & 1  & 0    & 256  & 0    & 1000 \\
% sims1 & n   & y   & n    & 0  & n    & 10  & 0    & 1000  & 1  & 0    & 256  & 0    & 1000 \\
% sims3 & n   & y   & n    & 0  & n    & 10  & 0    & -1000 & 1  & 0    & 256  & 4    & 1000 \\
% sims5 & n   & y   & y    & 0  & n    & 10  & 0    & 1000  & 1  & 0    & 256  & 0    & 1000 \\
% sims7 & n   & y   & y    & 0  & n    & 10  & 0    & -1000 & 1  & 0    & 256  & 4    & 1000 \\
% sims9 & n   & n   & n    & 0  & n    & 10  & 1000 & 0     & 1  & 0    & 256  & 4    & 1000 \\
% \hline\hline
% \end{tabular*}
% \end{table}

\begintt
                                   parameter
          -------------------------------------------------------------
strategy  row  mend  number  look  com    ct     rt  fill  pmode  dmode
-----------------------------------------------------------------------
default     y     n      -1     n   10     0      0     0      3      4
easy        y     n       0     n  100     0   1000     1      0      0
felsch0     n     n       0     n   10  1000      0     1      0      4
felsch1     n     n      -1     n   10  1000      0     0      3      4
hard        y     n      -1     n   10  1000      1     0      3      4
hlt         y     n       0     y   10     0   1000     1      0      0
purec       n     n       0     n  100  1000      0     1      0      4
purer       n     n       0     n  100     0   1000     1      0      0
sims1       y     n       0     n   10     0   1000     1      0      0
sims3       y     n       0     n   10     0  -1000     1      0      4
sims5       y     y       0     n   10     0   1000     1      0      0
sims7       y     y       0     n   10     0  -1000     1      0      4
sims9       n     n       0     n   10  1000      0     1      0      4
-----------------------------------------------------------------------
\endtt

The  various  parameters  occurring  in  this table  (which  are  also
available  via {\GAP})  are  explained in  groups  according to  their
functionality  in  Chapter  "Options  for the  ACE  Enumerator".   The
parameter names are  listed in the top row and  the strategy names are
in the first column.

Note  that  we  explicitly   (re)set  all  of  the  listed  enumerator
parameters in all of the  predefined strategies, even although some of
them have  no effect. For example,  the `fill' value  is irrelevant in
HLT mode.   The idea  behind this  is that, if  you later  change some
parameters individually, then the enumeration retains the `flavour' of
the  last   selected  predefined  strategy.   

Note also  that other parameters  which may effect an  enumeration are
left  untouched  by setting  one  of  the  predefined strategies;  for
example,  the values  of  `max' (see~"max")  and `asis'  (see~"asis").
These parameters have  an effect which is independant  of the selected
strategy.

Note that, apart from the `felsch0' and `sims9' strategies, all of the
strategies are distinct, although some are very similar.

In detail, the strategies are as follows:

\beginitems

\>`default'{parameter default}&
This selects the default strategy, which is based on the defaulted R/C
style;  see Section~"Enumeration Mode  and Style".   The idea  here is
that  we assume  that the  enumeration is  'easy' and  start out  in R
style.  If it turns  out not to be easy, then we  regard it as 'hard',
and switch  to CR  style, after performing  a lookahead on  the entire
table.

\>`easy'{parameter easy}&
If this strategy  is selected, we run in R style  (i.e., HLT) and turn
lookahead  and compaction  off.  For  small and/or  easy enumerations,
this mode is likely to be the fastest.

\>`felsch0'{parameter felsch0}&
This selects a pure Felsch strategy.


\>`felsch1'{parameter felsch1}&
selects a Felsch strategy with  all relators in the subgroup and turns
gap-filling on.

\>`hard'{parameter hard}&  In many  'hard' enumerations, a  mixture of
R-style  and  C-style  definitions,  all  tested  in  all  essentially
different positions, is appropriate.  This option selects such a mixed
strategy.  The  idea here  is that  most of the  work is  done C-style
(with the relators  in the subgroup and with  gap-filling active), but
that every  $1000$ C-style definitions  a further coset is  applied to
all  relators.*Guru  Notes:*  The  $1000/1$  mix  is  not  necessarily
optimal, and some experimentation may  be needed to find an acceptable
balance (see,  for example, \cite{HR1}).   Note also that,  the longer
the total length  of the presentation, the more work  needs to be done
for each coset application to  the relators; one coset application can
result in more than  $1000$ definitions, reversing the balance between
R-style and C-style definitions.

\>`hlt'{parameter hlt}&
Selects the standard HLT strategy.

\>`purec'{parameter purec}&
Sets  the  strategy  to  basic  C-style  (coset  table  based)  --  no
compaction, no gap-filling, no relators in subgroup.

\>`purer'{parameter purer}&
Sets the strategy  to basic R-style (relator based)  -- no Mendelsohn,
no compaction, no lookahead, no row-filling.

\>`sims1'{parameter sims1}
\>`sims3'{parameter sims3}
\>`sims5'{parameter sims5}
\>`sims7'{parameter sims7}
\>`sims9'{parameter sims9}&
In  his  book~\cite{Sim},  Sims discusses  ten  standard  enumeration
strategies.  These  are effectively HLT  (with and without  the `mend'
parameter -see~"mend")  and Felsch, all  either with or  without table
standardisation as the enumeration proceeds.
{\ACE} does not implement table standardisation during an enumeration,
although  incomplete tables  can  be standardised  and an  enumeration
continued.
The  five non-standardising  strategies  are implemented,  and can  be
selected by these options.  The number is the same as given in Section
5.5~of~\cite{Sim}.   With  care,  it  is  possible  to  duplicate  the
statistics   given  in~\cite{Sim};  some   examples  are   given  in
Section~{ex001}.
% again, in the appendix ...

%%%%%%%%%%%%%%%%%%%%%%%%%%%%%%%%%%%%%%%%%%%%%%%%%%%%%%%%%%%%%%%%%%%%%%

\Chapter{Information and Experimentation}

%%%%%%%%%%%%%%%%%%%%%%%%%%%%%%%%%%%%%%%%%%%%%%%%%%%%%%%%%%%%%%%%%%%%%%

\Section{Information about the Enumeration}

\beginitems

\>`silent'{parameter silent}& If the  coset enumeration did not finish
within  the preset  limits, an  error is  raised by  the  interface to
{\GAP}, unless the option `silent'  has been set, in which case `fail'
is  returned.  Note that  this is  a feature  of the  {\GAP} interface
only, it has no counterpart in the {\ACE} standalone.

\>`outfile:=<filename>'{parameter outfile}&

If this  option is used,  {\GAP} will keep  the input file  for {\ACE}
produced by he {\GAP} interface under the filename <filename>, so that
one  could  easily  use  the  {\ACE}  standalone  for  work  with  the
respective presentation,  without having to  create an input  file for
the standalone anew.  Clearly this is a feature  of the interface only
that has no counterpart in the {\ACE} standalone.

\>`mess'{parameter mess}
\>`mess:=<val>'{parameter mess}&
If only `mess' is given, messages are turned on. It is also possible to
assigne the message level.

\>`messfile:=<filename>'{parameter messfile}&
Messages are printed in file <filename> (the default output is on the
screen).

\enditems

By default,  or if the argument `mess'  = 0, {\ACE} prints  out only a
single line of information giving  the result of each enumeration.  If
`mess' is non-zero then the presentation and the parameters are echoed
at the start  of the run, and messages on  the enumeration's status as
it progresses are also printed  out.  The absolute value of `val' sets
the frequency of  the progress messages, with a  negative sign turning
hole monitoring on.

The result line gives the result of the call to the enumerator and some
basic statistics.  The statistics given are, in order: 
 
\beginlist
\item{--}  `a':   number of active cosets; 
\item{--}  `r':   number of applied cosets;
\item{--}  `h':   first (potentially) incomplete row;
\item{--}  `n':   next coset definition number; 
\item{--}  `h':   (if `mess'$ \< 0$), percentage of holes in the table;
\item{--}  `l':   number of main loop passes;
\item{--}  `c':   total CPU time;
\item{--}  `m':   maximum number of active cosets;
\item{--}  `t':   total number of cosets defined.
\endlist

The possible results are given in the following Table; any result not
listed represents an internal error and should be reported.

\begintt
result               level     meaning 
----------------------------------------------------------------------------
INDEX = x                0     finite index of `x' obtained
OVERFLOW                 0     out of table space
SG PHASE OVERFLOW        0     out of space (processing subgroup generators)
ITERATION LIMIT          0     `loop' limit triggered
TIME LIMT                0     `ti' limit triggered
HOLE LIMIT               0     `ho' limit triggered
INCOMPLETE TABLE         0     all cosets applied, but table has holes
MEMORY PROBLEM           1     out of memory (building data structures)
----------------------------------------------------------------------------
\endtt

% \begin{table}
% \hrule
% \caption{Possible enumeration results}
% \label{tab:rslts}
% \smallskip
% \renewcommand{\arraystretch}{0.875}
% \begin{tabular*}{\textwidth}{@{\extracolsep{\fill}}lll} 
% \hline\hline
% result & level & meaning \\
% \hline
% \ttt{INDEX = x}         & 0 & finite index of \ttt{x} obtained \\
% \ttt{OVERFLOW}          & 0 & out of table space \\
% \ttt{SG PHASE OVERFLOW} & 0 & out of space (processing subgroup
% 				generators) \\
% \ttt{ITERATION LIMIT}   & 0 & \ttt{loop} limit triggered \\
% \ttt{TIME LIMT}         & 0 & \ttt{ti} limit triggered \\
% \ttt{HOLE LIMIT}        & 0 & \ttt{ho} limit triggered \\
% \ttt{INCOMPLETE TABLE}  & 0 & all cosets applied, but table has holes \\
% \ttt{MEMORY PROBLEM}    & 1 & out of memory (building data structures) \\
% \hline\hline
% \end{tabular*}
% \end{table}

The  progress message  lines consist  of  an initial  tag, some  fixed
statistics, and  some variable statistics.  The  possible message tags
are listed in the folowing table, along with their meanings.


\begintt
----------------------------------------------------------------------
message   action      meaning
----------------------------------------------------------------------
AD             y      coset 1 application definition (`SG'/`RS' phase)
RD             y      R-style definition
RF             y      row-filling definition
CG             y      immediate gap-filling definition
CC             y*     coincidence processed
DD             y*     deduction processed
CP             y      preferred list gap-filling definition
CD             y      C-style definition
Lx             n      lookahead performed (type `x')
CO             n      table compacted
CL             n      complete lookahead (table as deduction stack)
UH             n      updated completed-row counter
RA             n      remaining cosets applied to relators
SG             n      subgroup generator phase
RS             n      relators in subgroup phase
DS             n      stack overflowed (compacted and doubled)
----------------------------------------------------------------------
\endtt

% \begin{table}
% \hrule
% \caption{Possible progress messages}
% \label{tab:prog}
% \smallskip
% \renewcommand{\arraystretch}{0.875}
% \begin{tabular*}{\textwidth}{@{\extracolsep{\fill}}lll} 
% \hline\hline
% message & action & meaning \\
% \hline
% \ttt{AD} & y  & coset \#1 application definition 
% 			(\ttt{SG}/\ttt{RS} phase) \\
% \ttt{RD} & y  & R-style definition \\
% \ttt{RF} & y  & row-filling definition \\
% \ttt{CG} & y  & immediate gap-filling definition \\
% \ttt{CC} & y* & coincidence processed \\
% \ttt{DD} & y* & deduction processed \\
% \ttt{CP} & y  & preferred list gap-filling definition \\
% \ttt{CD} & y  & C-style definition \\
% \ttt{Lx} & n  & lookahead performed (type \ttt{x}) \\
% \ttt{CO} & n  & table compacted \\
% \ttt{CL} & n  & complete lookahead (table as deduction stack) \\
% \ttt{UH} & n  & updated completed-row counter \\
% \ttt{RA} & n  & remaining cosets applied to relators \\
% \ttt{SG} & n  & subgroup generator phase \\
% \ttt{RS} & n  & relators in subgroup phase \\
% \ttt{DS} & n  & stack overflowed (compacted and doubled) \\
% \hline\hline
% \end{tabular*}
% \end{table}


The tags indicate the function  just completed by the enumerator.  The
tags with a  `y' in the `action' column  represent functions which are
aggregated and counted.

Every time this count overflows the value of `mess', a message line is
printed and the  count is zeroed.  Those tags flagged  with a `y*' are
only present if the appropriate option has been included in the build.

Tags with an `n' in the `action' column are not counted, and cause a
message line to be output every time they occur.
They also cause the action count to be reset.

The fixed  portion of  the statistics consists  of the `a',  `r', `h',
`n', `h', `l' and `c' values,  as for the result line, except that `c'
is the time since the last message line.

The variable portion of the statistics can be:

\beginlist

\item{--}  the `m' and `t' values, as for the result line;
\item{--}  `d', the current size of the deduction stack;
\item{--}  `s', `d' and `c' (with `DS' tag), the new stack size,
the non-redundant deductions retained, and the redundant deductions
discarded.

\endlist

*Notes:*
Hole monitoring  is expensive, so don't  turn it on  unless you really
need  it.   If  you  wish  to  print  out  the  presentation  and  the
parameters, but  not the progress messages, then  set `mess' non-zero,
but very large.  (You'll still  get the `SG', `DS', etc. messages, but
not the `RD', `DD', etc. ones.)  You can set `mess' to $1$, to monitor
all enumerator actions,  but be warned that this  can yield very large
output files.




%%%%%%%%%%%%%%%%%%%%%%%%%%%%%%%%%%%%%%%%%%%%%%%%%%%%%%%%%%%%%%%%%%%%%%%%%
\Section{Experimentation Possibilities}

\beginitems

\>`aep:=<val>'{parameter aep}&
<val> is  an integer $0 \leq  <val> \leq 7$. Runs  the enumeration for
equivalent presentations.

\enditems

The `aep'  (all equivalent  presentations) option runs  an enumeration
for combinations  of relator ordering, relator  rotations, and relator
inversions.

The mandatory  argument <val> is  considered as a binary  number.  Its
three bits  are treated as  flags, and control relator  rotations (the
$2^0$ bit),  relator inversions (the $2^1$ bit)  and relator orderings
(the $2^2$  bit); $1$  means 'active' and  $0$ means  'inactive'. (See
below for an example).

The `aep' option  first performs a 'priming run'  using the parameters
as they  stand.  In particular,  the `asis' and `mess'  parameters are
honoured.

It then  turns `asis' on and  `mess' off, and generates  and tests the
requested  equivalent presentations.  The  maximum and  minimum values
attained by  `m' (the maximum number  of coset numbers  defines at any
stage)  and  `t' (the  total  number  of  coset numbers  defined)  are
tracked, and each  time a new 'record' is found  the relators used and
the summary result line is printed.

The  order in  which the  equivalent presentations  are  generated and
tested has no particular  significance, but note that the presentation
as  given *after*  the initial  priming  run) is  the *last*
presentation to be generated and  tested, so that the group's relators
are left `unchanged' by running the `aep' option.

Use  with  the  `mess' option  to  see  the  result. 

As discussed by Cannon, Dimino, Havas and Watson \cite{CDHW} and Havas
and Ramsay  \cite{HR1} such  equivalent presentations can  yield large
variations in  the number of  cosets required in an  enumeration.  For
this command, we are interested in this variation.

At the end, some additional status information is printed: 


\beginlist
\item{--}  the number of runs which yielded a finite index; 
\item{--}  the total number of runs (excluding the priming run); and 
\item{--}  the range of values observed for `m' and `t'.
\endlist


As an example (drawn from the discussion in \cite{HR1}) consider the index
  $448$ enumeration of $(8,7 \mid 2,3) / \langle a^2,Ab \rangle$,
  where $$ (8,7 \mid 2,3) 
    = \langle a,b \mid a^8 = b^7 = (ab)^2 = (Ab)^3 = 1 \rangle . $$
There are $4!=24$ relator orderings and $2^4=16$ combinations of relator or
inverted relator.
Exponents are taken into account when rotating relators, so the relators
given give rise to 1, 1, 2 and 2 rotations respectively, for a total
of $1.1.2.2=4$ combinations.
So the command `aep` := $7$ would generate and test $24.16.4=1536$ 
equivalent presentations, while `aep' := $3$ would generate and test 
$16.4=64$ equivalent presentations.

*Notes:*
There is  no way  to stop  the `aep' option  before it  has completed,
other than killing the task.  So  do a reality check beforehand on the
size of  the search space and  the time for each  enumeration.  If you
are  interested  in finding  a  `good'  enumeration,  it can  be  very
helpful, in terms of running time,  to put a tight limit on the number
of cosets via the `max' parameter.
You may also have to set `com' := $100$ to prevent time-wasting attempts
to recover space via compaction.
This maximises throughput by causing the `bad' enumerations, which are in
the majority, to overflow quickly and abort.
If you wish to explore a very large search-space, consider firing up many
copies of {\ACE}, and starting each with a `random' equivalent
presentation.
Alternatively, you could use the `rep' command.

\beginitems

\>`rep:=<val>'{parameter rep}&
Runs the enumeration for random equivalent presentations. See the `aep'
option.

\enditems

The  `rep' (random  equivalent presentations)  option  complements the
`aep'  option.    It  generates  and  tests   some  random  equivalent
presentations.  The mandatory argument acts  as for `aep'.  It is also
possible to set the number  of presentations used (a default of eight)
by using the extended syntax 'rep':=[<val>,<number>].

The routine first  turns `asis' on and `mess'  off, and then generates
and   tests  the   requested  equivalent   presentations.    For  each
presentation  the  relators  used  and  the summary  result  line  are
printed.

*Notes:*
The  relator inversions  and  rotations are  'genuinely' random.   The
relator permuting is  a little bit of a kludge,  with the 'quality' of
the  permutations tending  to improve  with  successive presentations.
When the `rep' command completes,  the presentation active is the {\it
last} one generated.

*Guru Note:*
It might  appear that neglecting to restore  the original presentation
is an error.  In fact, it is a useful feature!  Suppose that the space
of  equivalent presentations is  too large  to exhaustively  test.  As
noted in the  entry for `aep', we can start  up multiple copies of
`aep'  at  random points  in  the  search-space.  Manually  generating
`random'  equivalent  presentations  to  serve as  starting-points  is
tedious and error-prone.  The `rep' option provides a simple solution;
simply run `rep' := $7$ before `aep' := $7$!

%%%%%%%%%%%%%%%%%%%%%%%%%%%%%%%%%%%%%%%%%%%%%%%%%%%%%%%%%%%%%%%%%%%%%%%

\Chapter{Using ACE Interactively}

\>StartACE( <name>, <fgens>, <rels>, <sgens> [:<options>] ) F

`StartACE' takes a string <name>, a list <fgens> of free generators,
a list <rels> of relators and a list <sgens> of subgroup generators,
both in terms of the free generators, and optionally <options>
(see~"CallAce" for the format of the latter).

The return value is a {\GAP} object that represents the running {\ACE},
it is used as an argument in all functions described below.
<name> is used to print the return value in all subsequent outputs.

\>LeaveACE( <aceobj> ) F

`LeaveACE' takes an object <aceobj> as returned by `StartACE'
(see~"StartACE"), and terminates the corresponding {\ACE} job.
%T return an exit status if possible?

\begintt
*GREG:

THERE SHOULD PERHAPS BE THREE SECTIONS, 

ONE GIVING COMMANDS THAT INTERACTIVELY 'STEER' ACE

ONE WITH FUNCTIONS THAT ALLOW  TO SEND QUESTIONS ABOUT THE COSET TABLE
TO ACE.

ONE THAT ALLOWS OPERATIONS USING THE (POSSIBLY INCOMPLETE) COSET TABLE
TO GET FURTHER INFORMATION.

I HAVE  INSERTED INTO PLACEHOLDERS  FOR THESE SECTIONS  THE RESPECTIVE
STANDALONE COMMANDS WHICH  I WOULD LIKE TO HAVE  AVAILABLE THROUGH THE
INTERFACE*
\endtt


%%%%%%%%%%%%%%%%%%%%%%%%%%%%%%%%%%%%%%%%%%%%%%%%%%%%%%%%%%%%%%%%%%%%%
\Section{Steering ACE Interactively}

\>ACECheck( <aceobj> ) F
\>ACERedo( <aceobj> ) F

An extant  enumeration is redone,  using the current  parameters.  Any
existing information in the table  is retained, and the enumeration is
restarted from coset \#1 (i.e., the subgroup).

*Notes:*

This option is really intended  for the case where additional relators
and/or subgroup  generators have been introduced.   The current table,
which may be  incomplete or exhibit a finite  index, is still 'valid'.
However, the new data may  allow the enumeration to complete, or cause
a collapse to a smaller index.

\>ACEContinue( <aceobj> ) F

Continue the  current enumeration,  building upon the  existing table.
If a previous run stopped without producing a finite index you can, in
principle, change any  of the parameters and continue  on.  Of course,
if you make any changes  which invalidate the current table, you won't
be allowed to continue, although you may be allowed to redo.


\>ACERecover( <aceobj> ) F

Invokes the compaction routine on  the table to recover the space used
by the dead  cosetnumbers. A `CO' message line is  printed if any rows
of the coset table were recovered,  and a `co' line if none were. 

*GREG: DOES THE NEXT SENTENCE MAKE SENSE FOR THE INTERFACE?*

This routine is  called automatically if the `cy',  `nc', `pr',or `st'
options are involved.


\>ACEAddGenerators( <aceobj>, <wordlist> ) F

Adds the words in the list <wordlist> to any subgroup generators already
present.
The enumeration must be restarted or redone, it cannot be continued.

\>ACEAddRelators( <aceobj>, <relatorlist> ) F

Adds the words in the list <relatorlist> to any relators already present.
The enumeration must be restarted or redone, it cannot be continued.

\>ACECosetCoincidence( <aceobj>, <int> ) F

Print out the representative of coset <int>, and add it to the
subgroup generators; i.e., equates this coset with coset \#1, the
subgroup.


%%%%%%%%%%%%%%%%%%%%%%%%%%%%%%%%%%%%%%%%%%%%%%%%%%%%%%%%%%%%%%%%%%%%%%%%
\Section{Interactive Questions}

% \>`dump' := [0/1/2[,0/1]]&
% 
% Dumps  the internal  variables of  {\ACE}.  The  first  argument selects
% which of  the three levels  of {\ACE} to  dump, and defaults  to Level
% $0$.  The second argument selects the level of detail, and defaults to
% $0$  (i.e., less  detail).  This  option  is intended  for gurus;  the
% source code should be consulted to see what the output means.

\>ACEPrintOptions( <aceobj> ) F

This command prints details of the options included  in the version of
{\ACE} you're  running; i.e.,  what compiler flags  were set  when the
executable  was built.   A  typical output,  illustrating the  default
build, is:

\begintt
Executable built:
  Sat Feb 27 15:57:59 EST 1999
Level 0 options:
  statistics package = on
  coinc processing messages = on
  dedn processing messages = on
Level 1 options:
  workspace multipliers = decimal
Level 2 options:
  host info = on
\endtt


\>ACEPrintDetails( <aceobj>[, <params> ) F

This  command  prints out  details  of  the  current presentation  and
parameters.  No second argument, or second argument `false', prints out
the group
and subgroup name, the group's relators and the subgroup's generators.
If the second argument is `true', then  the current setting of the
parameters is
also printed.  The printout is the  same as that produced at the start
of a run when messaging is on.

% *Notes:*
% The output is printed out in a form suitable for input, so that a record
% of a previous run can be used to replicate the run.
% Note that, due to the defaulting of some parameters and the special
% meaning attached to some values, a little care has to be taken in
% interpreting the parameters.
%T mention outfile option here?

*GREG: DOES THE FOLLOWING PASSAGE MAKE SENSE WITH THE INTERFACE?*
If you wish to *exactly* duplicate a run, you should use the output
of `sr' *after* the run completes.


\>ACEPrintStatistics( <aceobj> ) F

If the statistics package is compiled into the code (which it is by
default),
then print the statistics accumulated during the most recent enumeration.


%%%%%%%%%%%%%%%%%%%%%%%%%%%%%%%%%%%%%%%%%%%%%%%%%%%%%%%%%%%%%%%%%%%%%%%%
\Section{Operations with a Coset table}

\>ACECycles( <aceobj> ) F

returns a list of permutations corresponding to the generators,
i.e., the permutation representation.
%T only for complete tables?

\>ACENormalClosure( <aceobj>[, <conjugates>] ) F

If the second argument is missing or `false', test for normal closure.
If a subgroup  generator  does not  normalise  a  generator,  then this  is
returned.
If  the  argument  is `true',  then  any  non-normalising
conjugates are also added to the subgroup generators.

\>ACEOrder( <aceobj>, <int> ) F

This function searches for cosets with order modulo the subgroup a multiple
of the absolute value of the integer <int>.
If <int> is negative then the list of all cosets meeting the requirement is
returned.
If <int> is positive then the first coset meeting the requirement is
returned if such a coset exists, and `fail' otherwise.

\>ACEOrders( <aceobj> ) F

returns the list of orders modulo the subgroup of all cosets.


\>ACEStabilisingCosets( <aceobj>, <int> ) F

This function returns stabilising cosets of the subgroup.

\beginlist
\item{--} If <int> $> 0$, the list of the first <int> stabilising cosets
    is returned.
\item{--} If <int> $\< 0$, the list of pairs of the first <int> stabilising
    cosets, plus their representatives, is returned.
\item{--} If <int> = $0$, the list of pairs of all of the stabilising cosets,
    plus their representatives, is returned.
\endlist


\>ACEStandardTable( <aceobj> ) F

This function compacts and then standardises the table.
That is, for a given ordering of the generators in the columns of the
table, it produces a canonic table.
In such a table, a row-major scan encounters previously unseen cosets in
numeric order.

*GREG, WHAT DOES THE LAST SENTENCE MEAN???*

*Notes:*
In a canonic table, the coset representatives are in length plus (column
order) lexicographic order, and each is the minimum in this order.
Two tables are equivalent only if their canonic forms are the same.

*Guru Notes:*
In half of the ten standard enumeration strategies of Sims \cite{Sim},
the   table   is   standardised   repeatedly.    This   is   expensive
computationally, but can result  in fewer cosets being necessary.  The
effect of  doing this  can be investigated  in {\ACE}  by (repeatedly)
halting the enumeration, standardising the table, and continuing.

\>ACETraceWord( <aceobj>, <int>, <word> ) F

Traces <word> through the coset table, starting at coset number <int>,
and returns the final coset number if the trace completes,
and `fail' otherwise.


\>ACERandomCoincidences( <aceobj>, <index>[, <attempts> ) F

This function attempts  to  find  nontrivial subgroups  with  index  a
multiple  of the first  argument <index> by  repeatedly putting  random
cosets
coincident with coset \#1 and seeing what happens.  The starting coset
table  must  be non-empty,  but  need  not  be complete.   The  second
argument <attempts> puts  a limit on  the number of  attempts,
with a  default of
eight.    For   each  attempt,   we   repeatedly   add  random   coset
representatives  to the  subgroup and  redo the  enumeration.   If the
table becomes too small, the attempt is aborted, the original subgroup
generators  restored,  and  another   attempt  made.   If  an  attempt
succeeds, then the new set of subgroup generators is retained.

*Guru  Notes:*  A  coset  can  have  many  different  representatives.
Consider running  `st' before  `rc', to canonicise  the table  and the
representatives.


%%%%%%%%%%%%%%%%%%%%%%%%%%%%%%%%%%%%%%%%%%%%%%%%%%%%%%%%%%%%%%%%%%%%%%%%%
\Chapter{Examples}

\begintt
*GREG:

HERE I WANT EVENTUALLY TO INSERT MORE EXAMPLES USING THE INTERFACE,
THIS IS JUST THE ONE ALEXANDER HAD IN HIS FIRST SHORT DRAFT*
\endtt

The following example calls `ACE' for up to 800 cosets using
Mendelsohn style relator processing and sets the message level to 500
\begintt
gap> g:=PerfectGroup(2^5*60,2);;
gap> f:=FreeGeneratorsOfFpGroup(g);;
gap> CallACE(f,RelatorsOfFpGroup(g),f{[1]}:mendelsohn,max:=800,mess:=500);;
ACE 3.000        Fri Aug 20 16:05:01 1999
=========================================
Host information:
  name = muir
  #-- ACE 3.000: Run Parameters ---
Group Name: G;
Group Generators: abcdefg;
Group Relators: (c)^2, (d)^2, (e)^2, (f)^2, (g)^2, aag, (b)^3, (cd)^2, 
  (ef)^2, (ce)^2, (cf)^2, (de)^2, (df)^2, Acae, Adaf, Aeac, Afad, Bfbe, 
  gAga, gBgb, (gc)^2, (gd)^2, (ge)^2, (gf)^2, Bebfe, Bcbgfd, Bdbfedc, 
  (ab)^5;
Subgroup Name: H;
Subgroup Generators: a;
Wo:1000000; Max:500; Mess:500; Ti:-1; Ho:-1; Loop:0;
As:0; Path:0; Row:1; Mend:1; No:28; Look:0; Com:10;
C:0; R:0; Fi:13; PMod:3; PSiz:256; DMod:4; DSiz:1000;
  #--------------------------------
SG: a=1 r=1 h=1 n=2; l=1 c=+0.00; m=1 t=1
RD: a=321 r=68 h=1 n=412; l=5 c=+0.00; m=327 t=411
CL: a=302 r=107 h=1 n=501; l=7 c=+0.01; m=327 t=500
DD: a=302 r=107 h=1 n=501; l=8 c=+0.00; d=386
CO: a=302 r=83 h=139 n=303; l=9 c=+0.01; m=327 t=500
DD: a=379 r=101 h=237 n=388; l=11 c=+0.00; d=3
DD: a=456 r=101 h=266 n=465; l=11 c=+0.00; d=5
INDEX = 480 (a=480 r=101 h=489 n=489; l=12 c=0.02; m=480 t=686)
\endtt

If `ACE' is made the standard coset enumerator the same method of passing
arguments may be used with all other commands and will affect coset
enumerations. As an example we use the `ACE' enumerator to compute the
permutation representation of a perfect group from the data library:

\begintt
gap> TCENUM:=ACETCENUM;;
gap> PerfectGroup(IsPermGroup,16*60,1:max:=50,mess);
gap> PerfectGroup(IsPermGroup,16*60,1:max:=50,mess);
ACE 3.000        Fri Aug 20 16:06:16 1999
=========================================
Host information:
  name = muir
INDEX = 16 (a=16 r=36 h=1 n=36; l=3 c=0.00; m=30 t=35)
A5 2^4
\endtt

(If you run the examples time, machine name or the version number of the
binary might differ.)


%%%%%%%%%%%%%%%%%%%%%%%%%%%%%%%%%%%%%%%%%%%%%%%%%%%%%%%%%%%%%%%%%%%%%%%%%
%%
%E

