%%%%%%%%%%%%%%%%%%%%%%%%%%%%%%%%%%%%%%%%%%%%%%%%%%%%%%%%%%%%%%%%%%%%%%%%%
%%
%W  ace.tex         ACE documentation introduction       Alexander Hulpke
%W                                                      Joachim Neub"user
%W                                                            Greg Gamble
%%
%H  $Id$
%%
%Y  Copyright (C) 2000  Centre for Discrete Mathematics and Computing
%Y                      Department of Computer Science & Electrical Eng.
%Y                      University of Queensland, Australia.
%%

%%%%%%%%%%%%%%%%%%%%%%%%%%%%%%%%%%%%%%%%%%%%%%%%%%%%%%%%%%%%%%%%%%%%%%%%%
\Chapter{The ACE Package}

\index{ACE}
The ``Advanced Coset Enumerator'' {\ACE}:

\begintt
ACE coset enumerator (C) 1995-2001 by George Havas and Colin Ramsay
\endtt
\URL{http://www.itee.uq.edu.au/~cram/ce.html}

can  be called  from within  {\GAP} through  an interface,  written by
Alexander Hulpke and Greg Gamble, which is described in this manual.

The interface links to an external binary and therefore is only usable
under UNIX (see  Section~"Installing  the  ACE  Package"  for  how  to
install  {\ACE}).  It  will  not  work  on  Windows  or  a  pre-MacOSX
Macintosh. The current version  requires  at  least  {\GAP}~4.3.  (The
first versions supported {\GAP}~4.2.) {\ACE} can be used through  this
interface in a variety of ways:

\beginlist%unordered

\item{--} one may supplant the  usual  {\GAP}  coset  enumerator  (see
Section~"Using ACE as a Default for Coset Enumerations"),

\item{--}  one  may  generate  a  coset  table  using  {\ACE}  without
redefining the usual {\GAP} coset enumerator (see  Section~"Using  ACE
Directly to Generate a Coset Table"),

\item{--} one  may  simply  test  whether  a  coset  enumeration  will
terminate (see Section~"Using ACE Directly to  Test  whether  a  Coset
Enumeration Terminates"),

\item{--} one may  use  {\GAP}  to  write  a  script  for  the  {\ACE}
standalone  (see  Section~"Writing  ACE  Standalone  Input  Files   to
Generate a Coset Table"), and

\item{--} one may interact with  the  {\ACE}  standalone  from  within
{\GAP} (see Section~"Using ACE Interactively").  Among  other  things,
the   interactive   {\ACE}   interface   functions    (described    in
Chapter~"Functions for Using ACE Interactively") enable  the  user  to
search for subgroups of a group (see the note  of  Section~"Using  ACE
Interactively").

\endlist

Each of these ways gives the user access to a welter of options, which
are discussed in full  in  Chapters~"Options  for  ACE"  and~"Strategy
Options for ACE". Yet more options are provided in Appendix~"Other ACE
Options",  but  please  take  note  of  the  Appendix's   introductory
paragraph. Check out Appendix~"Examples" for numerous examples of  the
{\ACE} commands.

*Note*: Some care needs to be taken with  options;  be  sure  to  read
Section~"General Warnings  regarding  the  Use  of  Options"  and  the
introductory sections of Chapter~"Options for ACE" for  some  warnings
regarding them and a general discussion of their use, before using any
of the functions provided by this interface to the {\ACE} binary.

%%%%%%%%%%%%%%%%%%%%%%%%%%%%%%%%%%%%%%%%%%%%%%%%%%%%%%%%%%%%%%%%%%%%%%
\Section{Using ACE as a Default for Coset Enumerations}

After loading {\ACE} (see Section~"Loading the ACE Package"),  if  you
want to use the {\ACE} coset enumerator as a  default  for  all  coset
enumerations done by {\GAP} (which may also  get  called  indirectly),
you can achieve this  by  setting  the  global  variable  `TCENUM'  to
`ACETCENUM'.

\beginexample
gap> TCENUM:=ACETCENUM;;
\endexample

This    sets    the    function    `CosetTableFromGensAndRels'    (see
Section~"ref:Coset  Tables  and  Coset  Enumeration"  in  the   {\GAP}
Reference Manual) to be  the  function  `ACECosetTableFromGensAndRels'
(described in Section~"Using ACE Directly to Generate a Coset Table"),
which then can be called with all the options defined for  the  {\ACE}
interface, not just the options `max' and `silent'. If `TCENUM' is set
to `ACETCENUM' without any  further  action,  the  `default'  strategy
(option) of the {\ACE} enumerator will be used (see  Chapter~"Strategy
Options for ACE").

You can switch back to the coset  enumerator  built  into  the  {\GAP}
library by assigning `TCENUM' to `GAPTCENUM'.

\beginexample
gap> TCENUM:=GAPTCENUM;;
\endexample

%%%%%%%%%%%%%%%%%%%%%%%%%%%%%%%%%%%%%%%%%%%%%%%%%%%%%%%%%%%%%%%%%%%%%%
\Section{Using ACE Directly to Generate a Coset Table}

If, on the other hand you do not want to set up  {\ACE}  globally  for
your coset enumerations, you may call the {\ACE}  interface  directly,
which will allow you to decide for yourself, for each such call, which
options you want to use for running {\ACE}. Please note  the  warnings
regarding options in Section~"General Warnings regarding  the  Use  of
Options". The functions discussed in this and  the  following  section
(`ACECosetTableFromGensAndRels' and `ACEStats')  are  non-interactive,
i.e.~by their use, a file with your  input  data  in  {\ACE}  readable
format will be handed to {\ACE} and you will get the  answer  back  in
{\GAP} format. At that moment however the {\ACE}  job  is  terminated,
that is, you cannot send any further questions or requests  about  the
result of that job to {\ACE}. For an interactive use  of  {\ACE}  from
{\GAP} see Section~"Using ACE  Interactively"  and  Chapter~"Functions
for Using ACE Interactively".

Using the {\ACE} interface directly to generate a coset table is  done
by either of

\atindex{ACECosetTable}{@\noexpand`ACECosetTable'!non-interactive}
\>ACECosetTableFromGensAndRels( <fgens>, <rels>, <sgens> [: <options>] ) F
\>ACECosetTable( <fgens>, <rels>, <sgens> [: <options>] ) F

Here <fgens> is a list of free generators, <rels> a list of  words  in
these generators giving relators for a finitely presented  group,  and
<sgens> the list of subgroup generators, again expressed as  words  in
the free generators. All these are given in the standard {\GAP} format
(see Chapter~"ref:Finitely Presented Groups" of the  {\GAP}  Reference
Manual). Note that the 3-argument form  of  `ACECosetTable'  described
here is merely a synonym for `ACECosetTableFromGensAndRels', and  that
`ACECosetTable' may be called in a different  way  in  an  interactive
{\ACE}    session    (see    Sections~"Using    ACE     Interactively"
and~"ACECosetTable!interactive").

Behind the colon any  selection  of  the  options  available  for  the
interface (see Chapters~"Options for ACE"  and~"Strategy  Options  for
ACE") can be given, separated by commas like record components.  These
can be used e.g.~to preset limits of space and time  to  be  used,  to
modify input and output and to modify the enumeration procedure.  Note
that strategies are simply special options that set a  number  of  the
options, detailed in Chapter~"Options for ACE", all at once.

Please see Section~"General Warnings regarding the Use of Options" for
a discussion regarding global and local passing of  options,  and  the
non-orthogonal nature of {\ACE}'s options.

Each of `ACECosetTableFromGensAndRels' and `ACECosetTable'  calls  the
{\ACE} binary and, if successful, returns a standard coset table, as a
{\GAP} list of  lists.  At  the  time  of  writing,  two  coset  table
standardisations schemes were possible: `lenlex' and `semilenlex' (see
Section~"Coset Table Standardisation Schemes"). The user  may  control
which standardisation scheme is used by selecting either the  `lenlex'
(see~"option  lenlex")  or  `semilenlex'   (see~"option   semilenlex")
option; otherwise the table  is  standardised  according  to  {\GAP}'s
default standardisation scheme which is `semilenlex'  for  {\GAP}~4.2,
or the value of `CosetTableStandard' (which by  default  is  `lenlex')
for  {\GAP}~4.3  (or  later).  We  provide   `IsACEStandardCosetTable'
(see~"IsACEStandardCosetTable") to determine whether a table (list  of
lists)  is  standard  relative  to  {\GAP}'s  default  standardisation
scheme, or with the use of options (e.g.~`lenlex' or `semilenlex')  to
another standardisation scheme.

If the determination of a coset table is unsuccessful, then one of the
following occurs:

\beginlist%unordered

\item{--} with the `incomplete' option  (see~"option  incomplete")  an
incomplete coset table is returned (as a list of lists), with zeros in
positions where valid coset numbers could not be determined; or

\item{--} with the `silent' option (see~"option  silent"),  `fail'  is
returned; or

\item{--} a  `break'-loop  is  entered.  This  last   possibility   is
discussed in detail via the example that follows.

\endlist

The example given  below  is  the  call  for  a  presentation  of  the
Fibonacci group F(2,7) for  which  we  shall  discuss  the  impact  of
various options in Section~"Fun with ACEExample". Observe that in  the
example, no options are passed,  which  means  that  {\ACE}  uses  the
`default' strategy (see Chapter~"Strategy Options for ACE").

\beginexample
gap> F:= FreeGroup( "a", "b", "c", "d", "e", "x", "y");;
gap> a:= F.1;; b:= F.2;; c:= F.3;; d:= F.4;; e:= F.5;; x:= F.6;; y:= F.7;;
gap> fgens:= [a, b, c, d, e, x, y,];;
gap> rels:= [ a*b*c^-1, b*c*d^-1, c*d*e^-1, d*e*x^-1,
>             e*x*y^-1, x*y*a^-1, y*a*b^-1];;
gap> ACECosetTable(fgens, rels, [c]);;
\endexample

\atindex{break-loop}{@\noexpand`break'-loop}
In computing  the  coset  table,  `ACECosetTableFromGensAndRels'  must
first do a coset enumeration (which is where {\ACE} comes in!). If the
coset enumeration does not finish in the preset limits a  `break'-loop
is entered,  unless  the  `incomplete'  (see~"option  incomplete")  or
`silent' (see~"option silent") options is set. In  the  event  that  a
`break'-loop is entered, don't despair or be frightened  by  the  word
`Error'; by tweaking {\ACE}'s options via the `SetACEOptions' function
that becomes available in the `break'-loop and then  typing  `return;'
it may be possible to help {\ACE} complete the coset enumeration  (and
hence successfully compute the coset table); if not, you will  end  up
in the `break'-loop again, and you can have another go (or `quit;'  if
you've had enough). The  `SetACEOptions'  function  is  a  no-argument
function; it's there *purely* to pass *options* (which, of course, are
listed behind a colon (`:')  with  record  components  syntax).  Let's
continue the Fibonacci example above, redoing  the  last  command  but
with the option `max := 2'  (see~"option  max"),  so  that  the  coset
enumeration has only two coset numbers to play with and hence is bound
to fail to complete, putting us in a `break'-loop.

\atindex{ACECosetTable}{@\noexpand`ACECosetTable'!\noexpand`break'-loop example}
\beginexample
gap> ACECosetTable(fgens, rels, [c] : max := 2);
Error, no coset table ...
 the `ACE' coset enumeration failed with the result:
 OVERFLOW (a=2 r=1 h=1 n=3; l=5 c=0.00; m=2 t=2)
 called from
<function>( <arguments> ) called from read-eval-loop
Entering break read-eval-print loop ...
 try relaxing any restrictive options
 e.g. try the `hard' strategy or increasing `workspace'
 type: '?strategy options' for info on strategies
 type: '?options for ACE' for info on options
 type: 'DisplayACEOptions();' to see current ACE options;
 type: 'SetACEOptions(:<option1> := <value1>, ...);'
 to set <option1> to <value1> etc.
 (i.e. pass options after the ':' in the usual way)
 ... and then, type: 'return;' to continue.
 Otherwise, type: 'quit;' to quit to outer loop.
brk> SetACEOptions(: max := 0);
brk> return;
[ [ 1 ], [ 1 ], [ 1 ], [ 1 ], [ 1 ], [ 1 ], [ 1 ], [ 1 ], [ 1 ], [ 1 ], 
  [ 1 ], [ 1 ], [ 1 ], [ 1 ] ]
\endexample

Observe how the lines  after  the  ```Entering  break  read-eval-print
loop''' announcement tell you *exactly* what to  do  (for  {\GAP}  4.2
these  lines  are  instead   `Info'-ed   *before*   the   `break'-loop
announcement). At  the  `break'-loop  prompt  `brk>'  we  relaxed  all
restrictions on `max' (by re-setting it to 0) and typed  `return;'  to
leave the `break'-loop. The coset  enumeration  was  then  successful,
allowing the computation of what turned out  to  be  a  trivial  coset
table. Despite the fact that the eventual coset  table  only  has  one
line (i.e.~there is exactly one coset number)  {\ACE}  *did*  need  to
define more than 2 coset numbers. To  find  out  just  how  many  were
required before the final  collapse,  let's  set  the  `InfoLevel'  of
`InfoACE' (see~"SetInfoACELevel") to 2, so that the {\ACE} enumeration
result is printed.

\beginexample
gap> SetInfoACELevel(2);
gap> ACECosetTable(fgens, rels, [c]);
#I  INDEX = 1 (a=1 r=2 h=2 n=2; l=6 c=0.01; m=2049 t=3127)
[ [ 1 ], [ 1 ], [ 1 ], [ 1 ], [ 1 ], [ 1 ], [ 1 ], [ 1 ], [ 1 ], [ 1 ], 
  [ 1 ], [ 1 ], [ 1 ], [ 1 ] ]
\endexample

The enumeration result line is the `Info' line  beginning  ```\#I '''.
Appendix~"The Meanings of  ACE's  Output  Messages"  explains  how  to
interpret such output messages from {\ACE}. In particular, it explains
that `t=3127' tells us that a `t'otal number  of  3127  coset  numbers
needed to be defined before the final  collapse  to  1  coset  number.
Using some of the many options that {\ACE} provides, one  may  achieve
this  result  more  efficiently,  e.g.~with   the   `purec'   strategy
(see~"option purec"):

\beginexample
gap> ACECosetTable(fgens, rels, [c] : purec);
#I  INDEX = 1 (a=1 r=2 h=2 n=2; l=4 c=0.00; m=332 t=332)
[ [ 1 ], [ 1 ], [ 1 ], [ 1 ], [ 1 ], [ 1 ], [ 1 ], [ 1 ], [ 1 ], [ 1 ], 
  [ 1 ], [ 1 ], [ 1 ], [ 1 ] ]
\endexample

{\ACE} needs to define a `t'otal number of only  (relatively-speaking)
332 coset numbers before the final collapse to 1 coset number.

*Notes:* 
To initiate the coset enumeration,  the  `start'  option  (see~"option
start") is quietly inserted after the user's supplied options,  unless
the user herself supplies one  of  the  enumeration-invoking  options,
which are: `start', or one of its synonyms, `aep'  (see~"option  aep")
or `rep' (see~"option rep").

When  a  user  calls  `ACECosetTable'   with   the   `lenlex'   option
(see~"option  lenlex"),  occasionally  it  is  necessary  to   enforce
`asis'${}=1$ (see~"option asis"), which may be counter to the  desires
of the  user.  The  occasions  where  this  action  is  necessary  are
precisely those for which, for the  arguments  <gens>  and  <rels>  of
`ACECosetTable',   `IsACEGeneratorsInPreferredOrder'   would    return
`false'.

The non-interactive `ACECosetTable' and `ACECosetTableFromGensAndRels'
now use an iostream to communicate with the {\ACE} binary in order  to
avoid filling up a temporary directory with an incomplete coset  table
in  the  case  where  an  enumeration  overflows.  This  is  generally
advantageous. However, on some systems, it may *not* be  advisable  to
repeatedly  call  `ACECosetTable'  or   `ACECosetTableFromGensAndRels'
(e.g.~in a loop), since one may run out of the pseudo  ttys  used  for
iostreams. If you encounter this problem, consider using an adaptation
of  the  usage  of  the  interactive  forms  of  `ACECosetTable'   and
`ACEStats'                            (see~"ACECosetTable!interactive"
and~"ACEStats!interactive"), together with  `ACEStart'  initialisation
steps, that is sketched in the schema below. For the  following  code,
it  is  imagined  the  scenario  is  one  of  looping   over   several
possibilities of <fgens>, <rels> and <sgens>; the two special forms of
`ACEStart' used, allow one to continually re-use a single  interactive
{\ACE} process (i.e.~only *one* iostream is used).

\){\kernttindent}\# start the interactive ACE process with no group information
\){\kernttindent}procId := ACEStart(0);
\){\kernttindent}while <expr> do
\){\kernttindent\quad}fgens := ...; rels := ...; sgens := ...;
\){\kernttindent\quad}ACEStart(procId, fgens, rels, sgens : <options>);
\){\kernttindent\quad}if ACEStats(procId).index > 0 then
\){\kernttindent\quad\quad}table := ACECosetTable(procId);
\){\kernttindent\quad\quad}...
\){\kernttindent\quad}fi;
\){\kernttindent}od;

For an already  calculated  coset  table,  we  provide  the  following
function to determine whether or not it has been standardised.

\>IsACEStandardCosetTable( <table> [: <option>] ) F

returns `true' if <table> (a list of lists of  integers)  is  standard
according to {\GAP}'s default or  the  <option>  (either  `lenlex'  or
`semilenlex')  standardisation  scheme,  or  `false'  otherwise.   See
Section~"Coset  Table  Standardisation   Schemes"   for   a   detailed
discussion of the `lenlex' and `semilenlex' standardisation schemes.

*Note:*
Essentially, `IsACEStandardCosetTable'  extends  the  {\GAP}  function
`IsStandardized'.

\atindex{lenlex standardisation}{@\noexpand`lenlex' standardisation}
Users who wish their coset tables  to  use  `ACECosetTable'  with  the
`lenlex'    (see~"option     lenlex")     option,     which     causes
`lenlex' standardisation to occur at the {\ACE} (rather  than  {\GAP})
level, should be aquainted with the following function.

\atindex{IsACEGeneratorsInPreferredOrder}%
{@\noexpand`IsACEGeneratorsInPreferredOrder'!non-interactive}
\>IsACEGeneratorsInPreferredOrder( <gens>, <rels> ) F

returns `true' if <gens>, a list of free group generators, are  in  an
order which will  not  be  changed  by  {\ACE},  for  the  group  with
presentation $\langle <gens> \mid <rels>\rangle$, where  <rels>  is  a
list   of   relators   (i.e.~words   in   the   generators    <gens>).
`IsACEGeneratorsInPreferredOrder' may also be called  in  a  different
way         for         interactive          {\ACE}          processes
(see~"IsACEGeneratorsInPreferredOrder!interactive").

For a presentation with more than one generator, the  first  generator
of which is an involution and the second is  not,  {\ACE}  prefers  to
switch the  first  two  generators.  `IsACEGeneratorsInPreferredOrder'
returns `true' if the order of the  generators  <gens>  would  not  be
changed  within  {\ACE}  and  `false',  otherwise.  (Technically,   by
``involution'' above, we really mean ``a generator `x' for which there
is a relator in <rels> of form `x*x' or `x^2'''. Such a generator may,
of course, turn out to actually be the identity.)

*Guru Notes:*
If  `IsACEGeneratorsInPreferredOrder(<gens>,  <rels>)'  would   return
`false', it is possible to enforce a user's order  of  the  generators
within {\ACE}, by setting the option `asis' (see~"option asis")  to  1
and, by passing the relator that determines that `<gens>[1]' (which we
will assume is `x') has order  at  most  2,  as:  `x*x'  (rather  than
`x^2').  Behind  the  scenes  this  is  precisely  what  is  done,  if
necessary, when `ACECosetTable' is called with the `lenlex' option.

The user may avoid all the technicalities  by  either  not  using  the
`lenlex' option (and in {\GAP}~4.3 or later, allowing {\GAP}  to  take
care of the `lenlex' standardisation), or by swapping  the  first  two
generators          in           those           cases           where
`IsACEGeneratorsInPreferredOrder(<gens>,   <rels>)'    would    return
`false'.

%%%%%%%%%%%%%%%%%%%%%%%%%%%%%%%%%%%%%%%%%%%%%%%%%%%%%%%%%%%%%%%%%%%%%%
\Section{Using ACE Directly to Test whether a Coset Enumeration Terminates}

If you only want to  test, whether a coset enumeration terminates, and
don't want to  transfer the whole coset table  to {\GAP}, you can call

\atindex{ACEStats}{@\noexpand`ACEStats'!non-interactive}
\>ACEStats( <fgens>, <rels>, <sgens> [: <options>] ) F

which calls {\ACE} non-interactively to do the coset enumeration,  the
result of which is parsed and returned as a {\GAP} record with fields

\beginitems

\quad`index' &the index of the subgroup $\langle <sgens>  \rangle$  in
$\langle <fgens> \mid <rels> \rangle$, or $0$ if the enumeration  does
not succeed;

\quad`cputime' &the total CPU  time  used  as  an  integer  number  of
`cputimeUnits' (the next field);

\quad`cputimeUnits' &the units of the  `cputime'  field,  e.g.~`"10^-2
seconds"';

\indextt{activecosets}
\quad`activecosets' &the number of currently  *active*  (i.e.~*alive*)
coset numbers (see Section~"Coset Statistics Terminology");

\indextt{maxcosets}
\quad`maxcosets' &the *maximum* number of alive coset numbers  at  any
one  time  in   the   enumeration   (see   Section~"Coset   Statistics
Terminology"); and

\indextt{totcosets}
\quad`totcosets' &the  *total*  number  of  coset  numbers  that  were
defined   in   the   enumeration   (see   Section~"Coset    Statistics
Terminology").

\enditems

Options (see Chapters~"Options  for  ACE"  and~"Strategy  Options  for
ACE")   are   used    in    exactly    the    same    way    as    for
`ACECosetTableFromGensAndRels', discussed in the previous section; and
the same  warnings  alluded  to  previously,  regarding  options  (see
Section~"General Warnings regarding the Use of Options"), apply.

*Notes:*
To initiate the coset enumeration,  the  `start'  option  (see~"option
start") is quietly inserted after the user's supplied options,  unless
the user herself supplies one  of  the  enumeration-invoking  options,
which are: `start', or one of its synonyms, `aep'  (see~"option  aep")
or `rep' (see~"option rep").

The fields of the  `ACEStats'  record  are  determined  by  parsing  a
``results  message''  (see  Appendix~"The  Meanings  of  ACE's  Output
Messages") from {\ACE}.

`ACEStats' may also be called in a different  way  in  an  interactive
{\ACE} session (see~"ACEStats!interactive").

%%%%%%%%%%%%%%%%%%%%%%%%%%%%%%%%%%%%%%%%%%%%%%%%%%%%%%%%%%%%%%%%%%%%%%
\Section{Writing ACE Standalone Input Files to Generate a Coset Table}

If you want to use {\ACE} as a standalone with its own syntax, you can
write an {\ACE} standalone input file by calling `ACECosetTable'  with
three arguments (see "ACECosetTableFromGensAndRels")  and  the  option
`aceinfile := <filename>' (see~"option aceinfile"). This will keep the
input file for the {\ACE} standalone produced by the {\GAP}  interface
under the file name <filename> (and  just  return)  so  that  you  can
perform interactive work in the standalone.

%%%%%%%%%%%%%%%%%%%%%%%%%%%%%%%%%%%%%%%%%%%%%%%%%%%%%%%%%%%%%%%%%%%%%%
\Section{Using ACE Interactively}

An interactive {\ACE} process is initiated with the command

\>ACEStart( <fgens>, <rels>, <sgens> [:<options>] )!{introduction} F

whose    arguments    and    options    are     exactly     as     for
`ACECosetTableFromGensAndRels'  and  `ACEStats',   as   discussed   in
Sections~"Using ACE Directly to Generate a Coset Table" and~"Using ACE
Directly to Test whether a Coset Enumeration  Terminates".  The  usual
warnings  regarding  options  apply  (see  Section~"General   Warnings
regarding the Use of Options"). `ACEStart' has a number of other forms
(see~"ACEStart!details").

The return value is an integer (numbering from 1) which represents the
running process. (It is possible to have  more  than  one  interactive
process running at once.) The integer returned may be  used  to  index
which of these processes an  interactive  {\ACE}  function  should  be
applied to.

An interactive {\ACE} process is terminated with the command

\>ACEQuit( <i> )!{introduction} F

where <i> is the integer returned by `ACEStart' when the  process  was
begun.   `ACEQuit'   may   also   be   called   with   no    arguments
(see~"ACEQuit!details").

We discuss each of these commands, as well as the range  of  functions
which enable one to access  features  of  the  {\ACE}  standalone  not
available non-interactively, in depth, in Chapter~"Functions for Using
ACE interactively".

*Note:*

\index{coincidence}
{\ACE} not only allows one to do a coset enumeration of a group  by  a
given (and then fixed) subgroup but it also allows one to  search  for
subgroups by starting from a given one (possibly the trivial subgroup)
and then augmenting  it  by  adding  new  subgroup  generators  either
explicitly               via                `ACEAddSubgroupGenerators'
(see~"ACEAddSubgroupGenerators")   or   implicitly   by    introducing
*coincidences* (see `ACECosetCoincidence':  "ACECosetCoincidence",  or
`ACERandomCoincidences': "ACERandomCoincidences");  or  one  can  find
smaller   subgroups    by    deleting    subgroup    generators    via
`ACEDeleteSubgroupGenerators' (see~"ACEDeleteSubgroupGenerators").

%%%%%%%%%%%%%%%%%%%%%%%%%%%%%%%%%%%%%%%%%%%%%%%%%%%%%%%%%%%%%%%%%%%%%%
\Section{Accessing ACE Examples with ACEExample and ACEReadResearchExample}

There are a number of examples available in the `examples'  directory,
which may be accessed via

\>ACEExample() F
\>ACEExample( <examplename> [:<options>] ) F
\>ACEExample( <examplename>, <ACEfunc> [:<options>] ) F

where  <examplename>  is  a  string,  the  name  of  an  example  (and
corresponding file in the `examples' directory); and <ACEfunc> is  the
{\ACE} function with which the example is to be executed. 

If `ACEExample' is called with no arguments,  or  with  the  argument:
`"index"' (meant in the sense of ``list''),  or  with  a  non-existent
example  name,  a  list  of  available  examples  is  displayed.   See
Section~"Fun with ACEExample" where the list is displayed.

By  default,  examples  are  executed  via  `ACEStats'.  However,   if
`ACEExample' is called with a second argument (choose from the (other)
alternatives:   `ACECosetTableFromGensAndRels'    (or,    equivalently
`ACECosetTable'), or `ACEStart'), the example is executed  using  that
function, instead. Note that, whereas the first  argument  appears  in
double quotes (since it's a string),  the  second  argument  does  not
(since it's a function); e.g.~to execute example `"A5"' with  function
`ACECosetTable', one would type: `ACEExample("A5", ACECosetTable);'.

`ACEExample' also accepts user options, which  may  be  passed  either
globally  (i.e.~by  using  `PushOptions'  to  push   them   onto   the
`OptionsStack') or locally  behind  a  colon  after  the  `ACEExample'
arguments, and they are passed to `ACEStats' or <ACEfunc> as  if  they
were *appended* to the existing options of <examplename>; in this way,
the user may *over-ride* any or all of the options  of  <examplename>.
This is done by passing  an  option  `aceexampleoptions'  (see~"option
aceexampleoptions"), which sets up a mechanism to  reverse  the  usual
order in which options of recursively called functions are pushed onto
the `OptionsStack'. The option `aceexampleoptions'  is  *not*  a  user
option; it is intended only for *internal* use  by  `ACEExample',  for
the above purpose. In the portion of the  output  due  to  the  `echo'
option, if one has  passed  options  to  `ACEExample',  one  will  see
`aceexampleoptions' listed first and the result of the interaction  of
<examplename>'s options and the additional options.

Consider the example `"A5"'. The effect of running

\beginexample
gap> ACEExample("A5", ACEStart);
\endexample

is essentially equivalent to executing:

\beginexample
gap> file := Filename(DirectoriesPackageLibrary("ace", "examples"), "A5");;
gap> ACEfunc := ACEStart;;
gap> ReadAsFunction(file)();
\endexample

except that some internal ``magic'' of `ACEExample' edits the  example
file and displays equivalent commands a user ``would'' execute. If the
user  has  passed  options  to  `ACEExample'   these   appear   in   a
```\# User Options''' block after the original options of the  example
in the `Info' portion of the output. By comparing with the portion  of
the output from the `echo' option (unless the user has over-ridden the
`echo' option), the user may directly observe any over-riding  effects
of user-passed options.

Please see Section~"Fun with ACEExample" for some sample  interactions
with `ACEExample'.

*Notes*

Most examples use the `mess'  ($=  `messages'$)  option.  To  see  the
effect of this, it is recommended to do: `SetInfoACELevel(3);'  before
calling `ACEExample', with an example.

The coset table output from `ACEExample', when called with many of the
examples and with the {\ACE}  function  `ACECosetTableFromGensAndRels'
is often quite long. Recall that  the  output  may  be  suppressed  by
following the (`ACEExample') command with a double semicolon (`;;').

Also, try `SetInfoACELevel(2);' before  calling `ACEExample', with  an
example.

If you unexpectedly observe `aceexampleoptions' in your  output,  then
most probably you have unintentionally passed options  by  the  global
method, by having a non-empty `OptionsStack'. One possible  remedy  is
to use `FlushOptionsStack();' (see~"FlushOptionsStack"), before trying
your `ACEExample' call again.

As discussed in Section~"Interpretation  of  ACE  Options",  there  is
generally no sensible meaning that can  be  attributed  to  setting  a
strategy option (see Chapter~"Strategy Options for ACE")  to  `false';
if you wish to nullify the effect of a strategy option,  pass  another
strategy  option,  e.g.~pass  the  `default'  (see~"option   default")
strategy option.

Also provided are:

\> ACEReadResearchExample( <filename> ) F
\> ACEReadResearchExample() F

which perform `Read' (see Section~"ref:Read" in the  {\GAP}  Reference
Manual) on <filename> or, with no argument,  the  file  with  filename
`"pgrelfind.g"' in the `res-examples' directory;  e.g.~the  effect  of
running

\beginexample
gap> ACEReadResearchExample("pgrelfind.g");
\endexample

is equivalent to executing:

\beginexample
gap> Read( Filename(DirectoriesPackageLibrary("ace", "res-examples"),
>                   "pgrelfind.g") );
\endexample

The examples provided in the `res-examples'  directory  were  used  to
solve a genuine research problem, the results of which are reported in
\cite{CHHR01}. Please see Section~"Using  ACEReadResearchExample"  for
a detailed description and examples of its use.

\> ACEPrintResearchExample( <example-filename> ) F
\> ACEPrintResearchExample( <example-filename>, <output-filename>) F

print the ``essential'' contents of the file <example-filename> in the
`res-examples' directory to the terminal, or with two arguments to the
file  <output-filename>;  <example-filename>   and   <output-filename>
should be strings. `ACEPrintResearchExample' is provided  to  make  it
easy for users to copy and edit the examples for their own purposes.

%%%%%%%%%%%%%%%%%%%%%%%%%%%%%%%%%%%%%%%%%%%%%%%%%%%%%%%%%%%%%%%%%%%%%%
\Section{General Warnings regarding the Use of Options}

Firstly, let us mention (and we will remind you later) that an  {\ACE}
strategy is merely a special option of {\ACE} that sets  a  number  of
the options described in Chapter~"Options for ACE" all  at  once.  The
strategy options are discussed in their own chapter (Chapter~"Strategy
Options for ACE").

In Section~"Passing ACE Options", we describe the two  means  provided
by {\GAP} by which {\ACE} options may be passed. In  Section~"Warnings
regarding Options", we discuss how options ``left over'' from previous
calculations can upset subsequent calculations; and hence, to  ``clear
the  decks''   in   such   circumstances,   why   we   have   provided
`FlushOptionsStack'   (see~"FlushOptionsStack").   However,   removing
`OptionsStack' options does not remove the options  previously  passed
to  an  *interactive*  {\ACE}  process;  Section~"Warnings   regarding
Options" discusses that too.

Note,  that  the  {\ACE}  package   treatment   of   options   is   an
``enhancement'' over the general way {\GAP} treats  options.  Firstly,
the {\ACE} binary allows abbreviations and mixed case of  options  and
so  the  {\ACE}  package  does  also,   as   much   as   is   possible
(see~"Abbreviations and mixed case for ACE Options"). Secondly,  since
{\ACE}'s options are in general not orthogonal,  the  order  in  which
they are put to {\ACE} is, in general, honoured (see~"Honouring of the
order in which ACE Options are passed"). Thirdly, the {\ACE}  binary's
concept of a boolean option is slightly different to that of {\GAP}'s;
Section~"Interpretation of ACE Options" discusses, in particular,  how
an option detected by {\GAP} as `true' is passed to the {\ACE} binary.

Finally, Section~"What happens if no ACE Strategy Option or if no  ACE
Option is passed" discusses what happens if no option is selected.

%%%%%%%%%%%%%%%%%%%%%%%%%%%%%%%%%%%%%%%%%%%%%%%%%%%%%%%%%%%%%%%%%%%%%%
\Section{The ACEData Record}

\>`ACEData' V

is a {\GAP} record in which the essential data for an  {\ACE}  session
within {\GAP} is stored; its fields are:

\beginitems

\quad`binary' & the path of the {\ACE} binary;

\quad`tmpdir' & the path of the  temporary  directory  containing  the
non-interactive   {\ACE}   input   and   output   files   (also    see
"ACEDirectoryTemporary" below);

\quad`ni'     & the data record for a non-interactive {\ACE} process;

\quad`io'     & list  of  data  records  for  `ACEStart'   (see  below 
and~"ACEStart") processes;

\quad`infile' & the full path of the {\ACE} input file;

\quad`outfile'& the full path of the {\ACE} output file; and

\quad`version'& the  version  of  the  current  {\ACE}  binary.   More
detailed information regarding the compilation of the binary is  given
by `ACEBinaryVersion' (see~"ACEBinaryVersion").

\enditems

Non-interactive processes used to use files rather than streams (hence
the fields `infile' and `outfile' above;  these  may  disappear  in  a
later version of the {\ACE} package).

Each time an interactive {\ACE} process is  initiated  via  `ACEStart'
(see~"ACEStart"), an identifying number <ioIndex> is generated for the
interactive process and  a  record  `ACEData.io[<ioIndex>]'  with  the
following fields is created. A  non-interactive  process  has  similar
fields but is stored in the record `ACEData.ni'.

\beginitems

\quad`procId' & the identifying number of the process  (<ioIndex>  for
interactive processes, and 0 for a non-interactive process);

\quad`args'   & a record with fields: `fgens', `rels',  `sgens'  whose
values  are  the  corresponding   arguments   passed   originally   to
`ACEStart';

\quad`options'& the current options record of the interactive process;

\quad`acegens'& a list of strings representing the generators used  by
{\ACE} (if the names of the generators passed via the  first  argument
<fgens> of `ACEStart' were all lowercase alphabetic  characters,  then
`acegens' is the `String' equivalent of <fgens>,  i.e.~`acegens[1]   =
String(<fgens>[1])' etc.);

\quad`stream' & the IOStream opened for an interactive {\ACE}  process
initiated via `ACEStart'; and

\quad`enumResult' & 
the enumeration result (string) without the trailing  newline,  output
from {\ACE};

\quad`stats'  & a record as output by the function `ACEStats'.

\quad`enforceAsis' & 
a  boolean  that  is  set  to  `true'  whenever  the   `asis'   option
(see~"option asis") must  be  set  to  `1'.  It  is  usually  `false'.
See~"IsACEGeneratorsInPreferredOrder" *Guru Notes* for the details.

\enditems

\>ACEDirectoryTemporary( <dir> ) F

calls the UNIX command `mkdir'  to  create  <dir>,  which  must  be  a
string, and if  successful  a  directory  object  for  <dir>  is  both
assigned to `ACEData.tmpdir' and returned. The fields `ACEData.infile'
and `ACEData.outfile' are also set to be  files  in  `ACEData.tmpdir',
and on exit from {\GAP} <dir> is removed. Most users will  never  need
this command;  by  default,  {\GAP}  typically  chooses  a  ``random''
subdirectory of `/tmp' for  `ACEData.tmpdir'  which  may  occasionally
have limits on what  may  be  written  there.  `ACEDirectoryTemporary'
permits the user to choose a directory (object) where one  is  not  so
limited.


%%%%%%%%%%%%%%%%%%%%%%%%%%%%%%%%%%%%%%%%%%%%%%%%%%%%%%%%%%%%%%%%%%%%%%
\Section{Setting the Verbosity of ACE via Info and InfoACE}

\>`InfoACE' V

The output of the {\ACE} binary is, by default, not displayed. However
the user may choose to see some, or all, of this output. This is  done
via the `Info' mechanism  (see  Chapter~"ref:Info  Functions"  in  the
{\GAP} Reference Manual). For this purpose, there is  the  <InfoClass>
`InfoACE'. Each line of output from {\ACE} is directed to  a  call  to
`Info' and will be displayed for the user to see if the `InfoLevel' of
`InfoACE' is high enough. By default, the `InfoLevel' of `InfoACE'  is
1, and it is recommended that you leave it at this level,  or  higher.
Only messages which we think that the user will really want to see are
directed to `Info' at `InfoACE' level 1. To turn off  *all*  `InfoACE'
messaging, set the `InfoACE' level to 0  (see~"SetInfoACELevel").  For
convenience, we provide the function

\>InfoACELevel() F

which returns the current `InfoLevel' of `InfoACE', i.e.~it is  simply
a shorthand for `InfoLevel(InfoACE)'.

To set the `InfoLevel' of `InfoACE' we also provide

\>SetInfoACELevel( <level> ) F
\>SetInfoACELevel() F

which for a non-negative integer  <level>,  sets  the  `InfoLevel'  of
`InfoACE'    to    <level>    (i.e.~it    is    a    shorthand     for
`SetInfoLevel(InfoACE, <level>)'),  or  with  no  arguments  sets  the
`InfoACE' level to the default value 1.  Currently,  information  from
{\ACE} is directed to `Info' at four `InfoACE' levels: 1, 2, 3 and  4.
The command

\beginexample
gap> SetInfoACELevel(2);
\endexample

enables the display of the results line of an enumeration from {\ACE},
whereas

\beginexample
gap> SetInfoACELevel(3);
\endexample

enables the display of  all  of  the  output  from  {\ACE}  (including
{\ACE}'s  banner,  containing  the  host  machine   information);   in
particular,  the  progress  messages,  emitted  by  {\ACE}  when   the
`messages' option (see~"option messages") is set to a non-zero  value,
will be displayed via `Info'. Finally,

\beginexample
gap> SetInfoACELevel(4);
\endexample

enables the display of all the input  directed  to  {\ACE}  (behind  a
```ToACE> ''' prompt, so you can distinguish it  from  other  output).
The `InfoACE' level of 4 is really for gurus who are familiar with the
{\ACE} standalone.

%%%%%%%%%%%%%%%%%%%%%%%%%%%%%%%%%%%%%%%%%%%%%%%%%%%%%%%%%%%%%%%%%%%%%%
\Section{Acknowledgements}

Large parts of this manual,  in  particular  the  description  of  the
options for running {\ACE}, are directly copied  from  the  respective
descriptions in the manual~\cite{Ram99} for the standalone version  of
{\ACE} by Colin Ramsay.  Most  of  the  examples,  in  the  `examples'
directory and accessed  via  the  `ACEExample'  function,  are  direct
translations of Colin Ramsay's  `test\#\#\#.in'  files  in  the  `src'
directory.

Many thanks to Joachim Neub{\accent127u}ser who not only provided  one
of the early manual drafts and hence a template for the style  of  the
manual and conceptual hooks for the design of the  Package  code,  but
also meticulously proof-read  all  drafts  and  made  many  insightful
comments.

Many thanks also to Volkmar Felsch who, in testing the {\ACE} Package,
discovered a number of bugs, made a number of important suggestions on
code improvement, thoroughly checked all the  examples,  and  provided
the example found at the end of Section~"Steering  ACE  Interactively"
which  demonstrates  rather  nicely  how  one  can  use  the  function
`ACEConjugatesForSubgroupNormalClosure'.

Finally, we wish to acknowledge the contribution of Charles Sims.  The
inclusion of the `incomplete' (see "option incomplete")  and  `lenlex'
(see~"option lenlex") options, were made in response  to  requests  to
the  {\GAP}  developers  to  include  such  options  for  coset  table
functions. Also, the definition of `lenlex' standardisation  of  coset
tables (see Section~"Coset Table Standardisation Schemes"), is due  to
him.

%%%%%%%%%%%%%%%%%%%%%%%%%%%%%%%%%%%%%%%%%%%%%%%%%%%%%%%%%%%%%%%%%%%%%%
\Section{Changes from earlier versions}

A reasonably comprehensive history of  the  evolution  of  pre-release
versions of the {\ACE} Package, is contained in the file `CHANGES'  in
the `gap' directory. 

The `3.xxx' versions of the {\ACE} Package were compatible with {\GAP}
4.2, but were built for use with  {\GAP}  4.3.  However,  the  current
version of the {\ACE}  Package  requires  at  least  {\GAP}  4.3.  One
important change in {\GAP} from {\GAP} 4.2  to  {\GAP}  4.3  that  has
relevance for {\ACE} Package  users  is  the  change  of  the  default
standard  for  the  numbering  of  cosets  in  a  coset   table   (see
Section~"Coset Table Standardisation Schemes").

\>ACEPackageVersion() F

gives the current version of the {\ACE} Package (i.e.~the {\GAP}  code
component; the  function  `ACEBinaryVersion'  (see~"ACEBinaryVersion")
gives details of the C code compilation).

*History of release versions*

\beginlist

\item{`3.001'}
First release.

\item{`3.002'}
`ACEIgnoreUnknownDefault' (see~"ACEIgnoreUnknownDefault")  added.

\item{}
Option `continue'  changed  to  `continu'  (see~"option  continu")  --
`continue' is a keyword in {\GAP} 4.3.

\item{`3.003'}
Initial value of `ACEIgnoreUnknownDefault' is now  `true'  (previously
it was `false').

\item{}
`ACECosetTableFromGensAndRels' and the non-interactive `ACECosetTable'
(except when producing an input script via  option  `ACEinfile'),  and
the non-interactive `ACEStats' which previously  used  files  all  now
invoke iostreams .

\item{}
Option `pkgbanner' used to control  the  printing  of  the  banner  on
loading (see~"Loading the ACE Package").

\item{`4.0'}
Revised for {\GAP} 4.4, for which the option `pkgbanner' is no  longer
available (most of its function is now provided by the third  argument
of the `LoadPackage' function; see~"ref:LoadPackage").

\item{}
The package version number now has a single decimal  place,  to  avoid
being confused with the binary version number.

\item{`4.1'}
Added $M_{12}$ presentation to `res-examples/pgrelfind.g'.

\endlist

%%%%%%%%%%%%%%%%%%%%%%%%%%%%%%%%%%%%%%%%%%%%%%%%%%%%%%%%%%%%%%%%%%%%%%%%%
%%
%E
