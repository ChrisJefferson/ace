%%%%%%%%%%%%%%%%%%%%%%%%%%%%%%%%%%%%%%%%%%%%%%%%%%%%%%%%%%%%%%%%%%%%%%%%%
%%
%W  ex.tex             ACE standalone documentation          Colin Ramsay
%W                                                            Greg Gamble
%%
%H  $Id$
%%

%%  Colin Ramsay - 22 Oct 1999  is the true author of this document
%%  Greg Gamble  - 29 Feb 2000 made minor modifications:
%%                * put this header at the top for CVS

%%  ex.tex - The wrapper for the appendix of examples.
%%
%%  5   10   15   20   25   30   35   40   45   50   55   60   65   70   75
%%..|....|....|....|....|....|....|....|....|....|....|....|....|....|....|

In this appendix we give some examples of \ace\ runs.
A stand-alone discussion of some of the features of these runs is included,
  although parts of these runs are mentioned in the body of the text, as 
  illustrations of specific features of \ace's behaviour.
The \ttt{ex***.in} files supplied as part of this documentation can be
  used to run these examples, although an example may be presented as if it
  were generated interactively, and the output may be edited for reasons of
  space or perspicuity.
There may be minor variations in the exact format of the output, since
  \ace\ is continually being `improved'\kern-1pt.
Unless otherwise noted, all parameters are defaulted and the default
  build of \ace\ was used.
In multipart runs, note that parameters from an earlier part may carry
  across to a later one.
Note that some of the examples may require a machine with a large amount
  of memory.

\section{Getting started}\label{ex000}
%%%%%%%%%%%%%%%%%%%%%%%%%%%%%%%%%%%%%%%%%%%%%%%%%%%%%%%%%%%%%%%%%%%%%%%%%
%%
%W  ex000.tex          ACE standalone documentation          Colin Ramsay
%W                                                            Greg Gamble
%%
%H  $Id$
%%

%%  Colin Ramsay - 21 Sep 1999  is the true author of this document
%%  Greg Gamble  - 29 Feb 2000 made minor modifications:
%%                * put this header at the top for CVS

%%  The example describing input file ex000.in

This example uses input file \ttt{ex000.in}, and illustrates the basics of
  \ace.
Note how the input is generally insensitive to command synonyms,
  capitalisation, white space, and \ttt{:} \amp \ttt{;} characters.
%
When \ace\ starts up, it prints out its version, the date \amp time, and
  the name of the host on which it's running.
If we attempt to do an enumeration immediately we get an error, since the
  lack of generators means we can't build the (empty) coset table.

\bv\begin{verbatim}
ACE 3.000        Sat May  8 13:50:29 1999
=========================================
Host information:
  name = flute
end;
** ERROR (continuing with next line)
   can't start (no generators?)
\end{verbatim}\ev

After defining two generators, we can do an enumeration.
The default state is not to echo the presentation or print any messages;
  only the result line is printed.
The group is free, since there are no relators, and the subgroup is
 trivial.
So the enumeration overflows.

\bv\begin{verbatim}
gr:ab;          # A stupid comment
Begin
OVERFLOW (a=249998 r=83333 h=83333 n=249999; l=337 c=0.97; 
                                                       m=249998 t=249998)
\end{verbatim}\ev

The \ttt{sr} commands dumps out the presentation and the parameters for
  the run.
All of these are currently defaulted, apart from those dependent on there
  being two (non-involutionary) generators.

\bv\begin{verbatim}
sr:1;
  #-- ACE 3.000: Run Parameters ---
Group Name: G;
Group Generators: ab;
Group Relators: ;
Subgroup Name: H;
Subgroup Generators: ;
Wo:1000000; Max:249998; Mess:0; Ti:-1; Ho:-1; Loop:0;
As:0; Path:0; Row:1; Mend:0; No:0; Look:0; Com:10;
C:0; R:0; Fi:7; PMod:3; PSiz:256; DMod:4; DSiz:1000;
  #--------------------------------
\end{verbatim}\ev

This is the first part of the table.
Note that, as there are no relators, the table has separate columns for
  generator inverses.
So the default workspace of $1000000$ words allows
  a table of $249998 = 1000000/4 - 2$ cosets.
As row filling is on by default, the table is simply filled with
  cosets in order.
Note that a compaction phase is done before printing the table, but that
  this does nothing here (the lower-case \ttt{co} tag), since there are no 
 dead cosets.
The coset representatives are simply all possible freely reduced words, in
  length plus lexicographic order.

\bv\begin{verbatim}
pr:-1,12;
co: a=249998 r=83333 h=83333 n=249999; c=+0.00
 coset |      a      A      b      B   order   rep've
-------+---------------------------------------------
     1 |      2      3      4      5
     2 |      6      1      7      8       0   a
     3 |      1      9     10     11       0   A
     4 |     12     13     14      1       0   b
     5 |     15     16      1     17       0   B
     6 |     18      2     19     20       0   aa
     7 |     21     22     23      2       0   ab
     8 |     24     25      2     26       0   aB
     9 |      3     27     28     29       0   AA
    10 |     30     31     32      3       0   Ab
    11 |     33     34      3     35       0   AB
    12 |     36      4     37     38       0   ba
\end{verbatim}\ev

We now set things up to do the alternating group on five letters, of order
  $60$.
We turn messaging on, but set the interval high enough so that there will
  be no progress messages.

\bv\begin{verbatim}
Enum: A_5;
rel: a^2, b^3, ababababab;
subgr: trivial;
mess: 1000;    start;
\end{verbatim}\ev

The presentation and the parameters are echoed, the enumeration is
  performed, and then the results of the run are printed.
Note that the exponent of the \ttt{ababababab} word has been correctly
  deduced, and that \ttt{a} is treated as an involution.
So the table has only three columns now.
Definitions are HLT-style, and a total of $76$ cosets (incl.\@ the 
  subgroup) were defined.

\bv\begin{verbatim}
  #-- ACE 3.000: Run Parameters ---
Group Name: A_5;
Group Generators: ab;
Group Relators: (a)^2, (b)^3, (ab)^5;
Subgroup Name: trivial;
Subgroup Generators: ;
Wo:1000000; Max:333331; Mess:1000; Ti:-1; Ho:-1; Loop:0;
As:0; Path:0; Row:1; Mend:0; No:3; Look:0; Com:10;
C:0; R:0; Fi:6; PMod:3; PSiz:256; DMod:4; DSiz:1000;
  #--------------------------------
INDEX = 60 (a=60 r=77 h=1 n=77; l=3 c=0.00; m=66 t=76)
\end{verbatim}\ev

We now use a non-trivial subgroup, and monitor all the actions of the
  enumerator.

\bv\begin{verbatim}
Subgroup Name: C_5 ;
gen:ab;
Monit :1
END;
  #-- ACE 3.000: Run Parameters ---
Group Name: A_5;
Group Generators: ab;
Group Relators: (a)^2, (b)^3, (ab)^5;
Subgroup Name: C_5;
Subgroup Generators: ab;
Wo:1000000; Max:333331; Mess:1; Ti:-1; Ho:-1; Loop:0;
As:0; Path:0; Row:1; Mend:0; No:3; Look:0; Com:10;
C:0; R:0; Fi:6; PMod:3; PSiz:256; DMod:4; DSiz:1000;
  #--------------------------------
AD: a=2 r=1 h=1 n=3; l=1 c=+0.00; m=2 t=2
SG: a=2 r=1 h=1 n=3; l=1 c=+0.00; m=2 t=2
RD: a=3 r=1 h=1 n=4; l=2 c=+0.00; m=3 t=3
RD: a=4 r=2 h=1 n=5; l=2 c=+0.00; m=4 t=4
RD: a=5 r=2 h=1 n=6; l=2 c=+0.00; m=5 t=5
RD: a=6 r=2 h=1 n=7; l=2 c=+0.00; m=6 t=6
RD: a=7 r=2 h=1 n=8; l=2 c=+0.00; m=7 t=7
RD: a=8 r=2 h=1 n=9; l=2 c=+0.00; m=8 t=8
RD: a=9 r=2 h=1 n=10; l=2 c=+0.00; m=9 t=9
CC: a=8 r=2 h=1 n=10; l=2 c=+0.00; d=0
RD: a=9 r=5 h=1 n=11; l=2 c=+0.00; m=9 t=10
RD: a=10 r=5 h=1 n=12; l=2 c=+0.00; m=10 t=11
RD: a=11 r=5 h=1 n=13; l=2 c=+0.00; m=11 t=12
RD: a=12 r=5 h=1 n=14; l=2 c=+0.00; m=12 t=13
RD: a=13 r=5 h=1 n=15; l=2 c=+0.00; m=13 t=14
RD: a=14 r=5 h=1 n=16; l=2 c=+0.00; m=14 t=15
CC: a=13 r=6 h=1 n=16; l=2 c=+0.00; d=0
CC: a=12 r=6 h=1 n=16; l=2 c=+0.00; d=0
INDEX = 12 (a=12 r=16 h=1 n=16; l=3 c=0.00; m=14 t=15)
\end{verbatim}\ev

We now dump out the statistics accumulated during the run.
The run had \ttt{a=12} \amp \ttt{t=15}, so there must have been three
  coincidences (\ttt{qcoinc=3}).
Of these, two were primary coincidences (\ttt{rdcoinc=2}).
Since \ttt{t=15}, there must have been fourteen coset definitions;  one
  was during the application of coset \#1 (i.e., the subgroup) to the
  subgroup generator (\ttt{apdefn=1}), and the remainder during the
  application of the cosets to the relators (\ttt{rddefn=13}).

\bv\begin{verbatim}
STATistics;
  #- ACE 3.000: Level 0 Statistics --
cdcoinc=0 rdcoinc=2 apcoinc=0 rlcoinc=0 clcoinc=0
  xcoinc=2 xcols12=4 qcoinc=3
  xsave12=0 s12dup=0 s12new=0
  xcrep=6 crepred=0 crepwrk=0 xcomp=0 compwrk=0
xsaved=0 sdmax=0 sdoflow=0
xapply=1 apdedn=1 apdefn=1
rldedn=0 cldedn=0
xrdefn=1 rddedn=5 rddefn=13 rdfill=0
xcdefn=0 cddproc=0 cdddedn=0 cddedn=0
  cdgap=0 cdidefn=0 cdidedn=0 cdpdl=0 cdpof=0
  cdpdead=0 cdpdefn=0 cddefn=0
  #----------------------------------
\end{verbatim}\ev

Note how the pre-printout compaction phase now does some work (the
  upper-case \ttt{CO} tag), since there were coincidences, and hence dead
  cosets.
Note how \ttt{b/B} have been used as the first two columns, since these
  must be occupied by a generator/inverse pair or a pair of involutions.
The \ttt{a} column is also the \ttt{A} column, as \ttt{a} is an
  involution.

\bv\begin{verbatim}
  print  TABLE : -1, 12 ;
CO: a=12 r=13 h=1 n=13; c=+0.00
 coset |      b      B      a   order   rep've
-------+--------------------------------------
     1 |      3      2      2
     2 |      1      3      1       3   B
     3 |      2      1      4       3   b
     4 |      8      5      3       5   ba
     5 |      4      8      6       2   baB
     6 |      9      7      5       5   baBa
     7 |      6      9      8       3   baBaB
     8 |      5      4      7       5   bab
     9 |      7      6     10       5   baBab
    10 |     12     11      9       3   baBaba
    11 |     10     12     12       2   baBabaB
    12 |     11     10     11       3   baBabab
\end{verbatim}\ev

If we define the generator order to be that of the columns, then the table
  above is not in canonic form, and the coset representatives are not in
  order.
We now standardise the table; note the compaction phase before
  standardisation, although it does nothing in this particular case.
Note how, if we read through the table in row-major order, new cosets
  are introduced using the smallest available number, and that the 
  representatives are now in order and are minimal.

\bv\begin{verbatim}
st;
co/ST: a=12 r=13 h=1 n=13; c=+0.00
pr:-1,12;
co: a=12 r=13 h=1 n=13; c=+0.00
 coset |      b      B      a   order   rep've
-------+--------------------------------------
     1 |      2      3      3
     2 |      3      1      4       3   b
     3 |      1      2      1       3   B
     4 |      5      6      2       5   ba
     5 |      6      4      7       5   bab
     6 |      4      5      8       2   baB
     7 |      8      9      5       5   baba
     8 |      9      7      6       5   baBa
     9 |      7      8     10       3   babaB
    10 |     11     12      9       3   babaBa
    11 |     12     10     12       3   babaBab
    12 |     10     11     11       2   babaBaB
\end{verbatim}\ev

We now exit \ace, printing out the version and the date \amp time again.

\bv\begin{verbatim}
q
=========================================
ACE 3.000        Sat May  8 13:52:49 1999
\end{verbatim}\ev

%%%%%%%%%%%%%%%%%%%%%%%%%%%%%%%%%%%%%%%%%%%%%%%%%%%%%%%%%%%%%%%%%%%%%%%%%
%%
%E


\section{Emulating Sims}\label{ex001}
%%%%%%%%%%%%%%%%%%%%%%%%%%%%%%%%%%%%%%%%%%%%%%%%%%%%%%%%%%%%%%%%%%%%%%%%%
%%
%W  ex001.tex          ACE standalone documentation          Colin Ramsay
%W                                                            Greg Gamble
%%
%H  $Id$
%%

%%  Colin Ramsay - 21 Sep 1999  is the true author of this document
%%  Greg Gamble  - 29 Feb 2000 made minor modifications:
%%                * put this header at the top for CVS

%%  The example describing input file ex001.in

Here we demonstrate the various \ttt{sims} modes, and see if we can
  duplicate the \ttt{m} (maximum active cosets) and \ttt{t} (total cosets
  defined) values (see the input file \ttt{ex001.in}).
The ability to do so gives our confidence in the correctness of \ace\ a
  large boost.
We work with the formal inverses of the relators and subgroup generators
  from \cite{Sim94}, since definitions are made from the front in Sims'
  routines and from the rear in \ace.
We may also have to use the \ttt{asis} flag, to force the column order (by
  entering involutions as \ttt{xx}) and to preserve the relator ordering.
We match Sims' values for R style \amp R* style (\ttt{sims:1} \amp 
  \ttt{3}) and C style (\ttt{sims:9}), but may not do so if we use
  Mendelsohn (\ttt{sims:5} \amp \ttt{7}); this makes sense, since the
  order of processing cycled relators is not specified by Sims.

The input and output files for Example 5.2:

\bv\begin{verbatim}
gr: r,s,t;
rel: (r^tRR)^-1, (s^rSS)^-1, (t^sTT)^-1;
text: ;                               sr;
text: ** Sims:1 (cf. 1502/1550) ...;  sims:1;  end;
text: ** Sims:3 (cf. 673/673) ...;    sims:3;  end;
text: ** Sims:5 (cf. 1808/1864) ...;  sims:5;  end;
text: ** Sims:7 (cf. 620/620) ...;    sims:7;  end;
text: ** Sims:9 (cf. 588/588) ...;    sims:9;  end;
\end{verbatim}\ev

\bv\begin{verbatim}
  #-- ACE 3.000: Run Parameters ---
Group Name: G;
Group Relators: rrTRt, ssRSr, ttSTs;
Subgroup Name: H;
Subgroup Generators: ;
  #--------------------------------
** Sims:1 (cf. 1502/1550) ...
INDEX = 1 (a=1 r=2 h=2 n=2; l=3 c=0.01; m=1502 t=1550)
** Sims:3 (cf. 673/673) ...
INDEX = 1 (a=1 r=2 h=2 n=2; l=3 c=0.01; m=673 t=673)
** Sims:5 (cf. 1808/1864) ...
INDEX = 1 (a=1 r=2 h=2 n=2; l=3 c=0.01; m=1603 t=1603)
** Sims:7 (cf. 620/620) ...
INDEX = 1 (a=1 r=2 h=2 n=2; l=3 c=0.01; m=615 t=615)
** Sims:9 (cf. 588/588) ...
INDEX = 1 (a=1 r=2 h=2 n=2; l=4 c=0.01; m=588 t=588)
\end{verbatim}\ev

The input and output files for Example 5.3, $k=8$:

\bv\begin{verbatim}
gr: x,y;
rel: (xx)^-1, (y^3)^-1, ((xy)^7)^-1, ((xyxY)^8)^-1;
text: ;                                  sr;
text: ** Sims:1 (cf. 87254/128562) ...;  sims:1;  end;
text: ** Sims:3 (cf. 31678/32320) ...;   sims:3;  end;
text: ** Sims:5 (cf. 99632/178620) ...;  sims:5;  end;
text: ** Sims:7 (cf. 30108/31365) ...;   sims:7;  end;
text: ** Sims:9 (cf. 39745/39745) ...;   asis:1;  sims:9;  end;
\end{verbatim}\ev

\bv\begin{verbatim}
  #-- ACE 3.000: Run Parameters ---
Group Name: G;
Group Relators: XX, YYY, YXYXYXYXYXYXYX, yXYXyXYXyXYXyXYXyXYXyXYXyXYXyXYX;
Subgroup Name: H;
Subgroup Generators: ;
  #--------------------------------
** Sims:1 (cf. 87254/128562) ...
INDEX=10752 (a=10752 r=128563 h=1 n=128563; l=27 c=1.00; m=87254 t=128562)
** Sims:3 (cf. 31678/32320) ...
INDEX=10752 (a=10752 r=8005 h=32321 n=32321; l=10 c=0.81; m=31678 t=32320)
** Sims:5 (cf. 99632/178620) ...
INDEX=10752 (a=10752 r=168547 h=1 n=168547; l=24 c=1.50; m=96952 t=168546)
** Sims:7 (cf. 30108/31365) ...
INDEX=10752 (a=10752 r=5738 h=31673 n=31673; l=8 c=0.90; m=30420 t=31672)
** Sims:9 (cf. 39745/39745) ...
INDEX=10752 (a=10752 r=1 h=39746 n=39746; l=43 c=1.37; m=39745 t=39745)
\end{verbatim}\ev

The input and output files for Example 5.4:

\bv\begin{verbatim}
gr: a,b;
rel: (a^8)^-1, (b^7)^-1, ((ab)^2)^-1, ((Ab)^3)^-1;
gen: (a^2)^-1, (Ab)^-1;
asis:1;
text: ;                               sr;
text: ** Sims:1 (cf. 2174/2635) ...;  sims:1;  end;
text: ** Sims:3 (cf. 1199/1212) ...;  sims:3;  end;
text: ** Sims:5 (cf. 2213/2619) ...;  sims:5;  end;
text: ** Sims:7 (cf. 1258/1284) ...;  sims:7;  end;
text: ** Sims:9 (cf. 1302/1306) ...;  asis:0;  sims:9;  end;
\end{verbatim}\ev

\bv\begin{verbatim}
  #-- ACE 3.000: Run Parameters ---
Group Name: G;
Group Relators: AAAAAAAA, BBBBBBB, BABA, BaBaBa;
Subgroup Name: H;
Subgroup Generators: AA, Ba;
  #--------------------------------
** Sims:1 (cf. 2174/2635) ...
INDEX = 448 (a=448 r=2636 h=1 n=2636; l=4 c=0.02; m=2174 t=2635)
** Sims:3 (cf. 1199/1212) ...
INDEX = 448 (a=448 r=576 h=1213 n=1213; l=3 c=0.02; m=1199 t=1212)
** Sims:5 (cf. 2213/2619) ...
INDEX = 448 (a=448 r=2620 h=1 n=2620; l=4 c=0.03; m=2213 t=2619)
** Sims:7 (cf. 1258/1284) ...
INDEX = 448 (a=448 r=612 h=1285 n=1285; l=3 c=0.02; m=1258 t=1284)
** Sims:9 (cf. 1302/1306) ...
INDEX = 448 (a=448 r=1 h=1307 n=1307; l=5 c=0.02; m=1302 t=1306)
\end{verbatim}\ev

%%%%%%%%%%%%%%%%%%%%%%%%%%%%%%%%%%%%%%%%%%%%%%%%%%%%%%%%%%%%%%%%%%%%%%%%%
%%
%E


\section{Row filling}\label{ex002}
%%%%%%%%%%%%%%%%%%%%%%%%%%%%%%%%%%%%%%%%%%%%%%%%%%%%%%%%%%%%%%%%%%%%%%%%%
%%
%W  ex002.tex          ACE standalone documentation          Colin Ramsay
%W                                                            Greg Gamble
%%
%H  $Id$
%%

%%  Colin Ramsay - 21 Sep 1999  is the true author of this document
%%  Greg Gamble  - 29 Feb 2000 made minor modifications:
%%                * put this header at the top for CVS

%%  The example describing input file ex002.in

If all definitions are made by applying cosets to relators, then the coset
  table can contain holes, either because the form of the relators `hides'
  one of the generators from one of the cosets, or because one of the
  generators is not present in the relators.
The \ttt{row} and \ttt{mend} parameters can be used to deal with these
  sorts of situations.
Consider the following examples, drawn from \cite{War}; see the input file
  \ttt{ex002.in}.
Note that, although the \ttt{row} parameter is specifically intended to
  prevent the table containing holes, the \ttt{mend} parameter actually
  yields better enumeration statistics.
Note also the use of the \ttt{asis} parameter to control whether or not
  the presentation is reduced.

\bv\begin{verbatim}
asis:1;  pure r;  end;  
#-- ACE 3.000: Run Parameters ---
Group Name: infinite cyclic group;
Group Generators: xy;
Group Relators: yxyxY;
Subgroup Name: self (index 1);
Subgroup Generators: x;
Wo:1000000; Max:249998; Mess:1000000; Ti:-1; Ho:-1; Loop:0;
As:1; Path:0; Row:0; Mend:0; No:0; Look:0; Com:100;
C:0; R:1000; Fi:1; PMod:0; PSiz:256; DMod:0; DSiz:1000;
  #--------------------------------
OVERFLOW (a=249992 r=249996 h=1 n=249999; l=253 c=1.19; m=249992 t=249998)
pr:-1,12;
CO: a=249992 r=249990 h=1 n=249993; c=+0.33
 coset |      x      X      y      Y   order   rep've
-------+---------------------------------------------
     1 |      1      1      2      0
     2 |      4      3      5      1       0   y
     3 |      2      5      6      4       0   yX
     4 |      0      2      3      0       0   yx
     5 |      3      6      7      2       0   yy
     6 |      5      7      8      3       0   yXy
     7 |      6      8      9      5       0   yyy
     8 |      7      9     10      6       0   yXyy
     9 |      8     10     11      7       0   yyyy
    10 |      9     11     12      8       0   yXyyy
    11 |     10     12     13      9       0   yyyyy
    12 |     11     13     14     10       0   yXyyyy

pure r;  row:1;  end;
INDEX = 1 (a=1 r=2 h=2 n=2; l=3 c=0.00; m=12 t=17)
pure r;  mend:1;  end;
INDEX = 1 (a=1 r=2 h=2 n=2; l=3 c=0.00; m=5 t=6)

asis:0;  pure r;  end;  
#-- ACE 3.000: Run Parameters ---
Group Name: infinite cyclic group;
Group Generators: xy;
Group Relators: xyx;
Subgroup Name: self (index 1);
Subgroup Generators: x;
Wo:1000000; Max:249998; Mess:1000000; Ti:-1; Ho:-1; Loop:0;
As:0; Path:0; Row:0; Mend:0; No:0; Look:0; Com:100;
C:0; R:1000; Fi:1; PMod:0; PSiz:256; DMod:0; DSiz:1000;
  #--------------------------------
INDEX = 1 (a=1 r=2 h=2 n=2; l=3 c=0.00; m=1 t=1)

asis:1;  pure r;  end;  
#-- ACE 3.000: Run Parameters ---
Group Name: C_3;
Group Generators: xy;
Group Relators: xxxyxyXXX, yyyxyxYYY;
Subgroup Name: trivial (index 3);
Subgroup Generators: ;
Wo:1000000; Max:249998; Mess:1000000; Ti:-1; Ho:-1; Loop:0;
As:1; Path:0; Row:0; Mend:0; No:0; Look:0; Com:100;
C:0; R:1000; Fi:1; PMod:0; PSiz:256; DMod:0; DSiz:1000;
  #--------------------------------
OVERFLOW (a=181146 r=38770 h=1 n=249999; l=32 c=0.67; m=181146 t=249998)
pr:-1,16;
CO: a=181146 r=28074 h=1 n=181147; c=+0.28
 coset |      x      X      y      Y   order   rep've
-------+---------------------------------------------
     1 |      2      0      7      0
     2 |      3      1     15      0       0   x
     3 |      4      2     23      0       0   xx
     4 |     12      3      6      5       0   xxx
     5 |     35      6      4      0       0   xxxY
     6 |      5      0     31      4       0   xxxy
     7 |     47      0      8      1       0   y
     8 |     55      0      9      7       0   yy
     9 |     11     10     52      8       0   yyy
    10 |      9      0     72     11       0   yyyX
    11 |     63      9     10      0       0   yyyx
    12 |     20      4     14     13       0   xxxx
    13 |     89     14     12      0       0   xxxxY
    14 |     13      0     85     12       0   xxxxy
    15 |    101      0     16      2       0   xy
    16 |    109      0     17     15       0   xyy

pure r;  row:1;   end;
INDEX = 3 (a=3 r=468 h=1 n=468; l=3 c=0.00; m=343 t=467)
pure r;  mend:1;  end;
INDEX = 3 (a=3 r=29 h=29 n=29; l=3 c=0.00; m=21 t=28)

asis:0;  pure r;  end;  
#-- ACE 3.000: Run Parameters ---
Group Name: C_3;
Group Generators: xy;
Group Relators: yxy, xyx;
Subgroup Name: trivial (index 3);
Subgroup Generators: ;
Wo:1000000; Max:249998; Mess:1000000; Ti:-1; Ho:-1; Loop:0;
As:0; Path:0; Row:0; Mend:0; No:0; Look:0; Com:100;
C:0; R:1000; Fi:1; PMod:0; PSiz:256; DMod:0; DSiz:1000;
  #--------------------------------
INDEX = 3 (a=3 r=6 h=6 n=6; l=3 c=0.00; m=5 t=5)
\end{verbatim}\ev

%%%%%%%%%%%%%%%%%%%%%%%%%%%%%%%%%%%%%%%%%%%%%%%%%%%%%%%%%%%%%%%%%%%%%%%%%
%%
%E


\section{Equivalent presentations}

TBA: $F(2,7)$, using \ttt{rep} \amp \ttt{aep} ...

\section{Deduction queues}

TBA: ... (see test009)

\section{Large enumerations}

Suppose that the presentation given is such that the final coset table
  exceeds the 4\,Gbyte limit imposed by 32-bit machines; e.g., an index of
  $250 \times 10^6$\kern-2pt, with a 5-column table and 4 byte integers.
We are justified in regarding such an enumeration as `big'\kern-1.5pt, 
  since it will require more than 4\,Gbyte of storage no matter how
  efficiently it is performed.
Of course, even trivial enumerations may exceed this limit if they are
  very pathological (see, e.g., \cite{HR00}).
However, we have no (easy) way of knowing whether or not such enumerations
  can be done within the 4\,Gbyte limit, so we are hesitant to classify
  them as big.
\ace\ is 64-bit `aware'\kern-1.5pt, and can use more than 4\,Gbyte of
  memory if it is available.
Note however that the number of cosets (i.e., the number of rows in the
  coset table) is still limited by the size of a signed \ttt{int}.
So the maximum size of a table is $2^{31} - n$ cosets, where $n$ is
  probably $3$; one since we can't actually represent $+2147483648$, one
  since coset \#0 is not used, and one since we need to count one past the
  top of the table.

Some trivial group enumerations involving more than 1\,G total cosets 
  and 4\,Gbyte of memory were reported in \cite{HR00}.
However, the first big enumeration, in the above sense, done by \ace\ was
  the Thomson simple group.
This group has order $TBA$, and contains $TBA$ as an index $TBA$ subgroup.
TBA: ...

\section{Looping}

TBA: ...

\section{Use of \ttt{st}}\label{ex007}

TBA: ...

\section{Use of \ttt{cy}, \ttt{nc}, \ttt{cc} and \ttt{rc}} 

TBA: ...

%%%%%%%%%%%%%%%%%%%%%%%%%%%%%%%%%%%%%%%%%%%%%%%%%%%%%%%%%%%%%%%%%%%%%%%%%
%%
%E
