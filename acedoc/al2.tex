
% al2.tex - Colin Ramsay - 16 Jun 99
%
% The AL2 chapter.
%
%   5   10   15   20   25   30   35   40   45   50   55   60   65   70   75
% ..|....|....|....|....|....|....|....|....|....|....|....|....|....|....|

Level 2 of {\ace} is a complete, standalone application for generating and
  manipulating coset tables.
It can be used interactively, or can take its input from a script file.
It is reasonably robust and comprehensive, but no attempt has been made to
  make it ``industrial strength'' or to give it any of the features of, say,
  \textsc{Magma} \cite{BCP} or \textsf{GAP} \cite{Sch}.
Most of its features have been added in response to user requests, and it
  is assumed that the user is `competent'\kern-1.5pt.
One of the primary goals in developing {\ace} was to demonstrate how to
  correctly use {\ace} Levels $0$ and $1$; some care is taken to ensure
  that the user cannot generate `invalid' tables.

\section{Commands}\label{sec:cmd}

In this chapter we give details of all the commands available when
  using the interactive interface.

The first sections:\ref{sec:general} to~\ref{sec:defpres} show how to run
{\ace} and  how to define a presentation.
Sections~\ref{sec:process} to~\ref{sec:strat} explain ways how to drive the
enumeration machinery.
Section~\ref{sec:opwith} lists operations that work with a coset table,

Section~\ref{sec:tech} finally contains commands that control the computer
resources used.

If you are using {\ace} via another system interface, such as {\sf GAP}, you
can proceed immediately to section~\ref{sec:process}.

The subsection headings match the help screen produced by the \ttt{help}
  command. Section~\ref{sec:alphind} gives an alphabetic index of all of
  them.
%
Alternate forms of a command are separated by a \ttt{/}\kern-1pt, while
  any optional part of a command is denoted by \ttt{[\dots]}.
Case is not significant in command names, but that part of a command
  actually present must be correct, modulo white space.
%
Appendix~\ref{app:ex} contains many examples of how to correctly drive
  \ace.

Parameters to a command are supplied after a colon (\ttt{:}).  
Each command is terminated by a newline or a semicolon (\ttt{;}), except
  in some cases where the argument may be a list of words, in which case
  newlines are ignored and a semicolon is the only terminator.
(E.g., the \ttt{add gen}, \ttt{add rel}, \ttt{rel} and \ttt{gen}
  commands.)
In many cases the parameters are optional, and entering the command
  without an argument prints the parameter's current value.
If the no-argument form has a special meaning, this is noted in its entry
  below.
Where there is no danger of confusion, the \ttt{:} and/or the \ttt{;} can
  usually be dispensed with.

%%%%%%%%%%%%%%%%%%%%%%%%%%%%%%%%%%%%%%%%%%%%%%%%%%%%%%%%%%%%%%%%%%%%%%%%%
\section{General Commands}
\label{sec:general}

\subsection{\ttt{ai\ /\ alter\ i[nput]\ :\ [<filename>]\ ;}}
\label{cmd:ai}
\label{cmd:alter input}

By default, commands to {\ace} are read from the standard input file
  (i.e., the `keyboard'\kern-1.5pt, \ttt{stdin}).
The \ttt{ai} command closes the current input file, and opens
  \ttt{filename} as the source of commands.
If \ttt{filename} can't be opened, input reverts to \ttt{stdin}.

\tsc{Notes:}
If you switch to taking input from (another) file, remember to switch back
  before the end of that file; otherwise the \ttt{EOF} there will cause
  {\ace} to terminate.

\subsection{\ttt{ao\ /\ alter\ o[utput]\ :\ [<filename>]\ ;}}
\label{cmd:ao}
\label{cmd:alter output}

By default, output from {\ace} is sent to the standard output file
  (i.e., the `terminal'\kern-1.5pt, \ttt{stdout}).
The \ttt{ao} command closes the current output file, and opens
  \ttt{filename} for all future output.
If \ttt{filename} can't be opened, output reverts to \ttt{stdout}.

\subsection{\ttt{beg[in]\ /\ end\ /\ start\ ;}}
\label{cmd:begin}

Start an enumeration.
Any existing information in the table is cleared, and the enumeration
  starts from coset \#1 (i.e., the subgroup).

\subsection{\ttt{bye\ /\ exit\ /\ q[uit]\ ;}}
\label{cmd:bye}

The exits {\ace} nicely, printing the date and the time.
An \ttt{EOF} (end-of-file; i.e., \ttt{\^{}d}) has the same effect, so 
  proper termination occurs if {\ace} is taking its input from a script
  file.

\subsection{\ttt{check\ /\ redo\ ;}}
\label{cmd:check}
\label{cmd:redo}

An extant enumeration is redone, using the current parameters.
Any existing information in the table is retained, and the enumeration
  is restarted from coset \#1 (i.e., the subgroup).

\tsc{Notes:}
This option is really intended for the case where additional relators
  and/or subgroup generators have been introduced.
The current table, which may be incomplete or exhibit a finite index, is
  still `valid'\kern-1.5pt.
However, the new data may allow the enumeration to complete, or cause a
  collapse to a smaller index.

\subsection{\ttt{cont[inue]\ ;}}
\label{cmd:continue}

Continue the current enumeration, building upon the existing table.
If a previous run stopped without producing a finite index you can, in
  principle, change any of the parameters and continue on.
Of course, if you make any changes which invalidate the current table, you
  won't be allowed to continue, although you may be allowed to redo.

\subsection{\ttt{d[ump]\ :\ [0/1/2[,0/1]]\ ;}}
\label{cmd:dump}

Dumps the internal variables of \ace.
The first argument selects which of the three levels of {\ace} to dump, and
  defaults to Level $0$.
The second argument selects the level of detail, and defaults to $0$
  (i.e., less detail).
This option is intended for gurus; the source code should be consulted to
  see what the output means.

\subsection{\ttt{echo\ :\ [0/1]\ ;}}
\label{cmd:echo}

By default, {\ace} does not echo its commands.
If you wish it to do so, turn this feature on with \ttt{echo:1}.
This feature can be used to render output files from {\ace} less
  incomprehensible.

\subsection{\ttt{h[elp]\ ;}}
\label{cmd:help}

Prints the help screen; i.e., all the subsection headings in this section.
Note that this list is fairly long, so its top may disappear off the top 
  of the screen.
(In future versions something will be done about this.
Also, the help command will allow a command name as an argument and will
  print out a brief r\'{e}sum\'{e} of the specified command.)

\subsection{\ttt{loop[\ limit]\ :\ [/0/1..]\ ;}}
\label{cmd:loop limit}

The core enumerator is organised as a state machine, with each step
  performing an `action' (i.e., lookahead, compaction) or a block of
  actions (i.e., \ttt{ct} coset definitions, \ttt{rt} coset applications).
The number of passes through the main loop (i.e., steps) is counted, and
  the enumerator can make an early return when this count hits the
  value of \ttt{loop}.
A value of \ttt{0}, the default, turns this feature off.

\tsc{Guru Notes:}
You can do lots of really neat things using this feature, but you need
  some understanding of the internals of {\ace} to get real benefit from
  it.
Read the code!

\subsection{\ttt{mo[de]\ ;}}
\label{cmd:mode}

Prints the possible enumeration modes, as deduced from the command history
  since the last call to the enumerator; see Section~\ref{sec:style}.

\subsection{\ttt{opt[ions]\ ;}}
\label{cmd:options}

This command dumps details of the options included in the version of {\ace}
  you're running; i.e., what compiler flags were set when the executable
  was built.
A typical output, illustrating the default build, is:

\bv\begin{verbatim}
Executable built:
  Sat Feb 27 15:57:59 EST 1999
Level 0 options:
  statistics package = on
  coinc processing messages = on
  dedn processing messages = on
Level 1 options:
  workspace multipliers = decimal
Level 2 options:
  host info = on
\end{verbatim}\ev

\subsection{\ttt{par[ameters]\ ;}}
\label{cmd:parameters}

An old option, which did nothing.
It is included for backward compatability.
Pre-\acet\ scripts may contain this option, which is quietly ignored
  by \acet.

\subsection{\ttt{print\ det[ails]\ /\ sr\ :\ [0/1]\ ;}}
\label{cmd:print details}

This command prints out details of the current presentation and parameters.
No argument, or an argument of \ttt{0}, prints out the group and
  subgroup name, the group's relators and the subgroup's generators.
If the argument is \ttt{1}, then the current setting of the parameters is
  also printed.
The printout is the same as that produced at the start of a run when
  messaging is on.
See Appendix~\ref{app:ex} for some examples.

\tsc{Notes:}
The output is printed out in a form suitable for input, so that a record
  of a previous run can be used to replicate the run.
Note that, due to the defaulting of some parameters and the special
  meaning attached to some values, a little care has to be taken in
  interpreting the parameters.
If you wish to \tit{exactly} duplicate a run, you should use the output
  of \ttt{sr} \tit{after} the run completes.

\subsection{\ttt{rec[over]\ /\ contig[uous]\ ;}}
\label{cmd:recover}
\label{cmd:contiguous}

Invokes the compaction routine on the table to recover the space used by
  any dead cosets.
A \ttt{CO} message line is printed if any cosets were recovered, and a
  \ttt{co} line if none were.
This routine is called automatically if the \ttt{cy}, \ttt{nc}, \ttt{pr}
  or \ttt{st} options are invoked.

\subsection{\ttt{restart\ ;}}
\label{cmd:restart}

An old option, included for backward compatability.
Use the \ttt{check/redo} option instead.
Pre-\acet\ scripts may contain this option, which is quietly ignored
  by \acet.


\subsection{\ttt{style\ ;}}
\label{cmd:style}

Prints the current enumeration style, as deduced from the current \ttt{Ct}
  and \ttt{Rt} parameters; see Section~\ref{sec:style}.

\subsection{\ttt{stat[istics]\ /\ stats\ ;}}
\label{cmd:statistics}

If the statistics package is compiled into the code (which it is by
  default, see the \ttt{opt} command), then dump the statistics
  accumulated during the most recent enumeration.
See Section~\ref{ex000} for an example, and the \ttt{enum.c} source file
  for the meaning of the variables.

\subsection{\ttt{text\ :\ <string>\ ;}}
\label{cmd:text}

Just echoes the string.
This allows the output from a run driven by a script to be tarted up.

\subsection{\ttt{sys[tem]\ :\ <string>\ ;}}
\label{cmd:system}

Passes \ttt{string} to a shell, via the C library routine \ttt{system()}.

\subsection{\ttt{\#\ ...\ <newline>}}
\label{cmd:hash}

Any input between a sharp sign (\ttt{\#}) and the next newline is ignored.
This allows comments to be included anywhere in command scripts.

%%%%%%%%%%%%%%%%%%%%%%%%%%%%%%%%%%%%%%%%%%%%%%%%%%%%%%%%%%%%%%%%%%%%%%%%%
\section{Defining and Changing a Presentation}
\label{sec:defpres}

\subsection{\ttt{add\ gen[erators]\ /\ sg\ :\ <word\ list>\ ;}}
\label{cmd:add generators}

Adds the words in the list to any subgroup generators already present.
The enumeration must be restarted or redone, it cannot be continued.

\subsection{\ttt{add\ rel[ators]\ /\ rl\ :\ <relation\ list>\ ;}}
\label{cmd:add relators}

Adds the words in the list to any relators already present.
The enumeration must be restarted or redone, it cannot be continued.

\subsection{\ttt{cc\ /\ coset\ coinc[idence]\ :\ int\ ;}}
\label{cmd:cc}
\label{cmd:coset coincidence}

Print out the representative of coset \#\ttt{int}, and add it to the
  subgroup generators; i.e., equates this coset with coset \#1, the
  subgroup.

\subsection{\ttt{del\ gen[erators]\ /\ ds\ :\ <int\ list>\ ;}}
\label{cmd:del generators}

This command allows subgroup generators to be deleted from the
  presentation.
If the generators are numbered from one in the output of, say, the
  \ttt{sr} command, then the generators listed in \ttt{int\ list} are
  deleted.
\ttt{int\ list} must be a strictly increasing sequence.

\subsection{\ttt{del\ rel[ators]\ /\ dr\ :\ <int\ list>\ ;}}
\label{cmd:del relators}

As \ttt{del gen}, but for the group's relators.

\subsection{\ttt{enum[eration]\ /\ group\ name\ :\ <string>\ ;}}
\label{cmd:enumeration}

This command defines the name by which the current enumeration (i.e.,
  the group being used) will be identified in any printout.
It has no effect on the actual enumeration, and defaults to \ttt{G}.
An empty name is accepted; to see what the current name is, use the
  \ttt{sr} command.

\subsection{\ttt{gen[erators]\ /\ subgroup\ gen[erators]\ 
  :\ <word\ list>\ ;}}
\label{cmd:generators}
\label{cmd:subgroup generators}

By default, there are no subgroup generators and the subgroup is trivial.
This command allows a list of subgroup generating words to be entered.
The format is the same as for relators, except that `genuine' relations
  (i.e., $w_1 = w_2$) are not allowed.

\subsection{\ttt{gr[oup\ generators]:\ [<letter\ list>\ /\ int]\ ;}}
\label{cmd:gr}
\label{cmd:group generators}

This command introduces the group generators, which may be represented in
  one of two ways.
They may be presented as a list of lower-case letters, optionally
  separated by commas.
This is the usual method, and gives up to $26$ generators.
Subsequently, upper-case letters can be used, if desired, to stand for the
  inverse of their lower-case versions; 
  e.g., \ttt{A} for \ttt{a\^{}-1}, \ttt{B} for \ttt{b\^{}-1}, etc.
Alternatively, a positive integer can be used to indicate the number of
  generators.
For example, \ttt{gr:5} indicates that there are five generators, 
  designated \ttt{1}, \ttt{2}, \ttt{3}, \ttt{4} and \ttt{5}.

\tsc{Notes:}
Any use of the \ttt{gr} command which actually defines generators
  invalidates any previous enumeration, and stays in effect until the next
  \ttt{gr} command.
Any words for the group or subgroup must be entered using the nominated
  generator format, and all printout will use this format.
%
This command is not optional, nor is their any default.
A valid set of generators is the minimum information necessary before
  {\ace} will attempt an enumeration.

\tsc{Guru Notes:}
The columns of the coset table are allocated in the same order as the 
  generators are listed, insofar as this is possible, given that the first
  two columns must be a generator/inverse pair or a pair of involutions.
The ordering of the columns can, in some cases, effect the definition
  sequence of cosets and impact the statistics of an enumeration.

\subsection{\ttt{group\ relators\ /\ rel[ators]\ :\ <relation\ list>\ ;}}
\label{cmd:relators}
\label{cmd:group relators}

By default, or if an empty argument to this command is used, the group
  is free.
Otherwise, this command is used to introduce the group's defining
  relators.
In order to allow {\ace} to accept presentations from a variety of
  sources, many kinds of word representations are allowed. 
{\ace} accepts words in the nominated generators, allowing \ttt{*} for
  multiplication, \ttt{\^{}} for exponentiation and conjugation, and
  brackets for precedence specification. 
Round or square brackets may be used for commutation. 
(There is no danger of confusion between \ttt{[a,b]}/\ttt{(a,b)} and
  \ttt{(ab)}, since a \ttt{,} implies commutation, while no comma implies a
  word.)
If letter generators are used, multiplication and exponentiation signs
  (but \tit{not} conjugation signs) may be omitted; e.g., \ttt{a3} is the
  same as \ttt{a\^{}3} and \ttt{ab} is the same as \ttt{a*b}.
Also, the exponent \ttt{-1} can be abbreviated to \ttt{-},
  so \ttt{a-} stands for \ttt{A}.
The \ttt{*} can also be dropped for numeric generators; but of course two
  numeric generators, or a numeric exponent and a numeric generator, must
  be separated by whitespace.  
Remember that \ttt{A} stands for \ttt{a\^{}-1}, \ttt{a\^{}b} for
  \ttt{Bab} and \ttt{[a,b]} and \ttt{[a,b,c]} for \ttt{ABab} and
  \ttt{[[a,b],c]}.

\ttt{relation\ list} is a comma-separated list of relations.
A relation is either a word $w$ (equivalent to the relation $w=1$) or
  a set of equivalent words, $w_1=w_2=w_3$ say (equivalent to
  $w_1w^{-1}_2=1$ and $w_1w^{-1}_3=1$).
A (somewhat sketchy) BNF for the words is:

\bv\begin{verbatim}
  <word>    = <factor> { "*" | "/" <factor> }
  <factor>  = <element> [ ["^"] <integer> | "^" <element> ]
  <element> = <generator> ["'"]
            | "(" <word> { "," <word> } ")" ["'"]
            | "[" <word> { "," <word> } "]" ["'"]
\end{verbatim}\ev

\subsection{\ttt{subg[roup\ name]\ :\ <string>\ ;}}
\label{cmd:subgroup name}

This command defines the name by which the current subgroup will be
  identified in any printout.
It has no effect on the actual enumeration, and defaults to \ttt{H}.
An empty name is accepted; to see what the current name is, use the
  \ttt{sr} command.


%%%%%%%%%%%%%%%%%%%%%%%%%%%%%%%%%%%%%%%%%%%%%%%%%%%%%%%%%%%%%%%%%%%%%%%%%
\section{The enumeration process}
\label{sec:process}

By default the {\ace} coset enumerator chooses strategies that work in many 
cases, however if problems become more complicated you might have to select
specific strategies to get the coset enumeration to finish successfully and
quickly.

The most useful ones to start with are the {\tt easy} option which will lead to a
quick coset enumeration if the problem is easy and the {\tt hard} option if the
coset enumeration is expected to be problematic.

While it is possible to set parameters which affect just parts of the
enumeration process, there are also various predefined strategies, which
people have found to be useful. The options {\tt easy} and {\tt hard} we just
mentioned are just two of these, section~\ref{sec:strat} describes
them all.

%%%%%%%%%%%%%%%%%%%%%%%%%%%%%%%%%%%%%%%%%%%%%%%%%%%%%%%%%%%%%%%%%%%%%%%%%
\section{Options that Alter the {\ace} Input}

\subsection{\ttt{as[is]\ :\ [0/1]\ ;}}
\label{cmd:asis}
By default, {\ace} freely and cyclically reduces the relators, freely
  reduces the subgroup generators, and sorts relators and subgroup generators
  in length-increasing order.
If you do not want this, you can switch it off by setting the {\tt asis} option.

\paragraph{Notes:}
As well as allowing you to use the presentation as it is given, this is
useful for forcing definitions to be made in a prespecified order, by
introducing dummy (i.e., freely trivial) subgroup generators.  Note also
that the exact form of the presentation can have a significant impact on the
enumeration statistics; it is not the case that the default option always
yields the best enumeration.

\paragraph{Guru Notes:}
If \ttt{asis} is on, a (reduced) relator of the form \ttt{aa} or \ttt{AA}
  causes that generator to be treated as an involution.
In the relators and subgroup generators, the inverses of involutionary
  generators are automatically replaced with the generator.
If \ttt{asis} is off, only relators of the form \ttt{a\^{}2} cause the
  generator to be treated as an involution.
The forms \ttt{aa} and \ttt{a\^{}2} are preserved in any printout, so
  that you can track \ace's behaviour.

%%%%%%%%%%%%%%%%%%%%%%%%%%%%%%%%%%%%%%%%%%%%%%%%%%%%%%%%%%%%%%%%%%%%%%%%%
\section{Enumeration mode and style}\label{sec:style}

The majority of options affect certain aspects of the enumeration process.
Describing them we shall assume that the reader already knows the basic
ideas of coset enumeration, as can be found for example in
\cite{Neu}.

The first main decision is in which sequence definitions are made.
In HLT-type procedures
the order in which cosets are defined is explicitly prescribed by the order
in which (the subgroup generators and) the group
relators are processed. In standard
Felsch-type procedures coset definition is normally from the left/top
of the coset table towards the right/bottom --- that is, in order row by row. 
In fact a procedure needs to follow a method like this to some extent for 
the proofs that coset enumeration eventually terminates in the case of 
finite index.

However this still leaves us with plenty of freedom in definition
strategies, freedom which can be used to great advantage in Felsch-type
methods. Though it is not strictly necessary, Felsch-type programs generally
start off by ensuring that each of the given subgroup generators
forms a cycle at coset 1 before embarking on further enumeration.
The key to performance of coset enumeration procedures is 
good selection of the  next coset to be defined. Thus a preferred
definition strategy is incorporated into our coset table based
definition methods. The detailed control of these mechanisms
is by way of various parameters.

By default {\ace} uses a strategy which assumes that the enumeration is
easy, and if it turns out not to be so, {\ace} switches to a strategy
designed for more difficult enumerations. The other straightforward
options for beginning users are {\tt easy} and {\tt hard}, and about 10
further predefined strategies are available by name. Details are
to be found in section~\ref{sec:strat}.

Thus, {\tt easy} will quickly succeed or fail (in the context of the
given resources); {\tt default} may quickly succeed, but then try {\tt hard}
for a while; {\tt hard} will run more slowly from the beginning try to succeed.

The *enumeration style* is the balance between C-style definitions (i.e.,
coset table based, Felsch style) and R-style definitions (i.e., relator
based, HLT style), and is controlled by the values of the {\tt ct} and
{\tt rt}
parameters. These parameters can be set by the following options:

\subsection{\ttt{c[factor]\ /\ ct[\ factor]\ :\ [int]\ ;}}
\label{cmd:cfactor}
\label{cmd:ct factor}
\subsection{\ttt{r[factor]\ /\ rt[\ factor]\ :\ [int]\ ;}}
\label{cmd:rfactor}
\label{cmd:rt factor}

The absolute value of these parameters sets the number of
definitions (C-style) or coset applications (R-style) per pass through the
enumerator's main loop.  The sign of these parameters sets the style, and
the possible combinations are given in table~\ref{tab:sty}:

\begin{table}
\hrule
\caption{The styles}
\label{tab:sty}
\smallskip
\renewcommand{\arraystretch}{0.875}
\begin{tabular*}{\textwidth}{@{\extracolsep{\fill}}crrlc} 
\hline\hline
& \ttt{Rt} value & \ttt{Ct} value & style name & \\
\hline
& $<\!0$ & $<\!0$ & R/C & \\
& $<\!0$ & $0$    & R*  & \\
& $<\!0$ & $>\!0$ & Cr  & \\
& $0$    & $<\!0$ & C   & \\
& $0$    & $0$    & R/C (defaulted) & \\
& $0$    & $>\!0$ & C  & \\
& $>\!0$ & $<\!0$ & Rc & \\
& $>\!0$ & $0$    & R  & \\
& $>\!0$ & $>\!0$ & CR & \\
\hline\hline
\end{tabular*}
\end{table}

In R style all the definitions are made via relator scans; i.e., this is 
an HLT-like
mode.  In C style most definitions are made in the next empty coset table
slot and are (in principle) tested in all essentially different positions in
the relators; i.e., this is a Felsch-like mode.
However in C Style some definitions may be made following a preferred
definition strategy, see the {\tt pmode} and {\tt psize} options below.
In R/C style we run in R style until an overflow,
perform a lookahead on the entire table, and then switch to CR style.
Defaulted R/C style is the default style, (here we use R/C style with
{\tt ct} set to 1000 and {\tt rt} set to approximately $2000$ divided by the total
length of the relators) in an attempt to balance R and C definitions when
we switch to CR style.  Rc and Cr styles are like R and C styles, except
that a single C or R style pass (respectively) is done after the initial R
or C style pass.  R\* style makes definitions the same as R style, but tests
all definitions as for C style.  In CR style alternate passes of C style and
R style are performed.

\subsection{\ttt{no[\ relators\ in\ subgroup]\ :\ [-1/0..]\ ;}}
\label{cmd:no relators}
It is sometimes helpful to include the relators in the subgroup, in the
sense that they are applied to coset \#1 at the start of an enumeration.  A
value of 0 for this option  turns this feature off and the (default) 
argument of -1
includes all the relators.  A positive argument includes the specified
number of relators, in order.
The {\tt number} option affects only the C mode.

\subsection{\ttt{mend[elsohn]\ :\ [0/1]\ ;}}
\label{cmd:mendelsohn}
Mendelsohn style processing during relator scanning/closing is turned on by
giving this option.  Off is the default, and here coset applications are
done only at the start (and end) of a relator.  Mendelsohn on means that
coset applications are done at all cyclic permutations of the (base)
relator.  The effect of the Mendelsohn parameter is case-specific.  It can
mean the difference between success or failure, or it can impact the number
of cosets required, or it can have no effect on an enumeration's statistics.

*Note:* Processing all cyclic permutations of the relators can be very
time-consuming, especially if the presentation is large.  So, all other
things being equal, the Mendelsohn flag should normally be left off. 

\subsection{\ttt{f[factor]\ /\ fi[ll\ factor]\ :\ [0/1..]\ ;}}
\label{cmd:ffactor}
\label{cmd:fill factor}

Controls the preferred definition strategy by setting the fill factor.
Unless prevented by the fill factor,
gaps of length one found during deduction testing are
preferentially filled (see \cite{Hav}).  However, this potentially violates
the formal requirement that all rows in the table are eventually filled (and
tested against the relators).  The fill factor is used to ensure that some
constant proportion of the coset table is always kept filled.  Before
defining a coset to fill a gap of length one, the enumerator checks whether
{\tt fill} times the completed part of the table is at least the total size of
the table and, if not, fills coset table rows in standard order instead 
of filling gaps.

An argument of 0 selects the default value of $\lfloor 5(n+2)/4 \rfloor$,
where $n$ is the number of columns in the table.  
This default fill factor allows a moderate amount of gap-filling.
If {\tt fill} is 1, then there is no gap-filling.
This parameter applies only to C-mode.
A large value of {\tt fill} can cause what is in effect infinite looping
(resolved by the coset enumeration failing).
However, in general, a large value does work well.
The effects of the various gap-filling stategies vary widely.  
It is not clear which values are good general defaults or, indeed, whether
  any strategy is always ``not too bad''.

\subsection{\ttt{pd\ mo[de]\ /\ pmod[e]\ :\ [0/1..3]\ ;}}
\label{cmd:pd mode}
\label{cmd:pmode}
The value of the {\tt pmode} option determines which definitions
are preferred.  If the argument is 0, then Felsch style definitions are made
using the next empty table slot.  If the argument is non-zero, then gaps of
length one found during relator scans in Felsch style are preferentially
filled (subject to the value of {\tt fill}).  If the argument is 1, they are
filled immediately, and if it is 2, the consequent deduction is also made
immediately (of course, these are also put on the deduction stack).  If the
argument is 3, then the gaps of length one
are noted in the preferred definition queue.
Provided such a gap survives (and no coincidence occurs, which causes
the queue to be discarded) the next coset will be defined to fill the oldest
gap of length one.  The default value is either 0 or 3, depending on the
strategy selected (see section~\ref{sec:strat}). If you want to
know more details, read the code.

\subsection{\ttt{pd\ si[ze]\ /\ psiz[e]\ :\ [0/2/4/8/...]\ ;}}
\label{cmd:pd size}
\label{cmd:psize}
The preferred definition queue is implemented as a ring, dropping
  earliest entries.
Its size *must* be $2^n$, for some $n>0$.
An argument of 0 selects the default size of $256$.
Each queue slot takes two words (i.e., 8 bytes), and the queue can store
  up to $2^n-1$ entries.

\subsection{\ttt{row[\ filling]\ :\ [0/1]\ ;}}
\label{cmd:row filling}
When making HLT-style definitions, it is normal to scan each row of the
  coset table after its coset has been applied to all relators, and make 
  definitions to fill any holes encountered.
Failure to do so can cause even simple enumerations to overflow; see
  Section~\ref{ex002}.
To turn row filling off, use \ttt{row:0}.
  
\subsection{\ttt{look[ahead]\ :\ [0/1..4]\ ;}}
\label{cmd:lookahead}
Although HLT-style strategies are fast, they are local, in the
sense that the implications of any definitions/deductions made while
applying cosets may not become apparent until much later.  One way to
alleviate this problem is to perform lookaheads occasionally; that is, to
test the information in the table, looking for deductions or concidences.
{\ace} can perform a lookahead when the table overflows, before the compaction
routine is called.  Lookahead can be done using the entire table or only
that part of the table above the current coset, and it can be done R-style
(scanning cosets from the beginning of relators) or C-style (testing all
definitions in all essentially different positions).

The {\tt look} option sets the value for lookahead, a value of 0 disables
lookahead.
A value of 1 does a
partial table, R-style lookahead; 2 does all the table, C-style; 3 does all
the table, R-style; and 4 does a partial table, C-style.  The default is
either 0 or 1 depending on the strategy;
see Section~\ref{sec:strat}.

\subsection{\ttt{ded\ mo[de]\ /\ dmod[e]\ :\ [0..4]\ ;}}
\label{cmd:ded mode}
\label{cmd:dmode}
A completed table is only valid if every table entry has been tested in all
essentially different positions in all relators.  This testing can either be
done directly (Felsch strategy) or via relator scanning (HLT strategy).  If
it is done directly, then more than one deduction (i.e., table entry) can be
outstanding at any one time.  So the untested deductions are stored in a
stack.  Normally this stack is fairly small but, during a collapse, it can
become very large.

This command allows the user to specify how
deductions should be handled.  The value <val> has the following
interpretations: 0 discard deductions if there is no stack space left; in
addition purges redundant cosets off the top of the stack at every
coincidence. 
\begin{itemize}
\item[1] as 0, but purge redundant cosets off the top of the stack
  at every coincidence,
\item[2] again is as 0,
but purges all redundant cosets from the stack at 
every coincidence.
\item[3] discard the entire stack if it overflows;
\item[4] if the stack overflows, then double the stack size and purge all
  redundant cosets from the stack.
\end{itemize}

The default deduction mode is either $0$ or $4$, depending on the strategy
selected (see section~\ref{sec:strat}), and it is recommended that
you stay with the default.  If you want to know more details, read the code.

\paragraph{Notes:}
If deductions are discarded for any reason, then a final relator
check phase will
be run automatically at the end of the enumeration, if necessary, to check
the result.

\subsection{\ttt{ded\ si[ze]\ /\ dsiz[e]\ :\ [0/1..]\ ;}}
\label{cmd:ded size}
\label{cmd:dsize}
Sets the size of the (initial) allocation for the deduction stack.
The size is in terms of the number of deductions, with one deduction
  taking two words (i.e., 8 bytes).
The default size, of $1000$, can be selected by a value of 0.
See the {\tt dmod} entry for a (brief) discussion of deduction handling.

\subsection{\ttt{ho[le\ limit]\ :\ [-1/0..100]\ ;}}
\label{cmd:hole limit}

An experimental feature which allows an enumeration to be terminated when
  the percentage of holes in the table exceeds a given value.
In practice, calculating this is very expensive, and it tends to remain
  constant or decrease throughout an enumeration.
So the feature doesn't seem very useful.
The default value of \ttt{-1} turns this feature off.
If you want more details, read the source code.

%%%%%%%%%%%%%%%%%%%%%%%%%%%%%%%%%%%%%%%%%%%%%%%%%%%%%%%%%%%%%%%%%%%%%%%%%
\section{Predefined Strategies}\label{sec:strat}

The versatility of {\ace} means that it can be difficult to select appropriate
parameters when presented with a new enumeration.  The problem is compounded
by the fact that no generally applicable rules exist to predict, given a
presentation, which parameter settings are ``good''.  To help overcome this
problem, {\ace} contains various commands which select particular enumeration
strategies.  One or other of these strategies may work and, if not, the
results may indicate how the parameters can be varied to obtain a successful
enumeration.  The thirteen standard strategies are listed below, their
definition is given in table~\ref{tab:pred}.

The various parameters available via \textsf{GAP} are explained in groups
according to their functionality. The parameter names are listed in the top
row and the strategy names are in the first column.

Note that we explicitly (re)set all of the listed enumerator parameters in
  all of the predefined strategies, even although some of them have no
  effect.
For example, the \ttt{fill} value is irrelevant in HLT mode.
The idea behind this is that, if you later change some parameters
  individually, then the enumeration retains the `flavour' of the last
  selected predefined strategy.
Note also that other parameters which may effect an enumeration are left
  untouched by setting one of the predefined strategies; for example, the
  values of \ttt{max} and \ttt{asis}.
These parameters have an effect which is independant of the selected 
  strategy.

\begin{table}
\hrule
\caption{The predefined strategies}
\label{tab:pred}
\smallskip
\renewcommand{\arraystretch}{0.875}
\begin{tabular*}{\textwidth}{@{\extracolsep{\fill}}lrrrrrrrrrrrrr} 
\hline\hline
          & \multicolumn{13}{c}{parameter} \\ 
\cline{2-14}
strategy & path & row & mend & number & look & com & ct   & rt    & fill &
pmode & psize & dmode & dsize \\ 
\hline
def    & n   & y   & n    & -1 & n    & 10  & 0    & 0     & 0  & 3    & 256  & 4    & 1000 \\
easy   & n   & y   & n    & 0  & n    & 100 & 0    & 1000  & 1  & 0    & 256  & 0    & 1000 \\
fel:0  & n   & n   & n    & 0  & n    & 10  & 1000 & 0     & 1  & 0    & 256  & 4    & 1000 \\
fel:1  & n   & n   & n    & -1 & n    & 10  & 1000 & 0     & 0  & 3    & 256  & 4    & 1000 \\
hard   & n   & y   & n    & -1 & n    & 10  & 1000 & 1     & 0  & 3    & 256  & 4    & 1000 \\
hlt    & n   & y   & n    & 0  & 1    & 10  & 0    & 1000  & 1  & 0    & 256  & 0    & 1000 \\
pure c & n   & n   & n    & 0  & n    & 100 & 1000 & 0     & 1  & 0    & 256  & 4    & 1000 \\
pure r & n   & n   & n    & 0  & n    & 100 & 0    & 1000  & 1  & 0    & 256  & 0    & 1000 \\
sims:1 & n   & y   & n    & 0  & n    & 10  & 0    & 1000  & 1  & 0    & 256  & 0    & 1000 \\
sims:3 & n   & y   & n    & 0  & n    & 10  & 0    & -1000 & 1  & 0    & 256  & 4    & 1000 \\
sims:5 & n   & y   & y    & 0  & n    & 10  & 0    & 1000  & 1  & 0    & 256  & 0    & 1000 \\
sims:7 & n   & y   & y    & 0  & n    & 10  & 0    & -1000 & 1  & 0    & 256  & 4    & 1000 \\
sims:9 & n   & n   & n    & 0  & n    & 10  & 1000 & 0     & 1  & 0    & 256  & 4    & 1000 \\
\hline\hline
\end{tabular*}
\end{table}

Note that, apart from the {\tt felsch:0} and {\tt sims:9} strategies, all of the
strategies are distinct, although some are very similar.

In detail, the strategies are as follows:

\subsection{\ttt{def[ault]\ ;}}
\label{cmd:default}
This selects the default strategy, which is based on the defaulted R/C
style; see section~"Options that Modify the Enumeration Process".  The idea
here is that we assume that the enumeration is ``easy'' and start out in R
style.  If it turns out not to be easy, then we regard it as ``hard'', and
switch to CR style, after performing a lookahead on the entire table.

\subsection{\ttt{easy\ ;}}
\label{cmd:easy}
If this strategy is selected, we run in R style (i.e., HLT) and turn
lookahead and compaction off.  For small and/or easy enumerations, this mode
is likely to be the fastest.

\subsection{\ttt{hard\ ;}}
\label{cmd:hard}
In many
``hard'' enumerations, a mixture of R-style and C-style definitions, all
tested in all essentially different positions, is appropriate.  This option
selects such a mixed strategy.  The idea here is that most of the work is
done C-style (with the relators in the subgroup and with gap-filling
active), but that every $1000$ C-style definitions a further coset is
applied to all relators.

\paragraph{Guru Notes:}
The {\tt 1000/1} mix is not necessarily optimal, and some
  experimentation may be needed to find an acceptable balance (see, for
  example, \cite{HR1}).
Note also that, the longer the total length of the presentation, the more
  work needs to be done for each coset application to the relators; one
  coset application can result in more than $1000$ definitions, reversing
  the balance between R-style and C-style definitions.

\subsection{\ttt{fel[sch]\ :\ [0/1]\ ;}}
\label{cmd:felsch}
{\tt felsch:0} selects the Felsch strategy, while {\tt felsch:1} selects Felsch with all
relators in the subgroup and turns gap-filling on.

\subsection{\ttt{hlt\ ;}}
\label{cmd:hlt}
Selects the standard HLT strategy.

\subsection{\ttt{pure\ c[t]\ ;}}
\label{cmd:pure ct}
Sets the strategy to basic C-style (coset table based) -- no compaction, no
gap-filling, no relators in subgroup.

\subsection{\ttt{pure\ r[t]\ ;}}
\label{cmd:pure rt}
Sets the strategy to basic R-style (relator based) -- no Mendelsohn, no
compaction, no lookahead, no row-filling.

\subsection{\ttt{sims\ :\ 1/3/5/7/9\ ;}}
\label{cmd:sims}
In his book \cite{Sim}, Sims discusses ten standard enumeration strategies.
These are effectively HLT (with and without the {\tt mend} parameter) and
Felsch, all either with or without table standardisation as the enumeration
proceeds.

{\ace} does not implement table standardisation during an enumeration, 
although incomplete tables can be standardised and an enumeration
continued.

The five non-standardising strategies are implemented, and can be selected
by these options.  The number is the same as given in \S5.5~of~\cite{Sim}.
With care, it is possible to duplicate the statistics given in \cite{Sim};
some examples are given in Section~\ref{ex001}.

%%%%%%%%%%%%%%%%%%%%%%%%%%%%%%%%%%%%%%%%%%%%%%%%%%%%%%%%%%%%%%%%%%%%%%%%%
\section{Operations with a Coset Table}
\label{sec:opwith}

\subsection{\ttt{cy[cles]\ ;}}
\label{cmd:cycles}

Print out the table in cycles; i.e., the permutation representation.

\subsection{\ttt{nc\ /\ normal[\ closure]\ :\ [0/1]\ ;}}
\label{cmd:nc}
\label{cmd:normal closure}

If the argument is missing or \ttt{0}, test for normal closure.
If a subgroup generator does not normalise a generator, then this is
  printed out.
If the argument is \ttt{1}, then any non-normalising conjugates are
  also added to the subgroup generators.

\subsection{\ttt{oo\ /\ order[\ option]\ :\ int;}}
\label{cmd:oo}
\label{cmd:order option}

This option finds a coset with order a multiple of $|\ttt{int}|$ modulo
  the subgroup, and prints out its coset representative.
If $\ttt{int} < 0$, then all cosets meeting the requirement are printed.
If $\ttt{int} = 0$, then the orders of all cosets are printed.
If $\ttt{int} > 0$, then the first coset meeting the requirement is
  printed.

\subsection{\ttt{pr[int\ table]\ :\ [[-]int[,int[,int]]]\ ;}}
\label{cmd:print table}

Compact the table, and then print it out from the first to the second
  argument, in steps of the third argument.
If the first argument is negative, then the orders (if available) and
  representatives of the cosets are printed also.
The third argument defaults to one.
The one-argument form is equivalent to the two-argument form with the
  first argument used as the second and a first argument of \ttt{1}.
The no-argument form prints the entire table, without orders or
  representatives.

\subsection{\ttt{sc\ /\ stabil[ising\ cosets]\ :\ int\ ;}}
\label{cmd:sc}
\label{cmd:stabilizing coset}

This option prints out the stabilising cosets of the subgroup.
If $\ttt{int} > 0$, it prints the first \ttt{int} stabilising cosets.
If $\ttt{int} = 0$, it prints all of the stabilising cosets, plus their
  representatives.
If $\ttt{int} < 0$, it prints the first $|\ttt{int}|$ stabilising cosets,
  plus their representatives.

\subsection{\ttt{st[andard\ table]\ ;}}
\label{cmd:standard table}

This option compacts and then standardises the table.
That is, for a given ordering of the generators in the columns of the
  table, it produces a canonic table.
In such a table, a row-major scan encounters previously unseen cosets in
  numeric order; see Section~\ref{ex000} for an example.

\tsc{Notes:}
In a canonic table, the coset representatives are in length plus (column
  order) lexicographic order, and each is the minimum in this order.
Two tables are equivalent only if their canonic forms are the same.

\tsc{Guru Notes:}
In half of the ten standard enumeration strategies of Sims \cite{Sim}, the
  table is standardised repeatedly.
This is expensive computationally, but can result in fewer cosets being
  necessary.
The effect of doing this can be investigated in {\ace} by (repeatedly)
  halting the enumeration, standardising the table, and continuing.

\subsection{\ttt{tw\ /\ trace[\ word]\ :\ int,<word>\ ;}}
\label{cmd:tw}
\label{cmd:trace word}

Traces \ttt{word} through the coset table, starting at coset \ttt{int}.
Prints the final coset, if the trace completes.

%%%%%%%%%%%%%%%%%%%%%%%%%%%%%%%%%%%%%%%%%%%%%%%%%%%%%%%%%%%%%%%%%%%%%%%%%
\section{Technical Options for {\ace}}
\label{sec:tech}

The following options do not affect how the coset enumeration is done, but
how it uses the computer's resources. They might thus affect the runtime as
well as the range of problems that can be tackled on a given machine.

\subsection{\ttt{wo[rkspace]\ :\ [int[k/m/g]]\ ;}}
\label{cmd:workspace}
By default, {\ace} has a physical table size of $10^6$ words (i.e., 
  $4 \times 10^6$ bytes in the default 32-bit environment).
The number of cosets in the table is the table size divided by the number of
columns.  Although the number of cosets is limited to $2^{31}-1$ (if the C
{\tt int} type is 32 bits), the table size can exceed the $4$GByte 32-bit limit
if a suitable machine is used.

\paragraph{Notes:}
If the binary option is set (see the \ttt{opt} command), the multipliers
  are $1$, $2^{10}$\kern-2pt, $2^{20}$ and $2^{30}$ respectively.
The actual number of cosets in the table is $\mathrm{size}/4 -2$, rounded
  down to the nearest integer.
The $-2$ is to allow for possible rounding errors and the fact that coset
  \#0 is not used.

\subsection{\ttt{ti[me\ limit]:\ [-1/0/1..]\ ;}}
\label{cmd:time limit}
The \ttt{ti} command puts a time limit (in seconds) on the length of a run.
The default is 0 which is no time limit.
If the argument is $\ge0$ then the total elapsed time for this call is
checked at the end of each pass through the enumerator's main loop, and
if it's more than the limit the run is stopped and the current table
returned.

Note that a limit of $\ttt{0}$ performs exactly one pass through the main
  loop, since $\ttt{0} \ge \ttt{0}$.
If the enumerator is run in the continue mode, this allows a form of
  ``single-stepping''\kern-1.5pt.

The time limit is approximate, in the sense that the enumerator may run for
a longer, but never a shorter, time.  So, if there is, e.g., a big collapse
(so that the time round the loop becomes very long), then the run may run
over the limit by a large amount.

\paragraph{Notes:}
The time limit is CPU-time, not wall-time.
As in all timing under Unix, the clock's granularity (usually $10$ mSec)
  and the system load can affect the timing; so the number of main loop
  iterations in a given time may vary.

\subsection{\ttt{path[\ compression]\ :\ [0/1]\ ;}}
\label{cmd:path compression}
To correctly process multiple concidences, a union-find must be performed.
If both path compression and weighted union are used, then this can be
  done in essentially linear time (see, e.g., \cite{CLR}).
Weighted union alone, in the worst-case, is worse than linear, but is
  subquadratic.
In practice, path compression is expensive, since it involves many coset
  table accesses.
So, by default, path compression is turned off; it can be turned on by
  \ttt{path:1}.
It has no effect on the result, but may effect the running time and the
  internal statistics.

\paragraph{Guru Notes:}
The whole question of the best way to handle large coincidence forests is
  problematic.
Formally, {\ace} does not do a weighted union, since it is constrained to
  replace the higher-numbered of a coincident pair.
In practice, this seems to amount to much the same thing!
Turning path compression on cuts down the amount of data movement during
  coincidence processing at the expense of having to trace the paths and
  compress them.
In general, it does not seem to be worthwhile.

\subsection{\ttt{com[paction]\ :\ [0..100]\ ;}}
\label{cmd:compaction}
The key word \ttt{com} controls compaction of the coset table during an
  enumeration.
It sets the percentage of dead cosets needed to trigger
compaction. The default is 10 or 100, depending on the strategy used
(see section~\ref{sec:strat}).

Compaction recovers the space allocated to cosets which are flagged as dead
(i.e., which were found to be coincident with lower-numbered cosets).  It
results in a table where all the active cosets are numbered contiguously
from \#1, and with the remainder of the table available for new cosets.

The coset table is compacted when a coset definition is required, there is
  no space for a new coset available, and provided that the given 
  percentage of the coset table contains dead cosets.
For example, \ttt{com:20} means compaction will occur only if 20\% or more
  of the cosets in the table are dead.
The argument can be any integer in the range 0--100, and the default value
  is 10 or 100; see Section~\ref{sec:strat}.
An argument of 100 means that compaction is never performed, while an
  argument of 0 means always compact, no matter how few dead cosets there
  are (provided there is at least one, of course).

Compaction may be performed multiple times during an enumeration, and the
  table that results from an enumeration may or may not be compact,
  depending on whether or not there have been any coincidences since the
  last compaction (or from the start of the enumeration, if there have been
  no compactions).

If messaging is enabled (see~"Information About the Enumeration"), then a
progress message (labelled {\tt CO}) is printed each time the compaction routine
is actually called (as opposed to each time it is potentially called).

\paragraph{Notes:}
In some strategies (e.g., HLT) a lookahead phase may be run before
compaction is attempted.  In other strategies (e.g., {\tt Sims:3}) compaction may
be performed while there are outstanding deductions; since deductions are
discarded during compaction, a final lookahead phase will (automatically) be
performed.
%
Compacting a table ``destroys'' information and history, in the sense that
the coincidence list is deleted, and the table entries for any dead cosets
are deleted.

At Level 2, it is not possible to access the `data' in dead cosets; in
  fact, most options that require table data compact the table 
  automatically before they run.

\subsection{\ttt{max[\ cosets]\ :\ [0/2..]\ ;}}
\label{cmd:max cosets}
Sets the maximum number of cosets to be defined.
By default, all of the workspace is used, if necessary, in building the
coset table.  So the table size is an upper bound on how many cosets can be
active at any one time.  The {\tt max} option allows a limit to be placed on how
much of the physical table space is made available to the enumerator.
Enough space for at least two cosets (i.e., the subgroup and one other) must
be made available.  An argument of 0 selects all of the workspace.

%%%%%%%%%%%%%%%%%%%%%%%%%%%%%%%%%%%%%%%%%%%%%%%%%%%%%%%%%%%%%%%%%%%%%%%%%
\section{Information About the Enumeration and Experiments}
\label{sec:info}

\subsection{\ttt{mess[ages]\ /\ mon[itor]\ :\ [int]\ ;}}
\label{cmd:messages}
\label{cmd:monitor}
By default, or if the argument is \ttt{0}, {\ace} prints out only a single
  line of information giving the result of each enumeration.
It \ttt{mess} is non-zero then the presentation and the parameters are
  echoed at the start of the run, and messages on the enumeration's status
  as it progresses are also printed out.
The absolute value of \ttt{int} sets the frequency of the progress
  messages, with a negative sign turning hole monitoring on.
The initial printout of the presentation and the parameters is the same
  as that produced by the \ttt{sr:1} command; see Appendix~\ref{app:ex}
  for some examples.

The result line gives the result of the call to the enumerator and some
  basic statistics (see Appendix~\ref{app:ex} for some examples).
The possible results are given in Table~\ref{tab:rslts}; any result not
  listed represents an internal error and should be reported.
The statistics given are, in order: 
  \ttt{a}, number of active cosets; 
  \ttt{r}, number of applied cosets;
  \ttt{h}, first (potentially) incomplete row;
  \ttt{n}, next coset definition number; 
  \ttt{h} (if $\ttt{mess} < 0$), percentage of holes in the table;
  \ttt{l}, number of main loop passes;
  \ttt{c}, total CPU time;
  \ttt{m}, maximum active cosets;
  and \ttt{t}, total cosets defined.

\begin{table}
\hrule
\caption{Possible enumeration results}
\label{tab:rslts}
\smallskip
\renewcommand{\arraystretch}{0.875}
\begin{tabular*}{\textwidth}{@{\extracolsep{\fill}}lll} 
\hline\hline
result & level & meaning \\
\hline
\ttt{INDEX = x}         & 0 & finite index of \ttt{x} obtained \\
\ttt{OVERFLOW}          & 0 & out of table space \\
\ttt{SG PHASE OVERFLOW} & 0 & out of space (processing subgroup
				generators) \\
\ttt{ITERATION LIMIT}   & 0 & \ttt{loop} limit triggered \\
\ttt{TIME LIMT}         & 0 & \ttt{ti} limit triggered \\
\ttt{HOLE LIMIT}        & 0 & \ttt{ho} limit triggered \\
\ttt{INCOMPLETE TABLE}  & 0 & all cosets applied, but table has holes \\
\ttt{MEMORY PROBLEM}    & 1 & out of memory (building data structures) \\
\hline\hline
\end{tabular*}
\end{table}

The progress message lines consist of an initial tag, some fixed
  statistics, and some variable statistics.
The possible message tags are listed in Table~\ref{tab:prog}, along
  with their meanings.
The tags indicate the function just completed by the enumerator.
The tags with a `y' in the `action' column represent functions which are
  aggregated and counted.
Every time this count overflows the value of \ttt{mess}, a message line
  is printed and the count is zeroed.
Those tags flagged with a `y*' are only present if the appropriate option
  has been included in the build (see the \ttt{opt} command).
Tags with an `n' in the `action' column are not counted, and cause a
  message line to be output every time they occur.
They also cause the action count to be reset.

The fixed portion of the statistics consists of the \ttt{a}, \ttt{r}, 
  \ttt{h}, \ttt{n}, \ttt{h}, \ttt{l} and \ttt{c} values, as for the
  result line, except that \ttt{c} is the time since the last message
  line. 
The variable portion of the statistics can be:
  the \ttt{m} and \ttt{t} values, as for the result line;
  \ttt{d}, the current size of the deduction stack;
  \ttt{s}, \ttt{d} and \ttt{c} (with \ttt{DS} tag), the new stack size,
    the non-redundant deductions retained, and the redundant deductions
    discarded.

\begin{table}
\hrule
\caption{Possible progress messages}
\label{tab:prog}
\smallskip
\renewcommand{\arraystretch}{0.875}
\begin{tabular*}{\textwidth}{@{\extracolsep{\fill}}lll} 
\hline\hline
message & action & meaning \\
\hline
\ttt{AD} & y  & coset \#1 application definition 
			(\ttt{SG}/\ttt{RS} phase) \\
\ttt{RD} & y  & R-style definition \\
\ttt{RF} & y  & row-filling definition \\
\ttt{CG} & y  & immediate gap-filling definition \\
\ttt{CC} & y* & coincidence processed \\
\ttt{DD} & y* & deduction processed \\
\ttt{CP} & y  & preferred list gap-filling definition \\
\ttt{CD} & y  & C-style definition \\
\ttt{Lx} & n  & lookahead performed (type \ttt{x}) \\
\ttt{CO} & n  & table compacted \\
\ttt{CL} & n  & complete lookahead (table as deduction stack) \\
\ttt{UH} & n  & updated completed-row counter \\
\ttt{RA} & n  & remaining cosets applied to relators \\
\ttt{SG} & n  & subgroup generator phase \\
\ttt{RS} & n  & relators in subgroup phase \\
\ttt{DS} & n  & stack overflowed (compacted and doubled) \\
\hline\hline
\end{tabular*}
\end{table}

\tsc{Notes:}
Hole monitoring is expensive, so don't turn it on unless you really need
  it.
If you wish to print out the presentation and the parameters, but not
  the progress messages, then set \ttt{mess} non-zero, but very large.
(You'll still get the \ttt{SG}, \ttt{DS}, etc.\@ messages, but not the
  \ttt{RD}, \ttt{DD}, etc.\@ ones.)
You can set \ttt{mess} to \ttt{1}, to monitor all enumerator actions, but
  be warned that this can yield very large output files.

\subsection{\ttt{aep\ :\ 1..7\ ;}}
\label{cmd:aep}
The \ttt{aep} (all equivalent presentations) option runs an enumeration
for each
possible combination of relator ordering, relator rotations, and relator
inversions.

As discussed by Cannon, Dimino, Havas and Watson \cite{CDHW} and Havas
  and Ramsay \cite{HR1} such equivalent presentations can yield large
  variations in the number of cosets required in an enumeration.
For this command, we are interested in this variation.

The {\tt aep} option first performs a ``priming run'' using the parameters
  as they stand.
In particular, the {\tt asis} and {\tt mess} parameters are honoured.
It then turns {\tt asis} on and {\tt mess} off, and generates and tests
  the requested equivalent presentations.
The maximum and minimum values attained by {\tt maxcos} and {\tt totcos}
  are tracked, and each time a new {\tt record} is found the relators used and
  the summary result line is printed.
At the end, some additional status information is printed: 
  the number of runs which yielded a finite index; 
  the total number of runs (excluding the priming run); 
  and the range of values observed for {\tt maxcos} and {\tt totcos}.
The mandatory argument is considered as a binary number.
Its three bits are treated as flags, and control relator rotations (the
  $2^0$ bit), relator inversions (the $2^1$ bit) and relator orderings
  (the $2^2$ bit); ``1'' means ``active'' and ``0'' means
  ``inactive''.
The order in which the equivalent presentations are generated and tested
  has no particular significance, but note that the presentation as given
  (\kern-1.5pt\tit{after} the initial priming run) is the \tit{last}
  presentation to be generated and tested, so that the group's relators are
  left `unchanged' by running the \ttt{aep} option.

As an example (drawn from the discussion in \cite{HR1}) consider the index
  $448$ enumeration of $(8,7 \mid 2,3) / \langle a^2,Ab \rangle$,
  where $$ (8,7 \mid 2,3) 
    = \langle a,b \mid a^8 = b^7 = (ab)^2 = (Ab)^3 = 1 \rangle . $$
There are $4!=24$ relator orderings and $2^4=16$ combinations of relator or
  inverted relator.
Exponents are taken into account when rotating relators, so the relators
  given give rise to 1, 1, 2 and 2 rotations respectively, for a total
  of $1.1.2.2=4$ combinations.
So the command \ttt{aep:7} would generate and test $24.16.4=1536$ 
  equivalent presentations, while \ttt{aep:3} would generate and test 
  $16.4=64$ equivalent presentations.

\paragraph{Notes:}
There is no way to stop the \ttt{aep} option before it has completed,
  other than killing the task.
So do a reality check beforehand on the size of the search space and the 
  time for each enumeration.
If you are interested in finding a `good' enumeration, it can be very
  helpful, in terms of running time, to put a tight limit on the number of
  cosets via the \ttt{max} parameter.
(You may also have to set \ttt{com:100} to prevent time-wasting attempts
  to recover space via compaction.)
This maximises throughput by causing the `bad' enumerations, which are in
  the majority, to overflow quickly and abort.
If you wish to explore a very large search-space, consider firing up many
  copies of \ace, and starting each with a `random' equivalent
  presentation.
Alternatively, you could use the \ttt{rep} command.

\subsection{\ttt{rep\ :\ 1..7[,int]\ ;}}
\label{cmd:rep}
The {\tt rep} (random equivalent presentations) option complements the
  `aep' option.
It generates and tests some random equivalent presentations.
The mandatory argument acts as for {\tt aep}.
It is also possible to set the number of presentations used (a default of
eight) by using the extended syntax {\tt rep:=[<val>,<number>]}. 

The routine first turns {\tt asis} on and {\tt mess} off, and then 
  generates and tests the requested equivalent presentations.
For each presentation the relators used and the summary result line are
  printed.

\paragraph{Notes:}
The relator inversions and rotations are `genuinely' random.
The relator permuting is a little bit of a kludge, with the `quality' of
  the permutations tending to improve with successive presentations. 
When the \ttt{rep} command completes, the presentation active is the 
  \tit{last} one generated.

\paragraph{Guru Note:}
It might appear that neglecting to restore the original presentation is an
  error.
In fact, it is a useful feature!
Suppose that the space of equivalent presentations is too large to
  exhaustively test.
As noted in the entry for \ttt{aep}, we can start up multiple copies of
  \ttt{aep} at random points in the search-space.
Manually generating `random' equivalent presentations to serve as
  starting-points is tedious and error-prone.
The \ttt{rep} option provides a simple solution; simply run \ttt{rep:7}
  before \ttt{aep:7}!

\subsection{\ttt{rc\ /\ random\ coinc[idences]:\ int[,int]\ ;}}
\label{cmd:rc}
\label{cmd:random coincidences}

This option attempts to find nontrivial subgroups with index a multiple
  of the first argument by repeatedly putting random cosets coincident
  with coset \#1 and seeing what happens.
The starting coset table must be non-empty, but need not be complete.
The second argument puts a limit on the number of attempts, with a default
  of eight.
For each attempt, we repeatedly add random coset representatives to the 
  subgroup and redo the enumeration.
If the table becomes too small, the attempt is aborted, the original 
  subgroup generators restored, and another attempt made.
If an attempt succeeds, then the new set of subgroup generators is
  retained.

\tsc{Guru Notes:}
A coset can have many different representatives.
Consider running \ttt{st} before \ttt{rc}, to canonicise the table and the
  representatives.

\newpage
\section{Alphabetic Index of all Commands}
\label{sec:alphind}

{\small
\begin{minipage}[b]{6cm}
\begin{tabular}{ll}
\#&\ref{cmd:hash}\\
add generators&\ref{cmd:add generators}\\
add relators&\ref{cmd:add relators}\\
aep&\ref{cmd:aep}\\
ai&\ref{cmd:ai}\\
alter input&\ref{cmd:alter input}\\
alter output&\ref{cmd:alter output}\\
ao&\ref{cmd:ao}\\
asis&\ref{cmd:asis}\\
begin&\ref{cmd:begin}\\
cc&\ref{cmd:cc}\\
cfactor&\ref{cmd:cfactor}\\
check&\ref{cmd:check}\\
compaction&\ref{cmd:compaction}\\
contiguous&\ref{cmd:contiguous}\\
continue&\ref{cmd:continue}\\
coset coincidence\quad&\ref{cmd:coset coincidence}\\
ct factor&\ref{cmd:ct factor}\\
cycles&\ref{cmd:cycles}\\
ded mode&\ref{cmd:ded mode}\\
ded size&\ref{cmd:ded size}\\
default&\ref{cmd:default}\\
del generators&\ref{cmd:del generators}\\
del relators&\ref{cmd:del relators}\\
dmode&\ref{cmd:dmode}\\
dsize&\ref{cmd:dsize}\\
dump&\ref{cmd:dump}\\
eas&y\ref{cmd:easy}\\
echo&\ref{cmd:echo}\\
enumeration&\ref{cmd:enumeration}\\
felsch&\ref{cmd:felsch}\\
ffactor&\ref{cmd:ffactor}\\
fill factor&\ref{cmd:fill factor}\\
generators&\ref{cmd:generators}\\
group generators&\ref{cmd:group generators}\\
group relators&\ref{cmd:group relators}\\
gr&\ref{cmd:gr}\\
hard&\ref{cmd:hard}\\
help&\ref{cmd:help}\\
hlt&\ref{cmd:hlt}\\
hole limit&\ref{cmd:hole limit}\\
lookahead&\ref{cmd:lookahead}\\
loop limit&\ref{cmd:loop limit}\\
\end{tabular}
\end{minipage}
\begin{minipage}[b]{6cm}
\begin{tabular}{ll}
max cosets&\ref{cmd:max cosets}\\
mendelsohn&\ref{cmd:mendelsohn}\\
messages&\ref{cmd:messages}\\
mode&\ref{cmd:mode}\\
monitor&\ref{cmd:monitor}\\
nc&\ref{cmd:nc}\\
no relators&\ref{cmd:no relators}\\
normal closure&\ref{cmd:normal closure}\\
oo&\ref{cmd:oo}\\
options&\ref{cmd:options}\\
order option&\ref{cmd:order option}\\
parameters&\ref{cmd:parameters}\\
path compression&\ref{cmd:path compression}\\
pd mode&\ref{cmd:pd mode}\\
pd size&\ref{cmd:pd size}\\
pmode&\ref{cmd:pmode}\\
print details&\ref{cmd:print details}\\
print table&\ref{cmd:print table}\\
psize&\ref{cmd:psize}\\
pure ct&\ref{cmd:pure ct}\\
pure rt&\ref{cmd:pure rt}\\
random coincidences\quad&\ref{cmd:random coincidences}\\
rc&\ref{cmd:rc}\\
recover&\ref{cmd:recover}\\
redo&\ref{cmd:redo}\\
relators&\ref{cmd:relators}\\
rep&\ref{cmd:rep}\\
restart&\ref{cmd:restart}\\
rfactor&\ref{cmd:rfactor}\\
row filling&\ref{cmd:row filling}\\
rt factor&\ref{cmd:rt factor}\\
sc&\ref{cmd:sc}\\
sims&\ref{cmd:sims}\\
stabilizing coset&\ref{cmd:stabilizing coset}\\
standard table&\ref{cmd:standard table}\\
statistics&\ref{cmd:statistics}\\
style&\ref{cmd:style}\\
subgroup generators&\ref{cmd:subgroup generators}\\
subgroup name&\ref{cmd:subgroup name}\\
system&\ref{cmd:system}\\
text&\ref{cmd:text}\\
time limit&\ref{cmd:time limit}\\
trace word&\ref{cmd:trace word}\\
tw&\ref{cmd:tw}\\
workspace&\ref{cmd:workspace}\\
\end{tabular}
\end{minipage}
}
