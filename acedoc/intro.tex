
% intro.tex - Colin Ramsay - 16 Jun 99
%
% The introductory chapter.
%
%   5   10   15   20   25   30   35   40   45   50   55   60   65   70   75
% ..|....|....|....|....|....|....|....|....|....|....|....|....|....|....|

\ace\ is designed to work with partial tables, as well as complete tables
  exhibiting a finite index.
TBA:
Intended user groups ...

\ace\ is divided into three `levels'\kern-1.5pt.
The actual enumerator, called the ``core enumerator''\kern-2pt, is \ace\ 
  Level 0, while the standard driver for the enumerator, the ``core 
  wrapper''\kern-2pt, is \ace\ Level 1.
A stand-alone `example' application, called the ``interactive 
  interface''\kern-2pt, is \ace\ Level 2.
To assist those interested in the actual source code, the function and 
  variable names are prepended with \ttt{AL0\_}\kern+1pt, \ttt{AL1\_} \amp
  \ttt{AL2\_} respectively.
%
\ace\ also includes the ``proof table'' package (PT for short), which can
  be compiled into the executable if required.
The proof table cuts across the level structure, and can only be used as
  part of the interactive interface.
Function and variable names of the PT package are prepended with 
  \ttt{PT\_}\kern+1pt.

TBA: version history, 3.000 vs 3.001, ...
TBA: default build ...

\section{Administrivia}

It is assumed that \ace\ is run on a Unix-box of some description.

In order not to clutter-up the body of the text with examples, the bulk of
  these are gathered into a separate appendix.
These examples illustrate many of the features of \ace, and can also serve
  as a source of interesting enumerations.
Some are referred to in the text, but they can all be read independently.
\ace\ script input generating these examples is available in the 
  \ttt{ex---.in} files, as part of the documentation.

\section{Code}

You will note in the source code various sections preceded by a warning
  comment containing the \ttt{DTT} acronym.
This stands for ``debug/test/trace''\kern-2pt, and denotes code that was
  added temporarily for one reason or another.
None of this code should be active; i.e., it should all be commented out.
It does \tit{not} form part of the \ace\ distribution.
Of course, gurus will find this code intriguing, and will probably want
  to uncomment it to see what happens!

TBA:
The source code is heavily commented, and is considered to be part of the
  documentation.
Conceptually, coset enumeration is easy, but there are tricky details and
  subtle performance issues -- you need to read the (commented) code, and
  think, to appreciate these.
